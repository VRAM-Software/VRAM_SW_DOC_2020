\section{Informazioni generali}
    \subsection{Informazioni incontro}
        \begin{itemize}
 	    \item \textbf{Luogo}: Dipartimento di Matematica "Tullio Levi-Civita";
		\item \textbf{Data}: 2020-02-18;
		\item \textbf{Ora d'inizio}: 14.30;
		\item \textbf{Ora di fine}: 16.00;
		\item \textbf{Partecipanti}: 
		\begin{itemize}
			\item Corrizzato Vittorio;
			\item Dalla Libera Marco;
			\item Rampazzo Marco;
			\item Santagiuliana Vittorio;
			\item Schiavon Rebecca;
			\item Spreafico Alessandro;
			\item Toffoletto Massimo.
        \end{itemize}
        \end{itemize}
    \subsection{Argomenti trattati}
		Durante l'incontro i componenti del gruppo \textit{VRAM Software} hanno discusso riguardo alla correzione dei casi d'uso\glosp in seguito al confronto con il prof. Cardin:
        \begin{enumerate}
        	\item differenze tra UC13.2 e UC13.3;
        	\item schemi generali dei casi d'uso\glo;
            \item R1V1.2: requisiti formato del file;
            \item R1V4: software su un repository pubblico;
            \item UC1.1 e sotto-casi;
            \item UC1.4: conclusione addestramento;
            \item UC5 e UC11.
        \end{enumerate}
\section{Verbale}
	\subsection{Punto 1}
	Il prof. Cardin ha spiegato che per correggere l'errore segnalato in sede di revisione dei requisiti è necessario migliorare le descrizioni di UC13.2 e UC13.3 per sottolineare le differenze. Il gruppo procederà quindi a spiegare tali casi d'uso\glosp con maggior precisione.
	
    \subsection{Punto 2}
    Il prof. Cardin ha fatto notare che nei diagrammi che il gruppo \textit{VRAM Software} chiama "schemi generali" viene usata una notazione formale in maniera informale, questo non è accettabile. Il gruppo ha deciso quindi di procedere con la rimozione di suddetti schemi.
    
    \subsection{Punto 3}
    Riguardo a R1V1.2: il fatto che l'applicazione permetta all'utente di fornire dati in formato JSON è una funzionalità e non un vincolo. Un requisito di vincolo è una limitazione dell'applicativo (ad esempio la compatibilità con una certa versione di Firefox) cosa che non è il formato del file in input.
    
    \subsection{Punto 4}
    Il prof. Cardin ha spiegato perché R1V4 è un requisito di qualità: mettere il codice su un repository riguarda il processo, non apporta miglioramenti tangibili sul prodotto. Inoltre ha specificato che R1V4.1 non è in gerarchia con R1V4 e quindi va rinominato.
    
    \subsection{Punto 5}
    I sotto-casi di UC1.1 sono scarsamente informativi, perciò il gruppo, anche a seguito del consiglio del prof. Cardin, ha deciso di rimuoverli.
    
    \subsection{Punto 6}
    La conclusione dell'addestramento non può essere vista come una funzionalità, ma il sotto UC1.4.1 (visualizzazione del messaggio di conferma) sì e quindi deve diventare un caso d'uso\glosp a parte. L'interruzione dell'addestramento invece deve essere descritto in maniera diversa.
    
    \subsection{Punto 7}
    Il prof. Cardin ha chiarito le correzioni su UC5 e UC11: sono troppo poco informativi, bisogna quindi migliorare la descrizione e integrare con dei sotto-casi.
    
    
    
    
  
       