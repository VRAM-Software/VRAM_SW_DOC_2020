\section{Informazioni generali}
    \subsection{Informazioni incontro}
        \begin{itemize}
            \item \textbf{Luogo}: Dipartimento di Matematica "Tullio Levi-Civita";
            \item \textbf{Data}: 2020-01-10;
            \item \textbf{Ora d'inizio}: 10.00;
            \item \textbf{Ora di fine}: 11.00;
            \item \textbf{Partecipanti}: \begin{itemize}
                \item Corrizzato Vittorio;
                \item Dalla Libera Marco;
                \item Toffoletto Massimo;
                \item Prof. Vardanega Tullio (committente).
            \end{itemize}
        \end{itemize}
    \subsection{Argomenti trattati}
        Durante la sesta e ultima riunione prima della RR, il gruppo ha posto delle domande riguardanti il progetto\glosp al committente, segue un riassunto degli argomenti trattati:
        \begin{enumerate}
            \item chiarimenti sulla struttura dello \textit{Studio di Fattibilità};
            \item chiarimenti sullo scopo degli use case;
            \item chiarimenti sui riferimenti e i processi formativi;
            \item chiarimenti sui requisiti prestazionali;
            \item chiarimenti sulle \textit{Norme di Progetto} in particolare la normazione dei processi\glosp di sviluppo.
        \end{enumerate}
\section{Verbale}
    \subsection{Punto 1}
        Come primo punto sono state discusse le tematiche dello \textit{Studio di Fattibilità}. Il committente si è raccomandato che nella stesura di questo documento dobbiamo aver agito in modo maturo e quindi, oltre all'analisi e alla comprensione del capitolato\glo, le motivazioni della nostra scelta devono rispecchiare un ragionamento professionale. Il gruppo ha quindi revisionato il documento e le motivazioni date ad ogni capitolato\glo, arrivando comunque alla scelta di C4.
    \subsection{Punto 2}
        Successivamente sono state chieste delucidazioni riguardo allo scopo degli use case\glosp poiché si erano venuti a creare dei dubbi durante la loro ideazione. Il professor Vardanega ha consigliato una discussione con il professor Cardin, che tratta questa parte nel corso di studi, rimarcando comunque il fatto che gli use case\glosp fungono da strumenti aggiuntivi per trovare i requisiti del capitolato\glo, non devono quindi essere troppo specifici in quanto sarà compito del progettista, in revisioni più avanzate del progetto\glo, definirli in modo implementativo, infatti è stato ricordato che le piattaforme che utilizziamo oggi potrebbero subire dei cambiamenti e invalidare i casi d'uso che devono invece essere sempre conformi. Il gruppo ha quindi cominciato una revisione degli use case\glosp attendendo comunque un riscontro anche dal professor Cardin.
    \subsection{Punto 3}
        In seguito è stato domandato un chiarimento riguardo ai riferimenti e ai processi\glosp formativi. Il committente ha spiegato che questo tipo di processi\glosp è molto difficile da normare in questa parte del progetto\glo, ma è comunque utile da tenere presente in quanto permette una migliore organizzazione ed allineamento tra i membri del gruppo. È infatti preferibile che non tutti i componenti studino uno strumento, ma che, al contrario, sia solo un gruppo ristretto a condividere le sue conoscenze. In ogni caso questo tipo di riferimenti danno solo una linea guida generale e non vincolano strettamente il loro utilizzo. Il gruppo aveva già applicato alcuni di questi consigli, proseguirà comunque con una revisione della parte interessata nelle \textit{Norme di Progetto v. 1.1.1}.
    \subsection{Punto 4}
        Poi è stata discussa la stesura dei requisiti prestazionali. Il professor Vardanega ha affermato che questo tipo di requisiti non dipendono dal nostro gruppo, ma dalle esigenze del proponente che può decidere se dare degli obblighi o meno. Bisogna tenere presente che alcune prestazioni potrebbero comunque essere influenzate dal dispositivo che viene utilizzato. Il gruppo ha quindi studiato l'incidenza del dispositivo utilizzato nelle prestazioni, mentre ulteriori domande verranno poste al proponente e riportate in un verbale esterno.
    \subsection{Punto 5}
        Infine è stata posta una domanda sulla normazione dei processi\glosp di sviluppo. Il committente ha sottolineato la difficoltà di stilare questa sezione in questa parte del progetto\glo, consiglia quindi un approccio che punti ad aggiornare costantemente le \textit{Norme di Progetto} quando si presenteranno effettivamente i suddetti processi\glosp da normare. Questo way of working è in ogni caso consigliato per tutto il documento che è uno dei più soggetto a modifiche durante il progetto\glo. Il gruppo ha revisionato le \textit{Norme di Progetto} demandando alcune scelte alle revisioni successive.