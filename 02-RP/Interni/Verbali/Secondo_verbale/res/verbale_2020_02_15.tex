\section{Informazioni generali}
    \subsection{Informazioni incontro}
        \begin{itemize}
            \item \textbf{Luogo}: Videochiamata tramite Skype;
            \item \textbf{Data}: 2020-02-15;
            \item \textbf{Ora d'inizio}: 9.00;
            \item \textbf{Ora di fine}: 12.00;
            \item \textbf{Partecipanti}: \begin{itemize}
                \item Corrizzato Vittorio;
                \item Dalla Libera Marco;
                \item Rampazzo Marco;
                \item Santagiuliana Vittorio;
                \item Schiavon Rebecca;
                \item Spreafico Alessandro;
                \item Toffoletto Massimo.
            \end{itemize}
        \end{itemize}
    \subsection{Argomenti trattati}
		Durante la riunione, i componenti hanno discusso riguardo ai seguenti argomenti:
        \begin{enumerate}
        	\item domande da porre al committente Prof. Riccardo Cardin;
        	\item discussione riguardante la technology baseline;
            \item domande da porre al proponente \textit{Zucchetti};
        \end{enumerate}
\section{Verbale}
    \subsection{Punto 1}
    	Il gruppo ha iniziato la riunione discutendo dei commenti, ricevuti nell'esito della revisione dei requisiti, riguardanti i casi d'uso\glosp e i requisiti. Il gruppo non ha compreso a fondo alcune correzioni e quindi ha redatto una lista di domande da porre al prof. Cardin in un successivo incontro.
            
    \subsection{Punto 2}
        Il gruppo ha in seguito discusso su come approcciare la technology baseline tenendo conto anche dei suggerimenti dati dal prof. Vardanega nel colloquio del 2020/02/10. Il gruppo ha quindi deciso di fare possibilmente due o tre PoC incrementali tali che:
        \begin{itemize}
        	\item l’applicativo di addestramento si occuperà soltanto di caricamento dei file e restituirà un file di addestramento già fatto;
        	\item il plug-in leggerà il file di addestramento, i dati (prestabiliti) e ritornerà una previsione già calcolata a mano.
        \end{itemize}
    	L'ultima PoC verrà presentata al prof. Cardin.
       
     \subsection{Punto 3}
       Il gruppo ha infine deciso di richiedere un incontro con il proponente \textit{Zucchetti} al fine di discutere sulla technology baseline e in particolare: cosa si aspetta dal gruppo in questo momento, se vanno bene le decisioni prese al punto 2 ed eventuali consigli sulle tecnologie da usare.