\section{Informazioni generali}
    \subsection{Informazioni incontro}
        \begin{itemize}
            \item \textbf{Luogo}: Dipartimento di Matematica "Tullio Levi-Civita";
            \item \textbf{Data}: 2019-12-02;
            \item \textbf{Ora d'inizio}: 14.00;
            \item \textbf{Ora di fine}: 16.30;
            \item \textbf{Partecipanti}: \begin{itemize}
                \item Corrizzato Vittorio;
                \item Dalla Libera Marco;
                \item Rampazzo Marco;
                \item Santagiuliana Vittorio;
                \item Schiavon Rebecca;
                \item Spreafico Alessandro;
                \item Toffoletto Massimo.
            \end{itemize}
        \end{itemize}
    \subsection{Argomenti trattati}
        Nel secondo incontro, i componenti del gruppo hanno discusso delle seguenti tematiche:
        \begin{enumerate}
            \item divisione dei ruoli;
            \item aggiornamento e revisione delle \textit{Norme di Progetto};
            \item discussione riguardante lo \textit{Studio di Fattibilità};
            \item scrittura di una serie di domande da porre all'incontro con \textit{Zucchetti};
            \item discussione generale dei problemi incontrati finora dal gruppo.
        \end{enumerate}
\section{Verbale}
    \subsection{Punto 1}
        Come primo argomento è stata discussa la divisione dei ruoli per continuare la stesura dei documenti. In questo frangente del progetto\glosp sono attivi quattro ruoli: responsabile, amministratore, analista e verificatore. Suddette cariche sono state suddivise in questo modo:
        \begin{itemize}
            \item \textbf{Responsabile}: Dalla Libera Marco;
            \item \textbf{Amministratore}: Toffoletto Massimo;
            \item \textbf{Analisti}: Spreafico Alessandro, Schiavon Rebecca e Santagiuliana Vittorio;
            \item \textbf{Verificatori}: Rampazzo Marco e Corrizzato Vittorio. 
        \end{itemize}
    \subsection{Punto 2}
        Successivamente si è passati a un aggiornamento e revisione delle \textit{Norme di Progetto}. Sono state quindi corrette e ampliate le sezioni riguardanti lo \textit{Studio di Fattibilità}, gli strumenti da utilizzare e la documentazione. Infatti, Spreafico Alessandro si è proposto per la creazione di un template \LaTeX\xspace per uniformare tra loro i documenti e facilitarne la scrittura.
    \subsection{Punto 3}
        Dopo la discussione avvenuta nel precedente incontro e un'iniziale normazione dello \textit{Studio di Fattibilità} è cominciata la stesura di suddetto documento. Il gruppo ha quindi approfittato di questo incontro per rivedere ed ampliare alcuni concetti riguardanti i vari capitolati\glo. Ne è anche seguita una divisione della redazione del documento che ha portato i seguenti risultati:
        \begin{itemize}
            \item Dalla Libera Marco: approvazione documento;
            \item Schiavon Rebecca: analisi dei capitolati\glosp C1, C2 e stesura dell'introduzione al documento;
            \item Spreafico Alessandro: analisi dei capitolati\glosp C3 e C4;
            \item Santagiuliana Vittorio: analisi dei capitolati\glosp C5 e C6;
            \item Corrizzato Vittorio: verifica dei capitolati\glosp C1, C2, C3 e dell'introduzione al documento;
            \item Rampazzo Marco: verifica dei capitolati\glosp C4, C5 e C6;
        \end{itemize}
    \subsection{Punto 4}
        Infine, in vista di un incontro con l'azienda proponente \textit{Zucchetti} il gruppo ha stilato delle domande per il chiarimento di alcuni dubbi inerenti alla progettazione\glosp del capitolato\glo. Tali temi verranno in seguito trattati su un verbale esterno.
