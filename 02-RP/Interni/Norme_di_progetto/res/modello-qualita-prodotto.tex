\subsubsection{Modello di qualità del prodotto} 
Per descrivere il modello di qualità del prodotto\glosp software il nostro gruppo ha deciso di utilizzare lo standard ISO/IEC 25010, che determina le caratteristiche della qualità di un prodotto\glosp software che verranno valutate. Il modello di qualità del prodotto\glo, definito nello standard, è suddiviso nelle seguenti caratteristiche:
	\paragraph{Idoneità funzionale} \mbox{}\\[1mm]
	Per adeguatezza funzionale si intende la capacità di un prodotto\glosp software di fornire funzioni che soddisfano i requisiti impliciti ed espliciti, quando usati in un determinato contesto sotto specifiche condizioni. \\
	Le metriche\glosp sono:
	\begin{itemize}
		\item \textbf{Copertura funzionale}: percentuale di requisiti soddisfatti rispetto a tutti quelli individuati e si calcola con la formula
		\[C_F=(1-N_{RNS}/N_{RI})*100\]
		dove N$_{RNS}$ sono i requisiti non sviluppati e N$_{RI}$ i requisiti trovati durante l'\textit{Analisi dei Requisiti}.
	\end{itemize}
	%\paragraph{Efficienza prestazionale} \mbox{}\\
	%L'efficienza prestazionale è la caratteristica relativa alle prestazioni rispetto al numero di risorse usate sotto condizioni precise.
	%\begin{itemize}
	%	\item \textbf{Comportamento rispetto al tempo}: la capacità del software di eseguire le funzioni in un tempo di risposta, tempo processionale e un volume di produzione che soddisfa i requisiti;
	%	\item \textbf{Utilizzo delle risorse}: la capacità del software di eseguire le funzioni usando un numero di risorse disponibili che soddisfa i requisiti;
	%	\item \textbf{Capacità}: la capacità del software di avere limiti prestazionali massimi che soddisfano i requisiti.
	%\end{itemize}
	%\paragraph{Compatibilità} \mbox{}\\
	%La compatibilità è la caratteristica di un software di scambiare informazioni con altri prodotti\glo, ed eseguire le sue funzioni mentre viene condiviso lo stesso ambiente hardware o software.
	%\begin{itemize}
	%	\item \textbf{Coesistenza}: la capacità del software di eseguire le sue funzionalità senza impatti negativi mentre condivide lo stesso ambiente con un altro software;
	%	\item \textbf{Interoperabilità}: la capacità di due o più software di scambiare informazioni tra loro e usare tali dati per i loro scopi.
	%\end{itemize}

	\paragraph{Affidabilità} \mbox{}\\[1mm]
	Un prodotto\glosp software è considerato affidabile quando mantiene un livello di prestazioni costanti per un determinato intervallo di tempo sotto precise condizioni che possono variare nel tempo.\\
	Le metriche\glosp sono:
	\begin{itemize}
		\item \textbf{Robustezza agli errori}: percentuale di errori critici messa sotto controllo. Si calcola con la formula
		\[R_E=N_{ER}/N_{T}*100\]
		dove N$_ER$ sono gli errori rilevati e N$_T$ sono i test eseguiti.
	\end{itemize}
	%\paragraph{Sicurezza} \mbox{}\\
	%Per sicurezza in un prodotto\glosp software si intende il grado di protezione che il software ha sui dati per evitare che persone o servizi non autorizzati li ricevano.
	%\begin{itemize}
	%	\item \textbf{Confidenzialità}: la capacità del software di assicurare la confidenzialità dei dati alle persone che hanno l'autorizzazione di accedere a tali informazioni;
	%	\item \textbf{Integrità}: la capacità del software di evitare modifiche o accessi non autorizzati;
	%	\item \textbf{Non ripudio}: la capacità del software di dimostrare che eventi o azioni sono avvenuti, in modo da non ripudiarli in un secondo momento;
	%	\item \textbf{Responsabilità}: la capacità del software di risalire a una sola entità a partire da una azione di una entità;
	%	\item \textbf{Autenticità}: la capacità del software di provare l'identità delle risorse utilizzate.
	%\end{itemize}
	\paragraph{Manutenibilità} \mbox{}\\[1mm]
	Per manutenibilità si intende la capacità di un prodotto\glosp software di essere modificato e aggiornato senza l'introduzione di problemi e anomalie in seguito a cambiamenti dell'ambiente e/o dei requisiti.
	Le metriche\glosp sono:
	\begin{itemize}
		\item \textbf{Semplicità delle funzioni}: numero di parametri che richiede ogni metodo
		\item \textbf{Modularità del prodotto}\glo: numero di metodi presenti per ogni classe.
	\end{itemize}
	%\paragraph{Portabilità} \mbox{}\\ 
	%Per portabilità si intende la caratteristica del software di essere utilizzato in modo efficacie quando esso viene trasferito su diversi hardware o software.
	%\begin{itemize}
	%	\item \textbf{Adattabilità}: la capacità del software di essere adattato per hardware o software in continuo cambiamento;
	%	\item \textbf{Installabilità}: la capacità del software di essere installato o disinstallato efficacemente in un ambiete specifico;
	%	\item \textbf{Sostituibilità}: la capacità del software di rimpiazzare un altro prodotto\glosp con lo stesso scopo nello stesso ambiente.
	%\end{itemize}
