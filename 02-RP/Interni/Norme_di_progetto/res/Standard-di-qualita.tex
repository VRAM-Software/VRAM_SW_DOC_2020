\appendix
\section{Standard di qualità}
	\subsection{ISO/IEC 25010}
		Lo standard internazionale ISO/IEC 25010 è utilizzato per misurare la qualità del software.
		Per farlo utilizza una serie di parametri e caratteristiche che possono essere utilizzate per valutare la qualità del software, grazie alle metriche\glosp fornite dallo standard ISO/IEC 25023.
		\subsubsection{Metriche di qualità interna}
			La qualità interna del software è relativa alle attività di progettazione\glosp e codifica, essa rappresenta la qualità della struttura e del codice del prodotto\glo.
			La misurazione della qualità interna viene effettuata principalmente attraverso l'analisi statica\glosp del codice.
			Questi controlli sono utili perché semplificano il lavoro dei programmatori e permettono di apportare miglioramenti strutturali al software senza la necessità di eseguirlo.
			Con un buon livello di qualità interna si ottiene una solida struttura del prodotto\glo, come base per la qualità esterna e in uso.
		\subsubsection{Metriche di qualità esterna}
			La qualità esterna del software è relativa alle proprietà dinamiche e comportamentali del software. Essa viene infatti misurata attraverso l'analisi dinamica, ovvero tramite l'esecuzione del codice e lo svolgimento di test.
		\subsubsection{Metriche di qualità in uso}
			La qualità in uso è probabilmente la più difficile da misurare poiché è valutata in un ambito di utilizzo reale, con la partecipazione degli utenti.
			Essa dipende dalla qualità interna e da quella esterna; infatti per avere una buona qualità in uso è necessario un buon livello di qualità interna ed esterna.
