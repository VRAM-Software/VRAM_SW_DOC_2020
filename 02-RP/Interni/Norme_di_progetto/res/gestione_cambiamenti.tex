\subsection{Gestione dei cambiamenti}
\subsubsection{Scopo}
Lo scopo di questo processo\glosp è fornire un metodo tempestivo, disciplinato e documentato per assicurare che tutti i cambiamenti, compresi i problemi, riscontrati nel corso del progetto\glosp, siano identificati, analizzati e risolti.
\subsubsection{Aspettative}
L'applicazione di questo processo\glosp deve comportare maggior tempestività ed efficacia nell'individuazione e nella risoluzione dei problemi, con un conseguente aumento di efficacia ed efficienza del processo di verifica. Inoltre questo provesso\glosp deve permettere di individuare i problemi maggiormente ricorrenti.
%grazie anche al tracciamento dei problemi più frequenti.
\subsubsection{Descrizione} 
Il processo\glosp di gestione dei cambiamenti analizza e risolve tutti i cambiamenti programmati e improvvisi, quali i problemi, riscontrati durante l'esecuzione dei processi\glosp di sviluppo e di fornitura, indipendentemente dalla loro natura e dalla loro origine. 
\subsubsection{Attività}
\paragraph{Implementazione del processo}\mbox{}\\ [1mm]
Deve essere istanziato il processo\glosp di gestione dei cambiamenti per gestire ogni nuova modifica, comprese le non conformità ed i problemi non pianificati, rilevato nei prodotti\glosp software o nelle attività. Questo processo\glosp deve essere conforme alle seguenti proprietà:
\begin{itemize}
	\item il processo\glosp deve essere circolare e chiuso, assicurando che:
	\begin{itemize} 
		\item tutti i cambiamenti siano prontamente segnalati e gestiti tramite il processo di risoluzione dei cambiamenti;
		\item i cambiamenti vengano presi in carico e gestiti;
		\item vengano inviate delle notifiche per informare gli interessati della presenza dell'eventuale problema;
		\item le cause del problema vengano identificate, analizzate e, dove possibile, eliminate;
		\item la risoluzione e le decisioni prese siano archiviate e storicizzate;
		\item lo stato del cambiamento sia tracciato, aggiornato e comporti delle notifiche al cambio di stato;
		\item venga mantenuto un registro di tutti i problemi riscontrati.
	\end{itemize}
	\item Il processo\glosp definisce uno schema per categorizzare e dare priorità ai cambiamenti. Questo schema è composto da:
	\begin{itemize} 
		\item \textbf{Identificativo}: codice numerico progressivo per identificare il singolo cambiamento;
		\item \textbf{Tipologia}: denota il tipo di cambiamento e può essere:
		\begin{itemize}
			\item \textbf{Editoriale}: cambiamento che comporta modifiche editoriali\glosp sulla documentazione;
			\item \textbf{Tecnico}: cambiamento che comporta modifiche tecniche\glosp sulla documentazione;
			\item \textbf{Funzionale}: cambiamento che influenza le caratteristiche funzionali del prodotto\glosp software;
			\item \textbf{Conformità}: cambiamento che comporta violazioni di conformità del prodotto\glo software;
			\item \textbf{Validazione}: cambiamento riscontrato durante il processo\glosp di Validazione\glosp del prodotto\glosp software.
		\end{itemize}
		\item \textbf{Priorità}: può essere:
		\begin{itemize}
			\item bloccante;
			\item urgente;
			\item alta;
			\item media;
			\item bassa.
		\end{itemize}
		\item \textbf{Stato}: può essere:
		\begin{itemize}
			\item da fare;
			\item in corso;
			\item completato.
		\end{itemize}
	\end{itemize}
	\item I cambiamenti rilevati ed archiviati vengono analizzati al fine di identificare quelli più frequenti ed eventuali tendenze;
	\item La modalità di gestione e di risoluzione individuate per i cambiamenti e per i problemi sono analizzate e valutate al fine di verificare che siano stati effettivamente risolti e che le modifiche siano state correttamente implementate nei prodotti\glosp software e nelle attività. Inoltre l'analisi deve permettere di individuare dei pattern di risoluzione ed aiutare a determinare quando vengono introdotti ulteriori problemi.			
\end{itemize}

\paragraph{Risoluzione dei problemi}\mbox{}\\ [1mm]
Quando dei cambiamenti, comprese le non conformità ed i problemi, sono rilevati in un prodotto\glosp software o in un'attività, deve essere realizzato un report che li descriva. Questo report deve essere usato come componente del ciclo chiuso del punto §3.6.4.1: dall'individuazione dei cambiamenti al processo di indagine, analisi e risoluzione, fino all'individuazione della causa ed alla sua analisi per individuare le tendenze.
Per svolgere quanto descritto, ci si appoggia all'Issue Tracking System di GitHub. Esso consente di aggiungere delle issues che corrispondono ai cambiamenti o ai problemi che si devono svolgere e di assegnarle ad incaricati e responsabili. Dunque viene assegnata un'istanza dello schema precedentemente descritto e, in base alle priorità, si procederà alla loro risoluzione.

\subsubsection{Metriche di qualità}
Le metriche di qualità utilizzate per valutare la qualità della gestione dei cambiamenti sono:
\begin{itemize}
	\item \textbf{M20 Tempo medio di risoluzione degli errori}: indica il tempo medio di risoluzione degli errori segnalati. Se elevato, questo valore indica che la gestione dei cambiamenti non è adeguata;
	\begin{itemize}
		\item[] \textbf{formula}: $\frac{TRE}{NE}$ dove $TRE$ indica il tempo totale per la risoluzione degli errori e $NE$ il numero di errori risolti.
	\end{itemize} 
\end{itemize}		
