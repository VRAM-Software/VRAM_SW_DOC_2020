\section{Creazione diagrammi UML}
I diagrammi UML vengono vengono creati con il servizio online Lucidchart. Bisogna registrarsi e richiedere l'account studenti che rende disponibili più funzionalità rispetto a quello base. Una volta inserite le credenziali di accesso si deve procedere così:
\begin{itemize}
	\item selezionare il bottone "+ Document";
	\item realizzare il diagramma come normato in §\ref{UML};
	\item andare nella sezione "My Documents", passare con il puntatore sopra al diagramma appena creato;
	\item selezionare il bottone "Share" e poi "Get shareable link";
	\item incollare in link dentro un apposito documento dentro la cartella Google Drive del gruppo;
	\item una volta verificati, i diagrammi sono esportati in formato png e inseriti nei documenti.
\end{itemize}