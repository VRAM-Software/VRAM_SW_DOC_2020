\section{Utilizzo di Git}
Il repository\glosp Git è composto di varie sezioni chiamate \textit{branch}. Il loro scopo è garantire lo sviluppo isolato e parallelo di diversi compiti. Il branch principale si chiama master in esso si trovano tutte le funzionalità che sono state implementate ed approvate tramite verifica. Ad ogni nuova funzionalità che si vuole implementare, è necessario creare un nuovo branch e lavorare dentro di esso. Quando si è conclusa l'implementazione di una funzionalità, essa deve essere verificata e ciò viene fatto tramite il meccanismo chiamato pull request. Questo meccanismo impone una verifica obbligatoria da parte di un verificatore. Solo quando egli approverà la funzionalità, sarà possibile effettuare il merge nel ramo master.
Qui vengono elencate le regole da seguire per l'utilizzo di Git:
\begin{itemize}
	\item non si lavora mai sul branch master, in quanto nel master vanno solo i contenuti verificati;
	\item si deve usare un branch dedicato alla funzionalità su cui si sta lavorando;
	\item una volta verificato il contenuto si può procedere al merge nel branch master;
	\item prima di iniziare a lavorare bisogna scaricare in locale le modifiche effettuate al repository\glosp dagli altri componenti del gruppo;
	\item bisogna inserire un messaggio di commit chiaro ed esplicativo.
\end{itemize}