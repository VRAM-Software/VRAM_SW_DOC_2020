\section{Utilizzo di Git}
Il repository Git è composto di varie sezioni chiamate \textit{branch}. Il branch principale si chiama master.
Qui vengono elencate le regole da seguire per l'utilizzo di Git:
\begin{itemize}
	\item non si lavora mai sul branch master, in quanto nel master vanno solo i contenuti verificati;
	\item bisogna scegliere sempre il branch più appropriato su cui lavorare, e in caso non ci sia crearlo;
	\item prima di iniziare a lavorare bisogna scaricare in locale le modifiche effettuate al repository dagli altri componenti del gruppo;
	\item una volta verificato il contenuto questo può essere messo nel branch master;
	\item bisogna inserire un messaggio di commit chiaro ed esplicativo.
\end{itemize}