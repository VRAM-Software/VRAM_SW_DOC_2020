\section{Qualità di prodotto}
    Per misurare la qualità di prodotto\glosp il gruppo ha deciso di prendere come riferimento lo standard ISO/IEC 25010 che definisce un modello di qualità del prodotto\glosp attraverso un insieme di caratteristiche e pone le basi per lo standard ISO/IEC 25023 che si occupa della misurazione di queste specifiche. Di seguito sono elencate le voci che il gruppo ha ritenuto importanti in questo frangente del progetto\glo.
    \subsection{Documenti}
    	\subsubsection{Obiettivi}
    		\begin{itemize}
    			\item \textbf{Leggibilità}: grado di facilità con cui un documento viene letto;
    			\item \textbf{Correttezza}: grado di errori ortografici presenti nel documento.
    		\end{itemize}
	    \subsubsection{Metriche}
	    \textbf{M15 Indice di Gulpease}
	    \begin{itemize}
	    	\item \textbf{Valore preferibile}: 60 $\le I_{G} \le$ 100;
	    	\item \textbf{Valore accettabile}: 40 $\le I_{G} \le$ 100.
	    \end{itemize}
	    %\textbf{M09 correttezza ortografica}
	    %\begin{itemize}
	    %	\item \textbf{Valore preferibile}: 0;
	    %	\item \textbf{Valore accettabile}: 0.
	    %\end{itemize}
    \subsection{Copertura funzionale}
        \subsubsection{Obiettivi}
            \begin{itemize}
                \item \textbf{Completezza}: grado con cui l'insieme di funzioni copre tutte le specifiche attività e gli obiettivi dell'utente;
                \item \textbf{Correttezza}: grado con cui un prodotto\glosp o un sistema fornisce con la giusta precisione il risultato corretto;
                \item \textbf{Adeguatezza}: grado con cui le funzioni facilitano il compimento di attività e obiettivi specifici.
            \end{itemize}
        \subsubsection{Metriche}
            \textbf{M16 Percentuale di requisiti obbligatori soddisfatti}
                \begin{itemize}
                    \item \textbf{Valore preferibile}: 100\%;
                    \item \textbf{Valore accettabile}: 100\%.
                \end{itemize}
            \textbf{M17 Percentuale di requisiti desiderabili soddisfatti}
            \begin{itemize}
            	\item \textbf{Valore preferibile}: $\ge65\%$;
            	\item \textbf{Valore accettabile}: $\ge0\%$;
            \end{itemize}
        	\textbf{M18 Percentuale di requisiti opzionali soddisfatti}
        	\begin{itemize}
        		\item \textbf{Valore preferibile}: $\ge50\%$;
        		\item \textbf{Valore accettabile}: $\ge0\%$;
        	\end{itemize}
        	%\textbf{M24 Percentuale di test passati}
        	%\begin{itemize}
        	%	\item \textbf{Valore preferibile}: $100\%$;
        	%	\item \textbf{Valore accettabile}: $\ge 80\%$.
        	%\end{itemize}
    \subsection{Usabilità}
        \subsubsection{Obiettivi}
            \begin{itemize}
                \item \textbf{Apprendibilità}: grado con cui il prodotto\glosp o il sistema può essere appreso con efficacia, efficienza e soddisfazione da uno specifico utente;
                \item \textbf{Appropriatezza-Riconoscibilità}: grado con cui gli utenti possono riconoscere che un determinato prodotto\glosp o sistema è appropriato per i propri bisogni.
            \end{itemize}
        %\subsubsection{Metriche}
         %   \textbf{Completezza della documentazione}: percentuale delle funzioni descritta nella documentazione con un dettaglio tale da consentire all’utente di utilizzarle.
          %      \begin{itemize}
           %         \item \textbf{Misurazione}: $C_{DOC}=(N_{FD}/N_{FI})*100$ \\
            %        dove N$_{FD}$ sono le funzioni definite sulla documentazione e N$_{FI}$ sono le funzioni individuate nella documentazione;
             %       \item \textbf{Valore preferibile}: 100\%;
              %      \item \textbf{Valore accettabile}: 100\%.
               % \end{itemize}
            %\textbf{Completezza di descrizione}: percentuale degli scenari d’uso descritta nella documentazione effettivamente presenti nel prodotto\glosp finale.
             %   \begin{itemize}
               %     \item \textbf{Misurazione}: $C_{DESC}=(N_{UCI}/N_{UCE})*100$ \\
                %    dove N$_{UCI}$ è il numero di casi d'uso\glosp individuati e N$_{UCE}$ è il numero di casi d'uso effettivi del prodotto\glo;
                 %   \item \textbf{Valore preferibile}: 100\%;
                  %  \item \textbf{Valore accettabile}: 100\%.
                %\end{itemize}
    \subsection{Affidabilità}
        \subsubsection{Obiettivi}
            \begin{itemize}
                \item \textbf{Maturità}: grado con cui un sistema, un prodotto\glosp o un componente è affidabile durante le normali condizioni di servizio;
                \item \textbf{Tolleranza agli errori}: grado con cui un sistema, un prodotto\glosp o un componente riesce ad operare anche in presenza di errori hardware o software.
            \end{itemize}
        %\subsubsection{Metriche}
         %   \textbf{M25 Densità degli errori}
          %      \begin{itemize}
           %         \item \textbf{Valore preferibile}: 0\%;
            %        \item \textbf{Valore accettabile}: 10\%.
             %   \end{itemize}
    \subsection{Manutenibilità}
        \subsubsection{Obiettivi}
        \begin{itemize}
            \item \textbf{Analizzabilità}: grado di efficacia ed efficienza con cui è possibile valutare l'impatto su un prodotto\glosp o un sistema di un eventuale cambiamento (in una o più parti);
            \item \textbf{Modificabilità}: grado con cui un prodotto\glosp o un sistema può essere modificato efficacemente ed efficientemente, cioè senza introdurre difetti o degradando la qualità esistente.
        \end{itemize}
        %\subsubsection{Metriche}
         %   \textbf{M26 Structural fan-in}
          %      \begin{itemize}
           %         \item \textbf{Valore preferibile}: $ \ge 1$
            %        \item \textbf{Valore accettabile}: $ \ge 0$
             %   \end{itemize}
            %\textbf{M27 Structural fan-out}
             %   \begin{itemize}
              %      \item \textbf{Valore preferibile}: $0$
               %     \item \textbf{Valore accettabile}: $ \le 6$
                %\end{itemize}
            %\textbf{M08 Presenza di code smells}
            %\begin{itemize}
            %	\item \textbf{Valore preferibile}: $0$
            %	\item \textbf{Valore accettabile}: $0$
            %\end{itemize}
            
            \subsection{Tabella riassuntiva delle metriche adottate}
            \rowcolors{2}{gray!25}{gray!15}
            \begin{longtable} {
                >{}p{40mm}  
                >{}p{95mm}
                }
            \rowcolor{gray!50}
                \textbf{Caratteristiche} & \textbf{Metriche} \TBstrut \\ [2mm]
        
                Documenti &
                M08 Indice di Gulpease \newline
                M09 correttezza ortografica \TBstrut \\ [2mm]

                Copertura funzionale & 
                M21 Percentuale di requisiti obbligatori soddisfatti \newline
                M22 Percentuale di requisiti desiderabili soddisfatti \newline
                M23 Percentuale di requisiti opzionali soddisfatti \newline 
                M24 Percentuale di test passati \TBstrut \\ [2mm]
        
                Affidabilità &
                M25 Densità degli errori \TBstrut \\ [2mm]
        
                Manutenibilità &
                M26 Structural fan-in \newline
                M27 Structural fan-out \newline
                M32 Presenza di code smells \TBstrut \\ [2mm]

                \rowcolor{white}
                \caption{Tabella riassuntiva metriche\glosp adottate per la qualità di prodotto\glo}
            \end{longtable}