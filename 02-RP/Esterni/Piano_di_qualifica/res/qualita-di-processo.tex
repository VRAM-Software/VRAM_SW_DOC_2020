\section{Qualità di processo}
												
	\subsection{Processo di Sviluppo}
		\subsubsection{Analisi dei Requisiti} 
			\paragraph{Obiettivi di qualità}
			Per l'analisi dei requisiti gli obiettivi di qualità sono:
			\begin{itemize}
				\item \textbf{Individuazione completa dei requisiti}: ci prefiggiamo di fare fin da subito un'individuazione più completa possibile dei requisiti per evitare il cambiamento degli stessi nel tempo.
			\end{itemize}
			\paragraph{Metriche} \mbox{} \\ \\
				\textbf{M01 Scostamento dei requisiti individuati} 
				\begin{itemize}
					\item \textbf{Valore preferibile}: 0;
					\item \textbf{Valore accettabile}: 5;
				\end{itemize}
			
		\subsubsection{Progettazione}
			\paragraph{Obiettivi}
			 Per la progettazione gli obiettivi di qualità sono:
			 \begin{itemize}
			 	\item \textbf{Semplificazione della gerarchia}: ci prefiggiamo di mantenere la gerarchia il più semplice possibile, evitando ove possibile di aumentarne la profondità;
			 	\item \textbf{Usare un numero limitato di design pattern}: ci prefiggiamo di usare design pattern solo nel caso in cui questi effettivamente semplifichino la struttura del progetto.
			 \end{itemize}
			\paragraph{Metriche} \mbox{} \\ \\
				\textbf{M02 Accoppiamento tra le classi di oggetti} 
				\begin{itemize}
					\item \textbf{Valore preferibile}: $\le 1$;
					\item \textbf{Valore accettabile}: $\le 6$;
				\end{itemize}
				\textbf{M03 Profondità della gerarchia} 
				\begin{itemize}
					\item \textbf{Valore preferibile}: $\le 4$;
					\item \textbf{Valore accettabile}: $\le 7$;
				\end{itemize}			
				\textbf{M04 Numero di design pattern} 
				\begin{itemize}
					\item \textbf{Valore preferibile}: $5 \le x \le 6$; 
					\item \textbf{Valore accettabile}: $2 \le x \le 15$; 
				\end{itemize}
			
		\subsubsection{Codifica}
			\paragraph{Obiettivi}
			Per la codifica gli obiettivi di qualità sono:
			\begin{itemize}
				\item \textbf{Rispetto delle norme di codifica}: ci prefiggiamo di attenerci alle norme di codifica indicate nelle \textit{Norme di Progetto} al fine di garantire la comprensibilità del codice.
			\end{itemize}	 
			\paragraph{Metriche} \mbox{} \\ \\
			\textbf{M05 livello di annidamento} 
			\begin{itemize}
				\item \textbf{Valore preferibile}: $1 \le x \le 3$; 
				\item \textbf{Valore accettabile}: $1 \le x \le 7$; 
			\end{itemize}
			\textbf{M06 numero di parametri per metodo} 
			\begin{itemize}
				\item \textbf{Valore preferibile}: $ \le 3$;
				\item \textbf{Valore accettabile}: $ \le 5$;
			\end{itemize}			
			\textbf{M07 numero di metodi per classe} 
			\begin{itemize}
				\item \textbf{Valore preferibile}: $ \le 8$;
				\item \textbf{Valore accettabile}: $ \le 15$;
			\end{itemize}
	
			
	\subsection{Processi di supporto}			
		\subsubsection{Gestione della qualità}
			\paragraph{Obiettivi}
			Per la gestione della qualità gli obiettivi sono:
			\begin{itemize}
				\item \textbf{Mantenimento di adeguati standard di qualità}: ci prefiggiamo di mantenere degli adeguati standard di qualità in tutti i processi svolti e in ogni prodotto risultante;
				\item \textbf{Monitoraggio della qualità}: ci prefiggiamo di mantenere monitorata la qualità dei processi svolti e dei prodotti risultanti al fine di migliorare, ove necessario, i risultati.
			\end{itemize}	 
			\paragraph{Metriche} \mbox{} \\ \\
				\textbf{M10 Percentuale di metriche soddisfatte}
				\begin{itemize}
					\item \textbf{Valore preferibile}: $100\%$;
					\item \textbf{Valore accettabile}: $\ge 60\%$.
				\end{itemize}
			

	\subsection{Processi organizzativi}
		\subsubsection{Pianificazione delle risorse}
			\paragraph{Obiettivi}
			Per la pianificazione delle risorse gli obiettivi di qualità sono:
			\begin{itemize}
				\item \textbf{Rispetto delle tempistiche preventivate}: ci prefiggiamo di rispettare le tempistiche indicate nel preventivo nel documento \textit{Piano di Progetto};
				\item \textbf{Rispetto dei costi preventivati}: ci prefiggiamo di rispettare i costi indicati nel preventivo nel documento \textit{Piano di Progetto};
			\end{itemize}	
			\paragraph{Metriche} \mbox{} \\ \\
			\textbf{M12 Planned Value}
			\begin{itemize}
				\item \textbf{Valore preferibile}: $\ge$ 0;
				\item \textbf{Valore accettabile}: $\ge$0.
			\end{itemize}
			\textbf{M13 Earned Value}
			\begin{itemize}
				\item \textbf{Valore preferibile}: = PV;
				\item \textbf{Valore accettabile}: $\ge$ 0.
			\end{itemize}
			\textbf{M14 Actual cost}
			\begin{itemize}
				\item \textbf{Valore preferibile}: 0 $\le$ AC $\le$ PV;
				\item \textbf{Valore accettabile}: 0 $\le$ AC $\le$ budget totale.
			\end{itemize}
			\textbf{M15 Cost Performance Index}
			\begin{itemize}
				\item \textbf{Valore preferibile}: = 1;
				\item \textbf{Valore accettabile}: 0.95 $\le$ CPI $\le$ 1.05.
			\end{itemize}
			\textbf{M16 Schedule Performance Index}
			\begin{itemize}
				\item \textbf{Valore preferibile}: = 1;
				\item \textbf{Valore accettabile}: 0.95 $\le$ SPI $\le$ 1.05.
			\end{itemize}
			\textbf{M17 Estimated Cost at Completion}
			\begin{itemize}
				\item \textbf{Valore preferibile}: pari a quanto preventivato;
				\item \textbf{Valore accettabile}: preventivo-5\% $\le$ EAC $\le$ preventivo+5\%.
			\end{itemize}
			\textbf{M18 Schedule at Completion}
			\begin{itemize}
				\item \textbf{Valore preferibile}: pari a quanto preventivato;
				\item \textbf{Valore accettabile}: pari a quanto preventivato.
			\end{itemize}
		
		\subsubsection{Gestione dei rischi} 
			\paragraph{Obiettivi}
			Per la gestione dei rischi gli obiettivi di qualità sono:
			\begin{itemize}
				\item \textbf{Individuazione completa dei rischi}: ci prefiggiamo di fare fin da subito un'individuazione più completa possibile dei rischi per evitare il cambiamento degli stessi nel tempo, in modo da essere preparati nel caso si verificassero.
			\end{itemize}
			\paragraph{Metriche} \mbox{} \\ \\
			\textbf{M19 Rischi non preventivati} 
			\begin{itemize}
				\item \textbf{Valore preferibile}: 0;
				\item \textbf{Valore accettabile}: $ \le 5$;
			\end{itemize}

		\subsubsection{Verifica}
		\paragraph{Obiettivi}
		Per la verifica gli obiettivi di qualità sono:
		\begin{itemize}
			\item \textbf{Completezza della verifica}: ci prefiggiamo di svolgere una verifica completa su ogni parte dei nostri prodotti;
			\item \textbf{Efficacia della verifica}: ci prefiggiamo di risolvere ogni difetto rilevato dalla verifica al fine di renderla efficacie.
		\end{itemize}
		\paragraph{Metriche} \mbox{} \\ \\
		\textbf{M28 Code coverage}
		\begin{itemize}
			\item \textbf{Valore preferibile}: $100\%$
			\item \textbf{Valore accettabile}: $\ge 85\%$
		\end{itemize}
		\textbf{M29 Condition coverage}
		\begin{itemize}
			\item \textbf{Valore preferibile}: $100\%$
			\item \textbf{Valore accettabile}: $\ge 85\%$ 
		\end{itemize}
		\textbf{M30 Percentuale bug sistemati}
		\begin{itemize}
			\item \textbf{Valore preferibile}: $100\%$
			\item \textbf{Valore accettabile}: $100\%$  
		\end{itemize}
		\textbf{M31 Copertura dei test eseguiti}
		\begin{itemize}
			\item \textbf{Valore preferibile}: $100\%$
			\item \textbf{Valore accettabile}: $\ge 90\%$  
		\end{itemize}
			
		\subsubsection{Gestione dei cambiamenti}
		\paragraph{Obiettivi}
		Per la gestione dei cambiamenti gli obiettivi di qualità sono:
		\begin{itemize}
			\item \textbf{Efficacia nel tempo}: ci prefiggiamo di gestire ogni cambiamento necessario in tempi ragionevoli.
		\end{itemize}
		 	\paragraph{Metriche} \mbox{} \\ \\
		 	\textbf{M20 Tempo medio risoluzione errori}
			\begin{itemize}
				\item \textbf{Valore preferibile}: $\le 10$minuti;
				\item \textbf{Valore accettabile}: $\le 120$minuti.
			\end{itemize}

		  
			
				

