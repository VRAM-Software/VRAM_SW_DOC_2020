\section{Valutazioni per il miglioramento} 
Lo scopo di questa sezione è tracciare e riportare i problemi sorti durante il lavoro svolto dal gruppo per individuare delle soluzioni efficaci ed efficienti che permettano di migliorare la collaborazione e aumentare la qualità dei prodotti\glosp realizzati.
\\L'individuazione e l'analisi delle criticità e dei problemi è svolta dagli stessi membri del gruppo. Ognuno è incaricato di appuntarsi le criticità e i problemi che riscontra in modo che possano essere discussi nella riunione di gruppo successiva. Ad ogni riunione avviene quindi un confronto in cui si discute dei possibili miglioramenti da applicare per eliminare i problemi esistenti nel modo più efficace ed efficiente. Per problemi di particolare gravità o urgenza si organizzano delle riunioni apposite il prima possibile.
\\ \\Sono quindi tracciati i problemi riscontrati nei seguenti ambiti:

\begin{itemize}
	\item \textbf{Organizzazione}: problemi riguardanti l'organizzazione del lavoro e della comunicazione all'interno del gruppo;
	\item \textbf{Ruoli}: problemi riguardanti il corretto funzionamento dei ruoli;
	\item \textbf{Strumenti}: problemi riguardanti gli strumenti di lavoro utilizzati.
\end{itemize}
\subsection{Revisione dei requisiti (RR)}
	\subsubsection{Valutazioni sull'organizzazione}
		\paragraph{Organizzazione incontri}
			\begin{itemize}
				\item \textbf{Descrizione problema}: è stata riscontrata una certa difficoltà nell'organizzare incontri frequenti a cui fossero presenti tutti i membri del gruppo;
				\item \textbf{Soluzione individuata}: si è deciso di dare priorità a riunioni via Skype, utilizzando la condivisione degli schermi per poter collaborare in modo efficace ed efficiente.
			\end{itemize}
		\paragraph{Comunicazione via chat}
				\begin{itemize}
				\item \textbf{Descrizione problema}: la comunicazione via chat fra i membri del gruppo si è rivelata non sufficientemente collaborativa e tempestiva;
				\item \textbf{Soluzione individuata}: si è deciso, di comune accordo, di impegnarsi nell'essere più partecipi e propositivi nelle conversazioni. Come aiuto ogni membro del gruppo ha installato le apposite applicazioni di messaggistica abilitando notifiche prioritarie per i messaggi del gruppo.
				\end{itemize}
	\subsubsection{Valutazioni sui ruoli}
		\paragraph{Ripartizione equa delle attività da parte del responsabile}
		\begin{itemize}
			\item \textbf{Descrizione problema}: a causa dell'inesperienza dei membri del gruppo, il carico di lavoro non è sempre stato suddiviso in modo equo dal responsabile;
			\item \textbf{Soluzione individuata}: si è iniziato il prima possibile a monitorare il lavoro tramite il sistema di ticketing GitHub, così grazie alle project board e all'assegnazione delle attività il responsabile è riuscito a ripartizionare e monitorare in modo sempre migliore il carico di lavoro.
		\end{itemize}
		\paragraph{Pianificazione corretta delle milestone da parte del responsabile}
		\begin{itemize}
		\item \textbf{Descrizione problema}: a causa dell'inesperienza dei membri del gruppo, le milestone non sono sempre state fissate dal responsabile in date adeguate;
		\item \textbf{Soluzione individuata}: si è iniziato il prima possibile ad utilizzare le milestone ed i ticket su GitHub, così grazie alle project board il responsabile è riuscito a valutare in modo migliore i collocamenti delle milestone.
		\end{itemize}
		\paragraph{Rapporto fra verificatori ed analisti}
			\begin{itemize}
				\item \textbf{Descrizione problema}: nelle fasi iniziali, dato l'ancora scarso affiatamento fra i membri del gruppo, si è instaurata una situazione allievo-maestro fra analisti e verificatori;
				\item \textbf{Soluzione individuata}: si è deciso che le prime verifiche dovevano essere collettive, facendo partecipare più membri del gruppo alla verifica dei documenti favorendo così una discussione costruttiva.
			\end{itemize}
		\paragraph{Amministratore}
			\begin{itemize}
				\item \textbf{Descrizione problema}: il problema principale dell'amministratore è stato quello di aggiornare in modo tempestivo il documento \textit{Norme di Progetto} per normare le attività del gruppo senza rallentarle;
				\item \textbf{Soluzione individuata}: si è lavorato maggiormente a livello di pianificazione e se necessario è stato dato maggiore supporto all'amministratore durante l'aggiornamento delle norme.
			\end{itemize}
	\subsubsection{Valutazioni sugli strumenti}
		\paragraph{\LaTeX}
			\begin{itemize}
				\item \textbf{Descrizione problema}: la scarsa conoscenza di \LaTeX\xspace da parte dei membri del gruppo ha reso difficile avere una struttura uniforme e concorde in tutte le parti dei documenti;
				\item \textbf{Soluzione individuata}: si è creato il prima possibile un template \LaTeX\xspace stabile che contenesse le sezioni base e la struttura generale dei documenti. Inoltre i membri del gruppo più esperti hanno istruito gli altri sulle funzionalità di \LaTeX.
			\end{itemize}
		\paragraph{TeXStudio}
			\begin{itemize}
				\item \textbf{Descrizione problema}: il controllo linguistico di TeXStudio non è disponibile di default in italiano ed alcuni membri del gruppo hanno avuto difficoltà ad installarlo, anche a causa dei differenti sistemi operativi;
				\item \textbf{Soluzione individuata}: sono stati incaricati due membri del gruppo affinché trovassero una soluzione per Microsoft Windows ed una per Linux. Una volta individuate le soluzioni hanno istruito gli altri membri del gruppo su come installare e configurare il correttore linguistico in italiano.
			\end{itemize}
		\paragraph{Git}
			\begin{itemize}
				\item \textbf{Descrizione problema}: la scarsa esperienza nell'uso di Git ed in particolare dei branch ha inizialmente causato confusione e conflitti frequenti sui file;
				\item \textbf{Soluzione individuata}: si è deciso di organizzare un piccolo corso in cui i membri del gruppo più esperti hanno spiegato tramite esempi ed esercizi l'uso del sistema di versionamento\glosp Git e dei branch. 
			\end{itemize}
		\paragraph{GitHub}
			\begin{itemize}
			\item \textbf{Descrizione problema}: la scarsa esperienza nell'uso di GitHub ed in particolare delle pull-request ha inizialmente causato confusione e rallentato le attività di analisi e verifica;
			\item \textbf{Soluzione individuata}: si è deciso di organizzare un piccolo corso in cui i membri del gruppo più esperti hanno spiegato tramite esempi ed esercizi l'uso delle pull-request su GitHub. 
			\end{itemize}
		\paragraph{Slack}
			\begin{itemize}
				\item \textbf{Descrizione problema}: la scarsa esperienza nell'uso di Slack ed in particolare dei thread e dei canali ha causato difficoltà di comunicazione e disordine nelle conversazioni;
				\item \textbf{Soluzione individuata}: i membri più esperti del gruppo hanno istruito gli altri componenti ed hanno incentivato l'uso di thread e canali, andando a correggere i casi in cui la comunicazione non avveniva in modo corretto. 
			\end{itemize}
		\subsubsection{Applicazione dei miglioramenti dell'RR}
			Durante il periodo della revisione di progettazione il gruppo si è impegnato nel rispettare le soluzioni individuate ai problemi esposti in precedenza:
			\begin{itemize}
				\item \textbf{Organizzazione}: siamo riusciti ad utilizzare e ad organizzare più frequentemente le chiamate Skype attraverso Slack senza vincolarci ad essere tutti presenti presso gli spazi universitari;
				\item \textbf{Ruoli}: chi aveva il ruolo di responsabile in un determinato momento è riuscito ad utilizzare in modo appropriato l'Issue Tracking System di GitHub assegnando le attività in modo più equo e monitorando più efficacemente le tempistiche. Inoltre abbiamo migliorato la ripartizione delle ore dell'amministratore nella pianificazione di dettaglio del periodo di revisione di progettazione aggiungendo volta per volta le ore necessarie;
				\item \textbf{Strumenti}: Dopo aver fatto un'attività di istruzione degli strumenti sopra citati, durante il periodo di RP il gruppo non ha riscontrato problemi nell'utilizzo di tali strumenti, bensì ne abbiamo tratto molti benefici.
			\end{itemize}
	\subsection{Revisione di progettazione (RP)}
		\subsubsection{Valutazioni sull'organizzazione}
			\paragraph{Organizzazione dello svolgimento dei compiti}
				\begin{itemize}
					\item \textbf{Descrizione problema}: la presenza della sessione invernale degli esami durante questo periodo ha comportato lo svolgimento di molti compiti rispetto a quanto pianificato;
					\item \textbf{Soluzione individuata}: si è deciso di definire con più accuratezza gli impegni universitari e personali di ogni singolo componente per poter pianificare i compiti da svolgere in modo migliore.
				\end{itemize}
			\paragraph{Parallelizzazione del lavoro}
				\begin{itemize}
					\item \textbf{Descrizione problema}: la necessità di svolgere i compiti in tempi più stretti ha comportato uno svolgimento parallelo del Proof of Concept\glo. In questo modo non tutti i membri del gruppo hanno avuto ben chiare le componenti implementate;
					\item \textbf{Soluzione individuata}: si è deciso di fissare una chiamata Skype per aggiornare tutti e di stabilire che da questo momento, ad ogni nuovo incremento implementato, tutti ne debbano essere al corrente.
				\end{itemize}
		\subsubsection{Valutazioni sui ruoli}
			\paragraph{Analista}
				\begin{itemize}
					\item \textbf{Descrizione problema}: Dopo aver analizzato ulteriormente il problema assieme al proponenti ci siamo accorti della mancanza di numerosi requisiti funzionali e ciò ha portato ad uno scostamento notevole;
					\item \textbf{Soluzione individuata}: si è deciso di analizzare più approfonditamente il problema confrontandoci maggiormente con il proponente in modo da definire tutti i requisiti in modo definitivo ed evitare scostamenti in futuro.
				\end{itemize}
			\paragraph{Programmatore}
				\begin{itemize}
					\item \textbf{Descrizione problema}: il primo approccio all'attività di codifica ha fatto emergere l'inesperienza di alcuni componenti del gruppo nell'utilizzo dei linguaggi di programmazione, in particolare per l'implementazione degli algoritmi di predizione dalle librerie;
					\item \textbf{Soluzione individuata}: si è deciso di dedicare più tempo all'apprendimento delle nuove tecnologie, con l'appoggio anche del proponente, per riuscire a sviluppare il progetto risultando più efficaci ed efficienti.
				\end{itemize}
			\paragraph{Verificatore}
		\subsubsection{Valutazioni sugli strumenti}
			\paragraph{Studio dei nuovi strumenti}
				\begin{itemize}
					\item \textbf{Descrizione problema}: durante lo sviluppo del Proof of Concept\glosp abbiamo riscontrato la necessità di utilizzare nuovi strumenti che la maggior parte del gruppo non aveva mai utilizzato. Questo ha portato a numerose difficoltà e ad un rallentamento del lavoro;
					\item \textbf{Soluzione individuata}: si è deciso di eseguire un'analisi di tutti i possibili strumenti necessari da qui a alla fine del progetto\glosp e di dare l'incarico ad alcuni del gruppo di studiarli. Questi componenti inoltre devono provvedere a normare gli strumenti ed a spiegarli al resto del gruppo durante una chiamata collettiva.
				\end{itemize}