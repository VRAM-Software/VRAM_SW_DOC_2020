\section{Informazioni generali}
    \subsection{Informazioni incontro}
        \begin{itemize}
            \item \textbf{Luogo}: \textit{Zucchetti} - Sede di Padova in Via Giovanni Cittadella, 7;
            \item \textbf{Data}: 2020-02-19;
            \item \textbf{Ora d'inizio}: 14.30;
            \item \textbf{Ora di fine}: 15.40;
            \item \textbf{Partecipanti}: 
            \begin{itemize}
                \item Corrizzato Vittorio;
                \item Dalla Libera Marco;
                \item Santagiuliana Vittorio;
                \item Schiavon Rebecca;
                \item Spreafico Alessandro;
                \item Toffoletto Massimo;
                \item Piccoli Gregorio (proponente).
            \end{itemize}
        \end{itemize}
    \subsection{Argomenti trattati}
        In questo incontro con il proponente il gruppo ha presentato l'approccio che intende avere nella technology baseline e ha ricevuto un nel feedback anche considerazioni generali sul progetto; qui di seguito viene riportato un riassunto delle tematiche trattate:
        \begin{enumerate}
            \item formato input e output addestramento;
            \item funzionalità aggiuntive;
            \item errori dell'utente;
            \item definizione di uno standard per i file JSON;
            \item .
        \end{enumerate}
\section{Verbale}
        \subsection{Punto 1}
            Durante la discussione con il dottor Piccoli è emerso che è preferibile usare un file in formato CSV come input dell'addestramento, in quanto è un formato facilmente ottenibile partendo da un foglio di calcolo. In output invece verrà generato un file in formato JSON con i risultati dell'addestramento.
        \subsection{Punto 2}
            Il dottor Piccoli inoltre ci ha suggerito di usare (opzionalmente se il tempo lo permette) la tecnica del k-fold cross-validation sul campione di dati osservato. L'addestramento interno al plug-in invece ci ha consigliato di prenderlo in considerazione solo quando avremo soddisfatto tutti gli altri requisiti.
        \subsection{Punto 3}
            Ci ha poi assicurato che è compito dell'utente collegare il risultato dell'addestramento al predittore corretto e perciò non bisogna fare controlli a riguardo.
        \subsection{Punto 4}
            Il proponente \textit{Zucchetti} poi ci ha detto che bisogna definire uno standard per i file JSON tale per cui i file abbiano un identificativo ed una versione.
        \subsection{Punto 5}
            
        \subsection{Punto 6}
            