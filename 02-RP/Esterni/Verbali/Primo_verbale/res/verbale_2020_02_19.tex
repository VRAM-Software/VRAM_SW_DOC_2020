\section{Informazioni generali}
    \subsection{Informazioni incontro}
        \begin{itemize}
            \item \textbf{Luogo}: \textit{Zucchetti} - Sede di Padova in Via Giovanni Cittadella, 7;
            \item \textbf{Data}: 2020-02-19;
            \item \textbf{Ora d'inizio}: 14.30;
            \item \textbf{Ora di fine}: 15.40;
            \item \textbf{Partecipanti}: 
            \begin{itemize}
                \item Corrizzato Vittorio;
                \item Dalla Libera Marco;
                \item Rampazzo Marco;
                \item Santagiuliana Vittorio;
                \item Schiavon Rebecca;
                \item Spreafico Alessandro;
                \item Toffoletto Massimo;
                \item Piccoli Gregorio (proponente).
            \end{itemize}
        \end{itemize}
    \subsection{Argomenti trattati}
        In questo incontro con il proponente \textit{Zucchetti} il gruppo ha presentato l'approccio che intende avere nella technology baseline e ha ricevuto nel riscontro anche considerazioni generali sul progetto\glo; qui di seguito viene riportato un riassunto delle tematiche trattate:
        \begin{enumerate}
            \item formato input CSV;
            \item funzionalità aggiuntive;
            \item errori dell'utente;
            \item definizione di uno standard per i file JSON;
            \item definizione più precisa di UC5;
            \item rimozione di UC11.
        \end{enumerate}
\section{Verbale}
        \subsection{Punto 1}
            Durante la discussione con il dottor Piccoli è emerso che è preferibile avere anche la possibilità di usare un file in formato CSV come input dell'addestramento, in quanto è un formato facilmente ottenibile partendo da un foglio di calcolo.
        \subsection{Punto 2}
            Il dottor Piccoli inoltre ci ha suggerito di usare (opzionalmente e se il tempo lo permette) la tecnica del k-fold cross-validation sul campione di dati osservato. L'addestramento interno al plug-in invece ci ha consigliato di prenderlo in considerazione solo quando avremo soddisfatto tutti gli altri requisiti.
        \subsection{Punto 3}
            Ci ha poi assicurato che è compito dell'utente collegare il risultato dell'addestramento al predittore corretto e perciò non bisogna fare controlli a riguardo.
        \subsection{Punto 4}
            Il proponente \textit{Zucchetti} ha successivamente detto che bisogna definire uno standard per i file JSON tale per cui i file abbiano un identificativo ed una versione.
        \subsection{Punto 5}
        	Il gruppo aveva deciso dopo la video-chiamata del 2020/02/18 con il prof. Cardin che UC5 (Visualizzazione grafico di qualità delle previsioni) andava descritto in modo più specifico. Il dott. Piccoli ci ha suggerito di indicare \textit{precision}, \textit{recall} e \textit{$R^{2}$} per raggiungere l'obiettivo.
        \subsection{Punto 6}
        	Il dott. Piccoli ha fatto notare che i risultati della previsione andranno dentro a un database InfluxDB, quindi il gruppo \textit{VRAM Software} ha deciso di rimuovere UC11 (Visualizzazione dei risultati della previsione) poiché l'utente non li visualizza.