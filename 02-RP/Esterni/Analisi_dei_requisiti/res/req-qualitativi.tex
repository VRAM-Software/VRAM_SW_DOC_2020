\subsection{Requisiti qualitativi}
	\rowcolors{2}{gray!25}{gray!15}
	\begin{longtable} {
		>{\centering}p{24mm} 
		>{\centering}p{32mm}
		>{\centering}p{40mm} 
		>{}p{24.5mm}
		}
	\rowcolor{gray!50}
		\textbf{Requisito} & \textbf{Classificazione} & \textbf{Descrizione} & \textbf{Fonti} 	\TBstrut \\
		R1Q1 & Obbligatorio & La documentazione e il codice dovranno rispettare le norme indicate nelle \textit{Norme di Progetto v. 1.1.1} e nel \textit{Piano di Qualifica v. 1.1.1} & Capitolato \TBstrut \\ [2mm]
		R1Q2 & Obbligatorio & Lo sviluppo del codice dovrà seguire le indicazioni date dallo strumento di analisi statica del codice SonarJS\glo & Interno \TBstrut \\ [2mm]
		R1Q3 & Obbligatorio & Deve essere stilato un manuale utente & Capitolato \TBstrut \\ [2mm]
        R1Q4 & Obbligatorio & Deve essere stilato un manuale manutentore & Capitolato \TBstrut \\ [2mm]
        R1Q5 & Obbligatorio & Il codice dovrà essere rilasciato con licenza Apache 2\glo & Capitolato \TBstrut \\ [2mm]
		R2Q6 & Desiderabile & Il codice e la documentazione dovranno essere versionati attraverso una repository\glosp GitHub & Capitolato \TBstrut \\ [2mm]
		R1Q7 & Obbligatorio & La documentazione sarà redatta in lingua italiana & Interno \TBstrut \\ [2mm]
		R1Q8 & Obbligatorio & Il progetto\glosp deve essere caricato su un repository\glosp pubblico disponibile sul sito github.com & Capitolato  \TBstrut \\ [2mm]
		R1Q9 & Obbligatorio & Il codice sorgente del plug-in sviluppato deve essere open source e deve avere licenza Apache 2.0\glo & Interno  \TBstrut \\ [2mm]
		\rowcolor{white}
		\caption{Requisiti qualitativi}
	\end{longtable}