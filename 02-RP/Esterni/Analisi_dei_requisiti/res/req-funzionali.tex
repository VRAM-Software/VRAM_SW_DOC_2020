\section{Requisiti}
Per descrivere un requisito viene utilizzata la seguente struttura:
\begin{itemize}
	\item codice identificativo;
	\item classificazione;
	\item descrizione;
	\item fonti.
\end{itemize} 
Il \textbf{Codice Identificativo} sarà scritto in questo formato: \\
\textbf{R[Importanza][Tipologia][Codice]} \\
Dove:
\begin{itemize}
	\item \textbf{Importanza} può assumere i seguenti valori:
	\begin{itemize}
		\item 1: requisito obbligatorio;
		\item 2: requisito desiderabile;
		\item 3: requisito opzionale.
	\end{itemize}
	\item \textbf{Tipologia} può assumere i seguenti valori:
	\begin{itemize}
		\item F: funzionale;
		\item Q: prestazionale;
		\item P: qualitativo;
		\item V: vincolo.
	\end{itemize}
	\item\textbf{Codice}: numero progressivo identificativo strutturato nel formato: [codice\_padre].[codice\_figlio]
\end{itemize}
Le \textbf{Fonti} possono essere:
\begin{itemize}
	\item capitolato\glo: il requisito è stato quindi individuato dalla lettura del capitolato\glo;
	\item interno: il requisito è stato individuato ed aggiunto in seguito ad un'analisi interna;
	\item caso d'uso\glo: il requisito è stato individuato dallo studio di un caso d'uso\glo;
	\item proponente: il requisito è stato individuato in seguito ad un colloquio con il proponente.
\end{itemize}
\subsection{Requisiti funzionali}
	\rowcolors{2}{gray!25}{gray!15}
	\setcounter{table}{0}
	\begin{longtable} {
		>{}p{24mm} 
		>{}p{32mm}
		>{}p{40mm} 
		>{}p{24.5mm}
		}
	\rowcolor{gray!50}
		\textbf{Requisito} & \textbf{Classificazione} & \textbf{Descrizione} & \textbf{Fonti} 	\TBstrut \\
		R3F1 & Opzionale & L'utente deve poter addestrare gli algoritmi di predizione su Grafana\glo & Capitolato UC1 \TBstrut \\ [2mm]
		R3F1.1 & Opzionale & L'utente deve poter inserire un file in formato CSV contente i dati per l'addestramento di un'algoritmo di predizione & Interno UC1.1 \TBstrut \\ [2mm]
		R3F1.1.1 & Opzionale & L'utente deve poter selezionare il file in formato CSV dei dati per l'addestramento & Interno \TBstrut \\ [2mm]
		R3F1.1.2 & Opzionale & L'utente deve poter caricare il file in formato CSV dei dati per l'addestramento & Interno \TBstrut \\ [2mm]
		R3F1.5 & Opzionale & L'utente deve poter visualizzare il grafico a dispersione che contiene i dati per l'addestramento & Proponente UC1.5 \TBstrut \\ [2mm]
		R3F1.6 & Opzionale & L'utente deve poter inserire il file in formato JSON che contiene la configurazione di un addestramento eseguito precedentemente & Capitolato UC1.6 \TBstrut \\ [2mm]
		R3F1.6.1 & Opzionale & L'utente deve poter selezionare il file formato JSON di una configurazione precedente & Interno \TBstrut \\ [2mm]
		R3F1.6.2 & Opzionale & L'utente deve poter caricare il file formato JSON di una configurazione precedente & Interno \TBstrut \\ [2mm]
		R3F1.7 & Opzionale & L'utente deve poter inserire delle note che verranno scritte nel file JSON risultante dall'addestramento & Proponente UC1.7 \TBstrut \\ [2mm]
		R3F1.2 & Opzionale & L'utente deve poter scegliere il modello di predizione su cui eseguire l'addestramento & Capitolato UC1.2 \TBstrut \\ [2mm]
		R3F1.9 & Opzionale & L'utente deve poter scegliere il modello di predizione SVM\glosp su cui eseguire l'addestramento & Capitolato UC1.9 \TBstrut \\ [2mm]
		R3F1.10 & Opzionale & L'utente deve poter scegliere il modello di predizione RL\glosp su cui eseguire l'addestramento & Capitolato UC1.10 \TBstrut \\ [2mm]
		R3F1.11 & Opzionale & L'utente deve poter scegliere il modello di predizione da altri metodi tra cui la versione delle SVM\glosp adattate alla regressione, piccole reti neurali\glosp per la classificazione e regressioni esponenziali o logaritmiche. & Capitolato UC1.2 \TBstrut \\
		R3F1.3 & Opzionale & L'utente deve poter avviare l'addestramento dell'algoritmo di predizione & Interno UC1.3 \TBstrut \\ [2mm]
		R3F1.4 & Opzionale & L'utente deve poter addestrare l'addestramento dell'algoritmo di predizione & Interno UC1.4 \TBstrut \\ [2mm]
		R3F1.8 & Opzionale & Se l'addestramento va a buon fine, l'utente deve visualizzare un messaggio di conferma & Interno UC1.8 \TBstrut \\ [2mm]		
		R3F2 & Opzionale & L'utente deve visualizzare gli indici di qualità delle previsioni sul plug-in interno a Grafana\glosp & Capitolato UC2 \TBstrut \\ [2mm]
		R3F2.1 & Opzionale & L'utente deve visualizzare l'indice di qualità delle previsioni R$^{2}$\glo & Capitolato UC2.1 \TBstrut \\ [2mm]
		R3F2.2 & Opzionale & L'utente deve visualizzare gli indici di qualità delle previsioni Precision & Capitolato UC2.2 \TBstrut \\ [2mm]
		R3F2.3 & Opzionale & L'utente deve visualizzare gli indici di qualità delle previsioni Recall & Capitolato UC2.3 \TBstrut \\ [2mm]
		R3F3 & Opzionale & Se l'utente inserisce un file formato CSV non valido nel plug-in per l'addestramento interno, deve visualizzare un messaggio di errore & Interno UC3 \TBstrut \\ [2mm]
		R3F17 & Opzionale & Se l'utente inserisce un file formato JSON non valido nel plug-in per l'addestramento interno, deve visualizzare un messaggio di errore & Interno UC17 \TBstrut \\ [2mm]	
		R1F4 & Obbligatorio & L'utente deve poter addestrare gli algoritmi di previsione su un'applicazione esterna & Capitolato UC4 \TBstrut \\ [2mm]		
		R1F4.1 & Obbligatorio & L'utente deve poter inserire un file in formato CSV contente i dati per l'addestramento dell'algoritmo di predizione & Interno UC4.1 \TBstrut \\ [2mm]		
		R1F4.1.1 & Obbligatorio & L'utente deve poter selezionare un file CSV contente i dati per l'addestramento & Interno \TBstrut \\ [2mm]
		R1F4.1.2 & Obbligatorio & L'utente deve poter caricare un file CSV contente i dati per l'addestramento & Interno \TBstrut \\ [2mm]
		R1F4.8 & Obbligatorio & L'utente deve poter visualizzare il grafico a dispersione che contiene i dati per l'addestramento & Proponente UC4.6 \TBstrut \\ [2mm]
		R1F4.9 & Obbligatorio & L'utente deve poter inserire il file in formato JSON che contiene la configurazione di un addestramento eseguito precedentemente & Capitolato UC4.7 \TBstrut \\ [2mm]
		R1F4.9.1 & Obbligatorio & L'utente deve poter selezionare il file formato JSON di una configurazione precedente & Interno \TBstrut \\ [2mm]
		R1F4.9.2 & Obbligatorio & L'utente deve poter caricare il file formato JSON di una configurazione precedente & Interno \TBstrut \\ [2mm]
		R1F4.10 & Obbligatorio & L'utente deve poter inserire delle note che verranno scritte nel file JSON risultante dall'addestramento & Proponente UC4.8 \TBstrut \\ [2mm]	
		R1F4.2 & Obbligatorio & L'utente deve poter scegliere il modello di predizione su cui applicare l'addestramento & Capitolato UC4.2 \TBstrut \\ [2mm]
		R1F4.11 & Obbligatorio & L'utente deve poter scegliere il modello di predizione SVM\glosp su cui eseguire l'addestramento & Capitolato UC4.10 \TBstrut \\ [2mm]
		R1F4.12 & Obbligatorio & L'utente deve poter scegliere il modello di predizione RL\glosp su cui eseguire l'addestramento & Capitolato UC4.11 \TBstrut \\ [2mm]
		R3F4.3 & Opzionale & L'utente deve poter scegliere il modello di predizione da altri metodi tra cui la versione delle SVM adattate alla Regressione, piccole Reti Neurali per la classificazione e regressioni esponenziali o logaritmiche. & Capitolato UC4.2 \TBstrut \\ [2mm]				
		R1F4.4 & Obbligatorio & L'utente deve poter avviare l'addestramento dell'algoritmo di predizione & Interno UC4.3 \TBstrut \\ [2mm]
		R1F4.5 & Obbligatorio & L'utente deve poter chiudere l'addestramento dell'algoritmo di predizione & Interno UC4.4 \TBstrut \\ [2mm]		
		R2F4.6 & Desiderabile & Se l'addestramento va a buon fine, l'utente deve visualizzare un messaggio di conferma & Interno UC4.9 \TBstrut \\ [2mm]		
		R1F4.7 & Obbligatorio & L'utente deve ricevere il file JSON con i predittori per eseguire le previsioni & Capitolato UC4.5 \TBstrut \\ [2mm]
		R1F5 & Obbligatorio & L'utente deve poter visualizzare gli indici qualità delle previsioni eseguite sull'applicazione esterna & Capitolato UC5 \TBstrut \\ [2mm]
		R1F5.1 & Obbligatorio & L'utente deve poter visualizzare l'indice di qualità delle previsioni R$^{2}$\glo & Capitolato UC5.1 \TBstrut \\ [2mm]
		R1F5.2 & Obbligatorio & L'utente deve poter visualizzare gli indici di qualità delle previsioni Precision & Capitolato UC5.2 \TBstrut \\ [2mm]
		R1F5.3 & Obbligatorio & L'utente deve poter visualizzare gli indici di qualità delle previsioni Recall & Capitolato UC5.3 \TBstrut \\ [2mm]
		R2F6 & Desiderabile & Se l'utente inserisce un file CSV non valido deve visualizzare un messaggio di errore & Interno UC6 \TBstrut \\ [2mm]
		R2F18 & Desiderabile & Se l'utente inserisce un file JSON non valido deve visualizzare un messaggio di errore & Interno UC18 \TBstrut \\ [2mm]
		R1F19 & Obbligatorio & L'utente deve poter abilitare il plug-in & Interno UC19 \TBstrut \\ [2mm]
		R1F7 & Obbligatorio & L'utente deve poter avviare il plug-in & Capitolato UC7 \TBstrut \\ [2mm]
		R1F21 & Obbligatorio & L'utente deve poter visualizzare la dashboard\glosp fornita dal plug-in & Interno UC21 \TBstrut \\ [2mm]
		R1F8 & Obbligatorio & L'utente deve poter caricare all'interno del plug-in il file JSON contenente i dati risultanti dall'addestramento & Capitolato UC8 \TBstrut \\ [2mm]
		R1F9 & Obbligatorio & L'utente deve poter associare il predittore letto dal file JSON al flusso dati & Capitolato UC9 \TBstrut \\ [2mm]
		R1F9.1 & Obbligatorio & L'utente deve poter selezionare il predittore & Capitolato UC9.1 \TBstrut \\ [2mm]
		R1F9.2 & Obbligatorio & L'utente deve poter selezionare un flusso dati statico su cui eseguire le previsioni & Capitolato UC9.2 \TBstrut \\ [2mm]
		R3F9.3 & Opzionale & L'utente deve poter selezionare un flusso dati continuo su cui eseguire le previsioni & Capitolato UC9.2 \TBstrut \\ [2mm]
		R1F9.4 & Obbligatorio & L'utente deve poter collegare il predittore scelto al flusso dati & Capitolato UC9.3 \TBstrut \\ [2mm]
		R2F9.5 & Desiderabile & Se il collegamento del predittore al flusso dati va a buon fine l'utente deve visualizzare un messaggio di conferma & Capitolato UC9.4 \TBstrut \\ [2mm]
		R2F10 & Desiderabile & Se il collegamento del predittore al flusso dati non va a buon fine l'utente deve visualizzare un messaggio di errore & Interno UC10 \TBstrut \\ [2mm]
		R1F11 & Obbligatorio & L'utente deve poter visualizzare i risultati della previsione sotto forma di grafici all'interno di una dashboard\glosp configurata & Capitolato \TBstrut \\ [2mm]
		R1F12 & Obbligatorio & L'utente deve poter rimuovere il pannello del plug-in dalla dashboard\glo & Capitolato UC12 \TBstrut \\ [2mm]	
		R1F20 & Obbligatorio & L'utente deve poter disabilitare il plug-in & Interno UC20 \TBstrut \\ [2mm]	
		R2F13 &	Desiderabile & L'utente deve poter definire un alert nel pannello grafico di una dashboard\glo & Capitolato UC13 \TBstrut \\ [2mm]					
		R2F13.2 & Desiderabile & L'utente deve poter definire le regole di funzionamento di un alert\glo & Interno UC13.2 \TBstrut \\ [2mm]		
		R2F13.3 & Desiderabile & L'utente deve poter definire le condizioni di funzionamento di un alert\glo & Interno UC13.3 \TBstrut \\ [2mm]
		R2F13.4 & Desiderabile & L'utente deve poter definire alcuni comportamenti di un alert\glosp da seguire in casi speciali come l'assenza di dati & Interno UC13.4 \TBstrut \\ [2mm]		
		R2F14 &	Desiderabile & Se l'utente inserisce un input errato nella definizione di un alert\glosp deve visualizzare un messaggio di errore & Interno UC14 \TBstrut \\ [2mm]
		R2F15 &	Desiderabile & L'utente deve poter sospendere un alert\glosp bloccandone temporaneamente l'esecuzione & Interno UC15 \TBstrut \\ [2mm]		
		R2F16 & Desiderabile & L'utente deve poter rimuovere un alert\glosp dal pannello grafico & Interno UC16 \TBstrut \\ [2mm]	
		\rowcolor{white}
		\caption{Requisiti funzionali}
	\end{longtable}