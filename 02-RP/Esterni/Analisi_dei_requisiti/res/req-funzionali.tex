\section{Requisiti}
Per descrivere un requisito viene utilizzata la seguente struttura:
\begin{itemize}
	\item codice identificativo;
	\item classificazione;
	\item descrizione;
	\item fonti.
\end{itemize} 
Il \textbf{Codice Identificativo} sarà scritto in questo formato: \\
\textbf{R[Importanza][Tipologia][Codice]} \\
Dove:
\begin{itemize}
	\item \textbf{Importanza} può assumere i seguenti valori:
	\begin{itemize}
		\item 1: requisito obbligatorio;
		\item 2: requisito desiderabile;
		\item 3: requisito opzionale.
	\end{itemize}
	\item \textbf{Tipologia} può assumere i seguenti valori:
	\begin{itemize}
		\item F: funzionale;
		\item Q: prestazionale;
		\item P: qualitativo;
		\item V: vincolo.
	\end{itemize}
	\item\textbf{Codice}: numero progressivo identificativo strutturato nel formato: [codice\_padre].[codice\_figlio]
\end{itemize}
Le \textbf{Fonti} possono essere:
\begin{itemize}
	\item capitolato\glo: il requisito è stato quindi individuato dalla lettura del capitolato\glo;
	\item interno: il requisito è stato individuato ed aggiunto in seguito ad un'analisi interna;
	\item caso d'uso\glo: il requisito è stato individuato dallo studio di un caso d'uso\glo;
	\item proponente: il requisito è stato individuato in seguito ad un colloquio con il proponente.
\end{itemize}
\subsection{Requisiti funzionali}
	\rowcolors{2}{gray!25}{gray!15}
	\setcounter{table}{0}
	\begin{longtable} {
		>{}p{24mm} 
		>{}p{32mm}
		>{}p{40mm} 
		>{}p{24.5mm}
		}
	\rowcolor{gray!50}
		\textbf{Requisito} & \textbf{Classificazione} & \textbf{Descrizione} & \textbf{Fonti} 	\TBstrut \\
		R3F1 & Opzionale & L'utente deve poter addestrare gli algoritmi di predizione su Grafana\glo & Capitolato UC1 \TBstrut \\ [2mm]
		R3F1.1 & Opzionale & L'utente deve poter selezionare un file JSON contente i dati di testing per l'addestramento & Interno UC1.1 \TBstrut \\ [2mm]
		R3F1.1.1 & Opzionale & L'utente deve poter selezionare il file JSON & Interno UC1.1.1 \TBstrut \\ [2mm]
		R3F1.1.2 & Opzionale & L'utente deve poter caricare il file JSON & Interno UC1.1.2 \TBstrut \\ [2mm]
		R3F1.2 & Opzionale & L'utente deve poter scegliere il modello di predizione & Capitolato UC1.2 \TBstrut \\ [2mm]
		R3F1.3 & Opzionale & L'utente deve poter avviare l'addestramento dell'algoritmo & Interno UC1.3 \TBstrut \\ [2mm]
		R3F1.4 & Opzionale & L'utente deve poter chiudere l'addestramento dell'algoritmo & Interno UC1.4 \TBstrut \\ [2mm]
		R3F1.4.1 & Opzionale & Se l'addestramento va a buon fine l'utente deve visualizzare un messaggio di conferma & Interno UC1.4.1 \TBstrut \\ [2mm]		
		R3F2 & Opzionale & L'utente deve visualizzare l'indice della qualità delle previsioni & Capitolato UC2 \TBstrut \\ [2mm]
		R3F3 & Opzionale & Se l'utente inserisce un file JSON non valido deve visualizzare un messaggio di errore & Interno UC3 \TBstrut \\ [2mm]		
		R1F4 & Obbligatorio & L'utente deve poter addestrare gli algoritmi di previsione su un'applicazione esterna & Capitolato UC4 \TBstrut \\ [2mm]		
		R1F4.1 & Obbligatorio & L'utente deve poter inserire un file JSON contente i dati di testing per l'addestramento & Interno UC4.1 \TBstrut \\ [2mm]		
		R1F4.1.1 & Obbligatorio & L'utente deve poter selezionare un file JSON contente i dati di testing per l'addestramento & Interno UC4.1.1 \TBstrut \\ [2mm]
		R1F4.1.2 & Obbligatorio & L'utente deve poter caricare un file JSON contente i dati di testing per l'addestramento & Interno UC4.1.2 \TBstrut \\ [2mm]		
		R1F4.2 & Obbligatorio & L'utente deve poter scegliere il modello di predizione tra SVM\glosp e RL\glosp su cui applicare l'addestramento & Capitolato UC4.2 \TBstrut \\ [2mm]
		R3F4.3 & Opzionale & L'utente deve poter scegliere il modello di predizione da altri metodi tra cui la versione delle SVM adattate alla Regressione, piccole Reti Neurali per la classificazione e regressioni esponenziali o logaritmiche. & Capitolato UC4.2 \TBstrut \\ [2mm]				
		R1F4.4 & Obbligatorio & L'utente deve poter avviare l'addestramento dell'algoritmo & Interno UC4.3 \TBstrut \\ [2mm]
		R1F4.5 & Obbligatorio & L'utente deve poter chiudere l'addestramento dell'algoritmo & Interno UC4.4 \TBstrut \\ [2mm]		
		R1F4.5.1 & Obbligatorio & Se l'addestramento va a buon fine l'utente deve visualizzare un messaggio di conferma & Interno UC4.4.1 \TBstrut \\ [2mm]		
		R1F4.5.2 & Obbligatorio & L'utente deve riceve il file JSON con i parametri per le previsioni & Capitolato UC4.4.2 \TBstrut \\ [2mm]
		R1F5 & Obbligatorio & L'utente deve visualizzare il grafico della qualità delle previsioni & Capitolato UC5 \TBstrut \\ [2mm]
		R2F6 & Desiderabile & Se l'utente inserisce un file JSON non valido deve visualizzare un messaggio di errore & Interno UC6 \TBstrut \\ [2mm]
		
		R1F7 & Obbligatorio & L'utente deve poter avviare il plug-in & Capitolato UC7 \TBstrut \\ [2mm]
		R1F8 & Obbligatorio & L'utente deve poter caricare il file JSON di addestramento all'interno del plug-in & Capitolato UC8 \TBstrut \\ [2mm]
		R1F9 & Obbligatorio & L'utente deve poter associare i nodi letti dal file JSON al flusso dati & Capitolato UC9 \TBstrut \\ [2mm]
		R1F9.1 & Obbligatorio & L'utente deve poter selezionare i nodi & Capitolato UC9.1 \TBstrut \\ [2mm]
		R1F9.2 & Obbligatorio & L'utente deve poter selezionare un flusso dati statico su cui eseguire le previsioni & Capitolato UC9.2 \TBstrut \\ [2mm]
		R3F9.3 & Opzionale & L'utente deve poter selezionare un flusso dati continuo su cui eseguire le previsioni & Capitolato UC9.2 \TBstrut \\ [2mm]
		R1F9.4 & Obbligatorio & L'utente deve poter collegare i nodi scelti al flusso dati& Capitolato UC9.3 \TBstrut \\ [2mm]
		R1F9.5 & Obbligatorio & Se il collegamento dei nodi al flusso dati va a buon fine l'utente deve visualizzare un messaggio di conferma & Capitolato UC9.4 \TBstrut \\ [2mm]
		R2F10 & Desiderabile & Se il collegamento dei nodi al flusso dati non va a buon fine l'utente deve visualizzare un messaggio di errore & Interno UC10 \TBstrut \\ [2mm]
		R1F11 & Obbligatorio & L'utente deve poter visualizzare i risultati della previsione sotto forma di grafici all'interno di una dashboard\glosp configurata & Capitolato UC11 \TBstrut \\ [2mm]
		R1F12 & Obbligatorio & L'utente deve poter fermare l'esecuzione del plug-in rimuovendolo & Capitolato UC12 \TBstrut \\ [2mm]		
		
			R2F13 &	Desiderabile & L'utente deve poter definire un alert nel pannello grafico di una dashboard\glo & Capitolato UC13 \TBstrut \\ [2mm]			
		R2F13.1 & Desiderabile & L'utente deve poter inserire un alert & Capitolato UC13.1 \TBstrut \\ [2mm]		
		R2F13.2 & Desiderabile & L'utente deve poter definire le regole di funzionamento di un alert\glo & Interno UC13.2 \TBstrut \\ [2mm]		
		R2F13.3 & Desiderabile & L'utente deve poter definire le condizioni di funzionamento di un alert\glo & Interno UC13.3 \TBstrut \\ [2mm]
		R2F13.4 & Desiderabile & L'utente deve poter definire alcuni comportamenti speciali di un alert\glo come assenza di dati & Interno UC13.4 \TBstrut \\ [2mm]		
		R2F14 &	Desiderabile & Se l'utente inserisce un input errato nella definizione di un alert\glosp deve visualizzare un messaggio di errore &	Interno UC14 \TBstrut \\ [2mm]
		R2F15 &	Desiderabile & L'utente deve poter sospendere un alert\glo & Interno UC15 \TBstrut \\ [2mm]		
		R2F16 & Desiderabile & L'utente deve poter rimuovere un alert\glo & Interno UC16 \TBstrut \\ [2mm]	
		\rowcolor{white}
		\caption{Requisiti funzionali}
	\end{longtable}