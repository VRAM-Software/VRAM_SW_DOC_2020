\section{Tecnologie utilizzate}
	\subsubsection{Electron}
		Electron è un container per applicazioni basate su tecnologie web, restituendo un programma in finestra che funziona con tecnologie chromium.
	\subsubsection{React}
		React è un framework JavaScript che si occupa prevalentemente della gestione dell'interfaccia, permettendo di dividerla in componenti modulari e riutilizzabili.
	\subsubsection{AngularJS}
		Framework JavaScript che si basa su un pattern MVC/MVVM per la realizzazione di applicazioni web.
	\subsubsection{NodeJS}
		Offre le funzionalità di server per la web app nella fase di sviluppo.
	\subsubsection{JSX}
		Estensione del linguaggio JavaScript utilizzato per la struttura dei componenti dell'interfaccia grafica in React, il quale ritornerà codice HTML.
	\subsubsection{Jest}
		Framework per i test di unità dei componenti JavaScript.
	\subsubsection{SonarLint}
		Strumento per l'analisi statica del codice, permette di rilevare errori comuni e anomalie senza eseguire il codice.
	\subsubsection{NPM}
		Il Node Package Manager viene utilizzato per la gestione delle dipendente e la build automation.
		Richiede che all'interno del progetto sia presente il file package.json, che contiene informazioni riguardanti le dipendenze e gli script di build e test.
	\subsubsection{Plotly}
	\subsubsection{D3}
	\subsubsection{}
	\subsubsection{}
	\subsubsection{}


