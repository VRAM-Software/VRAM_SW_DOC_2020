\section{Installazione}
Per installare il prodotto\glosp è necessario scaricare il contenuto di questi moduli del repository:
\begin{itemize}
	\item 
\end{itemize}
Terminato lo scaricamento si può procedere all'istallazione

\subsection{Applicazione di addestramento esterna}
Per far partire l'applicazione esterna bisogna semplicemente avviare l'eseguibile fornito. Di seguito viene illustrato come.
	\subsubsection{Windows}
	Aprire la cartella BLABLABLA e fare doppio click sul file BLABLABLABLA.exe e si aprirà l'applicazione per l'addestramento.

	\subsubsection{macOS e Gnu/Linux}
	\begin{itemize}
		\item aprire la cartella BLABLABLA;
		\item assicurarsi che il file BLABLABLA.sh sia eseguibile in uno dei modi seguenti:
		\begin{itemize}
			\item fare click con il tasto destro sul file, selezionare proprietà e poi permessi e mettere la spunta su "eseguibile";
			\item aprire un terminale nella cartella BLABLABLA e dare il seguente comando COMANDOBELLO
		\end{itemize}
		\item aprire un terminale nella cartella BLABLABLA e dare il seguente comando ./COMANDOBELLO.sh .
	\end{itemize}

\subsection{Plug-in di Grafana}
Per poter utilizzare il plug-in bisogna copiare BLABLABLA all'interno della cartella /plugins nella root folder del server Grafana (nel caso tale cartella non fosse presente, il progetto dovrà essere copiato all'interno della cartella /data/plugins ). Infine è necessario abilitare il plug-in: una volta selezionato "BLABLABLA " dalla lista di plug-in disponibili, sarà sufficiente aprire la tab "Config" e cliccare sul pulsante "Enable".

