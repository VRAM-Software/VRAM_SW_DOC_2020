\section{Introduzione}
	\subsection{Scopo del documento}
		Il presente manuale ha lo scopo di illustrare le funzionalità e la procedura per l'utilizzo del prodotto\glo, che si compone di un plug-in per Grafana\glosp e di un applicativo per l'addestramento degli algoritmi di predizione.
	\subsection{Scopo del prodotto}
		Il capitolato\glosp C4 ha lo scopo di sviluppare un plug-in per Grafana\glosp scritto in linguaggio Javascript che, eseguendo gli algoritmi di support vector-machine (SVM\glo) o regressione lineare (RL\glo), sarà in grado di generare previsioni utili al flusso di monitoraggio di Grafana\glo. Se utilizzato in congiunzione con il sistema di alert\glosp di Grafana\glo, potrà essere lanciato un allarme qualora fosse rilevata una situazione di rischio.
		Il plug-in fa uso di un applicativo per l'addestramento esterno a Grafana\glosp degli algoritmi SVM\glosp, RL\glosp, ed eventualmente altri algoritmi di predizione. L'obiettivo è quindi quello di migliorare l'attività di monitoraggio svolta da operatori specializzati, aggiungendo previsioni ai dati raccolti sul campo.
	\subsection{Glossario}
		Questo documento è corredato di un \textit{Glossario v. 3.1.1} dove sono chiariti i termini potenzialmente ambigui. Le voci interessate sono identificate da una 'G' a pedice.
