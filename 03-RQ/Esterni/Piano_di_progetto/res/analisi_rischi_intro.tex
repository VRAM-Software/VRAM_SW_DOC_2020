\section{Analisi dei rischi}
%(mettere studio standish group tra riferimenti informativi e slide 06 tullio)

Un progetto\glosp software è soggetto a molti rischi che necessitano di essere analizzati e contenuti. Perciò il gruppo \textit{VRAM Software} ha deciso di seguire una procedura per gestirli composta dalle seguenti attività:

\begin{itemize}
	\item \textbf{Individuazione dei rischi}: attività di identificazione dei fattori di rischio per il nostro progetto\glo;
	\item \textbf{Analisi dei rischi}: analisi dei fattori di rischio individuati e definizione delle loro probabilità di attuazione e livello di gravità che comportano;
	\item \textbf{Pianificazione dell'attività di controllo}: definizione di attività di rilevazione riconoscimento, controllo e attuazione di contromisure per i rischi;
	\item \textbf{Controllo e monitoraggio}: definizione di metodologie per rilevare i rischi attualizzati e attivare le contromisure adeguate o per aggiornare la lista dei rischi.
\end{itemize}

Qui di seguito sono riportati in forma tabellare i rischi fino ad ora presi in esame.
