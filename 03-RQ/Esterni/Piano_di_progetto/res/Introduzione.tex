\section{Introduzione}
	\subsection{Scopo del documento}
		Questo documento descrive la disponibilità delle risorse e stabilisce come vengono assegnate ad attività e processi\glo. L'obiettivo è ottenere un'organizzazione efficiente, per farlo si utilizza una scansione temporale delle attività.
	\subsection{Scopo del prodotto}
		L'obiettivo del capitolato\glosp C4 è sviluppare un plug-in per Grafana\glo, che analizza una serie temporale di dati applicandovi gli algoritmi di Support Vector Machine (SVM\glo) o Regressione Lineare (RL\glo) e restituisce previsioni riguardo a comportamenti rischiosi da parte del sistema monitorato, possibilmente accompagnate da un indice di affidabilità. Assieme al plug-in, viene sviluppato un programma per la gestione dei parametri degli algoritmi di previsione che permette di allenare gli algoritmi di previsione con dei dati di test.
	\subsection{Glossario}
		Questo documento sarà corredato di un \textit{Glossario v. 22.0.0} dove saranno chiariti i termini potenzialmente ambigui.
		Le voci interessate saranno identificate da una 'G' a pedice.
	\subsection{Riferimenti}
		\subsubsection{Riferimenti normativi}
			\begin{enumerate}
				\item \textbf{Norme di Progetto}: \textit{Norme di Progetto v. 22.0.0};
				\item \textbf{Capitolato}\glosp \textbf{d'appalto C4 - Predire in Grafana}\glo: \url{https://www.math.unipd.it/~tullio/IS-1/2019/Progetto/C4.pdf}.
			\end{enumerate}
		\subsubsection{Riferimenti informativi}
			\begin{enumerate}
				\item \textbf{Piano di Qualifica}: \textit{Piano di Qualifica v. 22.0.0};
				\item \textbf{Slide gestione di progetto}\glo: \url{https://www.math.unipd.it/~tullio/IS-1/2019/Dispense/L06.pdf};
				\item \textbf{Ingegneria del software, Ian Sommerville, Pearson Education, Addison Wesley (Decima edizione)}: capitolo 19.1 .
			\end{enumerate}
		\subsection{Calendario delle attività}
		\begin{itemize}
			\item Revisione dei Requisiti: 14 Gennaio 2020;
			\item Revisione di Progettazione: 16 Marzo 2020;
			\item Revisione di Qualifica: 20 Aprile 2020;
			\item Revisione di Accettazione: 18 Maggio 2020;			
		\end{itemize}
