\section{Introduzione}
\subsection{Glossario}
Per rendere il documento il meno possibile prono ad ambiguità e incomprensioni è fornito nell'appendice §{METTERE SEZIONE!!!!} un glossario interno del documento che raccoglie tutti i termini bisognosi di spiegazione; tali termini sono individuati da una G a pedice.
\subsection{Contenuti}
In questo manuale vengono trattati:
\begin{itemize}
	\item gli strumenti necessari
	\item le tecnologie usate
	\item le modalità di installazione
	\item i requisiti necessari
	\item le scelte progettuali
	\item l'architettura software
\end{itemize}

\subsection{Scopo del manuale}
Questo manuale si rivolge allo sviluppatore che vuole mantenere o estendere il prodotto\glosp \textit{Predire in Grafana}. Lo scopo del documento è quindi di facilitare il lavoro di manutenzione ed estensione fornendo una spiegazione dettagliata del prodotto\glo.

\subsection{Scopo del prodotto}
Lo scopo del prodotto è implementare un plug-in di Grafana\glosp scritto in linguaggio Javascript che esegue gli algoritmi di Support Vector Machine SVM\glosp o Regressione Lineare RL\glo, i quali, leggendo da un file JSON la loro configurazione, saranno in grado di generare previsioni che potranno essere aggiunte al flusso di monitoraggio. Il plug-in prevede il superamento di determinati livelli di soglia per generare un allarme attraverso il meccanismo di Grafana\glosp detto alert\glo. Il plug-in utilizza un applicativo esterno per addestrare gli algoritmi di SVM\glosp ed RL\glosp o altri algoritmi di predizione. Dunque il plug-in permette un'attività di monitoraggio grazie alla quale gli operatori possono intervenire con ogni cognizione di causa sul sistema.

\subsection{Riferimenti}
\begin{itemize}
	\item https://nodejs.org/
	\item https://www.npmjs.com/
	\item https://reactjs.org/
	\item https://www.electronjs.org/
\end{itemize}
