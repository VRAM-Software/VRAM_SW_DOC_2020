\section{Introduzione}
\subsection{Glossario}
Per rendere il documento il meno possibile prono ad ambiguità e incomprensioni è fornito nell'appendice §{METTERE SEZIONE!!!!} un glossario interno del documento che raccoglie tutti i termini bisognosi di spiegazione; tali termini sono individuati da una G a pedice.
\subsection{Contenuti}
In questo manuale vengono trattati:
\begin{itemize}
	\item gli strumenti necessari
	\item le tecnologie usate
	\item le modalità di installazione
	\item le modalità di test
	\item i requisiti necessari
	\item le scelte progettuali
	\item l'architettura software
\end{itemize}

\subsection{Scopo del manuale}
Questo manuale si rivolge allo sviluppatore che vuole mantenere o estendere il prodotto\glosp \textit{Predire in Grafana}. Lo scopo del documento è quindi di facilitarne il lavoro fornendo una spiegazione dettagliata del prodotto\glo.

\subsection{Scopo del prodotto}
Lo scopo del prodotto è implementare un plug-in di Grafana\glosp scritto in linguaggio TypeScript che esegua un algoritmo di previsione  per generare previsioni che potranno essere aggiunte al flusso di monitoraggio e un applicativo esterno che crei un file JSON di configurazione adeguato tramite l'addestramento del suddetto algoritmo. \\
Gli algoritmi di previsione implementati sono Support Vector Machine e Regressione Lineare.
Lo scopo del nostro prodotto è dunque quello di permettere un'attività di monitoraggio grazie alla quale gli operatori possono intervenire con ogni cognizione di causa sul sistema.

\subsection{Riferimenti}
\begin{itemize}
	\item https://nodejs.org/
	\item https://www.npmjs.com/
	\item https://reactjs.org/
	\item https://www.electronjs.org/
\end{itemize}
