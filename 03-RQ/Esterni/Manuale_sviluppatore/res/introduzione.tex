\section{Introduzione}
\subsection{Scopo del documento}
Questo manuale si rivolge agli sviluppatori che devono mantenere o estendere il prodotto\glosp software \textit{Predire in Grafana}. Lo scopo del documento è quindi fornire una spiegazione dettagliata del prodotto\glosp completa di tutte le informazioni necessarie agli sviluppatori per svolgere le proprie attività.

\subsection{Scopo del prodotto}
Lo scopo del prodotto è implementare un plug-in per la piattaforma Grafana\glosp che implementi degli algoritmi capaci di generare previsioni, le quali potranno essere aggiunte al flusso di monitoraggio. Inoltre, come ausilio, un applicativo esterno alla piattaforma in grado di creare un file JSON di configurazione adeguato tramite l'addestramento del suddetto algoritmo. \\
Gli algoritmi di predizione implementati sono SVM\glosp e RL\glo.
Lo scopo del nostro prodotto è dunque quello di permettere un'attività di previsione grazie alla quale gli operatori possono monitorare e intervenire con ogni cognizione di causa sul sistema.

\subsection{Glossario}
Per rendere il documento il meno possibile prono ad ambiguità e incomprensioni è fornito nell'appendice §A un glossario interno al documento che raccoglie tutti i termini di cui è necessaria una spiegazione; tali termini sono individuati da una G a pedice.
%\subsection{Contenuti}
%In questo manuale vengono trattati:
%\begin{itemize}
%	\item gli strumenti necessari
%	\item le tecnologie usate
%	\item le modalità di installazione
%	\item le modalità di test
%	\item i requisiti necessari
%	\item le scelte progettuali
%	\item l'architettura software
%\end{itemize}

\subsection{Riferimenti}
\subsubsection{Installazione}
\begin{itemize}
	\item 
\end{itemize}
\subsubsection{Tecnologie}
\begin{itemize}
	\item 
\end{itemize}
\subsubsection{Legali}
\begin{itemize}
	\item 
\end{itemize}
\subsubsection{Informativi}
\begin{itemize}
	\item https://nodejs.org/
	\item https://www.npmjs.com/
	\item https://reactjs.org/
	\item https://www.electronjs.org/
\end{itemize}
