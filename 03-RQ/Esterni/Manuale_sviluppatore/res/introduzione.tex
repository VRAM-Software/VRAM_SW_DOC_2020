\section{Introduzione}
\subsection{Scopo del documento}
Questo manuale si rivolge agli sviluppatori che devono mantenere o estendere il prodotto\glosp software del progetto\glosp \textit{Predire in Grafana}. Lo scopo del documento è quindi fornire una spiegazione dettagliata e completa del prodotto\glo, con tutte le informazioni necessarie agli sviluppatori per svolgere le proprie attività.

\subsection{Scopo del prodotto}
Lo scopo del prodotto è implementare un plug-in per la piattaforma Grafana\glosp che implementi degli algoritmi di predizione che potranno essere aggiunte al flusso di monitoraggio. Inoltre, come ausilio, è implementato un applicativo esterno alla piattaforma, in grado di addestrare gli algoritmi di predizione e creare un corrispettivo file di configurazione in formato JSON. \\
Gli algoritmi di predizione implementati sono SVM\glosp e RL\glo.
Lo scopo del nostro prodotto è dunque quello di permettere un'attività di previsione su un flusso di dati, grazie alla quale gli operatori possono monitorare e intervenire con ogni cognizione di causa sul sistema.

\subsection{Glossario}
Per rendere il documento il meno possibile prono ad ambiguità e incomprensioni è fornito nell'appendice §A un glossario interno al documento che raccoglie tutti i termini di cui è necessaria una spiegazione. Tali termini sono individuati da una G a pedice.
%\subsection{Contenuti}
%In questo manuale vengono trattati:
%\begin{itemize}
%	\item gli strumenti necessari
%	\item le tecnologie usate
%	\item le modalità di installazione
%	\item le modalità di test
%	\item i requisiti necessari
%	\item le scelte progettuali
%	\item l'architettura software
%\end{itemize}

\subsection{Riferimenti}
\subsubsection{Installazione}
\begin{itemize}
	\item \url{https://nodejs.org/}
	\item \url{https://git-scm.com/}
	\item \url{https://jetbrains.com/webstorm}
	\item \url{https://grafana.com/get}
\end{itemize}
\subsubsection{Tecnologie}
\begin{itemize}
	\item \url{https://reactjs.org/}
	\item \url{https://electronjs.org/}
	\item \url{https://angularjs.org/}
	\item \url{https://grafana.com/}
	\item \url{https://influxdata.com/}
	\item \url{https://d3js.org/}
	\item \url{https://plotly.com/}
	\item \url{https://jestjs.io/}
	\item \url{https://www.sonarlint.org/}
\end{itemize}
\subsubsection{Legali}
\begin{itemize}
	\item \url{https://www.apache.org/licenses/LICENSE-2.0}
\end{itemize}
\subsubsection{Informativi}
\begin{itemize}
	\item \url{https://en.wikipedia.org/wiki/Linear_regression}
	\item \url{https://en.wikipedia.org/wiki/Support-vector_machine}
\end{itemize}
