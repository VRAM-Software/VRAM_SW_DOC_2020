\subsection{Configurazione ambiente applicativo esterno}
\subsubsection{Contenuto file: package.json}
Il file \verb|package.json | contiene tutte le informazioni e le dipendenze necessarie della nostra applicazione.
\begin{itemize}
    \item \textbf{"main"}: questo elemento ha come valore il percorso del file che si occupa di gestire il processo principale dell'applicazione Electron;
    \item \textbf{"dependencies"}: questo elemento contiene la seguente lista di pacchetti che sono necessari per il corretto funzionamento dell'applicazione:
        \begin{itemize}
            \item csvtojson;
            \item d3;
            \item electron-is-dev;
            \item ml-modules;
            \item react;
            \item react-dom;
            \item react-router-dom;
            \item react-scripts.
        \end{itemize}
    \item \textbf{"devDependencies"}: questo elemento contiene la seguente lista di pacchetti necessari per il corretto funzionamento dell'applicazione durante lo sviluppo:
        \begin{itemize}
            \item electron;
            \item electron-packager;
            \item coveralls;
            \item enzyme;
            \item enzyme-adapter-react-16;
            \item spectron.
        \end{itemize} 
    \item \textbf{"scripts"}: questo elemento contiene una lista di tutti i comandi, utili per uno sviluppatore, che possono essere eseguiti da linea di comando:
        \begin{itemize}
            \item \textbf{electron}: il comando seguente avvia un'istanza di Electron;
            \begin{verbatim}
            	npm run electron
            \end{verbatim}
            \item \textbf{start}: il comando seguente avvia un server Node.js alla porta 3000;
            \begin{verbatim}
	            npm start
            \end{verbatim}
            \item \textbf{build}: il comando seguente genera una cartella: \verb|build| che contiene i file di produzione di React;
            \begin{verbatim}
            	npm build
            \end{verbatim}
            \item \textbf{test}: il comando seguente esegue tutti i test dell'applicazione;\\
            \begin{verbatim}
            	npm test
            \end{verbatim}
            \item \textbf{package-mac}: il comando seguente genera una release di produzione per il sistema operativo: MacOS;
            \begin{verbatim}
            	npm run package-mac
            \end{verbatim}
            \item \textbf{package-win}: il comando seguente genera una release di produzione per il sistema operativo: Windows;
            \begin{verbatim}
            	npm run package-win
            \end{verbatim}
            \item \textbf{package-linux}: il comando seguente genera una release di produzione per una qualsiasi distribuzione di Linux.
            \begin{verbatim}
            	npm run package-linux
            \end{verbatim}
        \end{itemize}
\end{itemize}
