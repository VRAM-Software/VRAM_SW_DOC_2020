\subsection{Installazione applicativo esterno}
Nelle prossime sezioni verrà descritta la modalità con cui è possibile installare l'applicazione esterna.

\subsubsection{Installazione degli strumenti}
\paragraph{Strumenti}
Gli strumenti necessari per installare correttamente l'applicativo esterno sono:
\begin{itemize}
    \item \textbf{Node.js}: runtime Javascript che permette di eseguire codice Javascript fuori da un browser, l'installazione di npm (Node Package Manager) non è richiesta dato che viene installato automaticamente durante l'installazione di Node.js;
    \item \textbf{Git}: sistema di versionamento distribuito.
\end{itemize}

Questi strumenti possono essere installati con diverse metodologie, ci limiteremo a elencare alcuni esempi per non appesantire questo manuale con informazioni non strettamente pertinenti all'applicativo esterno.

\paragraph{Node}
Per installare il runtime Javascript Node.js si può visitare il sito \href{https://nodejs.org}. Qui è possibile trovare il download di Node.js per tutti i sistemi operativi. Alternativamente, su sistemi operativi basati su Linux è possibile utilizzare lo strumento di gestione di pacchetti fornito dal sistema operativo per installare i pacchetti del runtime Node.js. Di seguito un esempio:
\textbf{Debian/Ubuntu} \\
\begin{verbatim}
    apt-get install nodejs
\end{verbatim}

\paragraph{Git}
Per installare il sistema di versionamento distribuito Git si può visitare la pagina web \href{https://git-scm.com/downloads}. Qui è possibile trovare il download per MacOS, Windows e sistemi operativi basati su Linux/Unix. Alternativamente, su sistemi operativi basati su Linux è possibile utilizzare lo strumento di gestione di pacchetti fornito dal sistema operativo. Di seguito un esempio:
\textbf{Debian/Ubuntu} \\
\begin{verbatim}
    apt-get install git
\end{verbatim}

\subsubsection{Installazione dell'applicazione}
L'installazione dell'applicazione è suddivisa in questi due passaggi che descriveremo in dettaglio nei prossimi paragrafi.
\begin{enumerate}
    \item Clonare la repository da GitHub;
    \item Installare le dipendenze.
\end{enumerate}

\paragraph{Clonare la repository da GitHub}
Per clonare la repository dell'applicazione aprite un terminale e usate il comando \verb{cd} per muovervi in una cartella del vostro computer, eseguite il comando: \verb{git clone https://github.com/VRAM-Software/grafana_prediction.git} per clonare la repository principale del nostro prodotto. Infine con il comando \verb{cd ./prediction_configuration_utility} spostatevi nella cartella che contiene il codice sorgente dell'applicativo esterno.

\paragraph{Installare le dipendenze}
Per il corretto funzionamento dell'applicazione è necessario installare tutte le dipendenze elencate nel file \verb{package.json}, per farlo eseguite il comando \verb{npm install} da terminale nella cartella \verb{prediction_configuration_utility}. I pacchetti che verranno installati si suddividono in dipendenze e dipendenze sviluppatore, di seguito verrano spiegate le due categorie.
\textbf{Dipendenze} \\
Nella seguente tabella sono elencate tutte le dipendenze necessarie per la corretta esecuzione dell'applicazione.
\rowcolors{2}{gray!25}{gray!15}
	\setcounter{table}{0}
	\begin{longtable} {
		>{}p{50mm} 
		>{}p{50mm}
		}
	\rowcolor{gray!50}
    @testing-library/jest-dom & ^4.2.4 \TBstrut \\ [2mm]
    @testing-library/react & ^9.3.2 \TBstrut \\ [2mm]
    @testing-library/user-event & ^7.1.2 \TBstrut \\ [2mm]
    csvtojson & 2.0.10 \TBstrut \\ [2mm]
    d3 & ^5.15.0 \TBstrut \\ [2mm]
    electron-is-dev & ^1.1.0 \TBstrut \\ [2mm]
    ml-modules & "https://github.com/Max09081998/ml-modules/tarball/master" \TBstrut \\ [2mm]
    react & ^16.13.0 \TBstrut \\ [2mm]
    react-dom & ^16.13.0 \TBstrut \\ [2mm]
    react-router-dom & ^5.1.2 \TBstrut \\ [2mm]
    react-scripts & 3.4.0 \TBstrut \\ [2mm]

\textbf{Dipendenze sviluppatore} \\
Nella seguente tabella sono elencate tutte le dipendenze necessarie per la corretta esecuzione dell'applicazione durante lo sviluppo della stessa.
\rowcolors{2}{gray!25}{gray!15}
	\setcounter{table}{0}
	\begin{longtable} {
		>{}p{50mm} 
		>{}p{50mm}
		}
	\rowcolor{gray!50}
    electron & ^8.02 \TBstrut \\ [2mm]
    electron-packager & ^14.2.1 \TBstrut \\ [2mm]
    coveralls & ^3.0.9 \TBstrut \\ [2mm]

\subsubsection{Conclusione}
Seguendo i due passaggi descritti nei precedenti paragrafi avrete correttamente installato l'applicazione esterna.
