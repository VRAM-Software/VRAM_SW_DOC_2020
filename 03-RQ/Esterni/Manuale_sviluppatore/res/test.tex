\section{Test}
	\subsection{Applicativo di addestramento}
		\subsubsection{Esecuzione dei test}
			Una volta avviato il server NodeJS, i test di unità vengono avviati con il comando \textbf{npm test} e viene scelta l'opzione \textbf{a}, che avvierà i test con Enzyme e Jest. I risultati dei test vengono visualizzati direttamente nel terminale, inoltre i test vengono rieseguiti automaticamente alla modifica del codice.
		\subsubsection{Scrittura dei test}
			Per ogni componente viene creato un file di test, con il nome \textbf{<nome componente>.test.js}, all'interno del quale vengono definiti i test con Enzyme. Vengono verificati sia il corretto caricamento dei componenti che la correttezza del loro comportamento, assegnando uno stato e verificando i risultati in seguito alle chiamate dei metodi.
	\subsection{SonarCloud}
		SonarCloud misura la qualità del codice, ad ogni commit infatti analizza i sorgenti ed estrae indici su manutenibilità, complessità, affidabilità, duplicazione del codice, numero di bug e vulnerabilità. L'obiettivo è quello di avere codice sempre in buono stato e pronto al rilascio.
	\subsection{Coperatura dei test}
		La copertura dei test viene misurata da Coveralls, che ad ogni nuovo commit esegue il calcolo e ne mantiene uno storico. Per accedere alle statistiche sulla copertura dei test bisogna premere sull'apposito badge nelle repository\glo, che apre la rispettiva pagina sul sito di Coveralls.
