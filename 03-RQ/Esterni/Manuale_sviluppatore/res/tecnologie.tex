\section{Tecnologie utilizzate}
	\subsection{Electron}
		Electron è un framework che permette di sviluppare applicazioni desktop utilizzando tecnologie web, basandosi sul motore di rendering di chromium. Viene utilizzato dall'applicativo di addestramento come contenitore dell'app
	\subsection{React}
		React è un framework JavaScript che si occupa prevalentemente della gestione dell'interfaccia, permettendo di dividerla in componenti modulari e riutilizzabili.
	\subsection{AngularJS}
		AngularJS è un framework JavaScript che si basa su un pattern MVC/MVVM per la realizzazione di applicazioni web. È l'alternativa più popolare per lo sviluppo di plug-in per Grafana\glo.
	\subsection{NodeJS}
		NodeJS è una runtime JavaScript, ovvero un ambiente di esecuzione per codice JavaScript; viene impiegato nella fase di sviluppo come server web per l'applicazione esterna.
	\subsection{JSX}
		JSX è un'estensione sintattica del linguaggio JavaScript che viene utilizzata per definire la struttura e la logica dei componenti in React, includendo frammenti di codice con una sintassi simile al linguaggio HTML.
	\subsection{Jest}
		Jest è un framework per i test di unità dei componenti JavaScript, viene utilizzato con Enzyme per i test dell'applicativo di addestramento.
	\subsection{Enzyme}
		Enzyme è uno strumento di test JavaScript specifico per componenti React, si integra con Jest.
	\subsection{SonarLint}
		SonarLint è un plug-in per IDE che attraverso l'analisi statica\glosp del codice rileva anomalie ed errori comuni. Provvede quindi a segnalare queste vulnerabilità e proporre eventuali soluzioni.
	\subsection{NPM}
		Il Node Package Manager viene utilizzato per la gestione delle dipendenze e la build automation.
		Richiede che all'interno del progetto sia presente il file package.json, che contiene informazioni riguardanti le dipendenze e gli script di build e test.
	\subsection{D3}
		D3 è una libreria JavaScript per produrre visualizzazioni di dati dinamiche e interattive. Impiegata per creare i grafici nell'applicativo di addestramento.
	\subsection{Plotly}
		Plotly è una libreria JavaScript per la visualizzazione di dati, si basa su D3, viene utilizzata dal plug-in per produrre grafici in Grafana\glo.
	\subsection{TypeScript}
		TypeScript è un'estensione del linguaggio JavaScript che aggiunge il supporto ai tipi, viene usato nei plug-in di Grafana\glo.
	\subsection{JSON}
		JSON è un formato per lo scambio di dati basato sul linguaggio JavaScript, viene utilizzato per memorizzare risultati e informazioni degli addestramenti.
	\subsection{CSV}
		CSV è un formato per lo scambio di dati, viene usato per la memorizzazione dei dati di test.
	\subsection{InfluxDB}
		InfluxDB è un time series database, cioè una base di dati che memorizza le informazioni come coppie di valori tempo e dato. Proprio per questo motivo è una delle fonti di dati predilette da Grafana\glo.
	\subsection{Regressione lineare}
		La regressione lineare è un modello di regressione che, a partire da un insieme di dati rappresentabili come punti su un piano, restituisce una stima sul loro sviluppo sotto forma di retta.
	\subsection{Support-vector machines}
		Le support-vector machines sono modelli di apprendimento supervisionati che vengono utilizzati nell'ambito del machine learning\glo: a partire da una serie di dati questi vengono analizzati e viene effettuata una classificazione che li divide in due sottoinsiemi.



