\section{Tecnologie utilizzate}
	\subsection{Electron}
		Electron è un framework che permette di sviluppare applicazioni desktop utilizzando tecnologie web, basandosi sul motore di rendering di chromium. Viene utilizzato dall'applicativo di addestramento come contenitore e soprattutto come strumento per per lo scambio di messaggi asincroni tra View e ViewModel.
	\subsection{React}
		React è un framework JavaScript che si occupa prevalentemente della gestione dell'interfaccia grafica, permettendo di dividerla in componenti modulari e riutilizzabili.
	\subsection{AngularJS}
		AngularJS è un framework JavaScript che si basa su un pattern architetturale MVVM per la realizzazione di applicazioni web. È l'alternativa più popolare per lo sviluppo di plug-in per Grafana\glo.
	\subsection{NodeJS}
		NodeJS è una runtime JavaScript, ovvero un ambiente di esecuzione per codice JavaScript; viene impiegato nella fase di sviluppo come server web per l'applicazione esterna.
	\subsection{JSX}
		JSX è un'estensione sintattica del linguaggio JavaScript che viene utilizzata per definire la struttura e la logica dei componenti in React. Ci permette di scrivere elementi in linguaggio HTML direttamente in un file JavaScript.
	\subsection{Jest}
		Jest è un framework per i test di unità dei componenti JavaScript. Viene utilizzato per i test del plug-in e, combinato ad Enzyme, per i test dell'applicativo di addestramento.
	\subsection{Enzyme}
		Enzyme è uno strumento di test JavaScript specifico per componenti React che si integra con Jest.
	\subsection{SonarLint}
		SonarLint è un plug-in per IDE che attraverso l'analisi statica\glosp del codice rileva anomalie ed errori comuni. Provvede quindi a segnalare queste vulnerabilità e proporre possibili soluzioni.
	\subsection{NPM}
		Il Node Package Manager viene utilizzato per la gestione delle dipendenze e la build automation.
		Richiede che all'interno del progetto sia presente il file package.json, che contiene informazioni riguardanti le dipendenze, gli script di build e i test.
	\subsection{D3}
		D3 è una libreria JavaScript per produrre visualizzazioni di dati dinamiche e interattive. Viene impiegata per creare i grafici nell'applicativo di addestramento.
	\subsection{Plotly}
		Plotly è una libreria JavaScript per la visualizzazione di dati basata su D3. Viene utilizzata dal plug-in per produrre grafici in Grafana\glo.
	\subsection{TypeScript}
		TypeScript è un'estensione del linguaggio JavaScript che aggiunge il supporto ai tipi e viene utilizzato nei plug-in di Grafana\glo.
	\subsection{JSON}
		JSON è un formato per lo scambio di dati basato sul linguaggio JavaScript. Nel nostro prodotto viene utilizzato per memorizzare la configurazione ottimale degli algoritmi di predizione ottenuta dal loro addestramento.
	\subsection{CSV}
		CSV è un formato per lo scambio di dati e viene usato per la memorizzazione dei dati di test.
	\subsection{InfluxDB}
		InfluxDB è un time series database, cioè una base di dati che memorizza le informazioni come coppie di valori tempo e dato. Per questo motivo è una delle fonti di dati predilette da Grafana\glo.
	\subsection{Regressione lineare}
		La regressione lineare è un modello di regressione che, a partire da un insieme di dati rappresentabili come punti su un piano, restituisce una stima sul loro sviluppo sotto forma di retta.
	\subsection{Support-vector machines}
		Le support-vector machines sono modelli di apprendimento supervisionati utilizzati nell'ambito del machine learning\glo: a partire da una serie di dati, questi vengono analizzati e successivamente classificati. La classificazione effettuata li divide in due sottoinsiemi.



