\subsection{Configurazione ambiente applicativo esterno}
\paragraph{Contenuto file: \verb{package.json}}
Il file \verb{package.json} contiene tutte le informazioni e le dipendenze necessarie della nostra applicazione. Gli elementi del file json da evidenziare sono i seguenti:
\textbf{"main"} \\
Questo elemento ha come valore il percorso del file che si occupa di gestire il processo principale dell'applicazione Electron
\textbf{"dependencies"} \\
Questo elemento contiene la seguente lista di pacchetti che sono necessari per il corretto funzionamento dell'applicazione.
\begin{itemize}
    \item csvtojson
    \item d3
    \item electron-is-dev
    \item ml-modules
    \item react
    \item react-dom
    \item react-router-dom
    \item react-scripts

\end{itemize}
\textbf{"devDependencies"} \\
Questo elemento contiene la seguente lista di pacchetti che sono necessari per il corretto funzionamento dell'applicazione durante lo sviluppo.
\begin{itemize}
    \item electron
    \item electron-packager
    \item coveralls
\end{itemize}
\textbf{"scripts"} \\
Questo elemento contiene una lista di tutti i comandi, utili per uno sviluppatore, che possono essere eseguiti da linea di comando.
\begin{itemize}
    \item \textbf{electron}: \verb{npm run electron}, questo comando fa partire un'istanza di Electron;
    \item \textbf{start}: \verb{npm start}, questo comando fa partire un server NodeJs alla porta 3000;
    \item \textbf{build}: \verb{npm build}, questo comando genera una cartella: \verb{build} che contiene i file di produzione di React;
    \item \textbf{test}: \verb{npm test}, questo comando fa eseguire tutti i test dell'applicazione;
    \item \textbf{package-mac}: \verb{npm run package-mac}, questo comando genera una release di produzione per il sistema operativo: MacOS;
    \item \textbf{package-win}: \verb{npm run package-win}, questo comando genera una release di produzione per il sistema operativo: Windows;
    \item \textbf{package-linux}: \verb{npm run package-linux}, questo comando genera una release di produzione per una qualsiasi distribuzione di Linux.
\end{itemize}