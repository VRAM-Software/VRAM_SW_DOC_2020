\section{Estensibilità del prodotto}
	\subsection{Supporto a nuovi algortimi di addestramento}
	Al momento sono stati implementati gli algoritmi di addestramento e di predizione di Support Vector Machine lineare e di Regressione lineare. Grazie alla struttura dell'architettura è possibile aggiungere le varianti delle famiglie di algoritmi di Support Vector Machine e di Regressione quali non lineari, esponenziali e logaritmiche.
	Per realizzare ciò è sufficiente fare una nuova implementazione dell'interfaccia \textit{PerformTraining} con la relativa classe nel Model all'interno dell'applicazione e una nuova implementazione delle interfaccie \textit{PerformPrediction} e \textit{AlgorithmTrainer} nel plug-in.
	\subsection{Supporto a nuove tipologie di file}
		\subsubsection{Lettura}
		Viene offerta la possibilità di ampliare le tipologia di file che vengono utilizzate per la lettura dei dati ricevuti in input di addestramento nell'applicazione. Questo è possibile grazie all'implementazione del template method. Infatti è sufficiente estendere la classe astratta \textit{Read} e quindi implementare il metodo \textit{parser()} per eseguire il parse dei dati nel formato desiderato.
		\subsubsection{Scrittura}
		Viene offerta la possibilità di ampliare le tipologia di file che vengono utilizzate per la scrittura dei dati risultanti dall'addestramento nell'applicazione. Questo è possibile grazie all'implementazione del template method. Infatti è sufficiente estendere la classe astratta \textit{Write} e quindi implementare il metodo \textit{parser()} per eseguire il parse dei dati nel formato desiderato.