\section{Installazione}
\subsection{Installazione degli strumenti}
Gli strumenti che utilizziamo possono essere installati con diverse metodologie; tuttavia presentiamo alcuni esempi per non appesantire questo manuale con informazioni non strettamente pertinenti e superflue.
\subsubsection{Installazione di Node.js}
%\paragraph{Node}\mbox{}\\ [1mm]
Per installare il runtime Javascript Node.js si può visitare il sito \url{https://nodejs.org}. Qui è possibile trovare il download di Node.js per tutti i sistemi operativi. Alternativamente, nei sistemi operativi basati su Linux è possibile utilizzare lo strumento di gestione dei pacchetti fornito dal sistema operativo per installare quelli del runtime Node.js.
Di seguito un esempio per sistemi operativi: \textbf{Debian/Ubuntu}:
\begin{verbatim}
apt-get install nodejs
\end{verbatim}

\subsubsection{Installazione di Git}
Per installare il sistema di versionamento distribuito Git, si può visitare la pagina web \url{https://git-scm.com/downloads}. Qui è possibile trovare il download per MacOS, Windows e sistemi operativi basati su Linux/Unix. Alternativamente, nei sistemi operativi basati su Linux, è possibile utilizzare lo strumento di gestione dei pacchetti fornito. Di seguito un esempio per sistemi operativi basati Debian/Ubuntu:
\begin{verbatim}
apt-get install git
\end{verbatim}