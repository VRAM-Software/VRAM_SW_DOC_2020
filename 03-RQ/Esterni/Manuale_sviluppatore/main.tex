%File principale del documento su cui invocare la compilazione, vedi "istruzioni.txt" per più info

%Preambolo: la parte prima del \begin{document}
\documentclass[12pt,a4paper]{article} %formato del documento e grandezza caratteri

%Input del file metadata.tex della cartella locale "res/"
%lista di comandi presenti in template_latex.tex, da qui posso essere modificati secondo le esigenze

\newcommand{\DocTitle}{Piano di Progetto} %variabile usata dal file template_latex.tex per settare il titolo del documento
%\newcommand{\DocAuthor}{Progetto "Predire in Grafana"} %variabile usata dal file template_latex.tex per settare l'autore del documento
\newcommand{\DocDate}{08 gennaio 2020} %variabile usata dal file template_latex.tex; Impostata manualmente, altrimenti ad ogni compilazione viene messa la data del giorno di compilazione.
\newcommand{\DocDesc}{Piano di progetto del gruppo \textit{VRAM Software}} %variabile usata dal file template_latex.tex per settare la descrizione del documento
\newcommand{\ver}{1.1.1} %variabile usata dal file template_latex.tex per settare la versione del documento
\newcommand{\app}{Corrizzato Vittorio} %variabile usata dal file template_latex.tex per settare l'approvatore del documento
\newcommand{\red}{Corrizzato Vittorio \\ & Toffoletto Masismo \\ & Rampazzo Marco \\ & Santagiuliana Vittorio} %variabile usata dal file template_latex.tex per settare il redattore del documento
\newcommand{\test}{Santagiuliana Vittorio \\ & Spreafico Alessandro \\ & Schiavon Rebecca} %variabile usata dal file template_latex.tex per settare il verificatore del documento
\newcommand{\stat}{Approvato} %variabile usata dal file template_latex.tex per settare lo stato del documento
\newcommand{\use}{Esterno} %variabile usata dal file template_latex.tex per indicare l'uso del documento %Contiene le varibili che descrivono il documento

%Input di file di configurazione presi dalla cartella "Template-LaTeX/config/", uguali per tutti i documenti
%Attenzione bisogna impostare il percorso del file!
% Tutti i pacchetti usati, da inserire nel preambolo prima delle configurazioni

\usepackage[T1]{fontenc} %Permette la sillabazione su qualsiasi testo contenente caratteri
\usepackage[utf8]{inputenc} %Serve per usare la codifica utf-8
\usepackage[english,italian]{babel} %Imposta italiano lingua principale, inglese secondaria. Es. serve per far apparire "indice" al posto di "contents"

\usepackage{graphicx} %Serve per includere le immagini

\usepackage[hypertexnames=false]{hyperref} %Gestisce i riferimenti/link. Es. Serve per rendere clickabili le sezioni dell'indice

\usepackage{float} %Serve per migliore la definizione di oggetti fluttuanti come figure e tabelle. Es. poter usare l'opzione [H] nelle figure ovvero tenere fissate le immagini che altrimenti LaTeX si sposta a piacere.

\usepackage{listings} %Serve per poter mettere snippets di codice nel testo

\usepackage{lastpage} %Serve per poter introdurre un'etichetta a cui si può fare riferimento Es. piè di pagina; poter fare " \rfoot{\thepage\ di \pageref{LastPage}} "

\usepackage{fancyhdr} %Per header e piè di pagina personalizzati

%Sono alcuni package che potranno esserci utili in futuro
%\usepackage{charter}
%\usepackage{eurosym}
\usepackage{subcaption}
%\usepackage{wrapfig}
%\usepackage{background}
\usepackage{longtable} % tabella che può continuare per più di una pagina
\usepackage[table]{xcolor} % ho dovuto aggiungere table in modo da poter colorare le row della tabella, dava: undefined control sequences
%\usepackage{colortbl}

\usepackage{dirtree} % usato per creare strutte tree-view in stile filesystem
\usepackage{xspace} % usato per inserire caratteri spazio
\usepackage[official]{eurosym}
\usepackage{pdflscape} %Inclusione pacchetti
% Configurazioni varie, da inserire nel preambolo dopo i pacchetti

\hypersetup{hidelinks} %serve per nascondere riquadri rossi che circondano i link 

\lstset{literate= {à}{{\`a}}1 } %Permette di usare lettere accentate nei listings

\pagestyle{fancy} %Imposto stile pagina
\fancyhf{} %Reset, se lo tolgo LaTex mette impostazioni di default (p.es numerazione pagine di default)


\lhead{\includegraphics[scale=0.25]{img/logo_header.png}} %Left header che compare in ogni pagina
%\rhead{\leftmark} %Nome della top-level structure (p.es. Section in article o Chapter in book) in ogni pagina
\rhead{\DocTitle} %Right header

\newcommand{\glo}{$_G$} %Comando per aggiungere il pedice G
\newcommand{\glosp}{$_G$ } %Comando per aggiungere il pedice G con spazio

\newcommand\Tstrut{\rule{0pt}{2.6ex}} % top padding
\newcommand\Bstrut{\rule[-0.9ex]{0pt}{0pt}} % bottom padding
\newcommand{\TBstrut}{\Tstrut\Bstrut} % top & bottom padding

%Setto il colore dei link
%\hypersetup{
%	colorlinks,
%	linkcolor=[HTML]{404040},
%	citecolor={purple!50!black},
%	urlcolor={blue!50!black}
%}

%Tabelle e tabulazione (può tornare utile)
%\setlength{\tablcolsep}{10pt}
%\renewcommand{\arraystretch}{1.4}

%Comando per aggiungere le pagine di ogni sezione
%\newcommand{\newSection}[1]{%
%	\input{res/sections/#1}
%}

% Comandi per aggiungere padding a parole contenute nella tabella; è una specie di strut (un carattere invisibile)
%\newcommand\Tstrut{\rule{0pt}{2.6ex}} % top padding
%\newcommand\Bstrut{\rule[-0.9ex]{0pt}{0pt}} % bottom padding
%\newcommand{\TBstrut}{\Tstrut\Bstrut} % top & bottom padding  %Configurazione pacchetti

\begin{document}
	%Input del file "frontmatter" preso dalla cartella "Template-LaTeX/config/", uguale per tutti i documenti
	%Attenzione bisogna impostare il percorso del file!
	% #### FRONTESPIZIO (frontmatter) ####
\setlength{\headheight}{33pt} %Distanzia l'header
\pagenumbering{gobble} %Toglie il numero di pagina
\begin{titlepage}
	\begin{center}
		\vspace*{-2cm}
		\includegraphics[scale=0.6]{img/logo.png} \\ %Logo
		\vspace{0.4cm} %Aggiunge uno spazio verticale di 0.5 cm
		
		{\LARGE Progetto "Predire in Grafana"} \\ %Nome progetto
		\vspace{0.4cm} %Attenzione a mettere il punto e NON la virgola
		
		{\Huge \textbf{\DocTitle}} \\ %Titolo, prende variabile definita in metadata.tex
		\vspace{0.4cm}
		
		\DocDate \\ %Data, prende variabile definita in metadata.tex
		\vspace{0.4cm}
		
		%Allineamento colonne: l=left r=right c=center, 
		%va specificato per ogni colonna
		%Se si vuole la riga tra colonne mettere "|"
		
		\begin{tabular}{r | l} %Elementi colonne separate da "&", le righe finiscono con "\\"
			Versione             & \ver \\
			Approvazione         & \app \\ 
			Redazione            & \red \\
			Verifica             & \test \\
			Stato                & \stat \\
			Uso                  & \use \\
		    Destinato a          & Zucchetti \\
						         & Prof. Vardanega Tullio \\
						         & Prof. Cardin Riccardo \\
			Email di riferimento & vram.software@gmail.com
		\end{tabular}
		\vfill
		\textbf{Descrizione} \\
		\DocDesc
	\end{center}
\end{titlepage}
\clearpage

% #### Impostazione header, footer  e numerazione pagine ####
\pagenumbering{arabic} %Pagine con i numeri arabi + reset a 1
\renewcommand{\footrulewidth}{0.4pt} %Di default footrulewidth==0 e quindi è invisibile, di default \headrulewith==0.4pt
\rfoot{\thepage\ di \pageref{LastPage}} %Pagina n di m, con numeri Arabi; usa il pacchetto "lastpage", in caso non sia possibile usare tale pacchetto mettere al fondo dell'ultima pagina "\label{LastPage}"

% #### Tabella dei log ####
% \textbf = grassetto; \Large = font più grande
% \rowcolors{quanti colori alternare}{colore numero riga pari}{colore numero riga dispari}: colori alternati per riga
% \rowcolor{color}: cambia colore di una riga
% p{larghezza colonna}: p è un tipo di colonna di testo verticalmente allineata sopra, ci sarebbe anche m che è centrata a metà ma non è precisa per questo utilizzo TBStrut; la sintassi >{\centering} indica che il contenuto della colonna dovrà essere centrato
% \TBstrut fa parte di alcuni comandi che ho inserito in config.tex che permetto di aggiungere un po' di padding al testo
% \\ [2mm] : questra scrittura indica che lo spazio dopo una break line deve essere di 2mm
% 

%\setcounter{secnumdepth}{0}
%\hfill \break
%\textbf{\Large{Diario delle modifiche}} \\


\addtocontents{toc}{\protect\setcounter{tocdepth}{0}} %Inserire questo per escludere una sezione dall'indice.

\section*{Registro delle modifiche} %Asterisco per fare sezione non numerata
\rowcolors{2}{gray!25}{gray!15}
\begin{longtable} {
		>{\centering}p{17mm} 
		>{\centering}p{19.5mm}
		>{\centering}p{24mm} 
		>{\centering}p{30mm} 
		>{}p{32mm}}
	\rowcolor{gray!50}
	\textbf{Versione} & \textbf{Data} & \textbf{Nominativo} & \textbf{Ruolo} & \textbf{Descrizione} \TBstrut \\
	4.1.1 & 2020-03-02 & Rampazzo Marco & \textit{Responsabile di progetto} & Approvazione documento. \TBstrut \\ [2mm]
	3.2.4 & 2020-03-01 & Spreafico Alessandro e Santagiuliana Vittorio & \textit{Amministratore} e \textit{Verificatore} & Aggiornamento e verifica dei paragrafi §2.1.5, §2.2.5, §3.2.5, §3.3.5 e §3.4.6. \TBstrut \\ [2mm]
	3.2.3 & 2020-02-28 & Spreafico Alessandro e Schiavon Rebecca & \textit{Amministratore} e \textit{Verificatore} & Stesura e verifica dei paragrafi §B e §C. \TBstrut \\ [2mm]
	3.1.2 & 2020-02-20 & Dalla Libera Marco e Santagiuliana Vittorio & \textit{Amministratore} e \textit{Verificatore} & Aggiornamento e verifica dei paragrafi §2.1.5, §2.2.5, §3.2.5, §3.3.5 e §3.4.6. \TBstrut \\ [2mm]
	3.1.1 & 2020-02-16 & Rampazzo Marco & \textit{Responsabile di progetto} & Approvazione documento. \TBstrut \\ [2mm]
	2.3.2 & 2020-02-14 & Spreafico Alessandro e Toffoletto Massimo & \textit{Amministratore} e \textit{Verificatore} & Aggiornamento e verifica dei paragrafi §2.1.5, §2.2.5, §3.2.5, §3.3.5 e §3.4.6. \TBstrut \\ [2mm]
	2.2.2 & 2020-02-12 & Spreafico Alessandro e Toffoletto Massimo & \textit{Amministratore} e \textit{Verificatore} & Stesura e verifica dei paragrafi §3.3, §3.4.5, §3.6.5 e §4.1.4. \TBstrut \\ [2mm]
	2.1.1 & 2020-02-08 & Rampazzo Marco & \textit{Responsabile di progetto} & Approvazione documento. \TBstrut \\ [2mm]
	1.4.4 & 2020-02-06 & Dalla Libera Marco e Santagiuliana Vittorio & \textit{Amministratore} e \textit{Verificatore} & Aggiornamento e verifica dei paragrafi §2.1.5, §2.2.5, §3.2.5, §3.3.5 e §3.4.6. \TBstrut \\ [2mm]
	1.3.3 & 2020-02-04 & Dalla Libera Marco e Schiavon Rebecca & \textit{Amministratore} e \textit{Verificatore} & Stesura e verifica dei paragrafi §2.2.4.3 e §3.6. \TBstrut \\ [2mm]
	1.2.2 & 2020-02-02 & Dalla Libera Marco e Corrizzato Vittorio & \textit{Amministratore} e \textit{Verificatore} & Correzione della struttura del documento, dei riferimenti, dello stile tipografico, del registro delle modifiche e associazione più precisa delle metriche ai relativi processi tutto come segnalato dal committente. \TBstrut \\ [2mm]
	1.1.1 & 2020-01-11 & Corrizzato Vittorio & \textit{Responsabile di progetto} & Approvazione documento. \TBstrut \\ [2mm]
	0.8.4 & 2020-01-11 & Santagiuliana Vittorio e Schiavon Rebecca & \textit{Amministratore} e \textit{Verificatore} & Aggiornamento e verifica finale. \TBstrut \\ [2mm]
	0.8.3 & 2020-01-06 & Santagiuliana Vittorio e Schiavon Rebecca & \textit{Amministratore} e \textit{Verificatore} & Aggiornamento e verifica dei paragrafi §2.1.4.2, §2.1.4.3, §4.1.6 e §A. \TBstrut \\ [2mm]
	0.7.3 & 2020-01-02 & Rampazzo Marco e Santagiuliana Vittorio & \textit{Amministratore} e \textit{Verificatore} & Aggiornamento e verifica dei paragrafi §3.3, §3.3.6 e §3.4.5. \TBstrut \\ [2mm]
	0.6.2 & 2020-01-02 & Rampazzo Marco e Santagiuliana Vittorio & \textit{Amministratore} e \textit{Verificatore} & Aggiornamento e verifica dei paragrafi §2.2.4.1 e §2.1.4.3. \TBstrut \\ [2mm]
	0.5.2 & 2019-12-23 & Rampazzo Marco e Spreafico Alessandro & \textit{Amministratore} e \textit{Verificatore} & Stesura e verifica dei paragrafi §2.2.4.1, §2.1.4.2, §3.1.6 e §3.2.5. \TBstrut \\ [2mm]
	0.4.1 & 2019-12-04 & Toffoletto Massimo e Corrizzato Vittorio & \textit{Amministratore} e \textit{Verificatore} & Aggiornamento e verifica dei paragrafi §2.1.4.1 e §2.2.5. \TBstrut \\ [2mm]
	0.3.1 & 2019-11-23 & Corrizzato Vittorio e Schiavon Rebecca & \textit{Amministratore} e \textit{Verificatore} & Stesura e verifica dei paragrafi §3.4, §3.5 e §4. \TBstrut \\ [2mm]
	0.2.0 & 2019-11-23 & Corrizzato Vittorio e Schiavon Rebecca & \textit{Amministratore} e \textit{Verificatore} & Stesura e verifica dei paragrafi §3.1 e §3.2. \TBstrut \\ [2mm]
	0.1.0 & 2019-11-23 & Corrizzato Vittorio e Schiavon Rebecca & \textit{Amministratore} e \textit{Verificatore} & Stesura e verifica dei paragrafi §1, §2.1.4.1, §2.1.5 e §2.2.5. \TBstrut \\ [2mm]
\end{longtable}

\addtocontents{toc}{\protect\setcounter{tocdepth}{4}} %Inserire questo per ripristinare il normale inserimento delle sezioni nell'indice. 4 significa fino al paragrah
\clearpage

% #### INDICE (tableofcontents) ####
\tableofcontents %Provoca la stampa dell'indice
\clearpage

\setcounter{secnumdepth}{4} %Permette di andare fino alla profondità del paragraph con la numerazione delle sezioni %Imposta il frontespizio, l'indice, header e footer
	%Tutte le sezioni del documento
	\section{Introduzione}
\subsection{Scopo del documento}

\subsection{Scopo del prodotto}

\subsection{Glossario}

\subsection{Riferimenti}
\subsubsection{Informativi}
\begin{itemize}
	\item 
\end{itemize}
	\pagebreak
	\section{Requisiti di sistema}
Vengono riportati i requisiti minimi per lo sviluppo e l'esecuzione del nostro prodotto\glo. Essi sono uguali sia per l'applicazione esterna che per il plug-in.

\subsection{Requisiti hardware}
I requisiti hardware minimi che devono essere soddisfatti per garantire un corretto funzionamento sono:
\begin{itemize}
	\item RAM: 2 GB;
	\item Memoria interna libera: 5 GB;
	\item Processore: dual core.
\end{itemize}

\subsection{Sistemi operativi}
Il prodotto\glosp è stato sviluppato, testato e utilizzato sui seguenti sistemi operativi, perciò è garantita la compatibilità con:
\begin{itemize}
	\item distribuzioni GNU/Linux post 2019 basate su DEB o RPM;
	\item macOS v. 10.15;
	\item Windows v. 10.
\end{itemize}

\subsection{Browser compatibili}
Il plug-in è stato sviluppato, testato e utilizzato sui seguenti browser perciò ne è garantita la compatibilità:
\begin{itemize}
	\item Google Chrome v. 58;
	\item Mozilla Firefox v. 54;
	\item Microsoft Edge v. 14;
	\item Microsoft Internet Explorer v. 11;
	\item Safari v. 10;
	\item Opera v. 55.
\end{itemize}
Per l'applicazione esterna alla piattaforma Grafana\glosp viene utilizzato Chromium che è integrato, dunque non è necessario l'utilizzo di un browser.
Per ottenere un corretto funzionamento del prodotto è richiesta l'abilitazione di Javascript.

	\pagebreak
	\section{Installazione}
\subsection{Installazione degli strumenti}
Gli strumenti che utilizziamo possono essere installati con diverse metodologie; tuttavia presentiamo solo alcuni esempi per non appesantire questo manuale con informazioni non strettamente pertinenti e superflue.
\subsubsection{Installazione di Node.js}
%\paragraph{Node}\mbox{}\\ [1mm]
Per installare il runtime Javascript Node.js si può visitare il sito \url{https://nodejs.org}. Qui è possibile trovare il download di Node.js per tutti i sistemi operativi. Alternativamente, nei sistemi operativi basati su Linux è possibile utilizzare lo strumento di gestione dei pacchetti fornito dal sistema operativo per installare quelli del runtime Node.js.
Di seguito un esempio per sistemi operativi: \textbf{Debian/Ubuntu}:
\begin{verbatim}
apt-get install nodejs
\end{verbatim}

\subsubsection{Installazione di Git}
Per installare il sistema di versionamento distribuito Git, si può visitare la pagina web \url{https://git-scm.com/downloads}. Qui è possibile trovare il download per MacOS, Windows e sistemi operativi basati su Linux/Unix. Alternativamente, nei sistemi operativi basati su Linux, è possibile utilizzare lo strumento di gestione dei pacchetti fornito. Di seguito un esempio per sistemi operativi basati Debian/Ubuntu:
\begin{verbatim}
apt-get install git
\end{verbatim}
	\subsection{Installazione di Grafana}
Per installare Grafana\glo, visitare la pagina \url{https://grafana.com/get}. Qui è possibile trovare il download per MacOS, Windows e sistemi operativi basati su Linux/Unix.
\subsubsection{Eseguire il servizio WEB Grafana} Per eseguire il servizio WEB Grafana\glo, aprire la cartella "bin" dell'installazione Grafana\glosp ed a seconda del sistema operativo eseguire:
\begin{itemize}
	\item \textbf{Windows}: doppio click sul file "grafana-server";
	\item \textbf{Linux e Mac}: eseguire in una shell il comando:
		\begin{verbatim}
			./grafana-server web
		\end{verbatim}
\end{itemize}
Collegarsi quindi con un browser all'indirizzo \url{http://localhost:3000/}. Le credenziali richieste al primo avvio sono username "admin" e password "admin".
\begin{figure}[H] 	
	\begin{center}
		\includegraphics[width=10cm,height=\textheight,keepaspectratio]{img/grafana-login.png}
	\end{center}
	\caption{Pagina di login di Grafana}	
\end{figure}

%\subsection{Installazione plug-in Grafana}
%\subsubsection{Requisiti}
%\begin{itemize}
%	\item \textbf{Grafana\glo};
%	\item \textbf{Node.js}: runtime Javascript che permette di eseguire codice Javascript fuori da un browser. L'installazione di npm (Node Package Manager) non è richiesta dato che viene installato automaticamente durante l'installazione di Node.js;
%	\item \textbf{Git}: sistema di versionamento distribuito.
%\end{itemize}

\subsection{Installazione del plug-in}
\subsubsection{Clonare la repository da GitHub}%\mbox{}\\ [1mm]
Per clonare la repository dell'applicazione, aprire un terminale e usare il comando cd per spostarsi in una cartella a propria scelta ed eseguire il comando:
\begin{verbatim}
	git clone https://github.com/VRAM-Software/grafana_prediction.git
\end{verbatim}
Infine con il comando 
\begin{verbatim}
	cd ./grafana_prediction_plugin
\end{verbatim}
si accede alla cartella che contiene il codice sorgente del plug-in.

\subsubsection{Installare le dipendenze\glo}%\mbox{}\\ [1mm]
Per il corretto funzionamento dell'applicazione è necessario installare tutte le dipendenze\glosp elencate precedentemente. Per farlo eseguire il comando:
\begin{verbatim}
	npm install
\end{verbatim}
I pacchetti che vengono installati si suddividono in dipendenze e dipendenze sviluppatore.

\paragraph*{Dipendenze}\mbox{}\\ [1mm]
Nella seguente tabella sono elencate tutte le dipendenze necessarie per la corretta esecuzione del plug-in.
\mbox{}\\ [1mm]
\rowcolors{2}{gray!25}{gray!15}
\setcounter{table}{0}
\begin{longtable} {
		>{}p{65mm} 
		>{}p{30mm}
	}
	\rowcolor{gray!50}
	\textbf{Pacchetto} & \textbf{Versione} \TBstrut \\ [2mm]
	\verb|@akanass/rx-http-reques| & \verb|^3.3.0|  \TBstrut \\ [2mm]
	\verb|@babel/preset-env| & \verb|^7.9.0| \TBstrut \\ [2mm]
	\verb|influx| & \verb|^5.5.1| \TBstrut \\ [2mm]
	\verb|jquery| & \verb|^3.4.1| \TBstrut \\ [2mm]
	\verb|lodash| & \verb|^4.17.10| \TBstrut \\ [2mm]
	\verb|luxon| & \verb|^1.22.2| \TBstrut \\ [2mm]
	\verb|ml-modules|\footnote{Il pacchetto \texttt{ml-modules} usato nel nostro progetto è una versione modificata dell'ononimo pacchetto} & \verb|0.1.0| \TBstrut \\ [2mm]
	\verb|plotly.js| & \verb|^1.52.3| \TBstrut \\ [2mm]
	\verb|rxjs| & \verb|^6.5.5| \TBstrut \\ [2mm]
	\verb|scriptjs| & \verb|^2.5.9| \TBstrut \\ [2mm]
	\rowcolor{white}
	\caption{Dipendenze per eseguire applicazione}
\end{longtable}

\paragraph*{Dipendenze sviluppatore}\mbox{}\\ [1mm]
Nella seguente tabella sono elencate tutte le dipendenze necessarie per la corretta esecuzione del plug-in durante lo sviluppo.
\rowcolors{2}{gray!25}{gray!15}
\setcounter{table}{1}
\begin{longtable} {
		>{}p{65mm} 
		>{}p{30mm}
	}
	\rowcolor{gray!50}
	\textbf{Pacchetto} & \textbf{Versione} \TBstrut \\ [2mm]
	\verb|@grafana/data| & \verb|^next| \TBstrut \\ [2mm]
	\verb|@grafana/toolkit| & \verb|^6.7.1| \TBstrut \\ [2mm]
	\verb|@grafana/ui| & \verb|^next| \TBstrut \\ [2mm]
	\verb|@types/grafana| & \verb|0.1.0| \TBstrut \\ [2mm]
	\verb|@types/luxon| & \verb|^1.22.0| \TBstrut \\ [2mm]
	\verb|coveralls| & \verb|^3.0.9| \TBstrut \\ [2mm]
	\verb|grafana-sdk-mocks| & \verb|0.1.0| \TBstrut \\ [2mm]
	\rowcolor{white}
	\caption{Dipendenze specifiche per lo sviluppatore}
\end{longtable}
\mbox{}\\ [1mm]
Seguendo i due passaggi descritti nei precedenti paragrafi, il plug-in è installato correttamente.

\subsubsection{Eseguire il plug-in}%\mbox{}\\ [1mm]
Per eseguire il plug-in è necessario copiare il contenuto del repository "grafana\_prediction\_plugin", ad eccezione delle cartelle "node\_modules" e "coverage", all'interno della cartella "data/plugins" dell'installazione Grafana\glo. Accedendo all'interfaccia web si potrà quindi abilitare il plug-in dall'apposita sezione plug-in delle impostazioni.
\\
\\
\begin{figure}[H] 	
	\begin{center}
		\includegraphics[width=\textwidth,height=\textheight,keepaspectratio]{img/plugin-directory.png}
	\end{center}
	\caption{Selezione plug-in nella directory}	
\end{figure}

\subsubsection{Impacchettare il plug-in}%\mbox{}\\ [1mm]
Per generare una release di produzione\glosp basta eseguire il comando:
\begin{verbatim}
	npm run build	
\end{verbatim}
e successivamente creare un archivio zip dell'intero repository escludendo le cartelle "node\_modules" e "coverage". Questo plug-in potrà poi essere distribuito e installato da altri utenti. 

	\subsection{Installazione applicativo esterno}
\subsubsection{Requisiti}
\begin{itemize}
    \item \textbf{Node}: runtime Javascript che permette di eseguire codice Javascript fuori da un browser;
    \item \textbf{Git}: sistema di versionamento distribuito.
\end{itemize}

\subsubsection{Installazione degli strumenti}
\paragraph{Node}
Per installare il runtime Javascript Node.js andate sul sito \href{https://nodejs.org}. Qui potrete trovare il download di Node.js per tutti i sistemi operativi. Alternativamente, su sistemi operativi basati su Linux potrete utilizzare lo strumento d'installazione di pacchetti fornito per installare i pacchetti del runtime Node.js.
\textbf{Debian/Ubuntu} \\
\begin{verbatim}
    apt-get install nodejs
\end{verbatim}
\textbf{OpenSUSE} \\
L'alias: \verb{nodejs4} potrebbe variare a seconda della versione del vostro sistema operativo.
\begin{verbatim}
    zypper install nodejs4
\end{verbatim}
\textbf{Arch Linux} \\
\begin{verbatim}
    pacman -S nodejs npm
\end{verbatim}

\paragraph{Git}
Per installare il sistema di versionamento distribuito Git visitate la pagina \href{https://git-scm.com/downloads}. Qui potrete trovare il download per MacOS, Windows e sistemi operativi basati su Linux/Unix.

\subsubsection{Installazione dell'applicazione}
\paragraph{Clonare la repository da GitHub}
Per clonare la repository dell'applicazione aprite un terminale e usate il comando \verb{cd} per muovervi in una cartella del vostro computer, eseguite il comando: \verb{git clone https://github.com/VRAM-Software/grafana_prediction.git}. Infine con il comando \verb{cd ./prediction_configuration_utility} spostatevi nella cartella che contiene il codice sorgente dell'applicativo esterno.

\paragraph{Contenuto file: \verb{package.json}}
Il file \verb{package.json} contiene tutte le informazioni e i requisiti necessari della nostra applicazione. Le informazioni importanti sono i seguenti attributi:
\textbf{"main"} \\
Questo attributo ha come valore il percorso del file che si occupa di gestire il processo principale dell'applicazione Electron
\textbf{"dependencies"} \\
Questo attributo contiene la seguente lista di pacchetti che sono necessari per il corretto funzionamento dell'applicazione.
\begin{itemize}
    \item csvtojson
    \item d3
    \item electron-is-dev
    \item ml-modules
    \item react
    \item react-dom
    \item react-router-dom
    \item react-scripts
    \item @testing-library/jest-dom
    \item @testing-library/react
    \item @testing-library/user-event
\end{itemize}
\textbf{"devDependencies"} \\
Questo attributo contiene la seguente lista di pacchetti che sono necessari per il corretto funzionamento dell'applicazione durante lo sviluppo.
\begin{itemize}
    \item electron
    \item electron-packager
    \item coveralls
\end{itemize}
\textbf{"scripts"} \\
Questo attributo contiene una lista di tutti i comandi, utili per uno sviluppatore, che possono essere eseguiti da linea di comando.
\begin{itemize}
    \item \textbf{electron}: \verb{npm run electron}, questo comando fa partire un'istanza di Electron;
    \item \textbf{start}: \verb{npm start}, questo comando fa partire un server NodeJs alla porta 3000;
    \item \textbf{build}: \verb{npm build}, questo comando genera una cartella: \verb{build} che contiene i file di produzione di React;
    \item \textbf{test}: \verb{npm test}, questo comando fa eseguire tutti i test dell'applicazione;
    \item \textbf{package-mac}: \verb{npm run package-mac}, questo comando genera una release di produzione per il sistema operativo: MacOS;
    \item \textbf{package-win}: \verb{npm run package-win}, questo comando genera una release di produzione per il sistema operativo: Windows;
    \item \textbf{package-linux}: \verb{npm run package-linux}, questo comando genera una release di produzione per una qualsiasi distribuzione di Linux.
\end{itemize}

\paragraph{Installare le dipendenze}
Per il corretto funzionamento dell'applicazione è necessario installare tutte le dipendenze elencate precedentemente, per farlo eseguite il comando \verb{npm install}.

\paragraph{Eseguire l'applicazione}
Per eseguire l'applicazione eseguire in un terminale, nella cartella dell'applicazione, il comando \verb{npm start}. Questo comando farà partire un server Node.js nella porta 3000. A questo punto eseguite, in un altro terminale, il comando \verb{npm run electron} per far partire l'applicazione.

\paragraph{Impacchettare l'applicazione}
Per generare una release di produzione si hanno tre possibilità a seconda di quale sistema operativo si vuole utilizzare.
\textbf{Windows}: per il sistema operativo Windows. \\
\begin{verbatim}
    npm build
    npm run package-win 
\end{verbatim}
\textbf{MacOS}: per il sistema operativo MacOS \\
\begin{verbatim}
    npm build
    npm run package-mac 
\end{verbatim}
\textbf{Linux}: per una distribuzione linux \\
\begin{verbatim}
    npm build
    npm run package-linux 
\end{verbatim}

	\pagebreak
	\section{Configurazione dell'ambiente di lavoro}
\subsection{Configurazione ambiente di sviluppo integrato WebStorm}
La configurazione base di WebStorm prevede la corretta configurazione dei path di sistema e l'apertura di un progetto.
\subsubsection{Configurazione dei path di sistema}
Aprire le impostazioni di WebStorm dal suo menù "File" ed utilizzando la barra di ricerca cercare "Node.js and NPM". Verificare quindi che entrambe le voci "Node interpreter" e "Package manager" siano correttamente impostate.
\\
\\
\includegraphics[width=\textwidth,height=\textheight,keepaspectratio]{img/node-npm.png}
\subsubsection{Importazione di un progetto}
Dal menu "File" selezionare la voce "Open" e successivamente la root directory del repository desiderato.

\pagebreak
\subsection{Configurazione plug-in SonarLint per WebStorm}
\subsubsection{Configurazione globale}
Per configurare il plug-in SonarLint, aprire le impostazioni di WebStorm dal suo menù "File". Quindi nella sezione "Tools" selezionare la voce "SonarLint". Nella finestra "Visualizzare", selezionare il "+" per aggiungere una connessione al servizio WEB SonarCloud.
\\
\\
\includegraphics[width=\textwidth,height=\textheight,keepaspectratio]{img/connection.png}
\pagebreak
\\
Dare un nome al collegamento e selezionare SonarCloud, successivamente premere "Next".
\\
\\
\includegraphics[width=\textwidth,height=\textheight,keepaspectratio]{img/connection-name.png}
\\
Premere su "Create Token".
\\
\\
\includegraphics[width=\textwidth,height=\textheight,keepaspectratio]{img/token.png}
\\
Scegliere un nome e creare il token, copiarlo quindi nella finestra WebStorm precedente.
%\\
%\\
%\includegraphics[width=\textwidth,height=\textheight,keepaspectratio]{img/sonarcloud.png}
La connessione viene verificata ed è possibile visualizzarla nell'elenco delle connessioni.
\subsubsection{Configurazione per progetto}
Dopo aver terminato la configurazione globale è possibile configurare i singoli progetti. Per farlo, aprire le impostazioni di WebStorm dal suo menù "File". Quindi, nella sezione "Tools", espandere la voce "SonarLint" e selezionare "Project Settings".
\\
\\
\includegraphics[width=\textwidth,height=\textheight,keepaspectratio]{img/sonarlint-project.png}
Selezionare la connessione a SonarCloud desiderata ed inserire una delle chiavi seguenti, a seconda del progetto che si sta configurando:
\begin{itemize}
	\item \textbf{plug-in Grafana}: VRAM-Software\_grafana\_prediction\_plugin;
	\item \textbf{Applicativo esterno}: VRAM-Software\_prediction\_configuration\_utility.
\end{itemize}

\subsection{Configurazione ambiente plug-in Grafana}
\subsubsection{Contenuto file package.json}
Il file package.json contiene tutte le informazioni e i requisiti necessari della nostra applicazione. Le informazioni importanti sono i seguenti attributi:
\begin{itemize}
	\item dependencies: contiene la seguente lista di pacchetti che sono necessari per il corretto funzionamento dell'applicazione:
		\begin{itemize}
			\item jquery;
			\item lodash;
			\item moment;
			\item plotly.js;
			\item scriptjs.
		\end{itemize}
	\item devDependencies: questo attributo contiene la seguente lista di pacchetti che sono necessari per il corretto funzionamento dell'applicazione durante lo sviluppo:
		\begin{itemize}
			\item @grafana/data;
			\item @grafana/toolkit;
			\item @grafana/ui;
			\item @types/grafana;
			\item grafana-sdk-mocks;
			\item coveralls.
		\end{itemize}
	\item{scripts}: questo attributo contiene una lista di tutti i comandi, utili per uno sviluppatore, che possono essere eseguiti da linea di comando:
		\begin{itemize}
			\item \textbf{build}: il comando seguente genera una cartella chiamata dist che contiene i file di produzione del plug-in Grafana\glo;
			\begin{verbatim}
				npm run build
			\end{verbatim}
			\item \textbf{test}: il comando fa eseguire tutti i test automatici dell'applicazione;
			\begin{verbatim}
				npm test
			\end{verbatim}
			\item \textbf{dev}: il comando genera una release di debug da utilizzare durante lo sviluppo, eseguendo inoltre i linting integrati nella dipendenza npm @grafana/toolkit. Non esegue i test automatici;
			\begin{verbatim}
				npm run dev
			\end{verbatim}
			\item \textbf{ci-test}: il comando, pensato per essere eseguito nell'ambiente di continuous integration, fa eseguire tutti i test automatici dell'applicazione e calcola il code coverage del codice;
			\begin{verbatim}
				npm run ci-test
			\end{verbatim}
			\item \textbf{watch}: il comando esegue il comando "dev" in modalità "watch", ovvero segnalando in automatico gli errori di linting durante la scrittura del codice.
			\begin{verbatim}
				npm run watch
			\end{verbatim}
		\end{itemize}
\end{itemize}

	\subsection{Configurazione ambiente applicativo esterno}
\subsubsection{Contenuto file: package.json}%\mbox{}\\ [1mm]
Il file \verb|package.json | contiene tutte le informazioni e le dipendenze necessarie della nostra applicazione.
\begin{itemize}
    \item \textbf{"main"}: Questo elemento ha come valore il percorso del file che si occupa di gestire il processo principale dell'applicazione Electron;
    \item \textbf{"dependencies"}: Questo elemento contiene la seguente lista di pacchetti che sono necessari per il corretto funzionamento dell'applicazione:
        \begin{itemize}
            \item csvtojson;
            \item d3;
            \item electron-is-dev;
            \item ml-modules;
            \item react;
            \item react-dom;
            \item react-router-dom;
            \item react-scripts.
        \end{itemize}
    \item \textbf{"devDependencies"}: Questo elemento contiene la seguente lista di pacchetti necessari per il corretto funzionamento dell'applicazione durante lo sviluppo:
        \begin{itemize}
            \item electron;
            \item electron-packager;
            \item coveralls.
        \end{itemize} 
    \item \textbf{"scripts"}: Questo elemento contiene una lista di tutti i comandi, utili per uno sviluppatore, che possono essere eseguiti da linea di comando:
        \begin{itemize}
            \item \textbf{electron}: il comando seguente avvia un'istanza di Electron;
            \begin{verbatim}
            	npm run electron
            \end{verbatim}
            \item \textbf{start}: il comando seguente avvia un server Node.js alla porta 3000;
            \begin{verbatim}
	            npm start
            \end{verbatim}
            \item \textbf{build}: il comando seguente genera una cartella: \verb|build| che contiene i file di produzione di React;
            \begin{verbatim}
            	npm build
            \end{verbatim}
            \item \textbf{test}: il comando seguente esegue tutti i test dell'applicazione;\\
            \begin{verbatim}
            	npm test
            \end{verbatim}
            \item \textbf{package-mac}: il comando seguente genera una release di produzione per il sistema operativo: MacOS;
            \begin{verbatim}
            	npm run package-mac
            \end{verbatim}
            \item \textbf{package-win}: il comando seguente genera una release di produzione per il sistema operativo: Windows;
            \begin{verbatim}
            	npm run package-win
            \end{verbatim}
            \item \textbf{package-linux}: il comando seguente genera una release di produzione per una qualsiasi distribuzione di Linux.
            \begin{verbatim}
            	npm run package-linux
            \end{verbatim}
        \end{itemize}
\end{itemize}

	\pagebreak
	\section{Test}
	\subsection{Esecuzione dei test}
		\subsubsection{Applicativo di addestramento}
			Una volta avviato il server NodeJS, i test di unità con Enzyme e Jest vengono avviati con il comando \textbf{npm test} scegliendo l'opzione \textbf{a}. I loro risultati vengono visualizzati direttamente nel terminale. Inoltre i test vengono rieseguiti automaticamente ad ogni modifica del loro codice.
		\subsubsection{Plug-in}
			I test di unità con Jest vengono avviati con il comando \textbf{npm run test} e il loro esito viene visualizzato direttamente nel terminale. In alternativa, possono essere eseguiti tramite il comando \textbf{npm run ci-test} il quale, oltre all'esecuzione dei test, effettua il calcolo della copertura sul codice totale visualizzandola tramite una tabella nel terminale.
	\subsection{Scrittura dei test}
		Per ogni componente viene creato un file di test, con nome \textbf{<nome componente>.test.js} per l'applicativo esterno e \textbf{<nome componente>.test.ts} per il plug-in, all'interno del quale vengono definiti. L'obiettivo dei test di unità è verificare il funzionamento e la correttezza dei singoli componenti, 
		analizzandone la costruzione, il caricamento, l'elaborazione dei dati e le chiamate ad altri metodi.
	\subsection{SonarCloud}
		SonarCloud misura la qualità del codice. Infatti ad ogni commit analizza i sorgenti ed estrae indici su manutenibilità, complessità, affidabilità, duplicazione del codice, numero di bug e vulnerabilità. L'obiettivo è quello di avere codice sempre in buono stato e pronto al rilascio.
	\subsection{Copertura dei test}
		La copertura dei test viene controllata da Coveralls, che ad ogni nuova build eseguita dalla continuous integration esegue il calcolo e ne mantiene uno storico. Per accedere alle statistiche sulla copertura dei test bisogna premere sull'apposito badge nelle repository\glosp che apre la rispettiva pagina sul sito di Coveralls. Nel caso del plug-in, è possibile misurare la copertura anche tramite l'apposito comando di test.

	\pagebreak
	\section{Tecnologie utilizzate}
	\subsection{Electron}
		Electron è un framework che permette di sviluppare applicazioni desktop utilizzando tecnologie web, basandosi sul motore di rendering di chromium. Viene utilizzato dall'applicativo di addestramento come contenitore dell'app
	\subsection{React}
		React è un framework JavaScript che si occupa prevalentemente della gestione dell'interfaccia, permettendo di dividerla in componenti modulari e riutilizzabili.
	\subsection{AngularJS}
		Framework JavaScript che si basa su un pattern MVC/MVVM per la realizzazione di applicazioni web. È l'alternativa più popolare per lo sviluppo di plug-in per Grafana\glo.
	\subsection{NodeJS}
		È una runtime JavaScript, ovvero un ambiente di esecuzione per codice JavaScript; viene impiegato nella fase di sviluppo come server web per l'applicazione esterna.
	\subsection{JSX}
		Estensione del linguaggio JavaScript che viene utilizzata per definire la struttura dei componenti in React, presenta numerose similitudini al linguaggio HTML.
	\subsection{Jest}
		Framework per i test di unità dei componenti JavaScript, viene utilizzato con Enzyme per i test dell'applicativo di addestramento.
	\subsection{Enzyme}
		Strumento di test JavaScript specifico per componenti React, si integra con Jest.
	\subsection{SonarLint}
		Plug-in per IDE che attraverso l'analisi statica\glosp del codice rileva anomalie ed errori comuni. Provvede quindi a segnalare queste vulnerabilità e proporre eventuali soluzioni.
	\subsection{NPM}
		Il Node Package Manager viene utilizzato per la gestione delle dipendente e la build automation.
		Richiede che all'interno del progetto sia presente il file package.json, che contiene informazioni riguardanti le dipendenze e gli script di build e test.
	\subsection{D3}
		Libreria JavaScript per produrre visualizzazioni di dati dinamiche e interattive. Impiegata per creare i grafici nell'applicativo di addestramento.
	\subsection{Plotly}
		È un'altra libreria JavaScript per la visualizzazione di dati, si basa su D3, viene utilizzata dal plug-in per produrre grafici in Grafana\glo.
	\subsection{TypeScript}
		Estensione del linguaggio JavaScript che aggiunge il supporto ai tipi, viene usato nei plug-in di Grafana\glo.
	\subsection{JSON}
		JSON è un formato per lo scambio di dati derivato da JavaScript, viene utilizzato per memorizzare risultati e informazioni degli addestramenti.
	\subsection{CSV}
		È un altro formato per lo scambio di dati, viene usato per la memorizzazione dei dati di test.
	\subsection{InfluxDB}
		È un time series database, cioè una base di dati che memorizza le informazioni come coppie di valori tempo e dato. Proprio per questo motivo è una delle fonti di dati predilette da Grafana\glo.
	\subsection{Regressione Lineare}
		È un modello di regressione che, a partire da un insieme di dati rappresentabili come punti su un piano, restituisce una stima sul loro sviluppo sotto forma di retta.
	\subsection{Support-Vector Machines}
		Sono modelli di apprendimento supervisionati che vengono utilizzati nell'ambito del machine learning\glo: a partire da una serie di dati questi vengono analizzati e viene effettuata una classificazione che li divide in due sottoinsiemi.




	\pagebreak
	%ARCHITETTURA PLUG-IN
\section{Architettura del prodotto}
Il nostro prodotto\glosp è composto da un plug-in sviluppato per la piattaforma Grafana\glosp e un'applicazione ausiliaria, esterna a tale piattaforma. Perciò l'analisi dell'architettura è suddivisa in queste due componenti.
\subsection{Plug-in}
Per poter visualizzare la suddivisione delle componenti del plug-in e le dipendenze che sussistono tra loro ad alto livello, viene riportato il diagramma dei Package in allegato nel file \textit{diagramma-package-plug-in.png}.
\subsubsection{Progettazione architetturale}
Abbiamo deciso di utilizzare un design pattern architetturale Model-View-Controller (MVC) perché si adatta bene allo sviluppo software all'interno di Grafana\glo. In particolare, come si può vedere nella figura seguente, abbiamo la View che scambia informazioni sulle interazioni dell'utente con il Controller che a sua volta le trasforma in azioni sui dati eseguite da Model. Infine vi è una comunicazione tra Model e View per il costante aggiornamento di quest'ultima, grazie ad una funzionalità fornita dalla piattaforma Grafana\glo.
\mbox{}
\begin{landscape}
	\begin{figure}
		\begin{figure} [H]
			\includegraphics[width=\linewidth]{./img/Diagrammi/architettura-plug-in.png}
			\caption{Diagramma dell'architettura del plug-in}
		\end{figure}
	\end{figure}
\end{landscape}
Analizzando i componenti, la nostra architettura è così strutturata: 
\begin{itemize}
	\item \textbf{Model}: modulo che gestisce la business logic\glo. Più in dettaglio, contiene gli algoritmi di predizione dei dati che sono stati attualmente implementati e la scrittura del risultato delle predizioni su un database Influx;
	\item \textbf{View}: modulo che gestisce la presentation logic. Più in dettaglio, permette la creazione di un pannello grafico personalizzato all'interno di una dashboard Grafana\glo. Con questo pannello l'utente può selezionare le impostazioni degli algoritmi di predizione, i flussi di dati in ingresso, le impostazioni del grafico presente nel pannello e le impostazioni del database influx su cui scrivere i risultati;
	\item \textbf{Controller}: modulo che gestisce l'application logic. Più in dettaglio, trasforma i dati ottenuti dalle interazioni dell'utente ed i flussi dati ottenuti da grafana in un formato adatto per l'esecuzione delle azioni da parte del Model.
\end{itemize}
\subsubsection{Progettazione di dettaglio}
Di seguito viene descritta in dettaglio la progettazione del plug-in. In allegato viene inoltre fornito il file \textit{diagramma-classi-plug-in.png} contenente l'intero diagramma delle classi.
\paragraph{Model} \mbox{}
L'elemento principale contenuto all'interno del modello è il componente degli algoritmi di predizione. Abbiamo implementato due algoritmi di predizione: Support Vector Machine lineare\glosp e Regressione lineare\glo. Esse sono rappresentate rispettivamente dalle classi concrete \textit{SvmPrediction} e \textit{RlPrediction}. In senso più generale abbiamo riscontrato che per le famiglie di algoritmi di Support Vector Machine e di Regressione è possibile ricondursi ad un'interfaccia unica che abbiamo chiamato \textit{AlgorithmPrediction}. Essa fornisce il metodo astratto \textit{predict(data, configuration, influxParameter)} che è implementato dalle classi concrete dei singoli algoritmi.
Queste due classi hanno una dipendenza di tipo composizione con la classe concreta \textit{WriteInflux} che importa la libreria \textit{InfluxDB} e contiene le funzionalità di scrittura su un database Influx. Essa presenta due metodi per la scrittura sul database.
\mbox{}
\begin{landscape}
	\begin{figure}
		\begin{figure} [H]
			\includegraphics[width=\linewidth]{./img/Diagrammi/model-plug-in.png}
			\caption{Diagramma delle classi del Model}
		\end{figure}
	\end{figure}
\end{landscape}
\paragraph{View} \mbox{}
All'interno della View abbiamo inserito la componente Panel rappresentata dalla classe \textit{PanelCtrl} che estende \textit{MetricPanelControl} di Grafana\glo. Essa rappresenta il nostro pannello principale dal quale gli utenti possono interagire con il plug-in.
Ha una dipendenza di tipo composizione con un componente della libreria Plotly per la creazione del grafico personalizzato e con la classe \textit{SelectinfluxDBCtrl} per la selezione dalla datasource Grafana\glosp di tipo Influx su cui scrivere i risultati delle predizioni. In questo modo, sfruttiamo il meccanismo di Grafana\glosp che dalla scrittura dei dati sul database a questo componente, permette l'aggiornamento continuo della View.
Infine, sempre all'interno di \textit{PanelCtrl}, c'è una dipendenza verso il componente \textit{ProcessData} del Controller che ne permette il collegamento.
\mbox{}
\begin{landscape}
	\begin{figure}
		\begin{figure} [H]
			\includegraphics[width=\linewidth]{./img/Diagrammi/view-plug-in.png}
			\caption{Diagramma delle classi della View}
		\end{figure}
	\end{figure}
\end{landscape}
\paragraph{Controller} \mbox{}
All'interno del Controller viene implementata la trasformazione dei dati ricevuti dalla View, che rappresentano le interazioni dell'utente con il nostro pannello ed i flussi di dati ottenuti da Grafana\glo, ad azioni da eseguire sul Model.
Abbiamo deciso di implementare un design pattern strategy ottenendo la seguente struttura:
\begin{itemize}
	\item \textbf{ProcessData}: è una classe concreta che rappresenta il context del nostro strategy. Al suo interno, infatti, viene scelto quale algoritmo di predizione processare sulla base dei dati e delle richieste ricevute. Inoltre essa ha una dipendenza di tipo aggregazione con l'interfaccia \textit{PerformPrediction};
	\item \textbf{PerformPrediction}: è un'interfaccia che rappresenta la strategia astratta. Essa definisce il contratto da rispettare per poter implementare una nuova classe concreta che processa i dati da fornire ad un determinato algoritmo;
	\item \textbf{ProcessSvm}: è una classe concreta che implementa \textit{PerformPrediction} e rappresenta il componente che processa e trasforma i dati da fornire per l'esecuzione dell'algoritmo Svm\glosp sul Model;
	\item \textbf{ProcessSvm}: è una classe concreta che implementa \textit{PerformPrediction} e rappresenta il componente che processa e trasforma i dati da fornire per l'esecuzione dell'algoritmo Rl sul Model.
\end{itemize}
\mbox{}
\begin{landscape}
	\begin{figure}
		\begin{figure} [H]
			\includegraphics[width=\linewidth]{./img/Diagrammi/controller-plug-in.png}
			\caption{Diagramma delle classi del Controller}
		\end{figure}
	\end{figure}
\end{landscape}
Per spiegare meglio l'insieme di azioni compiute al fine di processare i dati per eseguire la predizione degli algoritmi, illustriamo un diagramma di sequenza che prende in esame SVM\glo. Il procedimento è indicativo anche per gli altri algoritmi.
\mbox{}
\begin{landscape}
	\begin{figure}
		\begin{figure} [H]
			\includegraphics[width=\linewidth]{./img/Diagrammi/ds-plug-in.png}
			\caption{Diagramma di sequenza di processo dei dati per SVM}
		\end{figure}
	\end{figure}
\end{landscape}

	% ...
	%\input{res/inserire nome sezione n} 
	
	\appendix
	\section{Glossario}

\subsection*{A}
\subsubsection*{Analisi statica}
Valutazione di un sistema o di un suo componente basato sulla sua forma, sulla sua struttura, sul suo contenuto e sulla documentazione di riferimento. Ciò significa che la valutazione avviene senza l'esecuzione del sistema o dell'oggetto dell'analisi.

\subsection*{B}
\subsubsection*{Binding}
Collegamento fra una entità di un software.
\subsubsection*{Business logic}
Logica di elaborazione che rende operativa un'applicazione, in altre parole implementa gli specifici algoritmi di manipolazione dei dati che caratterizzano l'applicazione.

\subsection*{C}
\subsubsection*{Callback}
Meccanismo tramite il quale un blocco di codice viene passato come parametro a dell'altro codice, quest'ultimo eseguirà il codice passato.

\subsection*{D}
\subsubsection*{Design pattern}
Soluzione progettuale generale e riusabile ad un problema ricorrente nell'ambito dell'ingegneria del software.
\subsubsection*{Dipendenze}
Lista di pacchetti che sono necessari a un software per funzionare.

\subsection*{G}
\subsubsection*{Grafana}
Software ad uso generico per la produzione di cruscotti informativi (dashboard in inglese) e composizione di grafici. Viene utilizzato come un'applicazione web.

\subsection*{M}
\subsubsection*{Machine learning}
Il machine learning (in italiano apprendimento automatico) è una branca dell'intelligenza artificiale che utilizza metodi statistici per migliorare progressivamente la performance di un algoritmo nell'identificare pattern nei dati.

\subsection*{P}
\subsubsection*{Path di sistema}
Percorso che parte dalla root directory\glosp e arriva fino al file desiderato
\subsubsection*{Prodotto}
Si definisce prodotto qualsiasi bene scambiabile sul mercato che può rispondere alle esigenze di un compratore. Un esempio di prodotto informatico è il software che è composto dal codice e dalla documentazione.	
	
\subsection*{R}
\subsubsection*{Release di debug}
Prodotto compilato destinato agli sviluppatori perché facilita l'individuazione di errori.

\subsubsection*{Release di produzione}
Prodotto compilato e pronto per essere usato dall'utente finale.


\subsubsection*{Repository}
Repository significa archivio. In un repository sono raccolti dati e informazioni in formato digitale, valorizzati e archiviati sulla base di metadati che ne permettono la rapida individuazione, anche grazie alla creazione di tabelle relazionali. Grazie alla sua peculiare architettura, un repository consente di gestire in modo ottimale anche grandi volumi di dati.
\subsubsection*{Root directory}
La root directory di una qualsiasi partizione, è la cartella più "alta" nella gerarchia delle cartelle.
\subsubsection*{RL}
Acronimo di Regressione lineare, algoritmo di machine learning\glosp che ha la funzione di prevedere un valore di una variabile dipendente (y) in base a una determinata variabile indipendente (x) secondo una relazione di tipo lineare.
\subsection*{S}
\subsubsection*{SVM}
Acronimo di Support Vector Machine; algoritmo di apprendimento automatico supervisionato che può essere utilizzato sia per scopi di classificazione che di regressione.

\subsection*{V}
\subsubsection*{Versionamento}
Il controllo di versione è un sistema che registra, nel tempo, le modifiche effettuate ad un file o ad una serie di file, permettendo così di recuperare una specifica versione dei file stessi in un secondo momento. Permette inoltre ad un team di collaborare in modo efficiente facilitando l'individuazione e la risoluzione di conflitti.


\end{document}