\section{Qualità di prodotto}
    Per misurare la qualità di prodotto\glosp il gruppo ha deciso di prendere come riferimento informativo lo standard ISO/IEC 25010 che definisce un modello di qualità del prodotto\glosp attraverso un insieme di caratteristiche definite invece dallo standard ISO/IEC 25023. Di seguito sono elencate le caratteristiche e le metriche che il gruppo ha ritenuto importanti in questo frangente del progetto\glo.
    \subsection{PRD-Q1 Documenti}
    	\subsubsection{CP-1 Leggibilità dei documenti}
    		Ci prefiggiamo di scrivere dei documenti facilmente leggibili.
	    \paragraph{Metriche di qualità}
	    \begin{longtable} {
	    		>{}p{80mm} 
	    		>{}p{25mm}
	    		>{}p{25mm}
	    	}
	    	\rowcolor{gray!50}
	    	\textbf{Metrica} & \textbf{Preferibile} & \textbf{Accettabile} \TBstrut \TBstrut \\
	    	M15 Indice di Gulpease & 60 $\le I_{G} \le$ 100 & 40 $\le I_{G} \le$ 100 \TBstrut \\ [2mm]
	    \end{longtable}
    	\subsubsection{CP-2 Correttezza dei documenti}
    		Ci prefiggiamo di scrivere dei documenti ortograficamente corretti secondo le regole della lingua italiana.
    	\paragraph{Metriche di qualità}
    	\begin{longtable} {
    			>{}p{80mm} 
    			>{}p{25mm}
    			>{}p{25mm}
    		}
    		\rowcolor{gray!50}
    		\textbf{Metrica} & \textbf{Preferibile} & \textbf{Accettabile} \TBstrut \TBstrut \\
    		M19 correttezza ortografica & 0 & 0 \TBstrut \\ [2mm]
    	\end{longtable}
    	

    \subsection{PRD-Q2 Appropriatezza funzionale}
        \subsubsection{CP-3 Completezza e adeguatezza dei requisiti}
        	Ci prefiggiamo di definire un insieme di requisiti che copra e faciliti il compimento di tutte le attività e gli obiettivi dell'utente;
        \paragraph{Metriche di qualità}
        \begin{longtable} {
        		>{}p{80mm} 
        		>{}p{25mm}
        		>{}p{25mm}
        	}
        	\rowcolor{gray!50}
        	\textbf{Metrica} & \textbf{Preferibile} & \textbf{Accettabile} \TBstrut \TBstrut \\
        	M16 Percentuale di requisiti obbligatori soddisfatti & $100\%$ & $100\%$ \TBstrut \\ [2mm]
        	M17 Percentuale di requisiti desiderabili soddisfatti & $\ge65\%$ & $\ge0\%$ \TBstrut \\ [2mm]
        	M18 Percentuale di requisiti opzionali soddisfatti & $\ge50\%$ & $\ge0\%$ \TBstrut \\ [2mm]
        \end{longtable}
    \subsubsection{CP-4 Correttezza delle funzionalità implementate}
    	Ci prefiggiamo di implementare correttamente le funzionalità rispetto ai requisiti definiti;
    	\paragraph{Metriche di qualità}
    	\begin{longtable} {
    			>{}p{80mm} 
    			>{}p{25mm}
    			>{}p{25mm}
    		}
    		\rowcolor{gray!50}
    		\textbf{Metrica} & \textbf{Preferibile} & \textbf{Accettabile} \TBstrut \TBstrut \\
    		M? Percentuale di test passati & $100\%$ & $\ge 80\%$ \TBstrut \\ [2mm]
    	\end{longtable}
    %\subsection{Usabilità}
     %   \subsubsection{Obiettivi}
      %      \begin{itemize}
       %         \item \textbf{Apprendibilità}: grado con cui il prodotto\glosp o il sistema può essere appreso con efficacia, efficienza e soddisfazione da uno specifico utente;
        %        \item \textbf{Appropriatezza-Riconoscibilità}: grado con cui gli utenti possono riconoscere che un determinato prodotto\glosp o sistema è appropriato per i propri bisogni.
         %   \end{itemize}
        %\subsubsection{Metriche}
         %   \textbf{Completezza della documentazione}: percentuale delle funzioni descritta nella documentazione con un dettaglio tale da consentire all’utente di utilizzarle.
          %      \begin{itemize}
           %         \item \textbf{Misurazione}: $C_{DOC}=(N_{FD}/N_{FI})*100$ \\
            %        dove N$_{FD}$ sono le funzioni definite sulla documentazione e N$_{FI}$ sono le funzioni individuate nella documentazione;
             %       \item \textbf{Valore preferibile}: 100\%;
              %      \item \textbf{Valore accettabile}: 100\%.
               % \end{itemize}
            %\textbf{Completezza di descrizione}: percentuale degli scenari d’uso descritta nella documentazione effettivamente presenti nel prodotto\glosp finale.
             %   \begin{itemize}
               %     \item \textbf{Misurazione}: $C_{DESC}=(N_{UCI}/N_{UCE})*100$ \\
                %    dove N$_{UCI}$ è il numero di casi d'uso\glosp individuati e N$_{UCE}$ è il numero di casi d'uso effettivi del prodotto\glo;
                 %   \item \textbf{Valore preferibile}: 100\%;
                  %  \item \textbf{Valore accettabile}: 100\%.
                %\end{itemize}
    %\subsection{Affidabilità}
     %   \subsubsection{Obiettivi}
      %      \begin{itemize}
       %         \item \textbf{Maturità}: grado con cui un sistema, un prodotto\glosp o un componente è affidabile durante le normali condizioni di servizio;
        %        \item \textbf{Tolleranza agli errori}: grado con cui un sistema, un prodotto\glosp o un componente riesce ad operare anche in presenza di errori hardware o software.
         %   \end{itemize}
        %\subsubsection{Metriche}
         %   \textbf{M25 Densità degli errori}
          %      \begin{itemize}
           %         \item \textbf{Valore preferibile}: 0\%;
            %        \item \textbf{Valore accettabile}: 10\%.
             %   \end{itemize}
    %\subsection{Manutenibilità}
     %   \subsubsection{Obiettivi}
      %  \begin{itemize}
       %     \item \textbf{Analizzabilità}: grado di efficacia ed efficienza con cui è possibile valutare l'impatto su un prodotto\glosp o un sistema di un eventuale cambiamento (in una o più parti);
        %    \item \textbf{Modificabilità}: grado con cui un prodotto\glosp o un sistema può essere modificato efficacemente ed efficientemente, cioè senza introdurre difetti o degradando la qualità esistente.
        %\end{itemize}
        %\subsubsection{Metriche}
         %   \textbf{M26 Structural fan-in}
          %      \begin{itemize}
           %         \item \textbf{Valore preferibile}: $ \ge 1$
            %        \item \textbf{Valore accettabile}: $ \ge 0$
             %   \end{itemize}
            %\textbf{M27 Structural fan-out}
             %   \begin{itemize}
              %      \item \textbf{Valore preferibile}: $0$
               %     \item \textbf{Valore accettabile}: $ \le 6$
                %\end{itemize}
            %\textbf{M08 Presenza di code smells}
            %\begin{itemize}
            %	\item \textbf{Valore preferibile}: $0$
            %	\item \textbf{Valore accettabile}: $0$
            %\end{itemize}
            
            \subsection{Tabella riassuntiva delle metriche adottate}
            \rowcolors{2}{gray!25}{gray!15}
            \begin{longtable} {
                >{}p{40mm}  
                >{}p{95mm}
                }
	            \rowcolor{gray!50}
	            \multicolumn{2}{c}{\textbf{PRD-Q1 Documenti}}\\
	            \rowcolor{gray!50}
	                \textbf{Caratteristiche} & \textbf{Metriche} \TBstrut \\ [2mm]
	        
	                CP-1 Leggibilità dei documenti &
	                M15 Indice di Gulpease \TBstrut \\ [2mm]
	                CP-2 Correttezza dei documenti &
	                M19 Correttezza ortografica \TBstrut \\ [2mm]
	
				\rowcolor{gray!50}
				\multicolumn{2}{c}{\textbf{PRD-Q2 Appropriatezza funzionale}}\\
				\rowcolor{gray!50}
				\textbf{Caratteristiche} & \textbf{Metriche} \TBstrut \\ [2mm]
	
	                CP-3 Completezza e adeguatezza dei requisiti & 
	                M16 Percentuale di requisiti obbligatori soddisfatti \newline
	                M17 Percentuale di requisiti desiderabili soddisfatti \newline
	                M18 Percentuale di requisiti opzionali soddisfatti \TBstrut \\ [2mm]
	                CP-4 Correttezza delle funzionalità implementate &
	                M? Percentuale di test passati \TBstrut \\ [2mm]
	        
	                %Affidabilità &
	                %M? Densità degli errori \TBstrut \\ [2mm]
	        
	                %Manutenibilità &
	                %M? Structural fan-in \newline
	                %M? Structural fan-out \newline
	                %M? Presenza di code smells \TBstrut \\ [2mm]

                \rowcolor{white}
                \caption{Tabella riassuntiva metriche\glosp adottate per la qualità di prodotto\glo}
            \end{longtable}