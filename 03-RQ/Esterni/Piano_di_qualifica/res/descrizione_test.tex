\section{Specifica dei test}
Il nostro gruppo ha scelto di adottare il Modello a V\glosp per garantire la qualità del nostro prodotto\glo. In particolare, questo modello prevede lo sviluppo dei test durante le attività di analisi dei requisiti, progettazione\glosp architetturale e progettazione\glosp di dettaglio oltre a validazione\glosp e collaudo.
In questo modo è possibile verificare la correttezza sia di tutti gli aspetti che compongono il progetto\glosp che delle singole parti sviluppate. Sono state individuate quattro tipologie di test:
\begin{itemize}
	\item test di accettazione;
	\item test di sistema;
	\item test di integrazione;
	\item test di unità.
\end{itemize}
Ogni volta che viene svolta un'attività viene definita una tabella con i test di una tipologia.
All'interno del documento \textit{Norme di Progetto v. 4.1.1} vengono definite le caratteristiche dei test e i codici che identificano univocamente i singoli test.

\subsection{Test di accettazione}
\addtocontents{toc}{\protect\setcounter{tocdepth}{0}} %Inserire questo per escludere una sezione dall'indice.

\rowcolors{2}{gray!25}{gray!15}
\setcounter{table}{0}
\begin{longtable} {
		>{}p{15mm} 
		>{}p{79.5mm}
		>{}p{15mm} 
		>{}p{15mm}
		>{}p{0mm}}
	\rowcolor{gray!50}
	\textbf{Codice} & \textbf{Descrizione} & \textbf{Stato} & \textbf{Esito} &\TBstrut \\
	TA1		& Verificare che l'utente possa addestrare gli algoritmi di previsione dei dati all'interno della piattaforma Grafana\glo & NI & NE  &\TBstrut \\ [2mm]
	TA1.1	& Verificare che l'utente possa selezionare e caricare dal proprio dispositivo un file JSON che contiene i dati di testing per l'addestramento& NI & NE  &\TBstrut \\ [2mm]
	TA1.2	& Verificare che l'utente possa scegliere il modello di predizione da utilizzare tra tutti quelli forniti & NI & NE  &\TBstrut \\ [2mm]
	TA1.3	& Verificare che l'utente possa avviare l'addestramento dell'algoritmo & NI & NE  &\TBstrut \\ [2mm]
	TA1.4	& Verificare che l'utente possa chiudere l'addestramento e visualizzare un messaggio di conferma se esso va a buon fine & NI & NE  &\TBstrut \\ [2mm]
	TA2		& Verificare che l'utente possa visualizzare l'indice di qualità delle previsioni & NI & NE  &\TBstrut \\ [2mm]
	TA3		& Verificare che, se l'utente inserisce un file JSON non valido, venga visualizzato un messaggio di errore & NI & NE  &\TBstrut \\ [2mm]
	TA4		& Verificare che l'utente possa addestrare gli algoritmi di previsione dei dati sull'applicazione esterna a Grafana\glo & NI & NE  &\TBstrut \\ [2mm]
	TA4.1	& Verificare che l'utente possa selezionare e caricare dal proprio dispositivo un file JSON che contiene i dati di testing per l'addestramento & NI & NE  &\TBstrut \\ [2mm]
	TA4.2	& Verificare che l'utente possa scegliere se addestrare il modello di predizione da utilizzare per l'addestramento tra tutti quelli forniti & NI & NE  &\TBstrut \\ [2mm]
	TA4.3	& Verificare che l'utente possa avviare l'addestramento dell'algoritmo & NI & NE  &\TBstrut \\ [2mm]
	TA4.4	& Verificare che l'utente possa chiudere l'addestramento dell'algoritmo e visualizzare il messaggio di conferma se esso è stato svolto correttamente & NI & NE  &\TBstrut \\ [2mm]
	TA4.5	& Verificare che al termine della procedura l'utente riceva dall'applicazione esterna un file JSON con i parametri per le previsioni & NI & NE  &\TBstrut \\ [2mm]
	TA5		& Verificare che l'utente possa visualizzare l'indice di qualità delle previsioni & NI & NE  &\TBstrut \\ [2mm]
	TA6		& Verificare che, se l'utente inserisce un file JSON non valido, venga visualizzato un messaggio di errore & NI & NE  &\TBstrut \\ [2mm]
	TA7		& Verificare che l'utente possa avviare il plug-in & NI & NE  &\TBstrut \\ [2mm]
	TA8		& Verificare che l'utente possa caricare il file JSON ottenuto dall'addestramento all'interno del plug-in & NI & NE  &\TBstrut \\ [2mm]
	TA9		& Verificare che l'utente possa associare i nodi letti dal file JSON al flusso dati & NI & NE  &\TBstrut \\ [2mm]
	TA9.1	& Verificare che l'utente possa inserire i nodi & NI & NE  &\TBstrut \\ [2mm]
	TA9.2	& Verificare che l'utente possa selezionare un flusso di dati statico su cui eseguire delle previsioni & NI & NE  &\TBstrut \\ [2mm]
	TA9.3	& Verificare che l'utente possa selezionare un flusso di dati continuo su cui eseguire delle previsioni & NI & NE  &\TBstrut \\ [2mm]
	TA9.4	& Verificare che l'utente possa collegare i nodi scelti al flusso di dati corrispondente & NI & NE  &\TBstrut \\ [2mm]
	TA9.5	& Verificare che l'utente possa visualizzare un messaggio che conferma il successo nel collegamento dei nodi al flusso dati & NI & NE  &\TBstrut \\ [2mm]
	TA10	& Verificare che, se il collegamento dei nodi al flusso dati non va a buon file, l'utente visualizzi un messaggio di errore & NI & NE  &\TBstrut \\ [2mm]
	TA11	& Verificare che l'utente possa visualizzare il grafico dei risultati della previsione all'interno di una dashboard\glosp precedentemente configurata & NI & NE  &\TBstrut \\ [2mm]
	TA12	& Verificare che l'utente possa fermare l'esecuzione del plug-in rimuovendolo dalla dashboard\glo & NI & NE  &\TBstrut \\ [2mm]
	TA13	& Verificare che l'utente possa definire un alert\glosp all'interno del pannello della dashboard\glosp su cui si è applicato il plug-in & NI & NE  &\TBstrut \\ [2mm]
	TA13.1	& Verificare che l'utente possa inserire un alert\glosp nel pannello della dashboard\glo & NI & NE  &\TBstrut \\ [2mm]
	TA13.2	& Verificare che l'utente possa definire le regole di funzionamento di un alert\glo & NI & NE  &\TBstrut \\ [2mm]
	TA13.3	& L'utente deve poter definire le condizioni di funzionamento di un alert\glo & NI & NE  &\TBstrut \\ [2mm]
	TA13.4	& L'utente deve poter definire il comportamento legato all'assenza di dati  & NI & NE  &\TBstrut \\ [2mm]
	TA14	& Verificare che l'utente visualizzi un messaggio di errore se viene inserito un input errato nella definizione di un alert\glo & NI & NE  &\TBstrut \\ [2mm]
	TA15	& Verificare che l'utente possa sospendere un alert\glo & NI & NE  &\TBstrut \\ [2mm]
	TA16	& Verificare che l'utente possa rimuovere un alert\glo & NI & NE  &\TBstrut \\ [2mm]
	\rowcolor{white}
	\caption{Test di accettazione}
\end{longtable}

\addtocontents{toc}{\protect\setcounter{tocdepth}{4}} %Inserire questo per ripristinare il normale inserimento delle sezioni nell'indice. 4 significa fino al paragrah

\subsection{Test di sistema}
\addtocontents{toc}{\protect\setcounter{tocdepth}{0}} %Inserire questo per escludere una sezione dall'indice.

\rowcolors{2}{gray!25}{gray!15}
\begin{longtable} {
		>{}p{15mm} 
		>{}p{79.5mm}
		>{}p{15mm} 
		>{}p{15mm}
		>{}p{0mm}}
	\rowcolor{gray!50}
	\textbf{Codice} & \textbf{Descrizione} & \textbf{Stato} & \textbf{Esito} &\TBstrut \\
	TS1 & Verificare che l'addestramento degli algoritmi produca un file JSON con i parametri per le previsioni & NI & NE  &\TBstrut \\ [2mm]
	TS2 & Verificare la corretta visualizzazione della bontà dei modelli di previsione a seguito dell'addestramento sui dati & NI & NE  &\TBstrut \\ [2mm]
	TS3 & Verificare che i nodi ricavati dal file JSON siano associati correttamente al flusso dati scelto in Grafana\glo & NI & NE  &\TBstrut \\ [2mm]
	TS4 & Applicare le previsioni su un flusso dati statico e visualizzare correttamente i dati ottenuti all'interno di un grafico contenuto nella dashboard\glo & NI & NE  &\TBstrut \\ [2mm]
	TS5 & Applicare le previsioni su un flusso dati continuo e visualizzare correttamente i dati ottenuti all'interno di un grafico contenuto nella dashboard\glo & NI & NE  &\TBstrut \\ [2mm]
	TS6 & Verificare che il sistema permetta all'utente inserire un alert\glo & NI & NE  &\TBstrut \\ [2mm]
	\rowcolor{white}
	\caption{Test di sistema}
\end{longtable}

\addtocontents{toc}{\protect\setcounter{tocdepth}{4}} %Inserire questo per ripristinare il normale inserimento delle sezioni nell'indice. 4 significa fino al paragrah

\subsection{Test di integrazione}
I test di integrazione verranno sviluppati in seguito alla progettazione\glosp architetturale.

\subsection{Test di unità}
\addtocontents{toc}{\protect\setcounter{tocdepth}{0}} %Inserire questo per escludere una sezione dall'indice.

\rowcolors{2}{gray!25}{gray!15}
\begin{longtable} {
		>{}p{15mm} 
		>{}p{79.5mm}
		>{}p{15mm} 
		>{}p{15mm}
		>{}p{0mm}}
	\rowcolor{gray!50}
	\textbf{Codice} & \textbf{Descrizione} & \textbf{Stato} & \textbf{Esito} &\TBstrut \\
	TU1		& Verificare che la scritta 'VRAM Software Applicativo Esterno - PoC 3' venga renderizzata per dimostrare il corretto funzionamento del metodo render() del componente App. & I & P &\TBstrut \\ [2mm]
	TU2		& Verificare che i due componenti utilizzati per l'input dei file vengano correttamente renderizzati dimostrando quindi la correttezza del metodo render() del componente App. & I & P &\TBstrut \\ [2mm]
	TU3		& Verificare che il componente Modal, cioè la finestra utilizzata durante il salvataggio del file JSON per cambiare il nome, non venga renderizzato quando viene inizializzata l'applicazione e quindi dimostrare la correttezza del metodo render() del componente App. & I & P &\TBstrut \\ [2mm]
	TU4		& Verificare che vengano renderizzati il grafico e l'input di testo per inserire le note, se sono presenti, nello stato del componente principale, i dati provenienti dal file CSV dato in input dall'utente e quindi dimostrare la correttezza del metodo render() del componente App. & I & P &\TBstrut \\ [2mm]
	TU5		& Verificare che non vengano renderizzati il grafico e l'input di testo per inserire le note, se non sono presenti i dati provenienti dal file CSV dato in input dall'utente e quindi dimostrare la correttezza del metodo render() del componente App. & I & P &\TBstrut \\ [2mm]
	TU6		& Verificare che il pulsante 'Inizia addestramento' sia disabilitato quando viene inizializzata l'applicazione e quindi dimostrare la correttezza del metodo render() del componente App. & I & P &\TBstrut \\ [2mm]
	TU7		& Verificare che il pulsante 'Inizia addestramento' sia abilitato quando vengono salvati nello stato del componente principale le informazioni del file CSV e quindi dimostrare la correttezza del metodo render() del componente App. & I & P &\TBstrut \\ [2mm]
	TU8		& Verificare che il pulsante 'Salva JSON' sia disabilitato quando viene inizializzata l'applicazione, in quanto non è ancora stato inserito un file CSV e non è stato eseguito l'addestramento. & I & P &\TBstrut \\ [2mm]
	TU9		& Verificare che il pulsante 'Salva JSON' sia abilitato dopo che è stato eseguito l'addestramento dei dati inseriti da utente e quindi dimostrare la correttezza del metodo render() del componente App. & I & P &\TBstrut \\ [2mm]
	TU10	& Verificare che non vengano renderizzati i path dei file CSV e JSON se non sono stati selezionati dall'utente e quindi dimostrare la correttezza del metodo render() del componente App. & I & P &\TBstrut \\ [2mm]
	TU11	& Verificare che venga renderizzato il path del file JSON se è stato precedentemente selezionato dall'utente e quindi dimostrare la correttezza del metodo render() del componente App. & I & P &\TBstrut \\ [2mm]
	TU12	& Verificare che venga renderizzato il path del file JSON se è stato precedentemente selezionato dall'utente e quindi dimostrare la correttezza del metodo render() del componente App. & I & P &\TBstrut \\ [2mm]
	TU13	& Verificare che venga aperta la finestra per scegliere il nome del file JSON dopo che l'utente ha cliccato il pulsante 'Salva JSON' e quindi dimostrare la correttezza del metodo handleOpenModal(event) del componente App. & I & P &\TBstrut \\ [2mm]
	TU14	& Verificare che venga chiusa la finestra per scegliere il nome del file JSON dopo che l'utente ha cliccato il pulsante chiudi nella medesima finestra e quindi dimostrare la correttezza del metodo handleCloseModal(event) del componente App. & I & P &\TBstrut \\ [2mm]
	TU15	& Verificare che venga chiusa la finestra per scegliere il nome del file JSON dopo che l'utente ha cliccato fuori dalla medesima finestra per dimostrare la correttezza del metodo handleCloseModal(event) del componente App. & I & P &\TBstrut \\ [2mm]
	TU16	& Verificare che venga cambiato lo stato del componente principale che indica il nome del file da salvare quando vengono effettuati cambiamenti all'input di testo della finestra per cambiare il nome del file JSON per dimostrare la correttezza del metodo handleChangeFileName(event) del componente App. & I & P &\TBstrut \\ [2mm]
	TU17	& Verificare che venga cambiato lo stato del componente principale che indica le note che possono essere inserite nel file JSON di output handleChangeNotes(event) del componente App. & I & P &\TBstrut \\ [2mm]
	TU18	& Verificare che, dopo che l'utente ha cliccato sul pulsante 'Salva JSON' della finestra salva con nome, venga mandato un segnale al processo principale di Electron per effettuare la scrittura del file. Così facendo, è verificata la correttezza del metodo handleSaveJSON(event) del componente App. & I & P &\TBstrut \\ [2mm]
	TU19	& Verificare che venga mandato un segnale al processo principale di Electron dopo che l'utente ha cliccato sul pulsante 'Inizia addestramento', per dimostrare la correttezza del metodo handleStartTraining() del componente App. & I & P &\TBstrut \\ [2mm]
	TU20	& Verificare che il metodo onChange(e) eseguito quando vengono selezionati i file di input, inserisca correttamente le informazioni del file JSON all'interno dello stato del componente principale App. & I & P &\TBstrut \\ [2mm]
	TU21	& Verificare che il metodo onChange(e) eseguito quando vengono selezionati i file di input, inserisca correttamente le informazioni del file CSV all'interno dello stato del componente principale App. & I & P &\TBstrut \\ [2mm]
	TU22	& Verificare che il metodo onChange(e) del componente App, eseguito quando vengono selezionati i file di input, chiami il metodo per tradurre il file CSV in JSON. & I & P &\TBstrut \\ [2mm]
	TU23	& Verificare che il componente Modal, cioè la finestra utilizzata durante il salvataggio del file JSON per cambiare il nome, venga renderizzato se la variabile di stato showModal ha valore 'true'. & I & P &\TBstrut \\ [2mm]
	TU24	& Verificare che quando viene cliccato il pulsante Salva JSON il componente Modal venga nascosto. & I & P &\TBstrut \\ [2mm]
	TU25	& Verificare che il component CheckBox, utilizzato per scegliere l'algoritmo di predizione, venga renderizzato in modo corretto per dimostrare la correttezza del metodo render() di App. & I & P &\TBstrut \\ [2mm]
	TU26	& Verificare che il componente CheckBox non venga renderizzato senza che venga inserito alcun dato in input dall'utente, per dimostrare la correttezza del metodo render() di App. & I & P &\TBstrut \\ [2mm]
	TU27	& Verificare che venga cambiato lo stato del componente principale che indica le note che possono essere inserite nel file JSON di output handleChangeNotesPredittore(event) del componente App. & I & P &\TBstrut \\ [2mm]
	TU28	& Verificare che l'algoritmo possa essere cambiato cliccando il pulsante del componente CheckBox e che quindi vada a cambiare lo stato del componente App, verificando quindi la correttezza del metodo handleChangeAlgorithm(). & I & P &\TBstrut \\ [2mm]
	TU29	& Verificare che l'algoritmo possa essere cambiato cliccando il testo del componente CheckBox e che quindi vada a cambiare lo stato del componente App, verificando quindi la correttezza del metodo handleChangeAlgorithm(). & I & P &\TBstrut \\ [2mm]
	TU30	& Verificare che quando viene effettuato l'addestramento il pulsante 'Inizia addestramento' cambi l'etichetta in 'Addestrando...' per dimostrare la correttezza del metodo render() di App. & I & P &\TBstrut \\ [2mm]
	TU31	& Verificare che il caricamento del file venga segnalato tramite il log della console sviluppatore, per verificare la correttezza del metodo onChange(e). & I & P &\TBstrut \\ [2mm]
	TU32	& Verificare che la scelta dell'algoritmo venga segnalata tramite il log della console sviluppatore, per verificare la correttezza del metodo handleChangeAlgorithm(e). & I & P &\TBstrut \\ [2mm]
	TU33	& Verificare la corretta renderizzazione del bottone che apre la finestra di dialogo per la selezione del file di test. & I & P &\TBstrut \\ [2mm]
	TU34	& Verificare il caricamento della finestra per la selezione del file. & I & P &\TBstrut \\ [2mm]
	TU35	& Verificare che la funzione passata come argomento venga chiamata al verificarsi dell'evento onChange nel selettore di file. & I & P &\TBstrut \\ [2mm]
	TU36	& Verificare che il bottone per aprire il selettore di file cambi colore per indicare che un file è stato selezionato. & I & P &\TBstrut \\ [2mm]
	TU37	& Verificare il corretto caricamento del componente ScatterPlot, utilizzato nella costruzione del grafico. & I & P &\TBstrut \\ [2mm]
	TU38	& Verificare il rendering dei botttoni per la chiusura e il salvataggio del file JSON. & I & P &\TBstrut \\ [2mm]
	TU39	& Verificare la chiamata alla funzione di chiusura al click del pulsante chiudi. & I & P &\TBstrut \\ [2mm]
	TU40	& Verificare la chiamata della funzione di saltavataggio al click del pulsante 'Salva JSON'. & I & P &\TBstrut \\ [2mm]
	TU41	& Verificare la chiusura della finestra di salvataggio quando perde il focus emulando un click nel background. & I & P &\TBstrut \\ [2mm]
	TU42	& Verificare il rendering del campo di input per il nome del file da salvare. & I & P &\TBstrut \\ [2mm]
	TU43	& Verificare la chiamata della funzione passata come parametro al verificarsi dell'evento onChange. & I & P &\TBstrut \\ [2mm]
	TU44	& Verificare che il componente '<line>' venga ritornato in seguito alla creazione del componente 'TrendLine'. & I & P &\TBstrut \\ [2mm]
	TU45	& Verificare che, con un set di dati di prova, vengano effettuate delle chiamate ai metodi della libreria D3 con le corrette coordinate che identificano inizio e fine della linea, calcolate a partire dai dati inseriti. & I & P &\TBstrut \\ [2mm]
	TU46	& Verificare che il rendering della casella di testo per le note dell'utente sia avvenuto con successo. & I & P &\TBstrut \\ [2mm]
	TU47	& Verificare che il metodo passato come parametro venga chiamato al verificarsi dell'evento onChange nella casella di testo per le note. & I & P &\TBstrut \\ [2mm]
	TU48	& Verificare la correttezza del metodo translateData, a partire da un set di dati passato come argomento, viene controllato il risutato ritornato dalla funzione. & I & P &\TBstrut \\ [2mm]
	TU49	& Verificare che il metodo 'train' del componente 'Trainer' chiami il metodo 'train' contenuto all'interno di 'ml-modules' con i corretti parametri 'data', 'labels' e 'options'. & I & P &\TBstrut \\ [2mm]
	TU50	& Verificare che il metodo 'render' del componente 'RenderCircles' crei correttamente degli elementi '<circle>' (tag SVG) dati i parametri 'data' e 'scale'. & I & P &\TBstrut \\ [2mm]
	TU51	& Verificare che il metodo 'render' del componente 'RenderCircles' crei degli elementi '<circle>' (tag SVG) di colore verde se il parametro 'label' passato ha valore 1. & I & P &\TBstrut \\ [2mm]
	TU52	& Verificare che il metodo 'render' del componente 'RenderCircles' crei degli elementi '<circle>' (tag SVG) di colore rosso se il parametro 'label' passato ha valore diverso da 1. & I & P &\TBstrut \\ [2mm]
	\rowcolor{white}
	\caption{Test di unità}
\end{longtable}
