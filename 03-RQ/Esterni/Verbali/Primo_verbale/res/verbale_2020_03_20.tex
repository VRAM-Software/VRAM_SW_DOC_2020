\section{Informazioni generali}
    \subsection{Informazioni incontro}
        \begin{itemize}
            \item \textbf{Luogo}: Videochiamata tramite Skype;
            \item \textbf{Data}: 2020-03-20;
            \item \textbf{Ora d'inizio}: 11.00;
            \item \textbf{Ora di fine}: 12.00;
            \item \textbf{Partecipanti}: 
            \begin{itemize}
                \item Dalla Libera Marco;
                \item Rampazzo Marco;
                \item Santagiuliana Vittorio;
                \item Schiavon Rebecca;
                \item Spreafico Alessandro;
                \item Toffoletto Massimo;
                \item Piccoli Gregorio (proponente).
            \end{itemize}
        \end{itemize}
    \subsection{Argomenti trattati}
        In questo incontro con il proponente \textit{Zucchetti} il gruppo ha presentato delle idee relative all'architettura del prodotto\glo, chiedendone un riscontro ed eventuali consigli; in seguito viene riportato un riassunto delle tematiche trattate:
        \begin{enumerate}
            \item aggiornamento sullo stato di avanzamento del prodotto\glo;
            \item discussione sull'architettura dell'applicativo esterno;
            \item discussione sull'architettura del plug-in;
            \item chiarimenti sul riaddestramento.
        \end{enumerate}
\section{Verbale}
        \subsection{Punto 1}
            La riunione è iniziata aggiornando il proponente, tramite una breve demo, sulle funzionalità del prodotto\glosp fino ad ora sviluppate. Il dottor Piccoli non ha segnalato anomalie nell'andamento di lavoro. Continueremo, quindi, lo sviluppo degli incrementi pianificati come previsto.
        \subsection{Punto 2}
            Dopodiché abbiamo discusso dell'architettura dell'applicativo esterno. Il proponente ha affermato che, dato l'utilizzo della libreria React, si potrebbe sfruttare la divisione di quest'ultimo in DOM virtuale ed elementi React, che generano di fatto una "doppia vista" dell'applicativo. Abbiamo deciso, dopo una discussione interna, di adottare il design pattern architetturale MVVM (Model-View-ViewModel).
        \subsection{Punto 3}
            In seguito abbiamo parlato dell'architettura del plug-in dove abbiamo presentato un'architettura a strati per gestire la piattaforma. Il dottor Piccoli ha approvato questa scelta consigliando di integrarla con un altro design pattern più specifico. Dopo averne discusso internamente, abbiamo deciso di utilizzare il design pattern architetturale MVC (model-view-controller) dove i vari elementi ospiteranno al loro interno una divisione a strati come pensato da noi in precedenza.
        \subsection{Punto 4}
            Infine sono state chieste delucidazione riguardanti il funzionamento del riaddestramento dell'algoritmo di predizione. Il proponente ha affermato che lo scopo di tale funzionalità è salvare le note e i commenti dell'utente sul risultato e i dati per poter effettuare lo stesso addestramento, eventualmente con valori diversi, più velocemente. Abbiamo quindi iniziato l'implementazione del riaddestramento tramite file JSON in questo modo.