\section{Informazioni generali}
    \subsection{Informazioni incontro}
        \begin{itemize}
            \item \textbf{Luogo}: Videochiamata tramite Skype;
            \item \textbf{Data}: 2020-04-09;
            \item \textbf{Ora d'inizio}: 11.00;
            \item \textbf{Ora di fine}: 11.15;
            \item \textbf{Partecipanti}: 
            \begin{itemize}
            	\item Corrizzato Vittorio;
                \item Dalla Libera Marco;
                \item Rampazzo Marco;
                \item Santagiuliana Vittorio;
                \item Schiavon Rebecca;
                \item Spreafico Alessandro;
                \item Toffoletto Massimo;
                \item Piccoli Gregorio (proponente).
            \end{itemize}
        \end{itemize}
    \subsection{Argomenti trattati}
        In questo incontro con il proponente \textit{Zucchetti} il gruppo ha presentato al dott. Piccoli lo stato di avanzamento del prodotto\glosp e ha ricevuto riscontri e consigli; in seguito viene riportato un riassunto delle tematiche trattate:
        \begin{enumerate}
            \item aggiornamento sullo stato di avanzamento del prodotto\glo;
            \item uso di dati di esempio troppo "perfetti";
            \item assegnazione manuale dei predittori alle classi dell'algoritmo SVM\glo.
        \end{enumerate}
\section{Verbale}
        \subsection{Punto 1}
            La riunione è iniziata aggiornando il proponente, tramite una breve demo, sulle funzionalità del prodotto\glosp fino ad ora sviluppate. Il dott. Piccoli non ha segnalato anomalie importanti nell'andamento del lavoro. Il gruppo quindi continuerà lo sviluppo degli incrementi pianificati come previsto.
        \subsection{Punto 2}
            Durante la demo il gruppo ha mostrato una prova di esecuzione dell'algoritmo SVM\glosp e il risultato dell'addestramento sembrava troppo preciso perché sono stati usati dati troppo di comodo, quindi durante la chiamata il gruppo ha modificato un set di dati per renderli più realistici e ha riavviato l'addestramento, dando quindi un risultato più realistico e soddisfacendo il proponente.
        \subsection{Punto 3}
            Infine il dott. Piccoli ha consigliato di permettere all'utente di poter scegliere a piacere l'assegnazione dei predittori alle classi dell'algoritmo SVM\glosp e quindi di poter assegnare il colore corrispondente dei punti presenti nel grafico.