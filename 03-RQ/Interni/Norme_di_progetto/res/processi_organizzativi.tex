\section{Processi organizzativi}
	\subsection{Gestione organizzativa}
		\subsubsection{Scopo}
			Lo scopo del processo\glosp di gestione organizzativa è definito dai seguenti punti:
			\begin{itemize}
				\item creazione di un modello organizzativo che specifica i rischi che possono verificarsi;
				\item definizione un modello di sviluppo da seguire;
				\item pianificazione del lavoro in base alle scadenze fissate;
				\item creazione e calcolo di un piano economico suddiviso tra i ruoli;
				\item definizione finale del bilancio sul totale delle spese.
			\end{itemize}
			Queste attività sono definite a cura del responsabile di progetto\glosp e redatte all'interno del documento \textit{Piano di Progetto v. 6.1.1}.
		\subsubsection{Aspettative}
			Gli obiettivi del processo\glosp di gestione organizzativa sono:
			\begin{itemize}
				\item coordinare i componenti del gruppo attraverso l'assegnazione di ruoli e compiti;
				\item coordinare la comunicazione tra i membri del gruppo con l'ausilio di strumenti che la rendano facile ed efficace;
				\item pianificare l'esecuzione delle attività in modo ragionevole;
				\item gestire le attività anche dal punto di vista economico;
				\item monitorare costantemente il team, i processi\glosp e i prodotti\glosp in modo efficace.
			\end{itemize}
		\subsubsection{Descrizione}
			Il processo\glosp di gestione organizzativa è composto dalle seguenti attività:
			\begin{itemize}
				\item definizione dello scopo;
				\item generazione delle istanze dei processi\glo;
				\item stima e pianificazione di risorse, costi e tempo per lo svolgimento del progetto\glo;
				\item assegnazione dei ruoli e, per ognuno di essi, delle attività;
				\item gestione del controllo e dell'esecuzione di tutte le attività;
				\item valutazione periodica dello stato e dell'esecuzione delle attività rispetto a quanto pianificato.
			\end{itemize}			
		\subsubsection{Attività}
		\paragraph{Pianificazione delle risorse}\mbox{}\\ [1mm]
		Per gestire le risorse disponibili per il nostro progetto\glosp dobbiamo monitorare il rispetto dei costi e dei tempi preventivati. A tal proposito definiamo i ruoli di progetto e il modo in cui vengono assegnati.
		\paragraph{Ruoli di progetto}
		
			\paragraph*{Assegnazione}\mbox{}\\ [1mm]
				I ruoli scelti corrispondono alla rispettiva figura aziendale e valgono per la durata di una milestone. Al termine di ognuna di esse, viene eseguita una rotazione dei ruoli così da permettere a ciascuno dei membri del gruppo di ricoprirli tutti almeno una volta.
				I ruoli previsti sono descritti di seguito.
			\paragraph*{Responsabile di progetto}\mbox{}\\ [1mm]
				La figura del responsabile di progetto\glosp è molto importante in quanto convergono su di esso le responsabilità di gestione, pianificazione, coordinamento e controllo dell'intero progetto\glo.
				Le sue mansioni sono:
				\begin{itemize}
					\item redazione dell'organigramma e del documento \textit{Piano di Progetto};
					\item controllo e coordinamento di risorse, componenti e attività del gruppo;
					\item relazione con il controllo di qualità interno al progetto\glo;
					\item analisi e gestione degli elementi più critici e i rischi;
					\item intermediazione nelle comunicazioni esterne ovvero con proponente e committente;
					\item elaborazione ed emissione di piani e scadenza;
					\item approvazione finale dei documenti.
				\end{itemize}
			\paragraph*{Amministratore di progetto}\mbox{}\\ [1mm]
				La figura dell'amministratore è responsabile dell'efficienza e dell'operatività dell'ambiente di sviluppo e della redazione e attuazione di piani e procedure di gestione per la qualità.
				Le sue mansioni sono:
				\begin{itemize}
					\item controllo versioni e configurazioni del prodotto\glo;
					\item gestione di tutta la documentazione del progetto\glo;
					\item direzione delle infrastrutture di supporto;
					\item risoluzione dei problemi sorti dalla gestione dei processi\glo;
					\item stesura del documento \textit{Norme di Progetto} per conto del responsabile di progetto\glo;
					\item collaborazione con il responsabile di progetto\glo alla reazione del documento \textit{Piano di Progetto}.
				\end{itemize}
			\paragraph*{Analista}\mbox{}\\ [1mm]
				La figura dell'analista è responsabile di tutta l'attività di analisi dei problemi e del dominio applicativo.
				Le sue mansioni sono:
				\begin{itemize}
					\item studio del dominio del problema;
					\item analisi della complessità, della fattibilità e dei requisiti del problema;
					\item stesura del documento \textit{Studio di Fattibilità};
					\item stesura del documento \textit{Analisi dei Requisiti}.
				\end{itemize}
			\paragraph*{Progettista}\mbox{}\\ [1mm]
				La figura del progettista è responsabile delle attività di progettazione\glo. Deve perciò gestire gli aspetti tecnologici e tecnici del progetto\glo.
				Le sue mansioni sono:
				\begin{itemize}
					\item redigere le specifiche tecniche del prodotto\glosp effettuando scelte il più efficienti e performanti possibili, sulla base delle tecnologie presenti;
					\item sviluppare l'architettura del prodotto\glosp tale da fornire una base stabile a manutenibile che soddisfi le specifiche tecniche.
				\end{itemize}
			\paragraph*{Programmatore}\mbox{}\\ [1mm]
				La figura del programmatore è responsabile dell'attività di codifica mirata alla realizzazione del prodotto\glosp compreso delle componenti necessarie allo svolgimento di test di verifica e validazione\glo.
				Le sue mansioni sono:
				\begin{itemize}
					\item eseguire l'attività di codifica sulla base del lavoro svolto da progettista;
					\item fornire dei componenti che permettano si svolgere test e validazione\glosp sul prodotto\glo.
				\end{itemize}
			\paragraph*{Verificatore}\mbox{}\\ [1mm]
				La figura del verificatore è responsabile di tutta l'attività di verifica.
				Si affida al documento \textit{Norme di Progetto} è definito anche lo standard da seguire per questa attività.
				Le sue mansioni sono:
				\begin{itemize}
					\item redazione della parte retrospettiva del \textit{Piano di Qualifica};
					\item ispezione dei documenti al fine di controllare che siano conformi allo standard previsto nelle \textit{Norme di Progetto};
					\item segnalazione di eventuali errori nelle parti di prodotto\glosp a chi ha responsabilità su di esse.
				\end{itemize}
			\paragraph{Esecuzione e controllo}
				Il compito di eseguire e monitorare la gestione organizzativa viene assegnato all'amministratore. Egli infatti deve analizzare e risolvere i problemi che sorgono, fornendo dei report sullo stato di avanzamento di progetto sia internamente che al committente.	
		\paragraph{Gestione dei rischi}\mbox{}\\ [1mm]
		La gestione e l'analisi dei rischi è un'attività a carico del responsabile di progetto\glosp che deve essere documentata nel documento \textit{Piano di Progetto v. 6.1.1}.
		La procedura per la gestione dei rischi è composta dai seguenti passi:
		\begin{itemize}
			\item individuazione dei rischi;
			\item analisi dei rischi;
			\item pianificazione dell'attività di controllo;
			\item controllo e monitoraggio.
		\end{itemize}
		Viene definita una codifica per l'identificazione dei fattori di rischio: R[tipo][X]
		\begin{itemize}
			\item tipo:
			\begin{itemize}
				\item \textbf{RT}: rischi tecnologici di lavoro e produzione del software;
				\item \textbf{RG}: rischi di gruppo;
				\item \textbf{RO}: rischi organizzativi delle attività da svolgere;
				\item \textbf{RR}: rischi legati ai requisiti e ai rapporti con gli stakeholder\glo;
				\item \textbf{RS}: rischi di stima di costi e tempi.
			\end{itemize}
			\item X: numero intero progressivo che inizia da 1.
		\end{itemize}
		Inoltre vengono definiti delle misure di probabilità e degli indici di gravità:
		\begin{itemize}
			\item probabilità:
			\begin{itemize}
				\item \textbf{Alta};
				\item \textbf{Media};
				\item \textbf{Bassa}.
			\end{itemize}
			\item Gravità:
			\begin{itemize}
				\item \textbf{1}: basso livello di gravità;
				\item \textbf{2}: medio livello di gravità;
				\item \textbf{3}: alto livello di gravità.
			\end{itemize}
		\end{itemize}
		\subsubsection{Procedure}
		Sono state definite un insieme di procedure che definiscono la gestione del coordinamento interno del gruppo.
		\paragraph{Gestione delle comunicazioni}
			\subparagraph{Comunicazioni interne}\mbox{}\\ [1mm]
				Le comunicazioni interne al gruppo avvengono tramite la piattaforma Slack nella quale è stato creato un workspace suddiviso nei seguenti canali tematici:
				\begin{itemize}
					\item \textbf{Documentazione}: canale dedicato alle discussioni in merito alle attività di stesura e verifica della documentazione;
					\item \textbf{Sviluppo}: canale dedicato alle discussioni in merito alle attività di analisi, progettazione\glosp e codifica del prodotto\glo;
					\item \textbf{Generale}: canale dedicato alle comunicazioni di servizio per il coordinamento di attività, incontri e impegni del gruppo.
				\end{itemize}
				Nei canali dedicati alla documentazione e allo sviluppo è stata attivata l'integrazione con GitHub per segnalare i nuovi commit e le pull request con un messaggio.
				Inoltre è stata eseguita l'integrazione con Google Calendar che permette di avere una sezione dedicata alle notifiche degli eventi programmati sul calendario e l'integrazione del plug-in Simple Poll che permette di eseguire dei sondaggi per prendere le decisioni.
			\subparagraph{Comunicazioni esterne}\mbox{}\\ [1mm]
				Le comunicazioni esterne avvengono tramite la posta elettronica all'indirizzo \url{vram.software@gmail.com}.
				Il responsabile di progetto\glosp ha l'incarico di gestire queste comunicazioni ed è tenuto ad informare tutti gli elementi del gruppo degli argomenti trattati qualora non fossero presenti.
		\paragraph{Gestione degli incontri}
			\subparagraph{Incontri interni}\mbox{}\\ [1mm]
				Gli incontri del gruppo sono accordati su Slack nel canale generale oppure nella precedente riunione e vengono annotati su Google Calendar. Essi sono identificati da:
				\begin{itemize}
					\item data;
					\item ora di inizio;
					\item durata prevista;
					\item elenco di attività da svolgere.
				\end{itemize}
				A questi incontri partecipano solo i membri del gruppo. Coloro i quali, per diverse motivazioni, non possono essere presenti, sono tenuti a segnalarlo in fase di definizione dell'incontro e sarà compito del responsabile aggiornarli successivamente.
				Gli incontri interni si distinguono in due tipologie:
				\begin{itemize}
					\item \textbf{incontri di persona}: incontri svolti principalmente presso le strutture dell'Università degli Studi di Padova nei quali il gruppo si incontra di persona e discute dell'ordine del giorno prefissato. Questa tipologia viene scelta quando, per impegni universitari, la maggior parte dei membri del gruppo si trova in luoghi vicini oppure per prendere decisioni molto importanti;
					\item \textbf{incontri telematici}: chiamate o video chiamate di gruppo svolte attraverso l'applicazione Skype. Questa tipologia viene scelta quando si necessita di un incontro, ma la maggior parte dei membri del gruppo si trova scomoda o impossibilitata a raggiungere le sedi dell'Università. 
				\end{itemize}
			\subparagraph{Verbali di incontri interni}\mbox{}\\ [1mm]
				Un segretario, nominato dal responsabile, è incaricato della redazione di un \textit{Verbale} della riunione che tiene traccia di:
				\begin{itemize}
					\item \textbf{Luogo};
					\item \textbf{Data};
					\item \textbf{Ora d'inizio};
					\item \textbf{Ora di fine};
					\item \textbf{Partecipanti};
					\item \textbf{Argomenti trattati}: argomenti all'ordine del giorno che sono trattati nell'incontro da verbalizzare;
					\item \textbf{Discussioni e decisioni prese}: discussioni e decisioni prese in merito a tematiche interne al gruppo.
				\end{itemize}
			\subparagraph{Incontri esterni}\mbox{}\\ [1mm]
				Gli incontri esterni vengono concordati se i membri del gruppo, il committente o il proponente lo ritengono necessario. In tal caso, sarà compito del responsabile di progetto\glosp definire data e ora dell'incontro, in accordo con le due parti, attraverso gli appositi canali di comunicazione descritti in precedenza.
				Gli incontri interni si distinguono in due tipologie:
				\begin{itemize}
					\item \textbf{incontri di persona}: incontri svolti di persona presso gli uffici del proponente o del committente, in base a chi è diretto il colloquio, nei quali ognuno espone problemi o dubbi da chiarire e discutendone. Questa tipologia viene scelta per avere dei riscontri immediati qualora tutte le parti fossero disponibili e ci fosse la necessità;
					\item \textbf{incontri telematici}: chiamate o video chiamate attraverso l'applicazione Skype, Hangouts o Zoom sulla base delle richieste del proponente o del committente, in base a chi è diretto il colloquio. Questa tipologia viene scelta quando si necessita di un incontro, ma le parti in causa si trovano impossibilitate ad incontrarsi di persona. 
				\end{itemize}
			\subparagraph{Verbali di incontri esterni}\mbox{}\\ [1mm]
				Un segretario, nominato dal responsabile, è incaricato della redazione di un \textit{Verbale} della riunione tenendo traccia di:
				\begin{itemize}
					\item \textbf{Luogo};
					\item \textbf{Data};
					\item \textbf{Ora d'inizio};
					\item \textbf{Ora di fine};
					\item \textbf{Partecipanti};
					\item \textbf{Argomenti trattati}: argomenti all'ordine del giorno che sono trattati nell'incontro da verbalizzare;
					\item \textbf{Verbale}: esposizione delle discussioni con il proponente e delle decisioni prese in merito.
				\end{itemize}
			\subparagraph{Incontri in situazione eccezionale}\mbox{}\\ [1mm]
			Il Rettore dell'Università degli Studi di Padova ha emesso delle disposizioni in merito alla pandemia in corso causata dal virus Covid-19 nelle quali è ufficializzata la chiusura delle sedi universitarie dal mese di marzo a data da destinarsi. Fino ad allora, tutti gli incontri interni ed esterni sono obbligatoriamente virtuali.
		\paragraph{Gestione degli strumenti di coordinamento}
			\subparagraph{Ticketing}\mbox{}\\ [1mm]
				Il sistema di ticketing permette ai membri del gruppo di gestire le tutte le attività del progetto\glo.
				Per eseguire il ticketing vengono utilizzati gli strumenti dell'Issue Tracking System presenti gratuitamente in GitHub ed il funzionamento è il seguente:
				il responsabile di progetto\glosp ha l'incarico di assegnare i compiti ai membri del gruppo attraverso le issue. Ognuna di esse contiene titolo, descrizione, milestone, le project board Kanban automatizzate associate e il destinatario.
				Le project board Kanban automatizzate permettono al responsabile di progetto\glosp di monitorare l'andamento delle attività tracciando lo stato di ogni singolo compito che può essere:
				\begin{itemize}
					\item To Do (da fare);
					\item In Progress (in lavorazione);
					\item Done (concluso).
				\end{itemize}
				L'associazione con le milestone\glosp definisce le scadenze delle singole issue.
				Inoltre è presente una comoda funzionalità di ricerca per le issue assegnate che ne agevola ulteriormente la gestione.
				
		\subsubsection{Metriche di qualità}
		\paragraph{PRC-Q5 Processo di gestione organizzativa}
		\subparagraph{OP-7 Pianificazione efficace delle risorse}
		\paragraph*{Obiettivo}\mbox{}\\ [1mm]
		L'obiettivo è pianificare le risorse per garantire un corretto avanzamento del progetto monitorando e rispettando costi e tempistiche.
		\paragraph*{Metriche}\mbox{}\\ [1mm]
		Le metriche\glosp utilizzate sono:
		\begin{itemize}		
			\item \textbf{M07 Planned value}: indica il valore dei lavoro pianificato fino a quel momento;
			\begin{itemize}
				\item[] \textbf{formula}: BAC $\cdot$ \%lavoro pianificato con BAC=budget totale;
			\end{itemize}
			
			\item \textbf{M08 Earned value}: indica il valore del lavoro svolto fino a quel momento;
			\begin{itemize}
				\item[] \textbf{formula}: BAC $\cdot$ \%lavoro svolto BAC=budget totale;
			\end{itemize} 
			
			\item \textbf{M09 Actual cost}: indica il denaro speso fino a quel momento; 
			\begin{itemize}
				\item[] \textbf{formula}: si calcola con un valore intero;
			\end{itemize}
			
			\item \textbf{M10 Cost performance index}: rappresenta un indice per i costi effettivi rispetto a quelli previsti, se è $\le$1 indica che si sta spendendo più di quanto preventivato;
			\begin{itemize}
				\item[] \textbf{formula}: $\frac{EV}{AC}$; 
			\end{itemize}
			
			\item \textbf{M11 Schedule performance index}: rappresenta un indice per i tempi effettivi rispetto a quelli previsti, se è $\le$1 indica che si sta impiegando più tempo di quanto preventivato;
			\begin{itemize}
				\item[] \textbf{formula}: $\frac{EV}{PV}$; 
			\end{itemize}
			
			\item \textbf{M12 Estimated cost at completion}: rappresenta il budget totale stimato del progetto\glosp con il CPI del momento;
			\begin{itemize}
				\item[] \textbf{formula}: $AC+\frac{BT-EV}{CPI}$; 
			\end{itemize} 
			
			\item \textbf{M13 Schedule at completion}: rappresenta il tempo totale stimato del progetto\glosp con SPI del momento;
			\begin{itemize}
				\item[] \textbf{formula}: $\frac{TT}{SPI}$; con TT=Tempo Totale. 
			\end{itemize} 
		\end{itemize}
	
		\subparagraph{OP-Q8 Prevenzione dei rischi}
		\paragraph*{Obiettivo}\mbox{}\\ [1mm]
		L'obiettivo è ed analizzare i rischi al fine di trovare metodologie efficaci per preventivarli e successivamente prevenirli.
		\paragraph*{Metriche}\mbox{}\\ [1mm]
		Le metriche\glosp utilizzate sono:
		\begin{itemize}
			\item \textbf{M14 Rischi non preventivati}: indica il numero di rischi non preventivati che si verificano. Un valore elevato indica una superficialità nell'individuazione dei rischi possibili;
			\begin{itemize}
				\item[] \textbf{formula}: si calcola con un valore intero partendo da 0 e andando a fare un incremento ogni volta che si verifica un imprevisto.
			\end{itemize}
		\end{itemize}
		\subsubsection{Strumenti di supporto}
			Gli strumenti utilizzati sono i seguenti:
			\begin{itemize}
				\item \textbf{Telegram}: strumento di messaggistica istantanea utilizzato inizialmente dal gruppo per il primo incontro;
				\item \textbf{Slack}: strumento utilizzato per le comunicazione interne del gruppo;
				\item \textbf{Google Drive}: strumento utilizzato per file che non sono oggetto di cambiamenti frequenti come il \textit{Glossario} o immagini di vario tipo;
				\item \textbf{Google Calendar}: strumento utilizzato per tracciare e semplificare la gestione degli impegni e fornire a tutti notifiche e promemoria per gli stessi;
				\item \textbf{Gmail}: strumento di posta elettronica utilizzato per le comunicazioni esterne.
				\item \textbf{Skype}: strumento utilizzato per effettuare chiamate o video chiamate per alcuni incontri interni o esterni;
				\item \textbf{Hangouts}: strumento utilizzare per effettuare chiamate e video chiamate per alcuni incontri esterni;
				\item \textbf{Zoom}: strumento utilizzato per effettuare chiamate e videochiamate per alcuni incontri esterni; 
				\item \textbf{Git}: sistema di controllo di versionamento\glo;
				\item \textbf{GitKraken, GitHub Desktop}: interfacce l'utilizzo di GitHub in locale;
				\item \textbf{GitHub}: strumento utilizzato per il proprio Issue Tracking System e sistema di versionamento\glosp dei file.
			\end{itemize}
	\subsection{Formazione}
		\subsubsection{Scopo}
			Lo scopo di questo processo\glosp è formare ogni membro del gruppo per garantire che ognuno abbia le conoscenze e le competenze adeguate allo svolgimento dei compiti assegnati.
		\subsubsection{Aspettative}
			L'aspettativa del gruppo, nel caso di questo processo\glo, è riuscire a ottimizzare i tempi non impiegando tutti i membri nell'apprendimento di uno stesso strumento o tecnologia, ma lavorare in modo parallelo, al fine di assimilare più concetti possibili per poi condividerli con gli altri componenti.
		\subsubsection{Descrizione}
			Il processo\glosp di formazione è composto da un insieme di attività che definiscono come vengono formati i componenti del gruppo.
			Lo studio necessario per svolgere i compiti e utilizzare gli strumenti viene principalmente eseguito in modo individuale e autonomo da un componente del gruppo definito periodicamente dal responsabile di progetto\glo. Al termine dello studio, egli deve comunicare ciò che ha imparato agli altri membri per garantire che tutti apprendano le nozioni.
			Inoltre, qualora un componente ritenga di non avere ancora sufficienti capacità o conoscenze per svolgere un compito assegnato, si deve rivolgere al responsabile di progetto\glosp che ha l'incarico di organizzare un seminario mirato per il suo apprendimento sia tramite un incontro fisico che in via telematica.
		\subsubsection{Attività}
		\paragraph{Condivisione del materiale}\mbox{}\\ [1mm]
			Il materiale del progetto\glosp è inserito all'interno di un sistema di versionamento\glosp che permette ai membri del gruppo di integrare le proprie conoscenze e apprendere dal lavoro altrui.
			Per ogni documento, il materiale utilizzato per la comprensione dei concetti e la stesura è indicato nei "Riferimenti Informativi".
			Inoltre sono presenti i "Riferimenti Normativi" dentro i quali è presente il documento \textit{Norme di Progetto v. 4.1.1} che tutti devono rispettare.
		\subsubsection{Strumenti di supporto}
		\begin{itemize}
		 	\item \textbf{Skype}: strumento utilizzato per effettuare chiamate o video chiamate per gli incontri di formazione in via telematica;
	 	\end{itemize}
			
