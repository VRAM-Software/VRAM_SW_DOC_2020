\subsection{PRC-Q7 Validazione}

\subsubsection{Scopo}
Lo scopo del processo\glosp di validazione\glosp è accertare che il prodotto\glo, per uno specifico uso, corrisponda alle attese.

\subsubsection{Aspettative}
Le aspettative del gruppo sono di normare adeguatamente il processo\glosp di validazione\glosp al fine di rilasciare un prodotto\glosp privo di errori gravi e capace di rispondere alle richieste del proponente.

\subsubsection{Descrizione}
Le attese sono quelle del committente e corrispondono, se stilata correttamente, all'\textit{Analisi dei Requisiti v. 3.1.1}. 
Per eseguire una validazione\glosp si deve avere un prodotto\glosp che è a un sufficiente livello di avanzamento, quindi il numero di validazioni svolte è molto minore rispetto al numero delle verifiche, che iniziano a venire eseguite prima.
Durante questo processo vengono svolti i test di accettazione tramite i quali il proponente valida il prodotto\glosp software.

\subsubsection{Attività}
	\paragraph{Test di accettazione}
		\paragraph*{Codifica del test}
		Per ogni test vengono definiti il codice identificativo, la descrizione, lo stato e l'esito.
		\begin{itemize}
			\item il codice identificativo è scritto in questo formato:\\
			\textbf{tipologia[codice\_padre].[codice\_figlio]}
			\begin{itemize}
				\item \textbf{tipologia}:
				\begin{itemize}
					\item \textbf{TA}: test di accettazione;
				\end{itemize}
				\item \textbf{codice\_padre}: numero che identifica univocamente il test;
				\item \textbf{codice\_figlio}: numero progressivo che indica i sotto-test che compongono il padre. 
			\end{itemize}
			\item descrizione: spiegazione concisa e completa di ciò che verifica il test;
			\item stato:
			\begin{itemize}
				\item \textbf{I}: implementato;
				\item \textbf{NI}: non implementato.
			\end{itemize}
			\item Esito:
			\begin{itemize}
				\item \textbf{P}: positivo;
				\item \textbf{N}: negativo;
				\item \textbf{NE}: non eseguito.
			\end{itemize}
		\end{itemize}
		\paragraph*{Descrizione} \mbox{}\\ [1mm]
		I test di accettazione, o di collaudo, vengono effettuati ad ogni pre-rilascio, quindi dopo l'esecuzione dei test di sistema. Essi servono a confermare il soddisfacimento dei requisiti da parte del committente; infatti questi tipi di test ne richiedono la presenza. Se questi test vengono superati significa che il prodotto\glosp software è pronto per il rilascio. 
		Per identificare un test di accettazione viene utilizzata una codifica con il seguente formato:\\
		\textbf{TA[codice]}
		Dove codice è un numero progressivo nel formato: [X].[Y] con X e Y numeri maggiori di zero.
%\subsection{Strumenti di supporto}