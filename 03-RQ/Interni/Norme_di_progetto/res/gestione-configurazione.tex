\subsection{PRC-Q4 Gestione della configurazione}
	\subsubsection{Scopo}
		L'obiettivo di questo processo\glosp è rendere documenti e codice sorgente univocamente identificati e facilmente riconoscibili, evidenziando versioni e modifiche. Vuole anche agevolare l'identificazione delle relazioni esistenti fra gli elementi e fa da supporto alla fase di verifica.
	\subsubsection{Aspettative}
		Questa sezione ha lo scopo supportare i processi\glosp di documentazione, di sviluppo e manutenzione del software rendendoli definiti e ripetibili.  
	\subsubsection{Descrizione} 
		Questo processo descrive come il gruppo ha gestito la configurazione degli strumenti e delle risorse utilizzate per svolgere il progetto\glo.
		Sarà quindi descritta la configurazione del repository\glosp su GitHub, del sistema di versionamento\glosp Git e dei servizi GitHub.
	\subsubsection{Attività}
		\paragraph{Versionamento}\mbox{}\\ [1mm]
		A partire dalla versione 7.0.0 del prodotto, è stato introdotto un nuovo sistema di versionamento univoco per l'intero prodotto, di seguito descritto.
		\newline
		Le modifiche effettuate ai documenti ed ai file contenenti codice sorgente sono identificate da un numero di versione, comune per l'interno prodotto, consultabile direttamente tramite le issue su Github, all'interno delle quali sono anche identificabili i singoli commit che compongono la modifica. 
		\newline
		Nello specifico caso dei documenti, il numero di versione possiede sempre una corrispondente riga nella tabella delle modifiche, presente nei documenti stessi, che descrive i cambiamenti avvenuti all'interno dei singoli file. I salti di versione in queste tabelle, dovuti all'aggiornamento degli altri sottoprodotti, sono indicati in modo sintetico all'interno delle stesse. Il numero di versione viene inoltre inserito nel nome dei documenti pdf generati.
		\newline
		L'insieme delle modifiche effettuate in un rilascio del prodotto è consultabile dall'utente finale su Github nella sezione rilasci del repository grafana\_prediction, dove ogni rilascio è identificato da un tag.
		\newline
		La numerazione per il versionamento\glosp del prodotto si base sulle modifiche applicate al repository\glosp principale grafana\_prediction, che contiene al suo interno gli altri repository\glosp come sottomoduli. Ad ogni modifica in esso effettuata il numero di versione del prodotto viene incrementato secondo lo schema seguente.
		
		\subparagraph*{Numerazione delle versioni del prodotto}
		Per identificare univocamente le versioni del prodotto utilizziamo una numerazione basata su 3 cifre nel formato X.Y.Z, la prima bozza ha versione 0.0.0 e le modifiche ai file aggiorneranno le cifre della versione nel modo seguente:
		\begin{itemize}
			\item \textbf{Z} : Questa cifra viene incrementata ogni volta che vengono effettuate delle modifiche al codice sorgente o ai documenti al fine di risolvere problemi riscontrati nel prodotto;
			\item \textbf{Y} : Questa cifra viene incrementata ogni volta che vengono effettuate delle modifiche al codice sorgente o ai documenti al fine di aggiungere nuove funzionalità di entità minore al prodotto(modifiche minori\glo), la nuova versione deve inoltre presentare piena compatibilità con le versioni precedenti. Un aggiornamento di questa cifra comporta l'azzeramento della cifra Z, es 1.1.0, 1.2.0 etc;
			\item \textbf{X} : Questa cifra viene incrementata ogni volta che vengono effettuate modifiche al codice sorgente o ai documenti al fine di aggiungere nuove funzionalità di entità maggiore al prodotto(modifiche maggiori\glo), ad esempio novità che cambino in modo importante il suo uso o le funzionalità che esso offre. Un aumento di versione di questo tipo può, se necessario, non essere retrocompatibile. Un aggiornamento di questa cifra comporta l'azzeramento delle cifre Y e Z, es 2.0.0, 3.0.0 etc.
		\end{itemize}
		La prima versione finale del prodotto ha numerazione 1.0.0.
		\newline
		Ogni incremento sulle singole cifre è rigorosamente un +1, non possono essere saltati numeri.
		
	\paragraph{Gestione delle modifiche}\mbox{}\\ [1mm]
		Al fine di monitorare e limitare le modifiche al ramo principale del repository\glo, master, è utilizzato il meccanismo di pull request fornito da GitHub. Ogni membro del gruppo può creare branch secondari, secondo il workflow\glosp feature branch, su cui effettuare modifiche, tuttavia per unirle al branch master è necessario aprire una pull request che dovrà essere revisionata dai verificatori tramite i servizi di revisione integrati in GitHub. Una volta revisionata positivamente è compito del responsabile del documento approvare la pull request ed effettuare quindi l'effettiva unione delle modifiche nel branch master.
		\newline
		In sintesi, per effettuare modifiche ai file sono previsti i seguenti passaggi:
		\begin{itemize}
			\item contattare il responsabile del file affinché autorizzi la modifica del file stesso;
			\item creare un branch secondario ed effettuare le modifiche al file;
			\item aprire una pull request per unire le modifiche al ramo master;
			\item i verificatori revisionano la pull request ed eventualmente richiedono aggiornamenti;
			\item completata la revisione il responsabile approva la pull request.
		\end{itemize}
	
		Il meccanismo sopra esposto si applica anche al codice sorgente. Inoltre in esso sono presenti anche dei sistemi di verifica automatica e di continuous integration che possono bloccare le pull request qualora le modifiche effettuate non rispettino i livelli di qualità desiderati o interrompano la compilazione o il superamento dei test automatici sul codice. In particolare, gli strumenti utilizzati a questo scopo sono Github Actions per implementare la continuous integration, JEst per l'esecuzione dei test automatici sul codice sorgente ed i servizi Coveralls e SonarCloud per controllare la qualità del codice sorgente.
		
	\paragraph{Repository}\mbox{}\\ [1mm]
		Per tenere traccia di versioni e modifiche fatte a documenti e codice è utilizzato il sistema di versionamento\glosp distribuito Git, che può essere utilizzato tramite riga di comando o utilità grafiche come GitHub Desktop o GitKraken.
		La struttura dei repository\glosp utlizzati è la seguente:
		\newline
		\dirtree{%
			.1 grafana\_prediction.
			.2 VRAM\_SW\_DOC\_2020.
			.2 grafana\_prediction\_plugin.
			.2 prediction\_configuration.
		}
		\mbox{}\\ % forza un newline
		grafana\_prediction è il repository\glosp principale e gli altri sono suoi sottomoduli, configurati tramite la funzionalità Git submodules di Git.
		Il repository\glosp principale, contenente i sottomoduli, è ospitato sul sito GitHub all'indirizzo: 
		\begin{center}
			\url{https://github.com/VRAM-Software/grafana_prediction}
		\end{center}
		\subparagraph*{Struttura del repository VRAM\_SW\_DOC\_2020}
		Al fine di fornire una navigazione agevole e standardizzata, il contenuto del repository\glosp è organizzato in modo gerarchico tramite directory secondo il seguente schema:
		\newline
		\dirtree{%
			.1 root.
				.2 RR.
					.3 Esterni.
					.3 Interni.
				.2 RP.
					.3 Esterni.
					.3 Interni.
				.2 RQ.
					.3 Esterni.
					.3 Interni.
				.2 RA.
					.3 Esterni.
					.3 Interni.
				.2 Guide.
				.2 Template.
					.3 Images.
					.3 config.
					.3 img.
					.3 res.
		}
		\mbox{}\\ % forza un newline
		Nel dettaglio:
		\begin{itemize}
			\item \textbf{RR}: contiene i sorgenti \LaTeX \xspace dei documenti relativi alla revisione dei requisiti;
			\item \textbf{RP}: contiene i sorgenti \LaTeX \xspace dei documenti relativi alla revisione di progettazione\glo;
			\item \textbf{RQ}: contiene i sorgenti \LaTeX \xspace dei documenti relativi alla revisione di qualifica;
			\item \textbf{RA}: contiene i sorgenti \LaTeX \xspace dei documenti relativi alla revisione di accettazione; 
			\item \textbf{Guide}: contiene brevi indicazioni interne su come usare i comandi \LaTeX \xspace usati nei documenti e su come riutilizzare il template per generare nuovi documenti;
			\item \textbf{Template}: contiene i sorgenti \LaTeX \xspace usati per generare la base comune di tutti i documenti.
		\end{itemize}
		È inoltre disponibile una directory condivisa su Google Drive, con la stessa struttura gerarchica sopra esposta, che contiene tutti e soli i file PDF in versione finale. Lo scopo di questa directory condivisa è facilitare la consultazione e la condivisione dei file PDF stessi.
		
	\paragraph{Tipologie di file non accettate}\mbox{}\\ [1mm]
		Tramite un apposito file .gitignore presente nella root directory della gerarchia vengono definiti i tipi di file non accettati all'interno del repository\glo. Vengono esclusi tutti i file di compilazione, compilati o temporanei in quanto il repository\glosp dovrebbe contenere solamente i seguenti formati di file:
		\begin{itemize}
			\item file sorgente \LaTeX \xspace con estensione .tex;
			\item file immagine, preferibilmente in formato .png;
			\item file testuali in formato .md o .txt;
			\item tutti i file di codice sorgente come .js, .ts etc.		
		\end{itemize}
	\subsubsection{Strumenti di supporto}
	\paragraph{GitHub}
	\subparagraph*{Strumenti GitHub utilizzati}\mbox{}\\ [1mm]
		In aggiunta ai servizi già elencati, al fine di migliorare efficacia ed efficienza, vengono utilizzate le funzionalità di "Issue Tracking System",  
		"Milestone" e "Project Board" integrate in GitHub. Ognuna di queste funzionalità viene usata solo da chi autorizzato, ad esempio rilasci di versioni, creazione e chiusura di milestone sono concesse solo al responsabile di progetto\glo.
		\newline
		Vengono inoltre utilizzate le "Labels" offerte dall'Issue Tracking System di Github per definire le tipologie di problemi e le loro priorità.
		\newline
		Per il repository\glosp contenente il codice sorgente del software sono utilizzate le GitHub Actions al fine di implementare la pratica della continuous integration che, a sua volta, utilizza strumenti quali JEst, SonarCloud e Coveralls per eseguire le verifiche automatiche sul codice sorgente.
	\paragraph{SonarCloud}
	I repository\glosp del codice sorgente sono collegati al servizio SonarCloud ed è inserita la sua analisi statica\glosp del codice sorgente nei controlli eseguiti dalla continuous integration. Le pagine sonarcloud dei repository sono disponibili al seguente indirizzo: 
	\begin{center}
		\url{https://sonarcloud.io/organizations/vram-software/projects}
	\end{center}
	\paragraph{Coveralls}
	Poiché SonarCloud, nella versione disponibile al momento dello svolgimento del progetto\glo, non supporta l'analisi del code coverage tramite Github Actions, utilizziamo il servizio offerto da Coveralls che abbiamo integrato nella continuous integration. Le pagine Coveralls dei repository sono disponibili ai seguenti indirizzi:
	\begin{center}
		\url{https://coveralls.io/github/VRAM-Software/grafana_prediction_plugin}
	\end{center}
	\begin{center}
	\url{https://coveralls.io/github/VRAM-Software/prediction_configuration_utility}
	\end{center}
	
		