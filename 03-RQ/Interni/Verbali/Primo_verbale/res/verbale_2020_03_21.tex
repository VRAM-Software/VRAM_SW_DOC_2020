\section{Informazioni generali}
    \subsection{Informazioni incontro}
        \begin{itemize}
            \item \textbf{Luogo}: Videochiamata tramite Zoom;
            \item \textbf{Data}: 2020-03-21;
            \item \textbf{Ora d'inizio}: 09.00;
            \item \textbf{Ora di fine}: 11.30;
            \item \textbf{Partecipanti}: \begin{itemize}
                \item Corrizzato Vittorio;
                \item Dalla Libera Marco;
                \item Rampazzo Marco;
                \item Santagiuliana Vittorio;
                \item Schiavon Rebecca;
                \item Spreafico Alessandro;
                \item Toffoletto Massimo.
            \end{itemize}
        \end{itemize}
    \subsection{Argomenti trattati}
        Durante la riunione, i componenti hanno discusso l'esito della revisione di progettazione:
        È stata stilata una lista di domande 
\section{Verbale}
    La discussione ha seguito il documento di esito, dividendosi per argomento
    \subsection{Presentazione}
        È stata segnalata la mancanza di una visione generale per il prodotto, è stato quindi deciso di porre più enfasi nell'esposizione dei componenti di cui il prodotto si compone e dei loro ruoli.
    \subsection{Verbali}
        Modifica della titolo del riepilogo finale in "Riepilogo delle decisioni".
    \subsection{Versionamento}
    \subsection{Norme di progetto}
        Il documento delle \textit{Norme di progetto} andrà riformato con i seguenti cambiamenti
        \begin{itemize}
            \item Aggiungere la normazione per gli strumenti di comunicazione \textit{Hangouts} e \textit{Zoom} ai \textit{Processi organizzativi};
            \item  Specificare le modalità con cui si organizzano gli incontri telematici;
            \item Spostare la sezione riguardante la \textit{Qualità di prodotto} in quanto non facente parte dei processi.
        \end{itemize}
    \subsection{Analisi dei requisiti}
        Richiesta una revisione della descrizione del caso d'uso n° 21, richiesti chiarimenti sulla numerazione dei casi d'uso.
    \subsection{Piano di progetto}
        In seguito alla segnalazione di un numero di incrementi eccessivo, si è deciso di accorpare alcuni incrementi.
    \subsection{Piano di qualifica}
        Gli elenchi puntati delle sezioni §2 e §3 andranno adattati ad una forma tabellare.
    \subsection{Chiarimenti}
        Durante l'analisi degli esiti della revisione di progettazione, sono sorti alcuni dubbi: è stata stilata una lista di punti da chiarire o approfondire nel colloquio con il committente, prof. Tullio Vardanega.
