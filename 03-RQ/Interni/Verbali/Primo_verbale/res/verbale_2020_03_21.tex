\section{Informazioni generali}
    \subsection{Informazioni incontro}
        \begin{itemize}
            \item \textbf{Luogo}: Videochiamata tramite Zoom;
            \item \textbf{Data}: 2020-03-21;
            \item \textbf{Ora d'inizio}: 09.00;
            \item \textbf{Ora di fine}: 11.30;
            \item \textbf{Partecipanti}: \begin{itemize}
                \item Corrizzato Vittorio;
                \item Dalla Libera Marco;
                \item Rampazzo Marco;
                \item Santagiuliana Vittorio;
                \item Schiavon Rebecca;
                \item Spreafico Alessandro;
                \item Toffoletto Massimo.
            \end{itemize}
        \end{itemize}
    \subsection{Argomenti trattati}
        Durante la riunione, i componenti si sono confrontati riguardo all'esito della revisione di progettazione: la discussione si è divisa per argomenti, seguendo il documento di esito.
\section{Verbale}
    \subsection{Punto 1}
        \subsubsection{Versionamento}
            Sono sorti dubbi inerenti al versionamento, sia riguardo alla numerazione che all'incremento del numero di versione nello sviluppo dei documenti. Non riuscendo a chiarire questi dubbi, è stato deciso di chiedere al professore ulteriori spiegazioni.
        \subsubsection{Verbali}
            È stato deciso di modificare il titolo del riepilogo finale in "Riepilogo delle decisioni" come indicato nelle segnalazioni.
        \subsubsection{Presentazione}
            È stata segnalata la mancanza di una visione generale per il prodotto, è stato quindi deciso di porre più enfasi nell'esposizione dei componenti di cui il prodotto si compone e dei loro ruoli.
        \subsubsection{Norme di progetto}
            Il documento delle \textit{Norme di progetto} andrà riformato con i seguenti cambiamenti:
            \begin{itemize}
                \item Aggiungere la normazione per gli strumenti di comunicazione \textit{Hangouts} e \textit{Zoom} ai \textit{Processi organizzativi};
                \item  Specificare le modalità con cui si organizzano gli incontri telematici;
                \item Spostare la sezione riguardante la \textit{Qualità di prodotto} dalla sua attuale collocazione in quanto fuorviante perché la \textit{Qualità di prodotto} non fa parte dei processi.
            \end{itemize}
        \subsubsection{Analisi dei requisiti}
            Come da richiesta, verrà rivista la descrizione del caso d'uso n. 21. Le segnalazioni riguardanti la numerazione dei casi d'uso ci sono sembrate contrastanti con indicazioni ricevute in precedenza, perciò è stato deciso di chiedere chiarimenti a riguardo.
        \subsubsection{Piano di progetto}
            In seguito alla segnalazione di una mancanza di centralità degli incrementi nella pianificazione, si è deciso di farsi guidare da essi nella pianificazione di dettaglio andando a definire un periodo per ogni incremento.
        \subsubsection{Piano di qualifica}
            Gli elenchi puntati delle sezioni §2 e §3 andranno riadattati in forma tabellare per facilitarne la lettura.
    \subsection{Punto 2}
        Durante l'analisi degli esiti della revisione di progettazione, sono sorti alcuni dubbi. È stata quindi stilata una lista di punti da chiarire o approfondire con il committente, prof. Tullio Vardanega.
