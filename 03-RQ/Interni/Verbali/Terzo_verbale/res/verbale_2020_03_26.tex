\section{Informazioni generali}
\subsection{Informazioni incontro}
\begin{itemize}
	\item \textbf{Luogo}: Videochiamata tramite Zoom;
	\item \textbf{Data}: 2020-03-26;
	\item \textbf{Ora d'inizio}: 21.00;
	\item \textbf{Ora di fine}: 22.15;
	\item \textbf{Partecipanti}: \begin{itemize}
		\item Corrizzato Vittorio;
		\item Dalla Libera Marco;
		\item Rampazzo Marco;
		\item Santagiuliana Vittorio;
		\item Schiavon Rebecca;
		\item Spreafico Alessandro;
		\item Toffoletto Massimo.
	\end{itemize}
\end{itemize}
\subsection{Argomenti trattati}
Durante la riunione, i componenti del gruppo si sono confrontati riguardo i seguenti punti:
\begin{itemize}
	\item implementazione RL\glosp nell'incremento 4;
	\item richiesta di colloquio con il prof. Cardin;.	
	\item pianificazione più dettagliata.
\end{itemize}

\section{Verbale}
\subsection{Punto 1}
Come pianificato, durante il secondo periodo del macro periodo di progettazione\glosp di dettaglio e codifica, è stato sviluppato l'incremento 4, cioè la continuazione, dallo scorso periodo, dell'implementazione dell'algoritmo di addestramento RL\glo. Quindi, dopo un'attenta verifica, la funzionalità richiesta è stata giudicata integrata correttamente.

\subsection{Punto 2}
Il gruppo ha discusso dell'architettura del prodotto\glosp, ma dopo la discussione sono rimasti dei dubbi sopratutto nell'impiego dei design pattern, quindi, dopo aver messo per iscritto la lista delle domande da porre, è stato deciso di richiedere un colloquio con il prof. Cardin per chiarire le problematiche sorte e poter di proseguire con la progettazione\glo.

\subsection{Punto 3}
È stato deciso di revisitare la pianificazione in modo da avere incrementi più dettagliati; in particolare nella pianificazione gli incrementi saranno corredati anche dai relativi requisiti al fine di avere un'integrazione migliore con le attività del gruppo e far sì che questi non siano più solo una parte di contorno dello sviluppo ma elementi che scandiscono precisamente il proseguimento dei lavori.
