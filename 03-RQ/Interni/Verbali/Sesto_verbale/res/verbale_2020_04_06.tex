\section{Informazioni generali}
    \subsection{Informazioni incontro}
        \begin{itemize}
            \item \textbf{Luogo}: Videochiamata tramite Skype;
            \item \textbf{Data}: 2020-04-06;
            \item \textbf{Ora d'inizio}: 09.00;
            \item \textbf{Ora di fine}: 10.00;
            \item \textbf{Partecipanti}: \begin{itemize}
                \item Corrizzato Vittorio;
                \item Dalla Libera Marco;
                \item Rampazzo Marco;
                \item Santagiuliana Vittorio;
                \item Schiavon Rebecca;
                \item Spreafico Alessandro;
                \item Toffoletto Massimo.
            \end{itemize}
        \end{itemize}
    \subsection{Argomenti trattati}
        Durante la riunione, i componenti hanno discusso riguardo i seguenti temi:
        \begin{itemize}
            \item implementazione incremento 14;
            \item gestione dei test sul codice;
            \item approfondimento dell'operazione di scrittura in InfluxDB da Grafana\glo.
        \end{itemize}
\section{Verbale}
    \subsection{Punto 1}
        È stato analizzato lo sviluppo dell'incremento 14 ed è risultata la corretta implementazione della funzionalità di associazione dei predittori al flusso dati.
    \subsection{Punto 2}
        È stata in seguito discussa la gestione dei test, che si dividono in test di unità e test di integrazione. Per quanto riguarda i primi, sono già in fase di sviluppo, ed in particolare l'applicazione di addestramento è già a buon punto, mentre per il plug-in la copertura risulta ancora non conforme alle metriche\glo. I test di integrazione invece richiedono approfondimento da parte di un componente del gruppo che andrà poi a condividere le conoscenze apprese ed implementare questi test. Il documento del \textit{Piano di Progetto} verrà inoltre aggiornato con i nuovi test sviluppati.
    \subsection{Punto 3}
        Infine è stato effettuato da un membro del gruppo un approfondimento per gli altri componenti mirato a comprendere meglio il processo\glosp di scrittura dei dati in InfluxDB. Queste indicazioni hanno portato alla decisione di adottare node-influx, un client InfluxDB da utilizzare nei browser e con Node.js .