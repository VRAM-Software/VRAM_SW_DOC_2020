\section{Informazioni generali}
    \subsection{Informazioni incontro}
        \begin{itemize}
            \item \textbf{Luogo}: Videochiamata tramite Skype;
            \item \textbf{Data}: 2020-04-06;
            \item \textbf{Ora d'inizio}: 09.00;
            \item \textbf{Ora di fine}: 10.00;
            \item \textbf{Partecipanti}: \begin{itemize}
                \item Corrizzato Vittorio;
                \item Dalla Libera Marco;
                \item Rampazzo Marco;
                \item Santagiuliana Vittorio;
                \item Schiavon Rebecca;
                \item Spreafico Alessandro;
                \item Toffoletto Massimo.
            \end{itemize}
        \end{itemize}
    \subsection{Argomenti trattati}
        Durante la riunione, i componenti hanno discusso riguardo l'organizzazione e lo svolgimento del V periodo della progettazione di dettaglio e codifica.
\section{Verbale}
    \subsection{Punto 1}
        Il responsabile ha suddiviso i compiti e li ha assegnati tramite l'\textit{issue tracking system} di GitHub:
        \begin{itemize}
            \item sviluppo l'incremento 14, comprensivo di funzionalità di associazione di fonti dati e predittori, elaborazione e visualizzazione dei dati;
            \item sesura del \textit{Manuale utente} e completamento del \textit{Manuale dello sviluppatore};
            \item stesura dell'allegato tecnico e preparazione dei diagrammi; 
            \item preparazione al colloquio della product baseline e composizione della presentazione.
        \end{itemize}
    \subsection{Punto 2}
        È stata in seguito discussa la gestione dei test, che si dividono in test di unità e test di integrazione. Per quanto riguarda i primi, sono già in fase di sviluppo, ed in particolare l'applicazione di addestramento è già a buon punto, mentre per il plug-in la copertura risulta scarsa. I test di integrazione invece richiedono approfondimento da parte di un componente del gruppo che andrà poi a condividere le conoscenze apprese ed implementare questi test. Il documento del \textit{Piano di progetto} verrà inoltre aggiornato con i nuovi test sviluppati.
