\section{Informazioni generali}
    \subsection{Informazioni incontro}
        \begin{itemize}
            \item \textbf{Luogo}: Videochiamata tramite Skype;
            \item \textbf{Data}: 2020-04-10;
            \item \textbf{Ora d'inizio}: 14.30;
            \item \textbf{Ora di fine}: 15.30;
            \item \textbf{Partecipanti}: \begin{itemize}
                \item Corrizzato Vittorio;
                \item Dalla Libera Marco;
                \item Rampazzo Marco;
                \item Santagiuliana Vittorio;
                \item Schiavon Rebecca;
                \item Spreafico Alessandro;
                \item Toffoletto Massimo.
            \end{itemize}
        \end{itemize}
    \subsection{Argomenti trattati}
        Durante la riunione, i componenti hanno discusso riguardo questi argomenti:
        \begin{itemize}
            \item implementazione incremento 16 (arresto addestramento);
            \item gestione test;
            \item discussione a seguito della product baseline.
        \end{itemize}
\section{Verbale}
    \subsection{Punto 1}
        La riunione è iniziata con la visione degli incarichi assegnati dal responsabile per arrivare al completamento dello sviluppo e chiusura dei documenti in vista della revisione di qualifica. Per quanto riguarda lo sviluppo, l'incremento 16, che prevede l'implementazione della funzionalità di arresto dell'addestramento, è stato completato. Verranno anche ultimati i test così da soddisfare il requisito di copertura. 
    \subsection{Punto 2}
        Si è deciso di introdurre un foglio di calcolo condiviso su Google Drive che conterrà titolo e breve descrizione per ogni test, così da avere una piattaforma più flessibile per facilitare il tracciamento dei nuovi test, che verranno quindi aggiunti al \textit{Piano di Progetto}.
    \subsection{Punto 3}
        È stato discusso l'esito della product baseline e spartiti i compiti per applicare i miglioramenti dedotti dalle segnalazioni del professor Cardin per la consegna della revisione di qualifica.
