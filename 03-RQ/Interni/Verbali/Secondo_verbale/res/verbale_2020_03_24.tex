\section{Informazioni generali}
    \subsection{Informazioni incontro}
        \begin{itemize}
            \item \textbf{Luogo}: Videochiamata tramite Zoom;
            \item \textbf{Data}: 2020-03-24;
            \item \textbf{Ora d'inizio}: 15.00;
            \item \textbf{Ora di fine}: 16.30;
            \item \textbf{Partecipanti}: \begin{itemize}
                \item Corrizzato Vittorio;
                \item Dalla Libera Marco;
                \item Rampazzo Marco;
                \item Santagiuliana Vittorio;
                \item Schiavon Rebecca;
                \item Spreafico Alessandro;
                \item Toffoletto Massimo.
            \end{itemize}
        \end{itemize}
    \subsection{Argomenti trattati}
        Durante la riunione, i componenti del gruppo si sono confrontati riguardo i seguenti punti:
		\begin{itemize}
			\item implementazione SVM\glosp nell'incremento 3;
			\item nuovo sistema di versionamento\glo;
			\item preventivi e consuntivi per i micro periodi.
		\end{itemize}
\section{Verbale}
    \subsection{Incremento 3}
    È stato deciso di sviluppare l'incremento 3, cioè l'implementazione dell'algoritmo di addestramento SVM\glo, durante il primo periodo del macro periodo di Progettazione di dettaglio e codifica.
    
    \subsection{Nuovo sistema di versionamento}
    Dopo la segnalazione in sede di RP, e un colloquio con il prof. Vardanega, il gruppo ha deciso di cambiare il sistema di versionamento\glosp in uso. Il nuovo sistema prevede un solo numero di versione unico per tutte le componenti del prodotto\glo; tale sistema è molto diverso da quello precedente, si rende quindi necessario uniformare il numero di versione del prodotto alla versione 7.0.0.
    
    \subsection{Preventivi e consuntivi per i micro periodi}
    Dopo la segnalazione in sede di RP, il gruppo ha deciso di creare dei micro periodi che corrispondono a un singolo incremento, ai fini di miglioramento adattivo della pianificazione residua. Ogni uno di essi è ha un suo preventivo e un suo consuntivo.

