\section{Informazioni generali}
    \subsection{Informazioni incontro}
        \begin{itemize}
            \item \textbf{Luogo}: Videochiamata tramite Zoom;
            \item \textbf{Data}: 2020-03-24;
            \item \textbf{Ora d'inizio}: 15.00;
            \item \textbf{Ora di fine}: 16.30;
            \item \textbf{Partecipanti}: \begin{itemize}
                \item Corrizzato Vittorio;
                \item Dalla Libera Marco;
                \item Rampazzo Marco;
                \item Santagiuliana Vittorio;
                \item Schiavon Rebecca;
                \item Spreafico Alessandro;
                \item Toffoletto Massimo.
            \end{itemize}
        \end{itemize}
    \subsection{Argomenti trattati}
        Durante la riunione, i componenti del gruppo si sono confrontati riguardo i seguenti punti:
		\begin{itemize}
			\item implementazione SVM\glosp nell'incremento 3;
			\item nuovo sistema di versionamento\glo;
			\item preventivi e consuntivi per i micro periodi.
		\end{itemize}
\section{Verbale}
    \subsection{Incremento 3}
    Come pianificato, durante il primo periodo del macro periodo di progettazione\glosp di dettaglio e codifica, è stato sviluppato l'incremento 3, cioè l'implementazione dell'algoritmo di addestramento SVM\glo ed una parziale integrazione dell'RL\glo, le funzionalità richieste sono state verificate e il suddetto incremento è stato valutato come completato dal gruppo.
    
    \subsection{Nuovo sistema di versionamento}
    Dopo la segnalazione in sede di RP, e un colloquio con il prof. Vardanega, il gruppo ha deciso di cambiare il sistema di versionamento\glosp in uso. Il nuovo sistema prevede un solo numero di versione unico per tutte le componenti del prodotto\glo; questo metodo è molto diverso da quello precedente, si rende quindi necessario uniformare il numero di versione del prodotto\glosp alla versione 7.0.0.
    
    \subsection{Preventivi e consuntivi per i micro periodi}
    Dopo la segnalazione in sede di RP, il gruppo ha deciso di migliorare il tracciamento dei costi e delle ore relative ad ogni micro periodo. A questo fine è stato deciso di aggiungere dei prospetti orari ed economici per ognuno di essi permettendo quindi un'analisi più trasparente e chiara delle risorse utilizzate.

