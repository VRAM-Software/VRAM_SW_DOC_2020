%File principale del documento su cui invocare la compilazione, vedi "istruzioni.txt" per più info

%Preambolo: la parte prima del \begin{document}
\documentclass[12pt,a4paper]{article} %formato del documento e grandezza caratteri

%Input del file metadata.tex della cartella locale "res/"
%lista di comandi presenti in template_latex.tex, da qui posso essere modificati secondo le esigenze

\newcommand{\DocTitle}{Piano di Progetto} %variabile usata dal file template_latex.tex per settare il titolo del documento
%\newcommand{\DocAuthor}{Progetto "Predire in Grafana"} %variabile usata dal file template_latex.tex per settare l'autore del documento
\newcommand{\DocDate}{08 gennaio 2020} %variabile usata dal file template_latex.tex; Impostata manualmente, altrimenti ad ogni compilazione viene messa la data del giorno di compilazione.
\newcommand{\DocDesc}{Piano di progetto del gruppo \textit{VRAM Software}} %variabile usata dal file template_latex.tex per settare la descrizione del documento
\newcommand{\ver}{1.1.1} %variabile usata dal file template_latex.tex per settare la versione del documento
\newcommand{\app}{Corrizzato Vittorio} %variabile usata dal file template_latex.tex per settare l'approvatore del documento
\newcommand{\red}{Corrizzato Vittorio \\ & Toffoletto Masismo \\ & Rampazzo Marco \\ & Santagiuliana Vittorio} %variabile usata dal file template_latex.tex per settare il redattore del documento
\newcommand{\test}{Santagiuliana Vittorio \\ & Spreafico Alessandro \\ & Schiavon Rebecca} %variabile usata dal file template_latex.tex per settare il verificatore del documento
\newcommand{\stat}{Approvato} %variabile usata dal file template_latex.tex per settare lo stato del documento
\newcommand{\use}{Esterno} %variabile usata dal file template_latex.tex per indicare l'uso del documento %Contiene le varibili che descrivono il documento

%Input di file di configurazione presi dalla cartella "Template-LaTeX/config/", uguali per tutti i documenti
%Attenzione bisogna impostare il percorso del file!
% Tutti i pacchetti usati, da inserire nel preambolo prima delle configurazioni

\usepackage[T1]{fontenc} %Permette la sillabazione su qualsiasi testo contenente caratteri
\usepackage[utf8]{inputenc} %Serve per usare la codifica utf-8
\usepackage[english,italian]{babel} %Imposta italiano lingua principale, inglese secondaria. Es. serve per far apparire "indice" al posto di "contents"

\usepackage{graphicx} %Serve per includere le immagini

\usepackage[hypertexnames=false]{hyperref} %Gestisce i riferimenti/link. Es. Serve per rendere clickabili le sezioni dell'indice

\usepackage{float} %Serve per migliore la definizione di oggetti fluttuanti come figure e tabelle. Es. poter usare l'opzione [H] nelle figure ovvero tenere fissate le immagini che altrimenti LaTeX si sposta a piacere.

\usepackage{listings} %Serve per poter mettere snippets di codice nel testo

\usepackage{lastpage} %Serve per poter introdurre un'etichetta a cui si può fare riferimento Es. piè di pagina; poter fare " \rfoot{\thepage\ di \pageref{LastPage}} "

\usepackage{fancyhdr} %Per header e piè di pagina personalizzati

%Sono alcuni package che potranno esserci utili in futuro
%\usepackage{charter}
%\usepackage{eurosym}
\usepackage{subcaption}
%\usepackage{wrapfig}
%\usepackage{background}
\usepackage{longtable} % tabella che può continuare per più di una pagina
\usepackage[table]{xcolor} % ho dovuto aggiungere table in modo da poter colorare le row della tabella, dava: undefined control sequences
%\usepackage{colortbl}

\usepackage{dirtree} % usato per creare strutte tree-view in stile filesystem
\usepackage{xspace} % usato per inserire caratteri spazio
\usepackage[official]{eurosym}
\usepackage{pdflscape} %Inclusione pacchetti
% Configurazioni varie, da inserire nel preambolo dopo i pacchetti

\hypersetup{hidelinks} %serve per nascondere riquadri rossi che circondano i link 

\lstset{literate= {à}{{\`a}}1 } %Permette di usare lettere accentate nei listings

\pagestyle{fancy} %Imposto stile pagina
\fancyhf{} %Reset, se lo tolgo LaTex mette impostazioni di default (p.es numerazione pagine di default)


\lhead{\includegraphics[scale=0.25]{img/logo_header.png}} %Left header che compare in ogni pagina
%\rhead{\leftmark} %Nome della top-level structure (p.es. Section in article o Chapter in book) in ogni pagina
\rhead{\DocTitle} %Right header

\newcommand{\glo}{$_G$} %Comando per aggiungere il pedice G
\newcommand{\glosp}{$_G$ } %Comando per aggiungere il pedice G con spazio

\newcommand\Tstrut{\rule{0pt}{2.6ex}} % top padding
\newcommand\Bstrut{\rule[-0.9ex]{0pt}{0pt}} % bottom padding
\newcommand{\TBstrut}{\Tstrut\Bstrut} % top & bottom padding

%Setto il colore dei link
%\hypersetup{
%	colorlinks,
%	linkcolor=[HTML]{404040},
%	citecolor={purple!50!black},
%	urlcolor={blue!50!black}
%}

%Tabelle e tabulazione (può tornare utile)
%\setlength{\tablcolsep}{10pt}
%\renewcommand{\arraystretch}{1.4}

%Comando per aggiungere le pagine di ogni sezione
%\newcommand{\newSection}[1]{%
%	\input{res/sections/#1}
%}

% Comandi per aggiungere padding a parole contenute nella tabella; è una specie di strut (un carattere invisibile)
%\newcommand\Tstrut{\rule{0pt}{2.6ex}} % top padding
%\newcommand\Bstrut{\rule[-0.9ex]{0pt}{0pt}} % bottom padding
%\newcommand{\TBstrut}{\Tstrut\Bstrut} % top & bottom padding  %Configurazione pacchetti

\begin{document}
	%Input del file "frontmatter" preso dalla cartella "Template-LaTeX/config/", uguale per tutti i documenti
	%Attenzione bisogna impostare il percorso del file!
	% #### FRONTESPIZIO (frontmatter) ####
\setlength{\headheight}{33pt} %Distanzia l'header
\pagenumbering{gobble} %Toglie il numero di pagina
\begin{titlepage}
	\begin{center}
		\vspace*{-2cm}
		\includegraphics[scale=0.6]{img/logo.png} \\ %Logo
		\vspace{0.4cm} %Aggiunge uno spazio verticale di 0.5 cm
		
		{\LARGE Progetto "Predire in Grafana"} \\ %Nome progetto
		\vspace{0.4cm} %Attenzione a mettere il punto e NON la virgola
		
		{\Huge \textbf{\DocTitle}} \\ %Titolo, prende variabile definita in metadata.tex
		\vspace{0.4cm}
		
		\DocDate \\ %Data, prende variabile definita in metadata.tex
		\vspace{0.4cm}
		
		%Allineamento colonne: l=left r=right c=center, 
		%va specificato per ogni colonna
		%Se si vuole la riga tra colonne mettere "|"
		
		\begin{tabular}{r | l} %Elementi colonne separate da "&", le righe finiscono con "\\"
			Versione             & \ver \\
			Approvazione         & \app \\ 
			Redazione            & \red \\
			Verifica             & \test \\
			Stato                & \stat \\
			Uso                  & \use \\
		    Destinato a          & Zucchetti \\
						         & Prof. Vardanega Tullio \\
						         & Prof. Cardin Riccardo \\
			Email di riferimento & vram.software@gmail.com
		\end{tabular}
		\vfill
		\textbf{Descrizione} \\
		\DocDesc
	\end{center}
\end{titlepage}
\clearpage

% #### Impostazione header, footer  e numerazione pagine ####
\pagenumbering{arabic} %Pagine con i numeri arabi + reset a 1
\renewcommand{\footrulewidth}{0.4pt} %Di default footrulewidth==0 e quindi è invisibile, di default \headrulewith==0.4pt
\rfoot{\thepage\ di \pageref{LastPage}} %Pagina n di m, con numeri Arabi; usa il pacchetto "lastpage", in caso non sia possibile usare tale pacchetto mettere al fondo dell'ultima pagina "\label{LastPage}"

% #### Tabella dei log ####
% \textbf = grassetto; \Large = font più grande
% \rowcolors{quanti colori alternare}{colore numero riga pari}{colore numero riga dispari}: colori alternati per riga
% \rowcolor{color}: cambia colore di una riga
% p{larghezza colonna}: p è un tipo di colonna di testo verticalmente allineata sopra, ci sarebbe anche m che è centrata a metà ma non è precisa per questo utilizzo TBStrut; la sintassi >{\centering} indica che il contenuto della colonna dovrà essere centrato
% \TBstrut fa parte di alcuni comandi che ho inserito in config.tex che permetto di aggiungere un po' di padding al testo
% \\ [2mm] : questra scrittura indica che lo spazio dopo una break line deve essere di 2mm
% 

%\setcounter{secnumdepth}{0}
%\hfill \break
%\textbf{\Large{Diario delle modifiche}} \\


\addtocontents{toc}{\protect\setcounter{tocdepth}{0}} %Inserire questo per escludere una sezione dall'indice.

\section*{Registro delle modifiche} %Asterisco per fare sezione non numerata
\rowcolors{2}{gray!25}{gray!15}
\begin{longtable} {
		>{\centering}p{17mm} 
		>{\centering}p{19.5mm}
		>{\centering}p{24mm} 
		>{\centering}p{30mm} 
		>{}p{32mm}}
	\rowcolor{gray!50}
	\textbf{Versione} & \textbf{Data} & \textbf{Nominativo} & \textbf{Ruolo} & \textbf{Descrizione} \TBstrut \\
	4.1.1 & 2020-03-02 & Rampazzo Marco & \textit{Responsabile di progetto} & Approvazione documento. \TBstrut \\ [2mm]
	3.2.4 & 2020-03-01 & Spreafico Alessandro e Santagiuliana Vittorio & \textit{Amministratore} e \textit{Verificatore} & Aggiornamento e verifica dei paragrafi §2.1.5, §2.2.5, §3.2.5, §3.3.5 e §3.4.6. \TBstrut \\ [2mm]
	3.2.3 & 2020-02-28 & Spreafico Alessandro e Schiavon Rebecca & \textit{Amministratore} e \textit{Verificatore} & Stesura e verifica dei paragrafi §B e §C. \TBstrut \\ [2mm]
	3.1.2 & 2020-02-20 & Dalla Libera Marco e Santagiuliana Vittorio & \textit{Amministratore} e \textit{Verificatore} & Aggiornamento e verifica dei paragrafi §2.1.5, §2.2.5, §3.2.5, §3.3.5 e §3.4.6. \TBstrut \\ [2mm]
	3.1.1 & 2020-02-16 & Rampazzo Marco & \textit{Responsabile di progetto} & Approvazione documento. \TBstrut \\ [2mm]
	2.3.2 & 2020-02-14 & Spreafico Alessandro e Toffoletto Massimo & \textit{Amministratore} e \textit{Verificatore} & Aggiornamento e verifica dei paragrafi §2.1.5, §2.2.5, §3.2.5, §3.3.5 e §3.4.6. \TBstrut \\ [2mm]
	2.2.2 & 2020-02-12 & Spreafico Alessandro e Toffoletto Massimo & \textit{Amministratore} e \textit{Verificatore} & Stesura e verifica dei paragrafi §3.3, §3.4.5, §3.6.5 e §4.1.4. \TBstrut \\ [2mm]
	2.1.1 & 2020-02-08 & Rampazzo Marco & \textit{Responsabile di progetto} & Approvazione documento. \TBstrut \\ [2mm]
	1.4.4 & 2020-02-06 & Dalla Libera Marco e Santagiuliana Vittorio & \textit{Amministratore} e \textit{Verificatore} & Aggiornamento e verifica dei paragrafi §2.1.5, §2.2.5, §3.2.5, §3.3.5 e §3.4.6. \TBstrut \\ [2mm]
	1.3.3 & 2020-02-04 & Dalla Libera Marco e Schiavon Rebecca & \textit{Amministratore} e \textit{Verificatore} & Stesura e verifica dei paragrafi §2.2.4.3 e §3.6. \TBstrut \\ [2mm]
	1.2.2 & 2020-02-02 & Dalla Libera Marco e Corrizzato Vittorio & \textit{Amministratore} e \textit{Verificatore} & Correzione della struttura del documento, dei riferimenti, dello stile tipografico, del registro delle modifiche e associazione più precisa delle metriche ai relativi processi tutto come segnalato dal committente. \TBstrut \\ [2mm]
	1.1.1 & 2020-01-11 & Corrizzato Vittorio & \textit{Responsabile di progetto} & Approvazione documento. \TBstrut \\ [2mm]
	0.8.4 & 2020-01-11 & Santagiuliana Vittorio e Schiavon Rebecca & \textit{Amministratore} e \textit{Verificatore} & Aggiornamento e verifica finale. \TBstrut \\ [2mm]
	0.8.3 & 2020-01-06 & Santagiuliana Vittorio e Schiavon Rebecca & \textit{Amministratore} e \textit{Verificatore} & Aggiornamento e verifica dei paragrafi §2.1.4.2, §2.1.4.3, §4.1.6 e §A. \TBstrut \\ [2mm]
	0.7.3 & 2020-01-02 & Rampazzo Marco e Santagiuliana Vittorio & \textit{Amministratore} e \textit{Verificatore} & Aggiornamento e verifica dei paragrafi §3.3, §3.3.6 e §3.4.5. \TBstrut \\ [2mm]
	0.6.2 & 2020-01-02 & Rampazzo Marco e Santagiuliana Vittorio & \textit{Amministratore} e \textit{Verificatore} & Aggiornamento e verifica dei paragrafi §2.2.4.1 e §2.1.4.3. \TBstrut \\ [2mm]
	0.5.2 & 2019-12-23 & Rampazzo Marco e Spreafico Alessandro & \textit{Amministratore} e \textit{Verificatore} & Stesura e verifica dei paragrafi §2.2.4.1, §2.1.4.2, §3.1.6 e §3.2.5. \TBstrut \\ [2mm]
	0.4.1 & 2019-12-04 & Toffoletto Massimo e Corrizzato Vittorio & \textit{Amministratore} e \textit{Verificatore} & Aggiornamento e verifica dei paragrafi §2.1.4.1 e §2.2.5. \TBstrut \\ [2mm]
	0.3.1 & 2019-11-23 & Corrizzato Vittorio e Schiavon Rebecca & \textit{Amministratore} e \textit{Verificatore} & Stesura e verifica dei paragrafi §3.4, §3.5 e §4. \TBstrut \\ [2mm]
	0.2.0 & 2019-11-23 & Corrizzato Vittorio e Schiavon Rebecca & \textit{Amministratore} e \textit{Verificatore} & Stesura e verifica dei paragrafi §3.1 e §3.2. \TBstrut \\ [2mm]
	0.1.0 & 2019-11-23 & Corrizzato Vittorio e Schiavon Rebecca & \textit{Amministratore} e \textit{Verificatore} & Stesura e verifica dei paragrafi §1, §2.1.4.1, §2.1.5 e §2.2.5. \TBstrut \\ [2mm]
\end{longtable}

\addtocontents{toc}{\protect\setcounter{tocdepth}{4}} %Inserire questo per ripristinare il normale inserimento delle sezioni nell'indice. 4 significa fino al paragrah
\clearpage

% #### INDICE (tableofcontents) ####
\tableofcontents %Provoca la stampa dell'indice
\clearpage

\setcounter{secnumdepth}{4} %Permette di andare fino alla profondità del paragraph con la numerazione delle sezioni %Imposta il frontespizio, l'indice, header e footer	
	
	
	%Tutte le sezioni del documento
	%\input{res/inserire nome sezione 1} 
	% ...
	%\input{res/inserire nome sezione n} 
	%\section{Requisiti funzionali}
	\rowcolors{2}{gray!25}{gray!15}
	\begin{longtable} {
		>{}p{24mm} 
		>{}p{32mm}
		>{}p{40mm} 
		>{}p{24.5mm}
		}
	\rowcolor{gray!50}
		\textbf{Requisito} & \textbf{Classificazione} & \textbf{Descrizione} & \textbf{Fonti} 	\TBstrut \\
		R3F1 & Opzionale & L'utente deve poter addestrare gli algoritmi di predizione su Grafana\glo & Capitolato UC1 \TBstrut \\ [2mm]
		R3F1.1 & Opzionale & L'utente deve poter selezionare un file JSON contente i dati di testing per l'addestramento & Interno UC1.1 \TBstrut \\ [2mm]
		R3F1.1.1 & Opzionale & L'utente deve poter selezionare il file JSON & Interno UC1.1.1 \TBstrut \\ [2mm]
		R3F1.1.2 & Opzionale & L'utente deve poter caricare il file JSON & Interno UC1.1.2 \TBstrut \\ [2mm]
		R3F1.2 & Opzionale & L'utente deve poter scegliere il modello di predizione & Capitolato UC1.2 \TBstrut \\ [2mm]
		R3F1.3 & Opzionale & L'utente deve poter avviare l'addestramento dell'algoritmo & Interno UC1.3 \TBstrut \\ [2mm]
		R3F1.4 & Opzionale & L'utente deve poter chiudere l'addestramento dell'algoritmo & Interno UC1.4 \TBstrut \\ [2mm]
		R3F1.4.1 & Opzionale & Se l'addestramento va a buon fine l'utente deve visualizzare un messaggio di conferma & Interno UC1.4.1 \TBstrut \\ [2mm]		
		R3F2 & Opzionale & L'utente deve visualizzare l'indice della qualità delle previsioni & Capitolato UC2 \TBstrut \\ [2mm]
		R3F3 & Opzionale & Se l'utente inserisce un file JSON non valido deve visualizzare un messaggio di errore & Interno UC3 \TBstrut \\ [2mm]		
		R1F4 & Obbligatorio & L'utente deve poter addestrare gli algoritmi di previsione su un'applicazione esterna & Capitolato UC4 \TBstrut \\ [2mm]		
		R1F4.1 & Obbligatorio & L'utente deve poter inserire un file JSON contente i dati di testing per l'addestramento & Interno UC4.1 \TBstrut \\ [2mm]		
		R1F4.1.1 & Obbligatorio & L'utente deve poter selezionare un file JSON contente i dati di testing per l'addestramento & Interno UC4.1.1 \TBstrut \\ [2mm]
		R1F4.1.2 & Obbligatorio & L'utente deve poter caricare un file JSON contente i dati di testing per l'addestramento & Interno UC4.1.2 \TBstrut \\ [2mm]		
		R1F4.2 & Obbligatorio & L'utente deve poter scegliere il modello di predizione tra SVM\glosp e RL\glosp su cui applicare l'addestramento & Capitolato UC4.2 \TBstrut \\ [2mm]
		R3F4.3 & Opzionale & L'utente deve poter scegliere il modello di predizione da altri metodi tra cui la versione delle SVM adattate alla Regressione, piccole Reti Neurali per la classificazione e regressioni esponenziali o logaritmiche. & Capitolato UC4.2 \TBstrut \\ [2mm]				
		R1F4.4 & Obbligatorio & L'utente deve poter avviare l'addestramento dell'algoritmo & Interno UC4.3 \TBstrut \\ [2mm]
		R1F4.5 & Obbligatorio & L'utente deve poter chiudere l'addestramento dell'algoritmo & Interno UC4.4 \TBstrut \\ [2mm]		
		R1F4.5.1 & Obbligatorio & Se l'addestramento va a buon fine l'utente deve visualizzare un messaggio di conferma & Interno UC4.4.1 \TBstrut \\ [2mm]		
		R1F4.5.2 & Obbligatorio & L'utente deve riceve il file JSON con i parametri per le previsioni & Capitolato UC4.4.2 \TBstrut \\ [2mm]
		R1F5 & Obbligatorio & L'utente deve visualizzare il grafico della qualità delle previsioni & Capitolato UC5 \TBstrut \\ [2mm]
		R2F6 & Desiderabile & Se l'utente inserisce un file JSON non valido deve visualizzare un messaggio di errore & Interno UC6 \TBstrut \\ [2mm]
	\end{longtable}
	%\section{Requisiti funzionali}
	\rowcolors{2}{gray!25}{gray!15}
	\begin{longtable} {
		>{}p{24mm} 
		>{}p{32mm}
		>{}p{40mm} 
		>{}p{24.5mm}
		}
	\rowcolor{gray!50}
		\textbf{Requisito} & \textbf{Classificazione} & \textbf{Descrizione} & \textbf{Fonti} 	\TBstrut \\
		R1F7 & Obbligatorio & L'utente deve poter creare una nuova dashboard\glosp su cui aggiungere i pannelli & Capitolato UC7 \TBstrut \\ [2mm]
		R1F8 & Obbligatorio & L'utente deve poter aggiungere un nuovo pannello su cui visualizzare i grafici dei propri dati & Capitolato UC8 \TBstrut \\ [2mm]
		R1F8.1 & Obbligatorio & L'utente deve poter creare un nuovo pannello vuoto & Interno UC8.1 \TBstrut \\ [2mm]
		R1F8.2 & Obbligatorio & L'utente deve poter impostare le interrogazioni da eseguire sui dati da analizzare per prelevare solo ciò che necessita & Interno UC8.2 \TBstrut \\ [2mm]
		R1F8.2.1 & Obbligatorio & L'utente deve poter selezionare la data source su cui effettuare le previsioni & Interno UC8.2.1 \TBstrut \\ [2mm]
		R1F8.2.2 & Obbligatorio & L'utente deve poter selezionare i predittori per eseguire le previsioni & Interno UC8.2.2 \TBstrut \\ [2mm]
		R1F8.2.3 & Obbligatorio & L'utente deve poter impostare i livelli di soglia sui dati, oltre i quali si possono generare degli allarmi & Interno UC8.2.3 \TBstrut \\ [2mm]
		R1F8.3 & Obbligatorio & L'utente deve poter impostare il grafico che preferisce all'interno del suo pannello & Capitolato UC8.3 \TBstrut \\ [2mm]
		R1F8.3.1 & Obbligatorio & L'utente deve poter impostare il tipo di grafico per il pannello & Interno UC8.3.1 \TBstrut \\ [2mm]
		R1F8.3.2 & Obbligatorio & L'utente deve poter impostare i parametri del grafico per una visualizzazione adatta alle sue necessità & Interno UC8.3.2 \TBstrut \\ [2mm]
		R1F9 & Obbligatorio & L'utente deve poter modificare un pannello esistente & Interno UC9 \TBstrut \\ [2mm]
		R1F9.1 & Obbligatorio & L'utente deve poter modificare le impostazioni sulle interrogazioni precedentemente definite & Interno UC9.1 \TBstrut \\ [2mm]
		R1F9.1.1 & Obbligatorio & L'utente deve poter modificare la data source su cui effettuare le previsioni & Interno UC9.1.1 \TBstrut \\ [2mm]
		R1F9.1.2 & Obbligatorio & L'utente deve poter modificare i predittori selezionati per eseguire le previsioni & Interno UC9.1.2 \TBstrut \\ [2mm]
		R1F9.1.3 & Obbligatorio & L'utente deve poter modificare i livelli di soglia sui dati precedentemente impostati & Interno UC9.1.3 \TBstrut \\ [2mm]
		R1F9.2 & Obbligatorio & L'utente deve poter cambiare il grafico per la visualizzazione dei dati all'interno del pannello & Interno UC9.2 \TBstrut \\ [2mm]
		R1F9.2.1 & Obbligatorio & L'utente deve poter modificare il tipo di grafico per il pannello & Interno UC9.2.1 \TBstrut \\ [2mm]
		R1F9.2.2 & Obbligatorio & L'utente deve poter modificare i parametri del grafico precedemente impostati & Interno UC9.2.2 \TBstrut \\ [2mm]
		R2F10 & Desiderabile & Deve essere visualizzato un errore se l'utente inserisce un livello di soglia non valido & Interno UC10 \TBstrut \\ [2mm]
		R2F11 & Desiderabile & Deve essere visualizzato un messaggio di errore se l'utente imposta un parametro di un grafico con un valore non valido & Interno UC11 \TBstrut \\ [2mm]
		R1F12 & Obbligatorio & L'utente deve poter eliminare un pannello qualora non lo necessiti più & Interno UC12 \TBstrut \\ [2mm]
		R1F13 & Obbligatorio & L'utente deve poter impostare il plug-in da utilizzare & Capitolato UC13 \TBstrut \\ [2mm]
		R1F13.1 & Obbligatorio & L'utente deve poter visualizzare la lista dei plug-in disponibili & Interno UC13.1 \TBstrut \\ [2mm]
        R1F13.2 & Obbligatorio & L'utente deve poter selezionare il plug-in da utilizzare & Interno UC13.2 \TBstrut \\ [2mm]
		R1F13.3 & Obbligatorio & L'utente deve poter selezionare l'istanza di Grafana a cui associare il plug-in e lavorare & Interno UC13.3 \TBstrut \\ [2mm]
		R3F14 & Opzionale & Il sistema deve fornisce degli indici di bontà delle previsioni prodotte e visualizzarli su dei grafici & Capitolato UC14 \TBstrut \\ [2mm]
	\end{longtable}
	%\section{Requisiti funzionali}
	\rowcolors{2}{gray!25}{gray!15}
	\begin{longtable} {
		>{}p{24mm} 
		>{}p{32mm}
		>{}p{40mm} 
		>{}p{24.5mm}
		}
	\rowcolor{gray!50}
		\textbf{Requisito} & 
		\textbf{Classificazione} & 
		\textbf{Descrizione} & 
		\textbf{Fonti} 	\TBstrut \\
		
		R2F13 &	Desiderabile & L'utente deve poter definire un alert nel pannello grafico di una dashboard\glo & Capitolato UC13 \TBstrut \\ [2mm]			
		R2F13.1 & Desiderabile & L'utente deve poter inserire un alert & Capitolato UC13.1 \TBstrut \\ [2mm]		
		R2F13.2 & Desiderabile & L'utente deve poter definire le regole di funzionamento di un alert\glo & Interno UC13.2 \TBstrut \\ [2mm]		
		R2F13.3 & Desiderabile & L'utente deve poter definire le condizioni di funzionamento di un alert\glo & Interno UC13.3 \TBstrut \\ [2mm]
		R2F13.4 & Desiderabile & L'utente deve poter definire alcuni comportamenti speciali di un alert\glo come assenza di dati & Interno UC13.4 \TBstrut \\ [2mm]		
		R2F14 &	Desiderabile & Se l'utente inserisce un input errato nella definizione di un alert\glosp deve visualizzare un messaggio di errore &	Interno UC14 \TBstrut \\ [2mm]
		R2F15 &	Desiderabile & L'utente deve poter sospendere un alert\glo & Interno UC15 \TBstrut \\ [2mm]		
		R2F16 & Desiderabile & L'utente deve poter rimuovere un alert\glo & Interno UC16 \TBstrut \\ [2mm]		
	\end{longtable}

	\section{Requisiti}
Per descrivere un requisito viene utilizzata la seguente struttura:
\begin{itemize}
	\item codice identificativo;
	\item classificazione;
	\item descrizione;
	\item fonti.
\end{itemize} 
Il \textbf{Codice identificativo} sarà scritto in questo formato: \\
\textbf{R[Importanza][Tipologia][Codice]} \\
Dove:
\begin{itemize}
	\item \textbf{Importanza} può assumere i seguenti valori:
	\begin{itemize}
		\item 1: requisito obbligatorio;
		\item 2: requisito desiderabile;
		\item 3: requisito opzionale.
	\end{itemize}
	\item \textbf{Tipologia} può assumere i seguenti valori:
	\begin{itemize}
		\item F: funzionale;
		\item Q: prestazionale;
		\item P: qualitativo;
		\item V: vincolo.
	\end{itemize}
	\item\textbf{Codice}: numero progressivo identificativo strutturato nel formato: [codice\_padre].[codice\_figlio]
\end{itemize}
Le \textbf{Fonti} possono essere:
\begin{itemize}
	\item capitolato\glo: il requisito è stato quindi individuato dalla lettura del capitolato\glo;
	\item interno: il requisito è stato individuato ed aggiunto in seguito ad un'analisi interna;
	\item caso d'uso\glo: il requisito è stato individuato dallo studio di un caso d'uso\glo;
	\item proponente: il requisito è stato individuato in seguito ad un colloquio con il proponente.
\end{itemize}
\subsection{Requisiti funzionali}
	\rowcolors{2}{gray!25}{gray!15}
	\setcounter{table}{0}
	\begin{longtable} {
		>{}p{24mm} 
		>{}p{32mm}
		>{}p{40mm} 
		>{}p{24.5mm}
		}
	\rowcolor{gray!50}
		\textbf{Requisito} & \textbf{Classificazione} & \textbf{Descrizione} & \textbf{Fonti} 	\TBstrut \\
		R3F1 & Opzionale & L'utente deve poter addestrare gli algoritmi di predizione su Grafana\glo & Capitolato UC1 \TBstrut \\ [2mm]
		R3F1.1 & Opzionale & L'utente deve poter inserire un file in formato CSV contente i dati per l'addestramento di un'algoritmo di predizione & Interno UC1.1 \TBstrut \\ [2mm]
		R3F1.1.1 & Opzionale & L'utente deve poter selezionare il file in formato CSV dei dati per l'addestramento & Interno \TBstrut \\ [2mm]
		R3F1.1.2 & Opzionale & L'utente deve poter caricare il file in formato CSV dei dati per l'addestramento & Interno \TBstrut \\ [2mm]
		R3F1.5 & Opzionale & L'utente deve poter visualizzare il grafico a dispersione che contiene i dati per l'addestramento & Proponente UC1.5 \TBstrut \\ [2mm]
		R3F1.6 & Opzionale & L'utente deve poter inserire il file in formato JSON che contiene la configurazione di un addestramento eseguito precedentemente & Capitolato UC1.6 \TBstrut \\ [2mm]
		R3F1.6.1 & Opzionale & L'utente deve poter selezionare il file formato JSON di una configurazione precedente & Interno \TBstrut \\ [2mm]
		R3F1.6.2 & Opzionale & L'utente deve poter caricare il file formato JSON di una configurazione precedente & Interno \TBstrut \\ [2mm]
		R3F1.7 & Opzionale & L'utente deve poter inserire delle note che verranno scritte nel file JSON risultante dall'addestramento & Proponente UC1.7 \TBstrut \\ [2mm]
		R3F1.2 & Opzionale & L'utente deve poter scegliere il modello di predizione su cui eseguire l'addestramento & Capitolato UC1.2 \TBstrut \\ [2mm]
		R3F1.9 & Opzionale & L'utente deve poter scegliere il modello di predizione SVM\glosp su cui eseguire l'addestramento & Capitolato UC1.9 \TBstrut \\ [2mm]
		R3F1.10 & Opzionale & L'utente deve poter scegliere il modello di predizione RL\glosp su cui eseguire l'addestramento & Capitolato UC1.10 \TBstrut \\ [2mm]
		R3F1.11 & Opzionale & L'utente deve poter scegliere il modello di predizione reti neurali\glosp su cui eseguire l'addestramento & Capitolato UC1.11 \TBstrut \\
		R3F1.12 & Opzionale & L'utente deve poter scegliere il modello di predizione regressioni esponenziali su cui eseguire l'addestramento & Capitolato UC1.12 \TBstrut \\
		R3F1.13 & Opzionale & L'utente deve poter scegliere il modello di predizione regressioni logaritmiche su cui eseguire l'addestramento & Capitolato UC1.13 \TBstrut \\
		R3F1.14 & Opzionale & L'utente deve poter scegliere il modello di predizione SVM\glosp adattata alla Regressione su cui eseguire l'addestramento & Capitolato UC1.14 \TBstrut \\
		R3F1.3 & Opzionale & L'utente deve poter avviare l'addestramento dell'algoritmo di predizione & Interno UC1.3 \TBstrut \\ [2mm]
		R3F1.4 & Opzionale & L'utente deve poter addestrare l'addestramento dell'algoritmo di predizione & Interno UC1.4 \TBstrut \\ [2mm]
		R3F1.8 & Opzionale & Se l'addestramento va a buon fine, l'utente deve visualizzare un messaggio di conferma & Interno UC1.8 \TBstrut \\ [2mm]		
		R3F2 & Opzionale & L'utente deve visualizzare gli indici di qualità delle previsioni sul plug-in interno a Grafana\glo & Capitolato UC2 \TBstrut \\ [2mm]
		R3F2.1 & Opzionale & L'utente deve visualizzare l'indice di qualità delle previsioni R$^{2}$\glo & Capitolato UC2.1 \TBstrut \\ [2mm]
		R3F2.2 & Opzionale & L'utente deve visualizzare gli indici di qualità delle previsioni Precision\glo & Capitolato UC2.2 \TBstrut \\ [2mm]
		R3F2.3 & Opzionale & L'utente deve visualizzare gli indici di qualità delle previsioni Recall\glo & Capitolato UC2.3 \TBstrut \\ [2mm]
		R3F3 & Opzionale & Se l'utente inserisce un file formato CSV non valido nel plug-in per l'addestramento interno, deve visualizzare un messaggio di errore & Interno UC3 \TBstrut \\ [2mm]
		R3F17 & Opzionale & Se l'utente inserisce un file formato JSON non valido nel plug-in per l'addestramento interno, deve visualizzare un messaggio di errore & Interno UC17 \TBstrut \\ [2mm]	
		R1F4 & Obbligatorio & L'utente deve poter addestrare gli algoritmi di previsione su un'applicazione esterna & Capitolato UC4 \TBstrut \\ [2mm]		
		R1F4.1 & Obbligatorio & L'utente deve poter inserire un file in formato CSV contente i dati per l'addestramento dell'algoritmo di predizione & Interno UC4.1 \TBstrut \\ [2mm]		
		R1F4.1.1 & Obbligatorio & L'utente deve poter selezionare un file CSV contente i dati per l'addestramento & Interno \TBstrut \\ [2mm]
		R1F4.1.2 & Obbligatorio & L'utente deve poter caricare un file CSV contente i dati per l'addestramento & Interno \TBstrut \\ [2mm]
		R1F4.8 & Obbligatorio & L'utente deve poter visualizzare il grafico a dispersione che contiene i dati per l'addestramento & Proponente UC4.6 \TBstrut \\ [2mm]
		R1F4.9 & Obbligatorio & L'utente deve poter inserire il file in formato JSON che contiene la configurazione di un addestramento eseguito precedentemente & Capitolato UC4.7 \TBstrut \\ [2mm]
		R1F4.9.1 & Obbligatorio & L'utente deve poter selezionare il file formato JSON di una configurazione precedente & Interno \TBstrut \\ [2mm]
		R1F4.9.2 & Obbligatorio & L'utente deve poter caricare il file formato JSON di una configurazione precedente & Interno \TBstrut \\ [2mm]
		R1F4.10 & Obbligatorio & L'utente deve poter inserire delle note che verranno scritte nel file JSON risultante dall'addestramento & Proponente UC4.8 \TBstrut \\ [2mm]	
		R1F4.2 & Obbligatorio & L'utente deve poter scegliere il modello di predizione su cui applicare l'addestramento & Capitolato UC4.2 \TBstrut \\ [2mm]
		R1F4.11 & Obbligatorio & L'utente deve poter scegliere il modello di predizione SVM\glosp su cui eseguire l'addestramento & Capitolato UC4.10 \TBstrut \\ [2mm]
		R1F4.12 & Obbligatorio & L'utente deve poter scegliere il modello di predizione RL\glosp su cui eseguire l'addestramento & Capitolato UC4.11 \TBstrut \\ [2mm]
		R3F4.13 & Opzionale & L'utente deve poter scegliere il modello di predizione reti neurali\glosp su cui eseguire l'addestramento & Capitolato UC4.12 \TBstrut \\
		R3F4.14 & Opzionale & L'utente deve poter scegliere il modello di predizione regressioni esponenziali su cui eseguire l'addestramento & Capitolato UC4.13 \TBstrut \\
		R3F4.15 & Opzionale & L'utente deve poter scegliere il modello di predizione regressioni logaritmiche su cui eseguire l'addestramento & Capitolato UC4.14 \TBstrut \\
		R3F4.16 & Opzionale & L'utente deve poter scegliere il modello di predizione SVM\glosp adattata alla Regressione su cui eseguire l'addestramento & Capitolato UC4.15 \TBstrut \\				
		R1F4.4 & Obbligatorio & L'utente deve poter avviare l'addestramento dell'algoritmo di predizione & Interno UC4.3 \TBstrut \\ [2mm]
		R1F4.5 & Obbligatorio & L'utente deve poter chiudere l'addestramento dell'algoritmo di predizione & Interno UC4.4 \TBstrut \\ [2mm]		
		R2F4.6 & Desiderabile & Se l'addestramento va a buon fine, l'utente deve visualizzare un messaggio di conferma & Interno UC4.9 \TBstrut \\ [2mm]		
		R1F4.7 & Obbligatorio & L'utente deve ricevere il file JSON con i predittori\glosp per eseguire le previsioni & Capitolato UC4.5 \TBstrut \\ [2mm]
		R1F5 & Obbligatorio & L'utente deve poter visualizzare gli indici qualità delle previsioni eseguite sull'applicazione esterna & Capitolato UC5 \TBstrut \\ [2mm]
		R1F5.1 & Obbligatorio & L'utente deve poter visualizzare l'indice di qualità delle previsioni R$^{2}$\glo & Capitolato UC5.1 \TBstrut \\ [2mm]
		R1F5.2 & Obbligatorio & L'utente deve poter visualizzare gli indici di qualità delle previsioni Precision\glo & Capitolato UC5.2 \TBstrut \\ [2mm]
		R1F5.3 & Obbligatorio & L'utente deve poter visualizzare gli indici di qualità delle previsioni Recall\glo & Capitolato UC5.3 \TBstrut \\ [2mm]
		R2F6 & Desiderabile & Se l'utente inserisce un file CSV non valido deve visualizzare un messaggio di errore & Interno UC6 \TBstrut \\ [2mm]
		R2F18 & Desiderabile & Se l'utente inserisce un file JSON non valido deve visualizzare un messaggio di errore & Interno UC18 \TBstrut \\ [2mm]
		R1F19 & Obbligatorio & L'utente deve poter abilitare il plug-in & Interno UC19 \TBstrut \\ [2mm]
		R1F7 & Obbligatorio & L'utente deve poter avviare il plug-in & Capitolato UC7 \TBstrut \\ [2mm]
		R1F21 & Obbligatorio & L'utente deve poter visualizzare la dashboard\glosp fornita dal plug-in & Interno UC21 \TBstrut \\ [2mm]
		R1F8 & Obbligatorio & L'utente deve poter caricare all'interno del plug-in il file JSON contenente i dati risultanti dall'addestramento & Capitolato UC8 \TBstrut \\ [2mm]
		R1F9 & Obbligatorio & L'utente deve poter associare il predittore\glosp letto dal file JSON al flusso dati & Capitolato UC9 \TBstrut \\ [2mm]
		R1F9.1 & Obbligatorio & L'utente deve poter selezionare il predittore\glo & Capitolato UC9.1 \TBstrut \\ [2mm]
		R1F9.2 & Obbligatorio & L'utente deve poter selezionare un flusso dati statico su cui eseguire le previsioni & Capitolato UC9.2 \TBstrut \\ [2mm]
		R3F9.3 & Opzionale & L'utente deve poter selezionare un flusso dati continuo su cui eseguire le previsioni & Capitolato UC9.2 \TBstrut \\ [2mm]
		R1F9.4 & Obbligatorio & L'utente deve poter collegare il predittore\glosp scelto al flusso dati & Capitolato UC9.3 \TBstrut \\ [2mm]
		R2F9.5 & Desiderabile & Se il collegamento del predittore\glosp al flusso dati va a buon fine l'utente deve visualizzare un messaggio di conferma & Capitolato UC9.4 \TBstrut \\ [2mm]
		R2F10 & Desiderabile & Se il collegamento del predittore\glosp al flusso dati non va a buon fine l'utente deve visualizzare un messaggio di errore & Interno UC10 \TBstrut \\ [2mm]
		R1F11 & Obbligatorio & L'utente deve poter visualizzare i risultati della previsione sotto forma di grafici all'interno di una dashboard\glosp configurata & Capitolato \TBstrut \\ [2mm]
		R1F12 & Obbligatorio & L'utente deve poter rimuovere il pannello del plug-in dalla dashboard\glo & Capitolato UC12 \TBstrut \\ [2mm]	
		R1F20 & Obbligatorio & L'utente deve poter disabilitare il plug-in & Interno UC20 \TBstrut \\ [2mm]	
		R2F13 &	Desiderabile & L'utente deve poter definire un alert nel pannello grafico di una dashboard\glo & Capitolato UC13 \TBstrut \\ [2mm]					
		R2F13.2 & Desiderabile & L'utente deve poter definire le regole di funzionamento di un alert\glo & Interno UC13.2 \TBstrut \\ [2mm]		
		R2F13.3 & Desiderabile & L'utente deve poter definire le condizioni di funzionamento di un alert\glo & Interno UC13.3 \TBstrut \\ [2mm]
		R2F13.4 & Desiderabile & L'utente deve poter definire alcuni comportamenti di un alert\glosp da seguire in casi speciali come l'assenza di dati & Interno UC13.4 \TBstrut \\ [2mm]		
		R2F14 &	Desiderabile & Se l'utente inserisce un input errato nella definizione di un alert\glosp deve visualizzare un messaggio di errore & Interno UC14 \TBstrut \\ [2mm]
		R2F15 &	Desiderabile & L'utente deve poter sospendere un alert\glosp bloccandone temporaneamente l'esecuzione & Interno UC15 \TBstrut \\ [2mm]		
		R2F16 & Desiderabile & L'utente deve poter rimuovere un alert\glosp dal pannello grafico & Interno UC16 \TBstrut \\ [2mm]	
		\rowcolor{white}
		\caption{Requisiti funzionali}
	\end{longtable}

	
	
	
	%Input del file "decision_table" della cartella locale "res/"
	%% \textbf = grassetto; \Large = font più grande
% \rowcolors{quanti colori alternare}{colore numero riga pari}{colore numero riga dispari}: colori alternati per riga
% \rowcolor{color}: cambia colore di una riga
% p{larghezza colonna}: p è un tipo di colonna di testo verticalmente allineata sopra, ci sarebbe anche m che è centrata a metà ma non è precisa per questo utilizzo TBStrut; la sintassi >{\centering} indica che il contenuto della colonna dovrà essere centrato
% \TBstrut fa parte di alcuni comandi che ho inserito in config.tex che permetto di aggiungere un po' di padding al testo
% \\ [2mm] : questra scrittura indica che lo spazio dopo una break line deve essere di 2mm
% 
%\setcounter{secnumdepth}{0}
%\hfill \break
%\textbf{\Large{Diario delle modifiche}} \\
\section{Riepilogo delle decisioni}
\rowcolors{2}{gray!25}{gray!15}
\begin{longtable} {
		>{\centering}p{17mm} 
		%>{\centering}p{19.5mm}
		%>{\centering}p{24mm} 
		%>{\centering}p{24mm} 
		>{}p{120mm}}
	\rowcolor{gray!50}
	\textbf{Codice} & \multicolumn{1}{c}{\textbf{Decisione}} \\
	% \\ %\TBstrut \\
	VI\_17.1 & Sviluppo dell'incremento numero 8. \TBstrut \\ [2mm]
	VI\_17.2 & Rappresentazione indici di qualità in rettangoli. \TBstrut \\ [2mm]
	VI\_17.3 & Scelta della suddivisione dei dati 2/3 addestramento e 1/3 calcolo degli indici di qualità. \TBstrut \\ [2mm]
\end{longtable} %Tabella delle decisioni (solo per i verbali)
\end{document}