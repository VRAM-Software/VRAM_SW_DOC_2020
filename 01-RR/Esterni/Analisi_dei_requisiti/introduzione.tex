\documentclass[a4]{article}
\usepackage{hyperref}
\begin{document}
	\section{Introduzione}
		\subsection{Scopo del documento}
			Questo documento contiene una descrizione dettagliata di tutti i requisiti impliciti ed espliciti che il prodotto dovrà avere. Il nostro gruppo ha individuato questi requisiti dopo un'analisi delle esigenze e delle fonti realizzata mediante: 
				\begin{itemize}
					\item Un'analisi del capitolato C4 e della sua presentazione; 
					\item Uno studio di \textit{Grafana} e di \textit{InfluxDB};
					\item Da incontri con il proponente \textit{Zucchetti}.
				\end{itemize}
		\subsection{Scopo del prodotto}
			Il prodotto da sviluppare è costituito da un plug-in per \textit{Grafana} e da una applicazione per gestire i parametri di previsione (da aggiungere se sarà standalone o no). I dati inseriti da utente vengono mandati alla applicazione che genera i parametri necessari per la previsione. Questi ultimi verranno in seguito mandati al nostro plug-in e quindi applicati ai dati inseriti dall'utente che vengono a loro volta aggiunti al database InfluxDB. Mediante \textit{Grafana} può quindi essere creata una dashboard contente i grafici necessari per rappresentare i dati, viene inoltre data la possibilità di generare degli alert che avvisino l'utente in caso di anomalie.
		
		\subsection{Glossario}
			Per facilitare la lettura dei documenti si è deciso di introdurre il documento: \textit{Glossario 1.1.1}. che contiene una spiegazione delle parole che potrebbero essere considerate ambigue. Queste vengono segnalate con una \textit{G} maiuscola a pedice.  
		\subsection{Riferimenti}
			\subsubsection{Riferimenti normativi}
				\begin{itemize}
					\item \textbf{Materiale didattico:}
					\begin{itemize}
						\item \textbf{Analisi dei requisiti:} \textit{Lezione T8 e T9} \\*
							\url{https://www.math.unipd.it/~tullio/IS-1/2019/Dispense/L08.pdf};
						\item \textbf{Diagrammi dei casi d'uso:} \textit{Lezione E3} \\*
							\url{https://www.math.unipd.it/~tullio/IS-1/2019/Dispense/E03.pdf}.
					\end{itemize}
					\item \textbf{Verbali esterni:} 
					\begin{itemize}
						\item \textit{Verbale esterno 2019-12-18};
					\end{itemize}
					\item \textbf{Capitolato d'appalto:} \textit{C4 - Zucchetti - Predire in Grafana} \\*
						\url{https://www.math.unipd.it/~tullio/IS-1/2019/Progetto/C4.pdf};
					\item \textbf{Studi di fattibilità:} \textit{Studio di fattibilità 1.0.0};
					\item \textbf{Norme di progetto:} \textit{Norme di progetto 1.0.0}.
				\end{itemize}
			\subsubsection{Riferimenti informativi}
				\begin{itemize}
					\item \textbf{Presentazione capitolato d'appalto} \\*
						\url{https://www.math.unipd.it/~tullio/IS-1/2019/Dispense/C4a.pdf};
					\item \textbf{Grafana} \\*
						\url{https://grafana.com/docs/grafana/latest/};
					\item \textbf{InfluxDB} \\*
						\url{https://docs.influxdata.com/influxdb/v1.7/}.
		
				\end{itemize}
\end{document}