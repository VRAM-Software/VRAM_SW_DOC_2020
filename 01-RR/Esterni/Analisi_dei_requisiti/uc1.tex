\documentclass{article}
\begin{document}
	\subsection{UC1 - Addestramento degli algoritmi di predizione interno a Grafana}
	\begin{itemize}
		\item \textbf{Attore primario:} utente;
		\item \textbf{Attore secondario:} Grafana;
		\item \textbf{Descrizione:} questo use case descrive una delle funzionalità offerte dal plugin di addestrare degli algoritmi di predizione utilizzando dei dati inseriti da un utente;
		\item \textbf{Scenario principale:} 
			\begin{enumerate}
				\item \textbf{UC1.1:} viene fornito all'utente un modo per eseguire l'upload di un file JSON che contiene i dati che si vuole utilizzare per addestrare gli algoritmi di predizione;
				\item \textbf{UC1.2:} se l'utente ha inserito un file JSON il sistema fornisce all'utente una scelta, tra SVM e RL, su quale algoritmo utilizzare per effettuare l'addestramento;
				\item \textbf{UC1.3:} l'utente sceglie quale algoritmo di predizione utilizzare, in seguito il sistema fornisce all'utente la funzionalità di avviare l'addestramento;
				\item \textbf{UC1.4:} alla fine dell'addestramento, Grafana otterrà i parametri di predizioni che verranno utilizzati.
			\end{enumerate}
		\item \textbf{Precondizione:} l'utente è autenticato nel sistema software Grafana ed è presente una istanza di Grafana cloud o locale dove è installato il plugin;
		\item \textbf{Postcondizione:} viene fornito all'utente la possibilità di effettuare il caricamento di un file JSON.
	\end{itemize}

	\subsubsection{UC1.1 - Inserimento del file in formato JSON contenente i dati di testing}
	\begin{itemize}
		\item \textbf{Attore primario:} utente;
		\item \textbf{Attore secondario:} Grafana;
		\item \textbf{Descrizione:} Grafana fornisce all'utente un modo per eseguire l'upload di un file JSON che contiene dati di testing per l'addestramento degli algoritmi di predizione;
		\item \textbf{Scenario principale:} l'utente esegue il caricamento di un file JSON nel sistema;
		\item \textbf{Estensione:}
			\begin{itemize}
				\item \textbf{UC1.1.1:} il sistema da conferma all'utente del corretto caricamento del file JSON;
				\item \textbf{UC1.1.2:} il sistema avverte l'utente, attraverso un errore, che il caricamento del file JSON non è avvenuto con successo.
			\end{itemize}
		\item \textbf{Precondizione:} il plugin è installato nell'istanza di Grafana dell'utente;
		\item \textbf{Postcondizione:} il file JSON, inserito da utente, è stato caricato nel sistema.
	\end{itemize}

	\paragraph{UC1.1.1 - Successo inserimento file JSON}
	\begin{itemize}
		\item \textbf{Attore primario:} utente;
		\item \textbf{Attore secondario:} Grafana;
		\item \textbf{Descrizione:} il caricamento del file JSON è stato effettuato e il sistema fornisce all'utente una conferma;
		\item \textbf{Scenario principale:} viene fornito all'utente, dopo il caricamento del file JSON, un messaggio di conferma;
		\item \textbf{Precondizione:} l'utente ha selezionato e caricato il file JSON;
		\item \textbf{Postcondizione:} il sistema fornisce all'utente un messaggio di conferma che il file JSON è stato correttamente caricato all'interno del sistema.
	\end{itemize}

	\paragraph{UC1.1.2 - Errore caricamento file JSON}
	\begin{itemize}
		\item \textbf{Attore primario:} utente;
		\item \textbf{Attore secondario:} Grafana;
		\item \textbf{Descrizione:} ;
		\item \textbf{Scenario principale:} ;
		\item \textbf{Precondizione:} ;
		\item \textbf{Postcondizione:} .
	\end{itemize}

	\subsubsection{UC1.2 - Selezione del modello di predizione su cui applicare l'addestramento}
	\begin{itemize}
		\item \textbf{Attore primario:} utente;
		\item \textbf{Attore secondario:} Grafana;
		\item \textbf{Descrizione:} l'utente può decidere quale modello di predizione utilizzare per applicare l'addestramento tra la Regressione Lineare e la Support Vector Machine;
		\item \textbf{Scenario principale:} l'utente seleziona un algoritmo per eseguire l'addestramento;
		\item \textbf{Precondizione:} il file JSON inserito da utente è stato correttamente inserito nel sistema;
		\item \textbf{Postcondizione:} il modello di predizione è stato correttamente scelto e il sistema fornisce all'utente un modo per avviare l'addestramento.
	\end{itemize}

	\subsubsection{UC1.3 - Avvio dell'addestramento}
	\begin{itemize}
		\item \textbf{Attore primario:} utente;
		\item \textbf{Attore secondario:} Grafana;
		\item \textbf{Descrizione:} viene fornito all'utente un modo per avviare l'addestramento;
		\item \textbf{Scenario principale:} l'utente dopo aver scelto quale algoritmo di predizione utilizzare, inizia l'addestramento; 
		\item \textbf{Precondizione:} l'utente ha scelto il modello di predizione da utilizzare
		\item \textbf{Postcondizione:} l'addestramento è stato iniziato con successo.
	\end{itemize}

	\subsection{UC1.4 - Conclusione dell'addestramento}
	\begin{itemize}
		\item \textbf{Attore primario:} utente;
		\item \textbf{Attore secondario:} Grafana;
		\item \textbf{Descrizione:} il sistema ha completato l'addestramento, il risultato, di successo o insuccesso verrà fornito all'utente;
		\item \textbf{Scenario principale:} l'utente riceve un messaggio sull'esito dell'addestramento;
		% non so se queste estensioni devono essere considerate come "Scenari alternativi"
		\item \textbf{Estensioni:}
			\begin{itemize}
				\item \textbf{UC1.4.1:} l'addestramento è stato completato con successo;
				\item \textbf{UC1.4.2:} l'addestramento è fallito.
			\end{itemize}
		\item \textbf{Precondizione:} l'addestramento è stato avviato con successo;
		\item \textbf{Postcondizione:} l'addestramento è stato concluso.
	\end{itemize}
	
	\paragraph{UC1.4.1 - Ricezione del file JSON contenente i parametri per la previsione}
	\begin{itemize}
		\item \textbf{Attore primario:} utente;
		\item \textbf{Attore secondario:} Grafana;
		\item \textbf{Descrizione:} l'addestramento ;
		\item \textbf{Scenario principale:} l'utente riceve un messaggio che conferma il corretto svolgimento dell'addestramento e che è stato ricevuto il file JSON che contiene il risultato dell'addestramento;
		\item \textbf{Precondizione:} l'addestramento è stato completato;
		\item \textbf{Postcondizione:} il sistema riceve il file JSON fornito dal plugin dopo l'addestramento.
	\end{itemize}

	\paragraph{UC1.4.2 - Addestramento fallito}
	\begin{itemize}
		\item \textbf{Attore primario:} utente;
		\item \textbf{Attore secondario:} Grafana;
		\item \textbf{Descrizione:} in caso di errori nell'addestramento, viene fornito all'utente un messaggio di errore;
		\item \textbf{Scenario principale:} l'utente riceve un messaggio di errore esplicativo per l'insuccesso dell'addestramento;
		\item \textbf{Precondizione:} l'addestramento è stato completato;
		\item \textbf{Postcondizione:} viene visualizzato un messaggio di errore e non viene ricevuto nessun file JSON.
	\end{itemize}
\end{document}