\documentclass{article}
\begin{document}
	\subsection{UC1 - Addestramento degli algoritmi di predizione interno a Grafana}
	\begin{itemize}
		\item \textbf{Codice identificativo}: UC1;
		\item \textbf{Titolo}: Addestramento degli algoritmi di predizione interno a Grafana;
		\item \textbf{Attore primario:} utente;
		\item \textbf{Attore secondario:} Grafana\glo;
		\item \textbf{Descrizione:} questo use case descrive una delle funzionalità offerte dal plugin di addestrare degli algoritmi di predizione utilizzando dei dati inseriti da un utente;
		\item \textbf{Scenario principale:} 
			\begin{enumerate}
				\item L'utente esegue il caricamento di un file JSON che contiene i dati che si vuole utilizzare per addestrare gli algoritmi di predizione (UC1.1);
				\item L'utente sceglie tra SVM\glosp e RL\glosp, su quale algoritmo utilizzare per effettuare l'addestramento (UC1.2);
				\item L'utente inizia l'addestramento (UC1.3);
				\item Alla fine dell'addestramento, Grafana\glosp ottiene i parametri di predizioni e l'utente riceve una conferma della corretta ricezione di tali dati (UC1.4).
			\end{enumerate}
		\item \textbf{Precondizione:} l'utente è autenticato nel sistema software Grafana\glosp ed è presente una istanza di Grafana\glosp cloud\glosp o locale dove è installato il plugin;
		\item \textbf{Postcondizione:} viene fornito all'utente la possibilità di effettuare il caricamento di un file JSON.
	\end{itemize}

	\subsubsection{UC1.1 - Inserimento del file in formato JSON contenente i dati di testing}
	\begin{itemize}
		\item \textbf{Codice identificativo}: UC1.1;
		\item \textbf{Titolo}: Addestramento degli algoritmi di predizione interno a Grafana;
		\item \textbf{Attore primario:} utente;
		\item \textbf{Attore secondario:} Grafana\glo;
		\item \textbf{Descrizione:} Grafana\glosp fornisce all'utente un modo per eseguire l'upload di un file JSON che contiene dati di testing per l'addestramento degli algoritmi di predizione;
		\item \textbf{Scenario principale:} l'utente esegue il caricamento di un file JSON nel sistema;
		\item \textbf{Estensione:}
			\begin{itemize}
				\item \textbf{UC1.3:} il sistema avverte l'utente, attraverso un errore, che il caricamento del file JSON non è avvenuto con successo.
			\end{itemize}
		\item \textbf{Precondizione:} il plugin è installato nell'istanza di Grafana\glosp dell'utente;
		\item \textbf{Postcondizione:} il file JSON, inserito da utente, è stato caricato nel sistema.
	\end{itemize}

	\subsubsection{UC1.2 - Selezione del modello di predizione su cui applicare l'addestramento}
	\begin{itemize}
		\item \textbf{Codice identificativo}: UC1.2;
		\item \textbf{Titolo}: Selezione del modello di predizione su cui applicare l'addestramento;
		\item \textbf{Attore primario:} utente;
		\item \textbf{Attore secondario:} Grafana\glo;
		\item \textbf{Descrizione:} l'utente può decidere quale modello di predizione utilizzare per applicare l'addestramento tra la Regressione Lineare e la Support Vector Machine;
		\item \textbf{Scenario principale:} l'utente seleziona un algoritmo per eseguire l'addestramento;
		\item \textbf{Precondizione:} il file JSON inserito da utente è stato correttamente inserito nel sistema;
		\item \textbf{Postcondizione:} il modello di predizione è stato correttamente scelto e il sistema fornisce all'utente un modo per avviare l'addestramento.
	\end{itemize}

	\subsubsection{UC1.3 - Avvio dell'addestramento}
	\begin{itemize}
		\item \textbf{Codice identificativo}: UC1.3;
		\item \textbf{Titolo}: Avvio dell'addestramento;
		\item \textbf{Attore primario:} utente;
		\item \textbf{Attore secondario:} Grafana\glo;
		\item \textbf{Descrizione:} viene fornito all'utente un modo per avviare l'addestramento;
		\item \textbf{Scenario principale:} l'utente inizia l'addestramento; 
		\item \textbf{Precondizione:} l'utente ha scelto il modello di predizione da utilizzare
		\item \textbf{Postcondizione:} l'addestramento è stato iniziato con successo.
	\end{itemize}

	\subsection{UC1.4 - Conclusione dell'addestramento}
	\begin{itemize}
		\item \textbf{Codice identificativo}: UC1.4;
		\item \textbf{Titolo}: Conclusione dell'addestramento;
		\item \textbf{Attore primario:} utente;
		\item \textbf{Attore secondario:} Grafana\glo;
		\item \textbf{Descrizione:} il sistema ha completato l'addestramento, il risultato, di successo o insuccesso verrà fornito all'utente;
		\item \textbf{Scenario principale:} l'utente riceve un messaggio sull'esito dell'addestramento;
		\item \textbf{Estensioni:}
			\begin{itemize}
				\item \textbf{UC1.4.1:} l'addestramento è fallito.
			\end{itemize}
		\item \textbf{Precondizione:} l'addestramento è stato avviato con successo;
		\item \textbf{Postcondizione:} l'addestramento è stato concluso con successo e il file JSON contenente i parametri di previsione, generati dall'addestramento, è presente nel sistema.
	\end{itemize}

	\paragraph{UC1.4.1 - Addestramento fallito}
	\begin{itemize}
		\item \textbf{Codice identificativo}: UC1.4.1;
		\item \textbf{Titolo}: Addestramento fallito;
		\item \textbf{Attore primario:} utente;
		\item \textbf{Attore secondario:} Grafana\glo;
		\item \textbf{Descrizione:} in caso di errori nell'addestramento, viene fornito all'utente un messaggio di errore;
		\item \textbf{Scenario principale:} l'utente riceve un messaggio di errore esplicativo per l'insuccesso dell'addestramento;
		\item \textbf{Precondizione:} l'addestramento è stato completato;
		\item \textbf{Postcondizione:} viene visualizzato un messaggio di errore e non viene ricevuto nessun file JSON.
	\end{itemize}
\end{document}