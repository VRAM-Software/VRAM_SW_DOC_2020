\section*{A}
\subsection*{Alert}
Abbreviazione di alert notification; comunicazione macchina-uomo per segnalazioni importanti e/o sensibili al tempo.

\subsection*{Analisi statica}
Valutazione di un sistema o di un suo componente basato sulla sua forma, sulla sua struttura, sul suo contenuto e sulla documentazione di riferimento. Ciò significa che la valutazione avviene senza l'esecuzione del sistema o dell'oggetto dell'analisi.

\subsection*{Amazon Web Services}
Amazon Web Services, Inc. (nota con la sigla AWS) è un'azienda statunitense di proprietà del gruppo Amazon che fornisce su un'omonima piattaforma on demand servizi di cloud computing tra i quali: Amazon DynamoDB, Amazon Elastic Container Service, AWS Elastic Transcoder, AWS Rekognition e AWS Sage Maker.

\subsection*{API}
Acronimo di Application Programming Interface è una "scatola nera", cioè un'entità che evita al programmatore di sapere come funziona internamente, atta all'espletamento di un dato compito; spesso tale termine designa le librerie software di un linguaggio di programmazione.

%\subsection*{Amazon DynamoDB}
%Servizio di database NoSQL interamente gestito.

%\subsection*{Amazon Elastic Container Service}
%Servizio di gestione dei container che consente di eseguire, arrestare e gestire container Docker in un cluster.

%\subsection*{AWS Elastic Transcoder}
%Servizio che permette di fare conversioni di video in diversi formati, utile ad esempio per convertire file video di alta qualità in file più leggeri, tali da essere adatti all'utilizzo su dispositivi mobili.

%\subsection*{AWS Rekognition}
%Servizio cloud di riconoscimento che permette di identificare oggetti, persone, testo, scenari e attività in immagini e video. 

%\subsection*{AWS Sage Maker}
%Servizio gestito da Amazon di machine learning che fornisce algoritmi di apprendimento automatico che sono ottimizzati per essere eseguiti in modo efficiente su enormi quantità di dati in un ambiente distribuito. 

\subsection*{Architettura a multiservizi}
Architettura strutturata come un insieme di servizi, per quanto possibile disaccoppiati, che collaborano.

\subsection*{Asincrono}
Uno scambio di messaggi si dice asincrono se il mittente spedisce il messaggio e continua ad effettuare le proprie operazioni senza attendere una riposta.

\subsection*{Async/await}
Feature di alcuni linguaggi che permette di strutturare funzioni non bloccanti asincrone come normali funzioni sincrone.

\clearpage
\section*{B}

%\subsection*{Back-end}
%Parte di un architettura software che si prende carico dell'elaborazione principale e dell'archivio dei dati. È contrapposta al front-end che è un interfaccia con la quale l'utente interagisce con il sistema.

%\subsection*{BAL}
%Business Application Language; API\glosp generata sul momento da un parser\glosp partendo da documenti Gherkin\glosp o da un BDL\glo.

%\subsection*{BDD}
%Acronimo di behavior-driven development; metodologia di sviluppo del software che usa pattern di linguaggio standardizzati per descrivere una sequenza di precondizioni, azioni, e risultati al fine di descrivere le interazioni tra l'utente  e il sistema. La forza del BDD è che fornisce agli sviluppatori software e agli analisti degli strumenti e un processo condivisi per collaborare nello sviluppo software.

%\subsection*{BDL}
%Acronimo di Business Domain Language; insieme di azioni(verbi), oggetti(nomi) e delle combinazioni dei precedenti (predicati) che sono estratti da documenti di uno specifico dominio di conoscenza.

%\subsection*{Big Data}
%Grandi [masse di] dati o megadati, indica genericamente una raccolta di dati così estesa in termini di volume, velocità e varietà da richiedere tecnologie e metodi analitici specifici per l'estrazione di valore o conoscenza. 

%\subsection*{Blockchain}
%Struttura dati condivisa e immutabile. È definita come un registro digitale le cui voci sono raggruppate in blocchi, concatenati in ordine cronologico, e la cui integrità è garantita dall'uso della crittografia. Sebbene la sua dimensione sia destinata a crescere nel tempo, è immutabile in quanto, di norma, il suo contenuto una volta scritto non è più né modificabile né eliminabile, a meno di non invalidare l'intera struttura. 

%\subsection*{Blockchain}
%Struttura dati condivisa e immutabile. È definita come un registro digitale le cui voci sono raggruppate in blocchi, concatenati in ordine cronologico, e la cui integrità è garantita dall'uso della crittografia. Sebbene la sua dimensione sia destinata a crescere nel tempo, è immutabile in quanto, di norma, il suo contenuto una volta scritto non è più né modificabile né eliminabile, a meno di non invalidare l'intera struttura. 
\subsection*{Business logic}
%Logica di elaborazione che rende operativa un'applicazione, in altre parole l'algoritmica che codifica le regole dettate dal mondo reale.
Logica di elaborazione che rende operativa un'applicazione, in altre parole implementa gli specifici algoritmi di manipolazione dei dati che caratterizzano l’applicazione.
\clearpage
\section*{C}

\subsection*{Capitolato d'appalto}
Spesso abbreviato in capitolato; Documento che contiene le condizioni e le modalità relative all'esecuzione di un contratto fra un committente e un fornitore.

%\subsection*{Classificazione}
%Tecnica di previsione tramite cui, a partire da un dato, ne costruisco una classe ovvero un insieme di valori che non sono più continui, ma discreti.

\subsection*{Cloud}
Indica un paradigma di erogazione di servizi offerti on-demand, come l'archiviazione e l'elaborazione, da un fornitore ad un cliente finale attraverso la rete Internet, senza che l'utente debba gestire in modo diretto l'infrastruttura necessaria.

\subsection*{Container}
Ambiente d'esecuzione virtuale che permette di isolare il codice, le configurazioni e le dipendenze di un'applicazione. Diversi container possono coesistere sullo stesso sistema operativo perché sono studiati in modo da essere indipendenti dal resto dell'ambiente.

%Feature\glosp di un sistema operativo per cui il kernel permette l'esistenza di vari spazi utente isolati chiamati container. I container per i programmi che vi girano sembrano dei computer veri, con la particolarità che i programmi vedono solo le risorse che sono state loro assegnate.

\subsection*{Controllo di versione}
Il controllo di versione è un sistema che registra, nel tempo, i cambiamenti ad un file o ad una serie di file, così da poter richiamare una specifica versione in un secondo momento.
Permette ad un team di collaborare in modo efficiente facilitando l'individuazione e la risoluzione di conflitti.

%https://git-scm.com/book/it/v2/Per-Iniziare-Il-Controllo-di-Versione

%Gestione di versioni multiple di un insieme di file che sono ospitati su un repository/glo. Permette di lavorare collaborativamente senza rischio di sovrascritture accidentali.


%\subsection*{Cloud Application Platform}
%È una Platform as a service (PaaS), cioè un modello di servizi cloud che fornisce agli utenti strumenti di sviluppo, di database e di management per le applicazioni.

%\subsection*{Cocumber}
%Un framework per test automatici di BDD\glo.

%\subsection*{Criptovaluta}
%Rappresentazione digitale di valore basata sulla crittografia. Le criptovalute utilizzano tecnologie di tipo peer-to-peer su reti i cui nodi risultano costituiti da computer di utenti, situati potenzialmente in tutto il globo. Su questi computer vengono eseguiti appositi programmi che svolgono funzioni di portamonete. Il controllo decentralizzato di ciascuna criptovaluta funziona attraverso una tecnologia di contabilità generalizzata, in genere una blockchain\glo, che funge da database di transazioni finanziarie pubbliche. 

%\subsection*{CSS}
%Linguaggio usato per definire la formattazione di documenti HTML. La sua introduzione è stata necessaria per separare i contenuti delle pagine HTML dalla loro formattazione o layout e permettere una programmazione più chiara e facile da utilizzare, sia per gli autori delle pagine stesse sia per gli utenti, garantendo contemporaneamente anche il riutilizzo di codice ed una sua più facile manutenzione.

\clearpage
\section*{D}

\subsection*{Dashboard}
In italiano cruscotto; interfaccia che permette all'utente di tenere sotto controllo gli indicatori più importanti dell'ambiente in cui sta lavorando. È caratteristica fondamentale l'aggiornamento automatico dei dati, senza che vi debba essere un'interazione con l'utente.

\subsection*{Dati operativi}
Misurazioni effettuate da sensori.

%\subsection*{DLs}
%Acronimo di Description Languages; linguaggi formali in questo contesto usati per descrivere una struttura web REST\glo.

\subsection*{DevOps}
DevOps (dalla contrazione inglese di development, "sviluppo", e operations, "messa in produzione") è un metodo di sviluppo del software che punta alla comunicazione, collaborazione e integrazione tra sviluppatori e operatori che erogano il servizio. La costante collaborazione tra i due attori porta ad un notevole miglioramento della qualità.

\clearpage
\section*{E}

\subsection*{Emulatore}
Software che permette ad un sistema (host) di comportarsi come un altro sistema (guest).
Questo avviene emulando via software l'hardware del sistema guest, in modo da eseguirne il software originale nella macchina host.

%\subsection*{Ethereum}
%Ethereum è un sistema operativo e una piattaforma decentralizzata di computazione basata su blockchain\glosp che utilizza smart contracts\glo. Ether è la criptovaluta\glosp usata sulla piattaforma.

\clearpage
\section*{F}

\subsection*{Fattori influenzanti}
Elementi che sono in correlazione a un certo dato, utili quindi per prevedere l'andamento di quest'ultimo.

\subsection*{Feature}
Una caratteristica peculiare di un software.

\subsection*{Framework}
Architettura logica di supporto su cui un software può essere progettato e realizzato, spesso facilitandone lo sviluppo da parte del programmatore. Talora è usato come sinonimo piattaforma software.

\subsection*{Funzione lambda}
Detta anche funzione anonima è una funzione definita, e possibilmente chiamata, senza essere legata ad un identificatore. Le funzioni anonime sono utili per passare come argomento una funzione di ordine superiore.


\clearpage
\section*{G}
%\subsection*{Gherkin}
%Linguaggio che promuove l'uso del behavior-driven development per tutto il personale, inclusi analisti e manager.

%\subsection*{Git}
%Git è un software di controllo di versione distribuito creato da Linus Torvalds nel 2005. Le sue caratteristiche di richiedere poche risorse, essere uno strumento molto focalizzato ed essere open-source\glosp lo rendono molto usato.

\subsection*{Grafana}
Software ad uso generico per la produzione di cruscotti informativi (dashboard\glosp in inglese) e composizione di grafici. Viene utilizzato come un'applicazione web.

\subsection*{GUI}
Acronimo dell'inglese Graphical User Interface, in italiano interfaccia grafica; in informatica è un tipo di interfaccia utente che consente l'interazione uomo-macchina in modo visuale 
sostituendo con rappresentazioni grafiche i comandi tipici di un'interfaccia a riga di comando.

%utilizzando rappresentazioni grafiche piuttosto che utilizzando i comandi tipici di un'interfaccia a riga di comando.

\clearpage
\section*{H}

%\subsection*{HipTest}
%Servizio (rinominato nel 2019 CucumberStudio) per scrivere feature\glosp BDD\glo.

%\subsection*{HTML5}
%Quinta versione di HTML (HyperText Markup Language), linguaggio di markup con lo scopo di dare una struttura logica ai  documenti ipertestuali.

\clearpage
\section*{I}
\subsection*{Intelligenza artificiale}
L'intelligenza artificiale è una disciplina appartenente all'informatica che studia i fondamenti teorici, le metodologie e le tecniche che consentono la progettazione di sistemi hardware e sistemi di programmi software capaci di fornire all'elaboratore elettronico prestazioni che, a un osservatore comune, sembrerebbero essere di pertinenza esclusiva dell'intelligenza umana.

%\subsection*{IoT}
%Acronimo di Internet of Things; sistema di dispositivi interconnessi con un identificatore unico abilitati a trasferire dati sulla rete Internet senza che sia richiesta interazione umana.

\clearpage
\section*{J}

%\subsection*{Java}
%Linguaggio di programmazione ad alto livello, orientato agli oggetti e a tipizzazione statica, specificamente progettato per essere il più possibile indipendente dalla piattaforma hardware di esecuzione (tramite compilazione in bytecode prima e interpretazione poi da parte di una macchina virtuale).

%\subsection*{JavaScript}
%JavaScript è un linguaggio di scripting orientato agli oggetti e agli eventi, comunemente utilizzato nella programmazione Web lato client per la creazione, in siti web e applicazioni web, di effetti dinamici interattivi tramite funzioni di script invocate da eventi innescati a loro volta in vari modi dall'utente sulla pagina web in uso (mouse, tastiera ...)

%\subsection*{JSON}
%Acronimo di JavaScript Object Notation, è un formato adatto all'interscambio di dati tra applicazioni. Pur essendo derivato dal JavaScipt è indipendente da esso. Come l'XML (che vorrebbe sostituire), usa una sintassi leggibile da un umano costituita da coppie di attributo-valore.

\clearpage
\section*{K}

%\subsection*{Kafka}
%Apache Kafka è una piattaforma open source di stream processing\glosp che mira a creare una piattaforma a bassa latenza ed alta velocità per la gestione di dati in tempo reale. 

\clearpage
\section*{L}

%\subsection*{LDAP}
%Acronimo di  Lightweight Directory Access Protocol; protocollo del livello applicazione del modello ISO/OSI per accedere e interrogare servizi di directory, cioè  qualsiasi raggruppamento di informazioni che può essere espresso come record di dati e organizzato in modo gerarchico. 

\clearpage
\section*{M}

\subsection*{Machine learning}
Il machine learning (in italiano apprendimento automatico) è una branca dell'intelligenza artificiale che utilizza metodi statistici per migliorare progressivamente la performance di un algoritmo nell'identificare pattern nei dati. 

\subsection*{Metadato}
Letteralmente "(dato) per mezzo di un (altro) dato", è un'informazione che descrive un insieme di dati. 

\clearpage
\section*{N}
%\subsection*{NLP}
%Acronimo di Natural Language Processing; processo di trattamento automatico mediante un calcolatore elettronico delle informazioni scritte o parlate in una lingua naturale.

%\subsection*{NodeJS}
%Node.js è una runtime per l'esecuzione di codice JavaScript lato server, necessaria perché in origine JavaScript veniva utilizzato principalmente lato client.

%\subsection*{Notifica push}
%Notifica che porta a conoscenza del destinatario di un messaggio senza che questo debba effettuare un'operazione di scaricamento (modalità pull).

\clearpage
\section*{O}

%\subsection*{Open source}
%Tipo di software o suo modello di sviluppo. Un software open source è reso tale per mezzo di una licenza attraverso cui i detentori dei diritti ne favoriscono la modifica, lo studio, l'utilizzo e la redistribuzione. Caratteristica principale dunque delle licenze open source è la pubblicazione del codice sorgente (da cui il nome). 

\clearpage
\section*{P}
%\subsection*{Parser}
%Programma che analizza un flusso di dati in modo da determinare la correttezza della sua struttura grazie ad una data grammatica formale. 

%\subsection*{Pattern di Publisher/Subscriber}
%Pattern di messaggistica dove i mittenti, chiamati publisher, inviano i messaggi non a uno specifico ricevente ma a una classe di riceventi. Allo stesso modo i riceventi, chiamati receiver, scelgono di ricevere messaggi di una certa classe di interesse senza sapere chi sono i  mittenti.

%In software architecture, publish–subscribe is a messaging pattern where senders of messages, called publishers, do not program the messages to be sent directly to specific receivers, called subscribers, but instead categorize published messages into classes without knowledge of which subscribers, if any, there may be. Similarly, subscribers express interest in one or more classes and only receive messages that are of interest, without knowledge of which publishers, if any, there are.

\subsection*{PoC}
Acronimo di Proof of Concept, si intende una realizzazione incompleta o abbozzata di un determinato progetto, allo scopo di provarne la fattibilità e dimostrarne la fondatezza.

%\subsubsection*{Plug-in}
%Programma non autonomo che interagisce con un altro programma per ampliarne o estenderne le funzionalità originarie; vengono chiamati anche con i sinonimi: add-in, add-on o estensione.

\subsection*{Predittore}
Dati o variabili su cui applico le tecniche di regressione o di classificazione per ottenere un dato la cui diretta rilevazione sarebbe impossibile o troppo onerosa.

\subsection*{Prodotto}
Si definisce prodotto qualsiasi bene scambiabile sul mercato che può rispondere alle esigenze di un compratore. Un esempio di prodotto informatico è il software che è composto dal codice e dalla documentazione.

\subsection*{Progetto}
Insieme di attività e compiti il cui fine è raggiungere determinati obiettivi con specifiche fissate utilizzando risorse limitate. Le attività principali sono: pianificazione, analisi dei requisiti, progettazione, realizzazione e manutenzione.

\subsection*{Progettazione}
Design in inglese; attività facente parte del processo di sviluppo. Si occupa di trovare \textbf{come} implementare i requisiti scaturiti dall'analisi.

\subsection*{Prodotto}
Si definisce prodotto qualsiasi bene scambiabile sul mercato che può rispondere alle esigenze di un compratore. Un prodotto informatico è composto dal software e dalla documentazione.

\subsection*{Progetto}
Insieme di attività e compiti il cui fine è raggiungere determinati obiettivi con specifiche fissate utilizzando risorse limitate. Le attività principali sono: pianificazione, analisi dei requisiti, progettazione e realizzazione.

\subsection*{Progettazione}
Design in inglese; attività facente parte del processo di sviluppo. Si occupa di trovare \textbf{come} implementare i requisiti scaturiti dall'analisi.

\subsection*{Promise}
Pattern non annoverato tra i design pattern classici, ma di rilievo nella programmazione JavaScript, in particolare nel supporto alla programmazione asincrona, cioè alla possibilità di eseguire attività in background che non interferiscono con il flusso di elaborazione principale. 

%\subsection*{Python}
%Python è un linguaggio di programmazione ad alto livello e multi-paradigma che ha tra i principali obiettivi dinamicità, semplicità e flessibilità.

\clearpage
\section*{Q}

\clearpage
\section*{R}

\subsection*{Repository}
%Abbreviato anche in repo; posto dove si possono immagazzinare e trovare cose, che in campo informatico sono file. Un repo è spesso gestito tramite un sistema di versionamento (p.es. Git).

Repository significa archivio. In un repository sono raccolti dati e informazioni in formato digitale, valorizzati e archiviati sulla base di metadati che ne permettono la rapida individuazione, anche grazie alla creazione di tabelle relazionali. Grazie alla sua peculiare architettura, un repository consente di gestire in modo ottimale anche grandi volumi di dati.

%\subsection*{Regressione}
%Tecnica di previsione tramite cui, a partire da un dato, ne produco un altro continuo, cioè un dato non discreto che ha valori quindi in $\mathbb{R}$

\subsection*{RL}
Acronimo di Regressione lineare, algoritmo di machine learning\glosp che ha la funzione di prevedere un valore di una variabile dipendente (y) in base a una determinata variabile indipendente (x) secondo una relazione di tipo lineare.

\subsection*{Reti neurali}
Nel campo dell'apprendimento automatico, una rete neurale artificiale  è un modello computazionale composto di "neuroni" artificiali, ispirato vagamente dalla semplificazione di una rete neurale biologica. Questi modelli matematici sono troppo semplici per ottenere una comprensione delle reti neurali biologiche, ma sono utilizzati per tentare di risolvere problemi ingegneristici di intelligenza artificiale.

\subsection*{REST}
Acronimo di REpresentational State Transfer; è un modello architetturale inventato da Roy Fielding per la creazione di servizi web.

\clearpage
\section*{S}

%\subsection*{Scalabilità}
%La scalabilità denota in genere la capacità di un sistema di aumentare o diminuire di scala in funzione delle necessità e disponibilità. Un sistema che gode di questa proprietà viene detto scalabile.

\subsection*{Serverless}
È un modello computazionale nel quale il fornitore opera il server e dinamicamente assegna risorse al cliente. Il prezzo del servizio è basato sull'effettiva quantità di risorse utilizzate, e non su unità di grandezza in precedenza stabilite e pagate.

\subsection*{Smart contract}
Protocolli informatici che facilitano, verificano, o fanno rispettare, la negoziazione o l'esecuzione di un contratto. Permettono l'esecuzione di transazioni credibili senza l'ausilio di terze parti.

\subsection*{Stakeholder}
Gli stakeholder (portatori di interesse in italiano) sono l’insieme di coloro che a vario titolo hanno influenza sul prodotto, sul progetto, sui processi, e sono formati da: la comunità degli utenti che usa il prodotto, il committente che compra il prodotto, il fornitore che sostiene i costi di realizzazione e da eventuali regolatori che verificano l’attuazione di processi.

\subsection*{SVM}
Acronimo di Support Vector Machine; algoritmo di apprendimento automatico supervisionato che può essere utilizzato sia per scopi di classificazione che di regressione.

\clearpage
\section*{T}

%\subsection*{Telegram}
%Telegram è un servizio di messaggistica istantanea e broadcasting.

%\subsection{Test di unità}
%Il test d’unità è una metodologia che permette di verificare il corretto funzionamento di singole unità di codice in determinate condizioni.Per unità si intende normalmente il minimo componente di un programma dotato di funzionamento autonomo.

%\subsection*{Time series database}
%Sistema software ottimizzato per l'immagazzinamento di paia di elementi tempo-valore.

%\subsection*{Time stamp}
%Una marca temporale (timestamp) è una sequenza di caratteri che rappresentano una data e/o un orario per accertare l'effettivo avvenimento di un certo evento. 

\subsection*{Toolkit}
Letteralmente «cassetta degli attrezzi» in informatica è utilizzato per riferirsi ad un insieme di strumenti software di base, in genere librerie, usati per facilitare e uniformare lo sviluppo di applicazioni derivate più complesse. 

\clearpage
\section*{U}

\clearpage
\section*{V}


\clearpage
\section*{W}

\subsection*{Workflow}
In italiano flusso di lavoro; l'attuazione di regole procedurali ben definite al fine di portare a termine una determinata azione. Il workflow può coinvolgere una o più persone.


\clearpage
\section*{X}

\clearpage
\section*{Y}

\clearpage
\section*{Z}
