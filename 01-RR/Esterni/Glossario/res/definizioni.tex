\section*{A}
\addcontentsline{toc}{section}{A}

\subsection*{Alert}
Abbreviazione di alert notification; comunicazione macchina-uomo per segnalazioni importanti e/o sensibili al tempo.

\subsection*{Analisi statica}
Valutazione di un sistema o di un suo componente basato sulla sua forma, sulla sua struttura, sul suo contenuto e sulla documentazione di riferimento. Ciò significa che la valutazione avviene senza l'esecuzione del sistema o dell'oggetto dell'analisi.

\subsection*{Apache 2}
La Licenza Apache 2.0 è una licenza di software libero non copyleft scritta dalla Apache Software Foundation (ASF) che obbliga gli utenti a preservare l'informativa di diritto d'autore e d'esclusione di responsabilità nelle versioni modificate. 

\subsection*{Amazon Web Services}
Amazon Web Services, Inc. (nota con la sigla AWS) è un'azienda statunitense di proprietà del gruppo Amazon che fornisce su un'omonima piattaforma on demand servizi di cloud computing tra i quali: Amazon DynamoDB, Amazon Elastic Container Service, AWS Elastic Transcoder, AWS Rekognition e AWS Sage Maker.

\subsection*{API}
Acronimo di Application Programming Interface è una "scatola nera", cioè un'entità che evita al programmatore di sapere come funziona internamente, atta all'espletamento di un dato compito; spesso tale termine designa le librerie software di un linguaggio di programmazione.

\subsection*{Architettura a multiservizi}
Architettura strutturata come un insieme di servizi, per quanto possibile disaccoppiati, che collaborano.

\subsection*{Attore}
Un attore è un soggetto che interagisce con il sistema, ha un obiettivo e per raggiungerlo sfrutta un caso d'uso\glo. 

\subsection*{Asincrono}
Uno scambio di messaggi si dice asincrono se il mittente spedisce il messaggio e continua ad effettuare le proprie operazioni senza attendere una risposta.

\subsection*{Async/await}
Feature di alcuni linguaggi che permette di strutturare funzioni non bloccanti asincrone come normali funzioni sincrone.

\clearpage
\section*{B}
\addcontentsline{toc}{section}{B}

\subsection*{Baseline}
Letteralmente "base di riferimento", indica il piano originale approvato di un progetto, di una parte di lavoro o di un'attività.

\subsection*{Black-Box}
I test di sistema black-box sono test dove si esercita il sistema immettendo input e osservando i valori degli output.
Non conosciamo (oppure non teniamo in conto): il codice sorgente, lo stato ed il funzionamento interno interno dell’applicazione.

\subsection*{BSD}
Le licenze software BSD sono una famiglia di licenze permissive senza copyleft. Molte sono considerate libere ed open source.

\subsection*{Business logic}
Logica di elaborazione che rende operativa un'applicazione, in altre parole implementa gli specifici algoritmi di manipolazione dei dati che caratterizzano l’applicazione.

\clearpage
\section*{C}
\addcontentsline{toc}{section}{C}

\subsection*{Capitolato d'appalto}
Spesso abbreviato in capitolato; documento che contiene le condizioni e le modalità relative all'esecuzione di un contratto fra un committente e un fornitore.

\subsection*{Caso d'uso}
Un caso d’uso è un insieme di scenari\glosp che hanno in comune un obiettivo per un attore\glo.

\subsection*{Cloud}
Indica un paradigma di erogazione di servizi offerti on-demand, come l'archiviazione e l'elaborazione, da un fornitore ad un cliente finale attraverso la rete Internet, senza che l'utente debba gestire in modo diretto l'infrastruttura necessaria.

\subsection*{Containerizzazione}
Tecnica che permette di incapsulare le applicazioni software all’interno di ambienti autonomi detti container\glo.

\subsection*{Container}
Ambiente d'esecuzione virtuale che permette di isolare il codice, le configurazioni e le dipendenze di un'applicazione. Diversi container possono coesistere sullo stesso sistema operativo perché sono studiati in modo da essere indipendenti dal resto dell'ambiente.

\subsection*{Controllo di versione}
Il controllo di versione è un sistema che registra, nel tempo, le modifiche effettuate ad un file o ad una serie di file, permettendo così di recuperare una specifica versione dei file stessi in un secondo momento.
Permette inoltre ad un team di collaborare in modo efficiente facilitando l'individuazione e la risoluzione di conflitti.

\clearpage
\section*{D}
\addcontentsline{toc}{section}{D}

\subsection*{Dashboard}
In italiano cruscotto; interfaccia che permette all'utente di tenere sotto controllo gli indicatori più importanti dell'ambiente in cui sta lavorando. È caratteristica fondamentale l'aggiornamento automatico dei dati, senza che vi debba essere un'interazione con l'utente.

\subsection*{Dati operativi}
Misurazioni effettuate da sensori.

\subsection*{DevOps}
DevOps (dalla contrazione inglese di development, "sviluppo", e operations, "messa in produzione") è un metodo di sviluppo del software che punta alla comunicazione, collaborazione e integrazione tra sviluppatori e operatori che erogano il servizio. La costante collaborazione tra i due attori porta ad un notevole miglioramento della qualità.

\subsection*{Diagrammi UML}
I Diagrammi UML sono costruiti usando UML (Unified Modeling Language), un linguaggio semiformale e grafico (basato su diagrammi). Sono usati per specificare, visualizzare, realizzare, modificare e documentare gli artefatti di un sistema (solitamente un sistema software).

\subsection*{Dispatcher}
Modulo del sistema operativo che gestisce la CPU passandone il controllo ai processi indicati dallo scheduler.

\subsection*{Driver}
Dall'inglese "autista" indica l'insieme di procedure software che consente la comunicazione con un dispositivo hardware.

\clearpage
\section*{E}
\addcontentsline{toc}{section}{E}

\subsection*{Emulazione}
Consiste nella procedura con cui un elaboratore riproduce in un sistema di elaborazione il funzionamento di un altro elaboratore di analoghe o inferiori capacità tecniche.

\clearpage
\section*{F}
\addcontentsline{toc}{section}{F}

\subsection*{Fattori influenzanti}
Elementi che sono in correlazione a un certo dato, utili quindi per prevedere l'andamento di quest'ultimo.

\subsection*{Feature}
Una caratteristica peculiare di un software.

\subsection*{Front end}
Parte visibile all'utente necessaria per la sua interazione con il programma.

\subsection*{Funzione lambda}
Detta anche funzione anonima è una funzione definita, e possibilmente chiamata, senza essere legata ad un identificatore. Le funzioni anonime sono utili per passare come argomento una funzione di ordine superiore.


\clearpage
\section*{G}
\addcontentsline{toc}{section}{G}

\subsection*{Grafana}
Software ad uso generico per la produzione di cruscotti informativi (dashboard\glosp in inglese) e composizione di grafici. Viene utilizzato come un'applicazione web.

\clearpage
\section*{H}
\addcontentsline{toc}{section}{H}

\clearpage
\section*{I}
\addcontentsline{toc}{section}{I}
	
\subsection*{Intelligenza artificiale}
L'intelligenza artificiale è una disciplina appartenente all'informatica che studia i fondamenti teorici, le metodologie e le tecniche che consentono la progettazione di sistemi hardware e sistemi di programmi software capaci di fornire all'elaboratore elettronico prestazioni che, a un osservatore comune, sembrerebbero essere di pertinenza esclusiva dell'intelligenza umana.

\clearpage
\section*{J}
\addcontentsline{toc}{section}{J}

\subsection*{JavaMX}
Java Management Extension (JMX o JavaMX) è una tecnologia Java che fornisce strumenti per gestire e monitorare applicazioni, oggetti di sistema, dispositivi e reti orientate ai servizi. Gli elementi monitorati saranno rappresentati da degli MBean (oggetti forniti da Java).

\subsection*{JMeter}
JMeter è un progetto Apache che può essere usato per caricare strumenti di verifica e per analizzare e misurare le prestazioni di una varietà di servizi, con una focalizzazione sulle applicazioni web.

\clearpage
\section*{K}
\addcontentsline{toc}{section}{K}

\clearpage
\section*{L}
\addcontentsline{toc}{section}{L}

\clearpage
\section*{M}
\addcontentsline{toc}{section}{M}

\subsection*{Machine learning}
Il machine learning (in italiano apprendimento automatico) è una branca dell'intelligenza artificiale che utilizza metodi statistici per migliorare progressivamente la performance di un algoritmo nell'identificare pattern nei dati. 

\subsection*{Metriche}
Una metrica software è uno standard per la misura di alcune proprietà del software o delle sue specifiche. 

\subsection*{MIT}
La MIT è una licenza software permissiva le cui uniche condizioni riguardano la tutela delle note sul copyright e sulla licenza. L'opera, le sue modifiche, o lavori più estesi basati su di essa possono essere distribuiti con una licenza diversa, anche proprietaria e anche senza codice sorgente. 

\subsection*{Modello a V}
Modello di sviluppo software, estensione del modello a cascata. Il modello invece di discendere lungo una linea retta, dopo la fase di programmazione risale con una tipica forma a V. Il modello dimostra la relazione tra ogni fase del ciclo di vita dello sviluppo del software e la sua fase di testing. 

\subsection*{Modifiche Editoriali}
Sono modifiche che riguardano principalmente gli stili utilizzati nel documento e l'aspetto generale dei risultati, ma includono anche grammatica, punteggiatura, ecc. Di norma questo genere di modifiche vengono effettuate in seguito alle verifiche dei documenti.

\subsection*{Modifiche Tecniche}
Sono modifiche tecniche quelle in cui si cambiano gli elementi del documento (come unità e/o valori per esempio) o in cui si aggiungono o eliminano interi paragrafi.

\clearpage
\section*{N}
\addcontentsline{toc}{section}{N}

\clearpage
\section*{O}
\addcontentsline{toc}{section}{O}

\clearpage
\section*{P}
\addcontentsline{toc}{section}{P}

\subsection*{PoC}
Acronimo di Proof of Concept, si intende una realizzazione incompleta o abbozzata di un determinato progetto, allo scopo di provarne la fattibilità e dimostrarne la fondatezza.

\subsection*{Predittore}
Dati o variabili su cui applico le tecniche di regressione o di classificazione per ottenere un dato la cui diretta rilevazione sarebbe impossibile o troppo onerosa.

\subsection*{Processo}
È un insieme di attività correlate e coese che trasformano dei bisogni in prodotti, consumando risorse. Un processo è sistematico, disciplinato e quantificabile.

\subsection*{Prodotto}
Si definisce prodotto qualsiasi bene scambiabile sul mercato che può rispondere alle esigenze di un compratore. Un esempio di prodotto informatico è il software che è composto dal codice e dalla documentazione.

\subsection*{Progettazione}
Design in inglese; attività facente parte del processo di sviluppo. Si occupa di trovare come implementare i requisiti scaturiti dall'analisi.

\subsection*{Progetto}
Insieme di attività e compiti il cui fine è raggiungere determinati obiettivi con specifiche fissate utilizzando risorse limitate. Le attività principali sono: pianificazione, analisi dei requisiti, progettazione, realizzazione e manutenzione.

\subsection*{Promise}
Pattern non annoverato tra i design pattern classici, ma di rilievo nella programmazione JavaScript, in particolare nel supporto alla programmazione asincrona, cioè alla possibilità di eseguire attività in background che non interferiscono con il flusso di elaborazione principale.

\clearpage
\section*{Q}
\addcontentsline{toc}{section}{Q}

\clearpage
\section*{R}
\addcontentsline{toc}{section}{R}

\subsection*{Repository}
Repository significa archivio. In un repository sono raccolti dati e informazioni in formato digitale, valorizzati e archiviati sulla base di metadati che ne permettono la rapida individuazione, anche grazie alla creazione di tabelle relazionali. Grazie alla sua peculiare architettura, un repository consente di gestire in modo ottimale anche grandi volumi di dati.

\subsection*{REST}
Acronimo di REpresentational State Transfer; è un modello architetturale inventato da Roy Fielding per la creazione di servizi web.

\subsection*{Reti neurali}
Nel campo dell'apprendimento automatico, una rete neurale artificiale  è un modello computazionale composto di "neuroni" artificiali, ispirato vagamente dalla semplificazione di una rete neurale biologica. Questi modelli matematici sono troppo semplici per ottenere una comprensione delle reti neurali biologiche, ma sono utilizzati per tentare di risolvere problemi ingegneristici di intelligenza artificiale.

\subsection*{RL}
Acronimo di Regressione lineare, algoritmo di machine learning\glosp che ha la funzione di prevedere un valore di una variabile dipendente (y) in base a una determinata variabile indipendente (x) secondo una relazione di tipo lineare.

\clearpage
\section*{S}
\addcontentsline{toc}{section}{S}

\subsection*{Serverless}
È un modello computazionale nel quale il fornitore opera il server e dinamicamente assegna risorse al cliente. Il prezzo del servizio è basato sull'effettiva quantità di risorse utilizzate, e non su unità di grandezza in precedenza stabilite e pagate.

\subsection*{Smart contract}
Protocolli informatici che facilitano, verificano, o fanno rispettare, la negoziazione o l'esecuzione di un contratto. Permettono l'esecuzione di transazioni credibili senza l'ausilio di terze parti.

\subsection*{Snake\_case}
Lo snake case, o snake\_case, è la pratica di scrivere gli identificatori separando le parole che li compongono tramite trattino basso (o underscore: \_), solitamente con le prime lettere delle singole parole in minuscolo, e la prima lettera dell'intero identificatore minuscola o maiuscola

\subsection*{SonarJS}
SonarJS è uno strumento di analisi statica del codice per il linguaggio JavaScript, sviluppato da SonarSource.

\subsection*{Stakeholder}
Gli stakeholder (portatori di interesse in italiano) sono l’insieme di coloro che a vario titolo hanno influenza sul prodotto, sul progetto, sui processi, e sono formati da: la comunità degli utenti che usa il prodotto, il committente che compra il prodotto, il fornitore che sostiene i costi di realizzazione e da eventuali regolatori che verificano l’attuazione di processi.

\subsection*{Stub}
Stub o anche metodo stub, è una porzione di codice utilizzata in sostituzione di altre funzionalità software in quanto può simulare il comportamento sia di codice esistente sia di codice ancora da sviluppare.

\subsection*{SVM}
Acronimo di Support Vector Machine; algoritmo di apprendimento automatico supervisionato che può essere utilizzato sia per scopi di classificazione che di regressione.

\clearpage
\section*{T}
\addcontentsline{toc}{section}{T}

\subsection*{Toolkit}
Letteralmente «cassetta degli attrezzi» in informatica è utilizzato per riferirsi ad un insieme di strumenti software di base, in genere librerie, usati per facilitare e uniformare lo sviluppo di applicazioni derivate più complesse. 

\clearpage
\section*{U}
\addcontentsline{toc}{section}{U}

\clearpage
\section*{V}
\addcontentsline{toc}{section}{V}

\subsection*{Validazione}
Conferma tramite prove oggettive che il software è conforme ai bisogni dell'utente e agli usi a cui è destinato; i requisiti implementati tramite il prodotto\glosp sono quindi consistenti.

\clearpage
\section*{W}
\addcontentsline{toc}{section}{W}

\subsection*{Workflow}
In italiano flusso di lavoro; l'attuazione di regole procedurali ben definite al fine di portare a termine una determinata azione. Il workflow può coinvolgere una o più persone.


\clearpage
\section*{X}
\addcontentsline{toc}{section}{X}

\clearpage
\section*{Y}
\addcontentsline{toc}{section}{Y}

\clearpage
\section*{Z}
\addcontentsline{toc}{section}{Z}
