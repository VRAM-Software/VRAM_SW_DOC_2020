\section{Introduzione}
	\subsection{Scopo del Documento}
		Questo documento descrive la disponibilità delle risorse e stabilisce come vengono assegnate ad attività e processi\glo. L'obiettivo è ottenere un'organizzazione efficiente, per farlo si utilizza una scansione temporale delle attività.
	\subsection{Scopo del Prodotto}
		L'obiettivo del capitolato\glosp C4 è quello di sviluppare un plug-in per Grafana\glo, il quale analizzerà una serie temporale di dati applicandovi gli algoritmi di Support Vector Machine (SVM\glo) o Regressione Lineare (RL\glo) e restituirà previsioni riguardo a comportamenti rischiosi da parte del sistema monitorato, possibilmente accompagnate da un indice di affidabilità. Assieme al plug-in verrà sviluppato un programma per la gestione dei parametri degli algoritmi di previsione, che permetterà di allenare gli algoritmi di previsione con dei dati di test.
	\subsection{Ambiguità e Glossario}
		Questo documento sarà corredato di un \textit{Glossario v. 1.1.1} dove saranno chiariti i termini potenzialmente ambigui.
		Le voci interessate saranno identificate da una 'G' a pedice.
	\subsection{Riferimenti}
		\subsubsection{Riferimenti Normativi}
			\begin{enumerate}
				\item \textbf{Norme di Progetto}: \textit{Norme di Progetto v. 1.1.1};
				\item \textbf{Slide gestione di progetto}: \url{https://www.math.unipd.it/~tullio/IS-1/2019/Dispense/L06.pdf};
				\item \textbf{Capitolato\glosp d'appalto C4 - Predire in Grafana}:  \url{https://www.math.unipd.it/~tullio/IS-1/2019/Progetto/C4.pdf}.
			\end{enumerate}
		\subsubsection{Riferimenti Informativi}
			\begin{enumerate}
				\item \textbf{Piano di Qualifica}: \textit{Piano di Qualifica v. 1.1.1};
				\item \textbf{Ingegneria del software, Ian Sommerville, Pearson Education, Addison Wesley (Decima edizione)}: capitolo 19.1;
				\item \textbf{Guide to the Software Engineering Body of Knowledge(SWEBOK), 2004}: \url{http://www.math.unipd.it/~	tullio/IS-1/2007/Approfondimenti/SWEBOK.pdf};
			\end{enumerate}
