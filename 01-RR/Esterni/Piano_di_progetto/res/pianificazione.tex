%Mettere al suo interno una suddivisione con i 4 periodi che corrispondono alle scadenze. Per ognuno suddividere ulteriormente in periodi di analisi da definire. Alla fine di ognuno dei 4 periodi principali(delle 4 scadenze) fare un Gantt
%per ogni attività bisogna trovare un sottoperiodo e per ognuno un grafico di Gantt
\section{Pianificazione} 
Il nostro gruppo, per rispettare le scadenze elencate nella sezione \textit{Calendario delle attività}, ha deciso di suddividere lo sviluppo del prodotto\glosp e della sua documentazione nelle seguenti quattro attività:
\begin{itemize}
	\item Analisi dei Requisiti;
	\item Progettazione\glosp architetturale;
	\item Progettazione\glosp di dettaglio e codifica;
	\item Validazione\glosp e collaudo.
\end{itemize}
La pianificazione ha lo scopo di gestire lo sviluppo del progetto\glosp suddividendolo in attività che, singolarmente, risultano più facili da realizzare. Per descrivere ogni attività, abbiamo deciso di suddividerle ulteriormente in periodi e abbiamo elencato i ruoli attivi durante lo svolgimento di ciascuno di essi.

\subsection{Analisi dei Requisiti}
L'analisi dei requisiti è la prima attività: inizia il 2019-11-14, giorno successivo alla formazione dei gruppi e termina il 2020-01-20, giorno precedente alla presentazione del progetto\glo. In questa attività ci siamo principalmente interessati all'analisi di tutte le informazioni riguardanti il prodotto\glosp che dobbiamo sviluppare, l'organizzazione delle attività e la suddivisione delle risorse.

\subsubsection{Ruoli attivi}
\begin{itemize}
	\item Responsabile di progetto\glo;
	\item Amministratore di progetto\glo;
	\item Analista;
	\item Verificatore.
\end{itemize}

\subsubsection{Periodi}
Abbiamo suddiviso l'attività della Analisi dei Requisiti nei seguenti sei periodi: \\*
\textbf{I periodo: dal 2019-11-14 al 2019-11-27}
\begin{itemize}
	\item \textbf{Discussione dei capitolati}\glo: discussione interna, analizzando i fattori positivi e negativi di ogni capitolato\glo, per indirizzarci alla scelta di quale progetto\glosp realizzare;
	\item \textbf{Normazione}: discussione interna in merito alle regole da seguire per lo sviluppo della documentazione del progetto\glosp. Iniziato il documento interno \textit{Norme di progetto v. 1.1.1} nelle sue sezioni riguardanti: \textit{Studio di Fattibilità}, strumenti da utilizzare, documentazione, gestione della configurazione, processo\glosp di verifica e processi\glosp organizzativi;
	\item \textbf{Ricerca di strumenti e tecnologie}: inizio della ricerca di gruppo e individuale su strumenti e tecnologie necessari allo sviluppo della documentazione del progetto\glo;
	\item \textbf{Definizione dei ruoli}: suddivisione dei ruoli per le attività prese in considerazione; 
	\item \textbf{Pianificazione delle attività}: gestione delle risorse disponibili, suddivisione e pianificazione di tutte le attività che devono essere svolte in questo periodo;
	\item \textbf{Verifica}: attività di controllo dei documenti realizzati durante questo periodo.
\end{itemize}

\textbf{II periodo: dal 2019-11-28 al 2019-12-08}
\begin{itemize}
	\item \textbf{Studio di Fattibilità}: formalizzazione della scelta del capitolato\glosp con la stesura del documento \textit{Studio di Fattibilità v. 1.1.1};
	\item \textbf{Normazione}: revisione e aggiornamento delle norme di progetto\glosp riguardanti \textit{Studio di Fattibilità} e strumenti da utilizzare;
	\item \textbf{Ricerca di strumenti e tecnologie}: continuazione della ricerca di strumenti e tecnologie da utilizzare per lo sviluppo del prodotto\glo;
	\item \textbf{Definizione dei ruoli}: suddivisione dei ruoli per le attività prese in considerazione; 
	\item \textbf{Pianificazione delle attività}: gestione delle risorse disponibili, suddivisione e pianificazione di tutte le attività che devono essere svolte in questo periodo;
	\item \textbf{Verifica}: attività di controllo dei documenti realizzati durante questo periodo.
\end{itemize}

\textbf{III periodo: dal 2019-12-09 al 2020-12-22}
\begin{itemize}
	\item \textbf{Normazione}: revisione e aggiornamento delle norme di progetto\glosp riguardanti \textit{Analisi dei Requisiti}, \textit{Piano di Progetto} e strumenti da utilizzare;
	\item \textbf{Ricerca di strumenti e tecnologie}: continuazione della ricerca di strumenti e tecnologie da utilizzare per lo sviluppo del prodotto\glo;
	\item \textbf{Pianificazione delle attività}: gestione delle risorse disponibili, suddivisione e pianificazione di tutte le attività che devono essere svolte in questo periodo;
	\item \textbf{Analisi dei requisiti}: individuazione dei requisiti del prodotto\glosp in seguito a: incontri interni, analisi dei casi d'uso, analisi del capitolato\glosp e incontro esterno col proponente.
	\item \textbf{Verifica}: attività di controllo dei documenti realizzati durante questo periodo.
\end{itemize}

\textbf{IV periodo: dal 2019-12-23 al 2020-01-01}
\begin{itemize}
	\item \textbf{Normazione}: revisione e aggiornamento delle norme di progetto\glosp riguardanti \textit{Analisi dei Requisiti}, \textit{Piano di Qualifica}, gestione della qualità e strumenti da utilizzare;
	\item \textbf{Gestione della qualità}: inizio della discussione interna in merito a come definire e mantenere uno standard per garantire la qualità di tutti i documenti realizzati;
	\item \textbf{Ricerca di strumenti e tecnologie}: continuazione della ricerca di strumenti e tecnologie da utilizzare per lo sviluppo del prodotto\glo;
	\item \textbf{Pianificazione delle attività}: gestione delle risorse disponibili, suddivisione e pianificazione di tutte le attività che devono essere svolte in questo periodo;
	\item \textbf{Definizione dei casi d'uso}: realizzazione dei casi d'uso del prodotto\glosp richiesto dal proponente;
	\item \textbf{Verifica}: attività di controllo dei documenti realizzati durante questo periodo.
\end{itemize}


\textbf{V periodo: dal 2020-01-02 al 2020-01-14}
\begin{itemize}
	\item \textbf{Normazione}: revisione e aggiornamento delle norme di progetto\glosp riguardanti: \textit{Piano di Qualifica}, \textit{Piano di Progetto} e strumenti da utilizzare;
	\item \textbf{Gestione della qualità}: individuazione di regole e metodi per mantenere e garantire la qualità del prodotto\glo;
	\item \textbf{Ricerca di strumenti e tecnologie}: continuazione della ricerca di strumenti e tecnologie da utilizzare per lo sviluppo del prodotto\glo;
	\item \textbf{Pianificazione delle attività}: gestione delle risorse disponibili, suddivisione e pianificazione di tutte le attività che devono essere svolte in questo periodo;
	\item \textbf{Analisi dei rischi}: discussione interna dei possibili rischi nella realizzazione del progetto\glo;
	\item \textbf{Stesura lettera di presentazione}: stesura della \textit{Lettera di Presentazione} in cui si propone una soluzione alla richiesta del proponente;
	\item \textbf{Verifica}: attività di controllo dei documenti realizzati durante questo periodo.
\end{itemize}

\textbf{VI periodo: dal 2020-01-15 al 2020-01-20}
\begin{itemize}
	\item \textbf{Preparazione alla discussione}: realizzazione della presentazione e preparazione individuale e di gruppo alla discussione.
\end{itemize}

\subsection{Progettazione architetturale}
La Progettazione\glosp architetturale è la seconda attività: inizia il 2020-01-22, giorno successivo alla prima revisione e finisce il 2020-02-20, giorno precedente alla seconda revisione. In questa attività ci siamo occupati della progettazione della codifica del codice del prodotto\glo.

\subsubsection{Ruoli attivi}
\begin{itemize}
	\item Responsabile di progetto\glo;
	\item Amministratore di progetto\glo;
	\item Analista;
	\item Progettista;
	\item Programmatore;
	\item Verificatore.
\end{itemize}

\subsubsection{Periodi}
\textbf{I periodo: dal 2020-01-22 al 2020-02-20}
\begin{itemize}
	\item \textbf{Pianificazione delle attività}: gestione delle risorse disponibili, suddivisione e pianificazione di tutte le attività che devono essere svolte in questo periodo;
	\item \textbf{Normazione}: revisione e, se necessario, aggiornamento delle norme di progetto\glo;
	\item \textbf{Gestione della qualità}: revisione delle metodologie per mantenere e garantire la qualità del prodotto\glo;
	\item \textbf{Ricerca di strumenti e tecnologie}: ricerca delle tecnologie necessarie allo sviluppo del progetto\glo;
	\item \textbf{Revisione dei casi d'uso}: revisione ed, eventualmente, modifica dei casi d'uso in seguito a indicazioni da parte del proponente;
	\item \textbf{Revisione dei requisiti}: revisione ed, eventualmente, modifica dei requisiti analizzati durante la prima attività in seguito a modifiche di casi d'uso o di altre indicazioni; 
	\item \textbf{Verifica}: attività di controllo dei documenti modificati durante questo periodo.
\end{itemize}

\textbf{II periodo: dal 2020-02-21 al 2020-03-08}
\begin{itemize}
	\item \textbf{Studio di strumenti e tecnologie}: studio delle tecnologie e degli strumenti necessari per lo sviluppo del prodotto\glosp richiesto dal proponente;
	\item \textbf{Normazione}: revisione e aggiornamento delle norme del progetto\glo;
	\item \textbf{Pianificazione delle attività}: gestione delle risorse disponibili, suddivisione e pianificazione di tutte le attività che devono essere svolte in questo periodo;
	\item \textbf{Progettazione}\glosp\textbf{proof of concept}\glo: pianificazione dello sviluppo di una proof of concept\glosp per dimostrare la fattibilità del prodotto\glo;
	\item \textbf{Codifica proof of concept}\glo: sviluppo del codice per realizzare il proof of concept\glosp precedentemente pianificato;
	\item \textbf{Stesura lettera di presentazione}: stesura della \textit{Lettera di Presentazione} in cui si propone una soluzione alla richiesta del proponente;
	\item \textbf{Verifica}: attività di controllo dei documenti realizzati durante questo periodo.
\end{itemize}

\textbf{III periodo: dal 2020-03-09 al 2020-03-15}
\begin{itemize}
	\item \textbf{Preparazione alla discussione}: realizzazione della presentazione e preparazione individuale e di gruppo alla discussione.
\end{itemize}

\subsection{Progettazione di dettaglio e codifica}
La Progettazione\glosp di dettaglio e codifica è la terza attività: inizia il 2020-03-17, giorno successivo alla seconda revisione e finisce il 2020-04-19, giorno precedente alla terza revisione. Durante lo svolgimento di questa attività ci occuperemo principalmente della codifica del codice del prodotto\glo.

\subsubsection{Ruoli attivi}
\begin{itemize}
	\item Responsabile di progetto\glo;
	\item Amministratore di progetto\glo;
	\item Progettista;
	\item Programmatore;
	\item Verificatore.
\end{itemize}

\subsubsection{Periodi}
\textbf{I periodo: dal 2020-03-17 al 2020-03-29}
\begin{itemize}
	\item \textbf{Normazione}: revisione e, se necessario, aggiornamento delle norme del progetto\glo;
	\item \textbf{Ricerca di strumenti e tecnologie}: ricerca e studio degli strumenti e tecnologie utilizzate per la codifica del codice;
	\item \textbf{Pianificazione delle attività}: gestione delle risorse disponibili, suddivisione e pianificazione di tutte le attività che devono essere svolte in questo periodo;
	\item \textbf{Progettazione}\glo: progettazione\glosp della struttura del codice in modo da realizzare un prodotto\glosp uniforme e coeso;
	\item \textbf{Codifica}: scrittura del codice del prodotto\glosp seguendo le indicazioni definite nel documento \textit{Norme di Progetto} e nella progettazione\glosp indicata sopra; 
	\item \textbf{Verifica}: attività di controllo dei documenti e del codice realizzati durante questo periodo.
\end{itemize}

\textbf{II periodo: dal 2020-03-30 al 2020-04-12}
\begin{itemize}
	\item \textbf{Normazione}: revisione e, se necessario, aggiornamento delle norme di progetto\glo;
	\item \textbf{Ricerca di strumenti e tecnologie}: ricerca e studio degli strumenti e tecnologie utilizzate per la codifica del codice;
	\item \textbf{Pianificazione delle attività}: gestione delle risorse disponibili, suddivisione e pianificazione di tutte le attività che devono essere svolte in questo periodo;
	\item \textbf{Codifica}: scrittura del codice del prodotto\glosp seguendo le indicazioni del documento \textit{Norme di Progetto} e della progettazione\glosp indicata sopra;
	\item \textbf{Scrittura manuale}: scrittura del \textit{Manuale d'Uso} per descrivere come deve l'utente finale deve utilizzare il software;
	\item \textbf{Stesura lettera di presentazione}: stesura della \textit{Lettera di Presentazione} in cui si propone una soluzione alla richiesta del proponente;
	\item \textbf{Verifica}: attività di controllo dei documenti realizzati durante questo periodo.
\end{itemize}

\textbf{III periodo: dal 2020-04-14 al 2020-04-19}
\begin{itemize}
	\item \textbf{Preparazione alla discussione}: realizzazione della presentazione e preparazione individuale e di gruppo alla discussione.
\end{itemize}

\subsection{Validazione e collaudo}
Validazione\glosp e collaudo è la quarta attività: inizia il 2020-04-21, giorno successivo alla terza revisione e si conclude il 2020-05-17, giorno precedente aell'ultima revisione. In questa attività ci occuperemo del collaudo e della validazione\glosp del prodotto\glosp per verificare che tutti i requisiti definiti come obbligatori nell'\textit{Analisi dei Requisiti} sono stati soddisfatti.

\subsubsection{Ruoli attivi}
\begin{itemize}
	\item Amministratore di progetto\glo;
	\item Progettista;
	\item Programmatore;
	\item Verificatore.
\end{itemize}

\subsubsection{Periodi}
\textbf{I periodo: dal 2020-04-21 al 2020-04-30}
\begin{itemize}
	\item \textbf{Normazione}: revisione e, se necessario, aggiornamento delle norme di progetto\glo;
	\item \textbf{Ricerca di strumenti e tecnologie}: ricerca e studio degli strumenti e tecnologie utilizzate per la codifica del codice;
	\item \textbf{Pianificazione delle attività}: gestione delle risorse disponibili, suddivisione e pianificazione di tutte le attività che devono essere svolte in questo periodo;
	\item \textbf{Gestione qualità}: revisione, e se necessario, le metodologie per la mantenere il livello di qualità prestabilito;
	\item \textbf{Revisione dei requisiti}: revisionati e, se necessario, aggiornato i requisiti del progetto\glo;
	\item \textbf{Verifica}: ultima attività svolta in cui si esegue il controllo dei documenti realizzati durante questo periodo.
\end{itemize}

\textbf{II periodo: dal 2020-05-01 al 2020-05-10}
\begin{itemize}
	\item \textbf{Revisione manuale d'uso}: revisione del \textit{Manuale d'Uso};
	\item \textbf{Codifica}: scrittura del codice del prodotto\glosp seguendo le indicazioni delle \textit{Norme di Progetto} e della progettazione\glosp indicata sopra;
	\item \textbf{Test e collaudo}: scrittura dei test necessari per il corretto funzionamento del prodotto\glo;
	\item \textbf{Stesura lettera di presentazione}: stesura della \textit{Lettera di Presentazione} che in cui si propone una soluzione alla richiesta del proponente;
	\item \textbf{Verifica}: attività di controllo di documenti e codice realizzati durante questo periodo.
\end{itemize}

\textbf{III periodo: dal 2020-05-12 al 2020-05-17}
\begin{itemize}
	\item \textbf{Preparazione alla discussione}: realizzazione della presentazione e preparazione individuale e di gruppo alla discussione.
\end{itemize}