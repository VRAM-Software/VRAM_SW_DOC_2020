%Mettere al suo interno una suddivisione con i 4 periodi che corrispondono alle scadenze. Per ognuno suddividere ulteriormente in periodi di analisi da definire. Alla fine di ognuno dei 4 periodi principali(delle 4 scadenze) fare un Gantt
%per ogni attività bisogna trovare un sottoperiodo e per ognuno un grafico di Gantt
\section{Pianificazione} 
Per rispettare le scadenze elencate nella sezione calendario attività il nostro gruppo ha deciso di suddividere lo sviluppo del prodotto\glosp e della sua documentazione nelle seguenti quattro attività:
\begin{itemize}
	\item Analisi dei Requisiti;
	\item Progettazione\glosp architetturale;
	\item Progettazione\glosp di dettaglio e codifica;
	\item Validazione\glosp e collaudo.
\end{itemize}
La pianificazione ha lo scopo di rendere gestibile lo sviluppo del progetto\glosp suddividendolo in attività che, singolarmente, sono più facilmente realizzabili. Per descrivere ogni attività abbiamo deciso di suddividere ulteriormente ogni attività in periodi e abbiamo elencato i ruoli attivi durante lo svolgimento di essa.

\subsection{Analisi dei Requisiti}
L'analisi dei requisiti è la prima attività che abbiamo definito, inizia il giorno dopo la formazione dei gruppi (2019-11-14) e termina un giorno prima della presentazione del progetto\glosp (2020-01-20). In questa attività ci siamo principalmente interessati all'analisi di tutte le informazioni riguardanti il prodotto\glosp che si vuole sviluppare e all'organizzazione e suddivisione delle risorse.

\subsubsection{Ruoli attivi}
\begin{itemize}
	\item Responsabile di progetto\glo;
	\item Amministratore di progetto\glo;
	\item Analista;
	\item Verificatore.
\end{itemize}

\subsubsection{Periodi}
Abbiamo suddiviso l'attività della Analisi dei Requisiti nei seguenti sei periodi. \\*
\textbf{I periodo: dal 2019-11-14 al 2019-11-27}
\begin{itemize}
	\item \textbf{Discussione dei capitolati}\glo: discussione interna, analizzando i fattori positivi e negativi di ogni capitolato\glo, per scegliere quale progetto\glosp realizzare;
	\item \textbf{Normazione}: discussione interna del gruppo riguardo alle regole da seguire per lo sviluppo della documentazione del progetto\glosp e iniziato il documento interno \textit{Norme di progetto v. 1.1.1} nelle sue sezioni riguardanti: \textit{Studio di Fattibilità}, strumenti da utilizzare, documentazione, gestione della configurazione, processo\glosp di verifica e processi\glosp organizzativi;
	\item \textbf{Ricerca di strumenti e tecnologie}: iniziato la ricerca di gruppo e individuale sugli strumenti e tecnologie da conoscere durante lo sviluppo della documentazione del progetto\glo;
	\item \textbf{Definizione dei ruoli}: suddivisi i ruoli per l'attività presa in considerazione; 
	\item \textbf{Pianificazione delle attività}: gestione delle risorse disponibili, suddivisione e pianificazione di tutte le attività che devono essere svolte in questo periodo;
	\item \textbf{Verifica}: ultima attività da svolgere in cui si esegue il controllo dei documenti realizzati durante questo periodo.
\end{itemize}

\textbf{II periodo: dal 2019-11-28 al 2019-12-08}
\begin{itemize}
	\item \textbf{Studio di Fattibilità}: formalizzata la scelta del capitolato\glosp con la stesura del documento \textit{Studio di Fattibilità v. 1.1.1};
	\item \textbf{Normazione}: revisionato e aggiornato le regole e le norme del progetto\glosp riguardanti: \textit{Studio di Fattibilità} e strumenti da utilizzare;
	\item \textbf{Ricerca di strumenti e tecnologie}: continuato la ricerca di strumenti e tecnologie da utilizzare per lo sviluppo del prodotto\glo;
	\item \textbf{Definizione dei ruoli}: suddivisi i ruoli per l'attività presa in considerazione; 
	\item \textbf{Pianificazione delle attività}: gestione delle risorse disponibili, suddivisione e pianificazione di tutte le attività che devono essere svolte in questo periodo;
	\item \textbf{Verifica}: ultima attività da svolgere in cui si esegue il controllo dei documenti realizzati durante questo periodo.
\end{itemize}

\textbf{III periodo: dal 2019-12-09 al 2020-12-22}
\begin{itemize}
	\item \textbf{Normazione}: revisionato e aggiornato le regole e le norme del progetto\glosp riguardanti: \textit{Analisi dei Requisiti}, \textit{Piano di Progetto} e strumenti da utilizzare;
	\item \textbf{Ricerca di strumenti e tecnologie}: continuato la ricerca di strumenti e tecnologie da utilizzare per lo sviluppo del prodotto\glo;
	\item \textbf{Pianificazione delle attività}: gestione delle risorse disponibili, suddivisione e pianificazione di tutte le attività che devono essere svolte in questo periodo;
	\item \textbf{Analisi dei requisiti}: individuazione dei requisiti del prodotto\glosp in seguito a: incontri interni, una analisi dei casi d'uso, una analisi del capitolato\glosp e da incontri esterni col proponente.
	\item \textbf{Verifica}: ultima attività da svolgere in cui si esegue il controllo dei documenti realizzati durante questo periodo.
\end{itemize}

\textbf{IV periodo: dal 2019-12-23 al 2020-01-02}
\begin{itemize}
	\item \textbf{Normazione}: revisionato e aggiornato le regole e le norme del progetto\glosp riguardanti: \textit{Analisi dei Requisiti}, \textit{Piano di Qualifica}, gestione della qualità e strumenti da utilizzare;
	\item \textbf{Gestione della qualità}: iniziata discussione interna riguardo a come mantenere uno standard per garantire la qualità di tutti i documenti realizzati;
	\item \textbf{Ricerca di strumenti e tecnologie}: continuato la ricerca di strumenti e tecnologie da utilizzare per lo sviluppo del prodotto\glo;
	\item \textbf{Pianificazione delle attività}: gestione delle risorse disponibili, suddivisione e pianificazione di tutte le attività che devono essere svolte in questo periodo;
	\item \textbf{Definizione dei casi d'uso}: realizzati i casi d'uso del prodotto\glosp richiesto dal proponente;
	\item \textbf{Verifica}: ultima attività da svolgere in cui si esegue il controllo dei documenti realizzati durante questo periodo.
\end{itemize}


\textbf{V periodo: dal 2020-12-31 al 2020-01-14}
\begin{itemize}
	\item \textbf{Normazione}: revisionato e aggiornato le regole e le norme del progetto\glosp riguardanti: \textit{Piano di Qualifica}, \textit{Piano di Progetto} e strumenti da utilizzare;
	\item \textbf{Gestione della qualità}: individuati le regole e i metodi per mantenere e garantire la qualità del prodotto\glo;
	\item \textbf{Ricerca di strumenti e tecnologie}: continuato la ricerca di strumenti e tecnologie da utilizzare per lo sviluppo del prodotto\glo;
	\item \textbf{Pianificazione delle attività}: gestione delle risorse disponibili, suddivisione e pianificazione di tutte le attività che devono essere svolte in questo periodo;
	\item \textbf{Analisi dei rischi}: discussione interna dei possibili rischi nella realizzazione del progetto\glo;
	\item \textbf{Stesura lettera di presentazione}: stesura della lettera che dovrà essere consegnata in cui si propone una soluzione alla richiesta del proponente;
	\item \textbf{Verifica}: ultima attività da svolgere in cui si esegue il controllo dei documenti realizzati durante questo periodo.
\end{itemize}

\textbf{VI periodo: dal 2020-01-15 al 2020-01-20}
\begin{itemize}
	\item \textbf{Preparazione alla discussione}: realizzazione della presentazione e preparazione individuale e di gruppo alla discussione.
\end{itemize}

\subsection{Progettazione architetturale}
La seconda attività da noi individuata è la Progettazione\glosp architetturale, inizia il 2020-01-22 e finisce il 2020-02-20. In questa attività ci siamo occupati della progettazione della codifica del codice del prodotto\glo.

\subsubsection{Ruoli attivi}
\begin{itemize}
	\item Responsabile di progetto\glo;
	\item Amministratore di progetto\glo;
	\item Analista;
	\item Progettista;
	\item Programmatore;
	\item Verificatore.
\end{itemize}

\subsubsection{Periodi}
\textbf{I periodo: dal 2020-01-22 al 2020-02-20}
\begin{itemize}
	\item \textbf{Pianificazione delle attività}: gestione delle risorse disponibili, suddivisione e pianificazione di tutte le attività che devono essere svolte in questo periodo;
	\item \textbf{Normazione}: revisionato e, se necessario, aggiornato le regole e le \textit{Norme di Progetto};
	\item \textbf{Gestione della qualità}: revisionato le metodologie per mantenere la qualità del prodotto\glo;
	\item \textbf{Ricerca di strumenti e tecnologie}: ricerca delle tecnologie da conoscere per lo sviluppo del progetto\glo;
	\item \textbf{Revisione dei casi d'uso}: revisionato, e in caso modificato, i casi d'uso in seguito a indicazioni da parte del proponente;
	\item \textbf{Revisione dei requisiti}: revisionato, e in caso modificato, i requisiti analizzati durante la prima attività in seguito a modifiche degli use case o da altre indicazioni; 
	\item \textbf{Verifica}: ultima attività svolta in cui si esegue il controllo dei documenti modificati durante questo periodo.
\end{itemize}

\textbf{II periodo: dal 2020-02-21 al 2020-03-08}
\begin{itemize}
	\item \textbf{Studio di strumenti e tecnologie}: studio delle tecnologie e degli strumenti necessari per lo sviluppo del prodotto\glosp richiesto dal proponente;
	\item \textbf{Normazione}: aggiornato e revisionato le norme e le regole del progetto\glo;
	\item \textbf{Pianificazione delle attività}: gestione delle risorse disponibili, suddivisione e pianificazione di tutte le attività che devono essere svolte in questo periodo;
	\item \textbf{Progettazione}\glosp\textbf{proof of concept}\glo: pianificato lo sviluppo di una proof of concept\glosp per dimostrare la fattibilità del prodotto\glo;
	\item \textbf{Codifica proof of concept}\glo: sviluppato codice per realizzare il proof of concept\glosp precedentemente pianificato;
	\item \textbf{Stesura lettera di presentazione}: stesura della lettera che dovrà essere consegnata in cui si propone una soluzione alla richiesta del proponente;
	\item \textbf{Verifica}: ultima attività svolta in cui si esegue il controllo dei documenti realizzati durante questo periodo.
\end{itemize}

\textbf{III periodo: dal 2020-03-09 al 2020-03-15}
\begin{itemize}
	\item \textbf{Preparazione alla discussione}: realizzazione della presentazione e preparazione individuale e di gruppo alla discussione.
\end{itemize}

\subsection{Progettazione di dettaglio e codifica}
La terza attività individuata dal nostro gruppo: Progettazione\glosp di dettaglio e codifica, inizia il 2020-03-17 e finisce il 2020-04-19. Durante lo svolgimento di questa attività ci occuperemo principalmente della codifica del codice del prodotto\glo.

\subsubsection{Ruoli attivi}
\begin{itemize}
	\item Responsabile di progetto\glo;
	\item Amministratore di progetto\glo;
	\item Progettista;
	\item Programmatore;
	\item Verificatore.
\end{itemize}

\subsubsection{Periodi}
\textbf{I periodo: dal 2020-03-17 al 2020-03-29}
\begin{itemize}
	\item \textbf{Normazione}: revisionato e, se necessario, aggiornato le regole e le norme del progetto\glo;
	\item \textbf{Ricerca di strumenti e tecnologie}: ricerca e studio degli strumenti e tecnologie utilizzate per la codifica del codice;
	\item \textbf{Pianificazione delle attività}: gestione delle risorse disponibili, suddivisione e pianificazione di tutte le attività che devono essere svolte in questo periodo;
	\item \textbf{Progettazione}\glo: progettazione\glosp della struttura del codice in modo da realizzare un prodotto\glosp uniforme e coeso;
	\item \textbf{Codifica}: scrittura del codice del prodotto\glosp seguendo le indicazioni delle \textit{Norme di Progetto} e della progettazione\glosp indicata sopra; 
	\item \textbf{Verifica}: ultima attività svolta in cui si esegue il controllo dei documenti e del codice realizzati durante questo periodo.
\end{itemize}

\textbf{II periodo: dal 2020-03-30 al 2020-04-12}
\begin{itemize}
	\item \textbf{Normazione}: revisionato e, se necessario, aggiornato le regole e le \textit{Norme di Progetto};
	\item \textbf{Ricerca di strumenti e tecnologie}: ricerca e studio degli strumenti e tecnologie utilizzate per la codifica del codice;;
	\item \textbf{Pianificazione delle attività}: gestione delle risorse disponibili, suddivisione e pianificazione di tutte le attività che devono essere svolte in questo periodo;
	\item \textbf{Codifica}: scrittura del codice del prodotto\glosp seguendo le indicazioni delle \textit{Norme di Progetto} e della progettazione\glosp indicata sopra;
	\item \textbf{Scrittura manuale}: scrittura del \textit{Manuale d'Uso} per descrivere come deve essere utilizzato il software;
	\item \textbf{Stesura lettera di presentazione}: stesura della lettera che dovrà essere consegnata in cui si propone una soluzione alla richiesta del proponente;
	\item \textbf{Verifica}: ultima attività svolta in cui si esegue il controllo dei documenti realizzati durante questo periodo.
\end{itemize}

\textbf{III periodo: dal 2020-04-14 al 2020-04-19}
\begin{itemize}
	\item \textbf{Preparazione alla discussione}: realizzazione della presentazione e preparazione individuale e di gruppo alla discussione.
\end{itemize}

\subsection{Validazione e collaudo}
La quarta e ultima attività: Validazione\glosp e collaudo, inizia il 2020-04-21 e si conclude il 2020-04-30. In questa attività ci occuperemo del collaudo e della validazione\glosp del prodotto\glosp per verificare che tutti i requisiti definiti come obbligatori nell'\textit{Analisi dei Requisiti} sono stati soddisfatti.

\subsubsection{Ruoli attivi}
\begin{itemize}
	\item Amministratore di progetto\glo;
	\item Progettista;
	\item Programmatore;
	\item Verificatore.
\end{itemize}

\subsubsection{Periodi}
\textbf{I periodo: dal 2020-04-21 al 2020-04-30}
\begin{itemize}
	\item \textbf{Normazione}: revisionato e, se necessario, aggiornato le regole e le \textit{Norme di Progetto};
	\item \textbf{Ricerca di strumenti e tecnologie}: ricerca e studio degli strumenti e tecnologie utilizzate per la codifica del codice;
	\item \textbf{Pianificazione delle attività}: gestione delle risorse disponibili, suddivisione e pianificazione di tutte le attività che devono essere svolte in questo periodo;
	\item \textbf{Gestione qualità}: revisionato, se necessario, le metodologie per la mantenere il livello di qualità prestabilito;
	\item \textbf{Revisione dei requisiti}: revisionati e, se necessario, aggiornato i requisiti del progetto\glo;
	\item \textbf{Verifica}: ultima attività svolta in cui si esegue il controllo dei documenti realizzati durante questo periodo.
\end{itemize}

\textbf{II periodo: dal 2020-05-01 al 2020-05-10}
\begin{itemize}
	\item \textbf{Revisione manuale d'uso}: revisione del \textit{Manuale d'Uso};
	\item \textbf{Codifica}: scrittura del codice del prodotto\glosp seguendo le indicazioni delle \textit{Norme di Progetto} e della progettazione\glosp indicata sopra;
	\item \textbf{Test e collaudo}: scrittura dei test necessari per testare il corretto funzionamento del prodotto\glo;
	\item \textbf{Stesura lettera di presentazione}: stesura della lettera che dovrà essere consegnata in cui si propone una soluzione alla richiesta del proponente;
	\item \textbf{Verifica}: ultima attività svolta in cui si esegue il controllo dei documenti e del codice realizzati durante questo periodo.
\end{itemize}

\textbf{III periodo: dal 2020-05-12 al 2020-05-17}
\begin{itemize}
	\item \textbf{Preparazione alla discussione}: realizzazione della presentazione e preparazione individuale e di gruppo alla discussione.
\end{itemize}