%Mettere al suo interno una suddivisione con i 4 periodi che corrispondono alle scadenze. Per ognuno suddividere ulteriormente in periodi di analisi da definire. Alla fine di ognuno dei 4 periodi principali(delle 4 scadenze) fare un Gantt
%per ogni attività bisogna trovare un sottoperiodo e per ognuno un grafico di Gantt
\section{Pianificazione} 
Per rispettare le scadenze elencate in: [link a scadenze]; il nostro gruppo ha deciso di suddividere lo sviluppo del prodotto e della sua documentazione nelle seguenti quattro attività:
\begin{itemize}
	[queste 4 sono attività]
	\item Analisi dei Requisiti;
	\item Progettazione architetturale;
	\item Progettazione di dettaglio e codifica;
	\item Validazione\glosp e collaudo.
\end{itemize}
La pianificazione ha lo scopo di rendere gestibile lo sviluppo del progetto suddividendolo in attività che, singolarmente, sono più facilmente realizzabili. Per descrivere ogni attività abbiamo deciso di suddividere ulteriormente ogni attività in periodi e abbiamo elencato i ruoli attivi durante lo svolgimento di essa.

\subsection{Analisi dei Requisiti}
L'analisi dei requisiti è la prima attività che abbiamo definito, inizia il giorno dopo la formazione dei gruppi (2019-11-14) e termina un giorno prima della presentazione del progetto (2020-01-20).
{descrivere cos'è l'analisi dei requisiti}

[Ruoli attivi]
\begin{itemize}
	\item Responsabile di progetto;
	\item Amministratore di progetto;
	\item Analista;
	\item Verificatore.
\end{itemize}

Abbiamo suddiviso l'attività della Analisi dei Requisiti nei seguenti quattro periodi.
\textbf{I periodo: dal 2019-11-14 al 2019-12-12}
\begin{itemize}
	\item Discussione dei capitolati: discussione interna scegliere quale capitolato\glosp realizzare;
	\item Studio di Fattibilità: formalizzata la scelta del capitolato\glo , analizzando i fattori positivi e negativi di ogni capitolato\glo ;
	\item Normazione: discussione interna del gruppo riguardo alle regole da seguire per lo sviluppo della documentazione del progetto e iniziato il documento interno: \textit{Norme di Progetto v1.1.1};
	\item Ricerca di strumenti e tecnologie: iniziato la ricerca di gruppo e individuale sugli strumenti e tecnologie da conoscere durante lo sviluppo della documentazione del progetto;
	\item Definizione dei ruoli: suddivisi i ruoli per l'attività presa in considerazione; 
	\item Pianificazione delle attività: gestione delle risorse disponibili, suddivisione e pianificazione di tutte le attività che devono essere svolte in questo periodo;
	\item Verifica: ultima attività da svolgere in cui si esegue il controllo dei documenti realizzati durante questo periodo.
\end{itemize}

\textbf{II periodo: dal 2019-12-22 al 2020-01-04}
\begin{itemize}
	\item Normazione: aggiunte modifiche al documento: \textit{Norme di progetto v1.1.1} in seguito alla scrittura dei primi documenti e definite regole riguardo allo sviluppo del progetto;
	\item Gestione della qualità: iniziata discussione interna riguardo a come mantenere uno standard per garantire la qualità di tutti i documenti realizzati;
	\item Ricerca di strumenti e tecnologie: continuato la ricerca di strumenti e tecnologie da utilizzare per lo sviluppo del prodotto;
	\item Pianificazione delle attività: gestione delle risorse disponibili, suddivisione e pianificazione di tutte le attività che devono essere svolte in questo periodo;
	\item Definizione dei casi d'uso: realizzati i casi d'uso del prodotto richiesto dal proponente;
	\item Analisi dei rischi: discussione interna dei possibili rischi nella realizzazione del progetto;
	\item Verifica: ultima attività da svolgere in cui si esegue il controllo dei documenti realizzati durante questo periodo.
\end{itemize}

\textbf{III periodo: dal 2020-01-05 al 2020-01-13}
\begin{itemize}
	\item Normazione: aggiornato il documento: \textit{Norme di progetto v1.1.1} dopo la verifica dei documenti precedentemente realizzati;
	\item Gestione della qualità: individuati le regole e i metodi per mantenere e garantire la qualità del prodotto;
	\item Ricerca di strumenti e tecnologie: ricerca delle tecnologie e strumenti utili per lo sviluppo del prodotto;
	\item Pianificazione delle attività: gestione delle risorse disponibili, suddivisione e pianificazione di tutte le attività che devono essere svolte in questo periodo;
	\item Analisi dei requisiti: individuazione dei requisiti del prodotto in seguito a: incontri interni, una analisi dei casi d'uso, una analisi del capitolato e da incontri esterni col proponente: \textit{Zucchetti}.
	\item Stesura lettera di presentazione: realizzazione della lettera che dovrà essere consegnata in cui si propone una soluzione al proponente;
	\item Verifica: ultima attività da svolgere in cui si esegue il controllo dei documenti realizzati durante questo periodo.
\end{itemize}

\textbf{IV periodo: dal 2020-01-15 al 2020-01-20}
\begin{itemize}
	\item Preparazione alla discussione: realizzazione della presentazione e preparazione individuale e di gruppo alla discussione.
\end{itemize}

[diagrammi di Gantt]

\subsection{Progettazione architetturale}
[definire Introduzione + scadenze ( 2020-01-22 / 2020-03-15 ) + ruoli attivi]

[Ruoli attivi]
\begin{itemize}
	\item Responsabile di progetto;
	\item Amministratore di progetto;
	\item Analista;
	\item Progettista;
	\item Programmatore;
	\item Verificatore.
\end{itemize}

\textbf{I periodo: dal 2020-01-22 al 2020-02-20}
\begin{itemize}
	\item Pianificazione delle attività: gestione delle risorse disponibili, suddivisione e pianificazione di tutte le attività che devono essere svolte in questo periodo;
	\item Normazione: revisionato le norme e regole definite nell'attività precedente (Analisi dei Requisiti);
	\item Gestione della qualità: revisionato le metodologie per mantenere la qualità del prodotto;
	\item Ricerca di strumenti e tecnologie: ricerca delle tecnologie da conoscere per lo sviluppo del progetto;
	\item Use case:; % ancora use case?
	\item Analisi dei requisiti:; % ancora analisi dei requisiti?
	\item Verifica: ultima attività svolta in cui si esegue il controllo dei documenti realizzati durante questo periodo.
\end{itemize}

\textbf{II periodo: dal 2020-02-21 al 2020-03-08}
\begin{itemize}
	\item Studio di strumenti e tecnologie: studio delle tecnologie e degli strumenti necessari per lo sviluppo del prodotto richiesto dal proponente;
	\item Normazione: aggiornato e revisionato le regole del progetto, definite nel documento: \textit{Norme di progetto v1.1.1};
	\item Pianificazione delle attività: gestione delle risorse disponibili, suddivisione e pianificazione di tutte le attività che devono essere svolte in questo periodo;
	\item Progettazione proof of concepts:;
	\item Codifica proof of concepts:;
	\item Lettera di presentazione:;
	\item Verifica: ultima attività svolta in cui si esegue il controllo dei documenti realizzati durante questo periodo.
\end{itemize}

\textbf{III periodo: dal 2020-03-09 al 2020-03-15}
\begin{itemize}
	\item Preparazione alla discussione: realizzazione della presentazione e preparazione individuale e di gruppo alla discussione.
\end{itemize}

[diagrammi di Gantt]

\subsection{Progettazione di dettaglio e codifica}
[definire Introduzione + scadenze ( 2020-03-17 / 2020-04-19) + ruoli attivi]

[Ruoli attivi]
\begin{itemize}
	\item Responsabile di progetto:;
	\item Amministratore di progetto:;
	\item Progettista:;
	\item Programmatore:;
	\item Verificatore.
\end{itemize}

\textbf{I periodo: dal 2020-03-17 al 2020-03-29}
\begin{itemize}
	\item Normazione:;
	\item Ricerca di strumenti e tecnologie:;
	\item Pianificazione delle attività: gestione delle risorse disponibili, suddivisione e pianificazione di tutte le attività che devono essere svolte in questo periodo;
	\item Progettazione:;
	\item Codifica:;
	\item Verifica: ultima attività svolta in cui si esegue il controllo dei documenti realizzati durante questo periodo.
\end{itemize}

\textbf{II periodo: dal 2020-03-30 al 2020-04-12}
\begin{itemize}
	\item Normazione:;
	\item Ricerca di strumenti e tecnologie:;
	\item Pianificazione delle attività: gestione delle risorse disponibili, suddivisione e pianificazione di tutte le attività che devono essere svolte in questo periodo;
	\item Codifica:;
	\item Scrittura manuale:;
	\item Lettera di presentazione:;
	\item Verifica: ultima attività svolta in cui si esegue il controllo dei documenti realizzati durante questo periodo.
\end{itemize}

\textbf{III periodo: dal 2020-04-14 al 2020-04-19}
\begin{itemize}
	\item Preparazione alla discussione: realizzazione della presentazione e preparazione individuale e di gruppo alla discussione.
\end{itemize}

[diagrammi di Gantt]

\subsection{Validazione e collaudo}
[definire Introduzione + scadenze (2020-04-21/	2020-05-17) + ruoli attivi]

[Ruoli attivi]
\begin{itemize}
	\item Amministratore di progetto:;
	\item Progettista:;
	\item Programmatore:;
	\item Verificatore.
\end{itemize}

\textbf{I periodo: dal 2020-04-21 al 2020-04-30}
\begin{itemize}
	\item Normazione:;
	\item Ricerca di strumenti e tecnologie:;
	\item Pianificazione delle attività: gestione delle risorse disponibili, suddivisione e pianificazione di tutte le attività che devono essere svolte in questo periodo;
	\item Gestione qualità:;
	\item Analisi dei requisiti:;
	\item Verifica: ultima attività svolta in cui si esegue il controllo dei documenti realizzati durante questo periodo.
\end{itemize}

\textbf{II periodo: dal 2020-05-01 al 2020-05-10}
\begin{itemize}
	\item Scrittura manuale:;
	\item Codifica:;
	\item Test e collaudo:;
	\item Lettera di presentazione:;
	\item Verifica: ultima attività svolta in cui si esegue il controllo dei documenti realizzati durante questo periodo.
\end{itemize}

\textbf{III periodo: dal 2020-05-12 al 2020-05-17}
\begin{itemize}
	\item Preparazione alla discussione: realizzazione della presentazione e preparazione individuale e di gruppo alla discussione.
\end{itemize}

[diagrammi di Gantt]



