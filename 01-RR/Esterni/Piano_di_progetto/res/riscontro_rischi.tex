\appendix
\section{Riscontro dei rischi}
    \subsection{Attualizzazione dei rischi 2020-01-14}
        \rowcolors{2}{gray!25}{gray!15}
	    \begin{longtable} {
		    >{}p{10mm} 
		    >{}p{24mm}
		    >{}p{32mm} 
            >{}p{32mm}
		    }
	    \rowcolor{gray!50}
        \textbf{ID} & \textbf{Tempistiche} & \textbf{Descrizione} & \textbf{Manutenzione migliorativa}	\TBstrut \\
        RT1 & Creazione del template \LaTeX e della repository\glo & Nella prima fase del progetto\glosp ci sono state difficoltà dovute all'apprendimento di \LaTeX, il cui funzionamento era sconosciuto a tutti i membri del gruppo, e alla gestione di GitHub, noto a tutti i componenti, ma mai utilizzato al pieno delle sue possibilità & Per quanto riguarda \LaTeX è stato necessario un breve periodo di autoapprendimento da parte di alcuni componenti del gruppo, mentre per GitHub i membri del gruppo che hanno partecipato al corso di Tecnologie Open Source hanno contribuito a proporre delle soluzioni studiate a lezione  \TBstrut \\ [2mm]
        RG1 & \textit{Glossario} & È stato oggetto di una lunga discussione il metodo di stesura del \textit{Glossario}, soprattutto la scelta dei termini da inserire in esso & Alla fine di una condivisione di opinioni il responsabile ha delineato delle linee guida per la scelta dei lemmi da inserire nel \textit{Glossario} che hanno messo d'accordo i componenti del gruppo \TBstrut \\ [2mm]
        RG2 & Rischio permanente & I membri del gruppo sono dovuti andare incontro ad inevitabili sovrapposizioni di impegni durante la stesura dei documenti, tale rischio ha influenzato l'incidenza di RS1 & Ogni componente, per quanto possibile, ha cercato di ritagliarsi del tempo per rispettare le milestone e, eventualmente, ha chiesto assistenza ad altri componenti più disponibili \TBstrut \\ [2mm]
        RO2 & Gestione della repository\glo & Collegandosi a RT1 la scarsa esperienza nell'utilizzo di GitHub ha generato lacune nella corretta gestione dell'issue tracking system integrato nella piattaforma & I membri del gruppo che hanno seguito il corso di Tecnologie Open Source, e quindi più pratici nei sistemi di versionamento\glo, hanno risolto le inconsistenze individuate in GitHub \TBstrut \\ [2mm]
        RO3 & Stesura primi documenti & Con la stesura dei primi documenti alcuni componenti hanno trovato difficoltà ad adattarsi ai vari ruoli assegnati & I membri del gruppo si sono assistiti a vicenda cercando di colmare le debolezze evidenziate sotto certi aspetti dando delle linee guida sui processi\glosp di analisi e di verifica dei documenti \TBstrut \\ [2mm]
        RR1 & \textit{Studio di Fattibilità} e \textit{Analisi dei Requisiti} & Non avendo esperienza sugli strumenti da utilizzare certe specifiche richieste non sono state chiare o sono state mal interpretate & Per ovviare a questo problema sono stati indetti due incontri con il proponente che hanno contribuito a risolvere i dubbi sorti durante l'analisi dei documenti \TBstrut \\ [2mm]
        RS1 & Rischio permanente & Data la scarsa esperienza nella gestione di una tale mole di lavoro e gli impegni personali dei componenti certe milestone sono risultate imprecise e difficilmente rispettabili & Pur essendo stato un problema persistente man mano si è diminuita l'imprecisione delle scadenze inoltre, il proseguimento del lavoro in gruppo dovrebbe portare ad avere maggiore consapevolezza dei tempi e quindi il calo di incidenza del rischio \TBstrut \\ [2mm]
        \rowcolor{white}
        \caption{Attualizzazione dei rischi 2020-01-14}
        \end{longtable}