\section{Rischi tecnologici}
	\rowcolors{2}{gray!25}{gray!15}
	\setcounter{table}{0}
	\begin{longtable}{
		>{}p{10mm} 
		>{}p{24mm}
		>{}p{32mm} 
        >{}p{32mm}
        >{}p{32mm}
		}
	\rowcolor{gray!50}
		\textbf{ID} & \textbf{Nome} & \textbf{Descrizione} & \textbf{Rilevazione} & \textbf{Contromisure} 	\TBstrut \\
		RT1 & Inesperienza tecnologica & Inizialmente il funzionamento di alcune tecnologie e strumenti necessari per la realizzazione del capitolato\glosp sarà sconosciuto ai componenti del gruppo & I componenti del gruppo dovranno informarsi in anticipo sulle tecnologie interessate nella realizzazione del progetto\glosp e comunicare al responsabile possibili lacune tecniche & Questo rischio può essere eluso attraverso una fase di autoapprendimento preliminare, coadiuvata eventualmente da consultazioni con il proponente \TBstrut \\ [2mm]
        RT2 & Limitazioni hardware e software & Gli strumenti hardware e software in dotazione ai componenti del gruppo potrebbero non avere una potenza computazionale sufficiente per la realizzazione del prodotto\glo & I componenti del gruppo dovranno aver cura di consultare i requisiti minimi di sistema degli strumenti da utilizzare e comunicare eventuali inadeguatezze al responsabile & Per evitare questo rischio i compiti più onerosi dal punto di vista computazionale possono essere delegati a membri del gruppo con strumenti più performanti o, in alternativa, si potrà fare uso dei mezzi forniti dai laboratori di informatica universitari \TBstrut \\ [2mm]
        RT3 & Guasti hardware e software & Gli strumenti hardware e software in dotazione ai componenti del gruppo potrebbero guastarsi data la mole di lavoro e l'installazione di nuovi software, facendo perdere dati importanti & I membri del gruppo dovranno monitorare costantemente l'operatività del proprio sistema e comunicare eventuali anomalie al responsabile & Il lavoro svolto deve essere oggetto periodico di backup oltre ad essere versionato tramite GitHub \TBstrut \\ [2mm]
		RT4 & Dipendenze del sistema operativo & I membri del gruppo possono lavorare sia su Windows che su Linux, questo può causare errori di rappresentazione dei dati quando questi passano da un sistema operativo all'altro & I componenti del gruppo devono essere a conoscenza delle piattaforme di lavoro dei proprio colleghi prevedendo eventuali problemi & Bisognerà avere particolari accorgimenti quando avverrà uno sviluppo multipiattaforma inoltre, prima dello sviluppo, sarà compito del gruppo uniformarsi con strumenti disponibili per tutti e con compatibilità estesa \TBstrut \\ [2mm]
		\rowcolor{white}
		\caption{Rischi tecnologici}
	\end{longtable}
\section{Rischi di gruppo}
	\rowcolors{2}{gray!25}{gray!15}
	\begin{longtable} {
		>{}p{10mm} 
		>{}p{24mm}
		>{}p{32mm} 
        >{}p{32mm}
        >{}p{32mm}
		}
	\rowcolor{gray!50}
		\textbf{ID} & \textbf{Nome} & \textbf{Descrizione} & \textbf{Rilevazione} & \textbf{Contromisure} 	\TBstrut \\
    RG1 & Dinamiche interne al gruppo & Essendo il gruppo formato in maniera casuale si potrebbero creare incomprensioni e difficoltà comunicative tra i membri & È compito del responsabile supervisionare le interazioni tra il componenti e rilevare eventuali criticità & Nel momento in cui ci fossero indecisioni o dubbi che il gruppo non riesce a risolvere, sarà compito del responsabile trovare un punto di incontro che minimizzi i contrasti tra i membri \TBstrut \\ [2mm]
    RG2 & Impegni personali & Tutti i membri del gruppo hanno altri esami da svolgere in contemporanea al progetto\glosp e alcuni hanno anche una carriera lavorativa; questo potrebbe togliere tempo ai compiti di progetto\glosp assegnati & I membri devono avvisare subito gli altri componenti degli obblighi che devono assolvere nel periodo del progetto\glo & Il gruppo si coordinerà per far sì che il carico di lavoro e le riunioni siano il più possibile compatibili con gli impegni personali di ognuno \TBstrut \\ [2mm]
	RG3 & Inesperienza progettuale & Nessun membro del gruppo ha mai preso parte ad un progetto\glosp di grandi dimensioni e con alta precisione richiesta nei documenti & I membri devono costantemente tener traccia del lavoro svolto e comunicare le eventuali lacune al responsabile & I componenti del gruppo devono prendere parte alle lezioni di Ingegneria del Software ed applicare, nel modo migliore, le conoscenze apprese, in caso di ulteriori problemi il confronto tra i membri o con il docente può risolvere i rimanenti dubbi \TBstrut \\ [2mm]
	\rowcolor{white}
	\caption{Rischi di gruppo}
	\end{longtable}
\section{Rischi organizzativi}
	\rowcolors{2}{gray!25}{gray!15}
	\begin{longtable} {
		>{}p{10mm} 
		>{}p{24mm}
		>{}p{32mm} 
        >{}p{32mm}
        >{}p{32mm}
		}
	\rowcolor{gray!50}
		\textbf{ID} & \textbf{Nome} & \textbf{Descrizione} & \textbf{Rilevazione} & \textbf{Contromisure} 	\TBstrut \\
		RO1 & Divisione errata del lavoro & A causa dell'inesperienza riguardo i compiti da svolgere il carico di lavoro potrebbe essere diviso non ugualmente tra i vari membri & Il membro che si trovasse in difficoltà nello svolgimento puntuale del suo compito deve comunicare tempestivamente data evenienza al responsabile & Il responsabile, una volta individuata la criticità, si adopererà per ripartire il compito tra i membri meno oberati  \TBstrut \\ [2mm]
		RO2 & Mantenimento issue tracking system & Per dividere meglio il lavoro tra i membri viene usato un issue tracking system che, però, deve rimanere sempre aggiornato togliendo i compiti completati e aggiungendo quelli nuovi & Nel momento in cui un membro del gruppo noti delle discrepanze nell'assegnazione dei compiti segnalati sull'issue tracking system dovrà riportarlo al responsabile & Dopo ogni riunione un membro del gruppo sarà addetto ad aggiornare l'issue tracking system con i nuovi compiti e relativi assegnatari inoltre, ogni componente dovrà adoperarsi per gestire consistentemente i proprio issue \TBstrut \\ [2mm]
		RO3 & Rotazione dei ruoli & La rotazione periodica dei ruoli nel gruppo potrebbe portare difficoltà nell'individuare, di volta in volta, i vari addetti & Nel momento in cui più membri vengano interpellati per problemi non a loro carico l'errore verrà segnalato al responsabile & I vari ruoli verranno riportati nei verbali delle riunioni (momento in cui si attua la rotazione) inoltre, è compito del responsabile essere sempre al corrente delle funzioni di ogni membro e farlo presente in caso di problemi \TBstrut \\ [2mm]
		\rowcolor{white}
		\caption{Rischi organizzativi}
	\end{longtable}
\section{Rischi dei requisiti}
	\rowcolors{2}{gray!25}{gray!15}
	\begin{longtable} {
		>{}p{10mm} 
		>{}p{24mm}
		>{}p{32mm} 
        >{}p{32mm}
        >{}p{32mm}
		}
	\rowcolor{gray!50}
		\textbf{ID} & \textbf{Nome} & \textbf{Descrizione} & \textbf{Rilevazione} & \textbf{Contromisure}	\TBstrut \\
		RR1 & Incomprensione dei requisiti & Essendo il gruppo inesperto i requisiti presentati potrebbero non essere compresi in modo immediato & Sarà cura di tutto il gruppo, fin dai primi momenti, studiare i requisiti e riportare ai membri eventuali difficoltà & Alcuni dubbi possono essere risolti tramite il confronto tra i componenti, per altri si procederà a contattare il proponente per avere ulteriori chiarimenti \TBstrut \\ [2mm]
		RR2 & Volatilità dei requisiti & Il proponente, per motivi non meglio precisati, potrebbe modificare le caratteristiche del software richiesto & Sarà il proponente, nel caso lo decida, a comunicare al gruppo eventuali variazioni nei vincoli di progetto\glo & Il gruppo dovrà rivedere la precedente \textit{Analisi dei Requisiti} per aggiungere le nuove richieste e trovare celermente una soluzione ottimale per esse \TBstrut \\ [2mm]
		RR3 & Difficoltà di comunicazione con il proponente & Il tramite del gruppo con l'azienda \textit{Zucchetti} potrebbe, per vari impegni professionali, ritardare la corrispondenza con il gruppo & Il responsabile dovrà tenere traccia dei contatti con l'azienda e rilevare tempestivamente potenziali problemi comunicativi & Il responsabile dovrà sollecitare il proponente per una risposta, in caso di ulteriore indisponibilità il gruppo proseguirà ad una analisi interna degli eventuali dubbi e ad una loro risoluzione \TBstrut \\ [2mm]
		\rowcolor{white}
		\caption{Rischi dei requisiti}
	\end{longtable}
\section{Rischi di stima}
	\rowcolors{2}{gray!25}{gray!15}
	\begin{longtable} {
		>{}p{10mm} 
		>{}p{24mm}
		>{}p{32mm} 
		>{}p{32mm}
		>{}p{32mm}
		}
	\rowcolor{gray!50}
		\textbf{ID} & \textbf{Nome} & \textbf{Descrizione} & \textbf{Rilevazione} & \textbf{Contromisure} 	\TBstrut \\
		RS1 & Rispetto delle milestone & Per tutti i rischi sopracitati le milestone impostate su GitHub potrebbero essere impostate non propriamente & Avvicinandosi alle scadenze sarà compito del responsabile controllare il lavoro fatto ed eventuali ritardi & Il responsabile, se si prospettano ritardi sulla milestone, dovrà in qualche modo (ad esempio ripartendo i compiti) evitare che ciò accada; attraverso l'esperienza maturata sarà poi più facile la creazione di scadenze attendibili \TBstrut \\ [2mm]
		RS2 & Prospetto dei costi & Avendo poca esperienza nel mondo lavorativo e nessuna esperienza nella gestione totalitaria di un progetto\glosp la valutazione dei conseguenti costi potrebbe risultare inadatta & Tale problema sorge nel momento della stesura del \textit{Piano di Progetto} e sarà compito degli analisti di tale documento segnalare eventuali difficoltà al responsabile & Nel momento della stesura del \textit{Piano di Progetto} dovrà essere svolta un'analisi approfondita delle ore di lavoro di ogni membro in relazione ai ruoli ricoperti \TBstrut \\ [2mm]
		\rowcolor{white}
		\caption{Rischi di stima}
	\end{longtable}
\section{Incidenza e gravità dei rischi}
	\rowcolors{2}{gray!25}{gray!15}
	\begin{longtable} {
		>{}p{24mm} 
		>{}p{32mm}
		>{}p{32mm} 
		}
	\rowcolor{gray!50}
		\textbf{ID} & \textbf{Incidenza} & \textbf{Gravità}	\TBstrut \\
		RT1 & Alta & 2 \TBstrut \\ [2mm]
		RT2 & Bassa & 2 \TBstrut \\ [2mm]
		RT3 & Bassa & 2 \TBstrut \\ [2mm]
		RT4 & Media & 2 \TBstrut \\ [2mm]
		RG1 & Media & 2 \TBstrut \\ [2mm]
		RG2 & Alta & 3 \TBstrut \\ [2mm]
		RG3 & Alta & 3 \TBstrut \\ [2mm]
		RO1 & Bassa & 1 \TBstrut \\ [2mm]
		RO2 & Media & 1 \TBstrut \\ [2mm]
		RO3 & Media & 1 \TBstrut \\ [2mm]
		RR1 & Media & 2 \TBstrut \\ [2mm]
		RR2 & Bassa & 3 \TBstrut \\ [2mm]
		RR3 & Bassa & 3 \TBstrut \\ [2mm]
		RS1 & Alta & 3 \TBstrut \\ [2mm]
		RS2 & Alta & 1 \TBstrut \\ [2mm]
		\rowcolor{white}
		\caption{Incidenza e gravità dei rischi}
	\end{longtable}
