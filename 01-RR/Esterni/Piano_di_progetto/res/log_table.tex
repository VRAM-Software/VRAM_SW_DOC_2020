\setcounter{secnumdepth}{0}
%\hfill \break
%\textbf{\Large{Diario delle modifiche}} \\
\section{Introduzione}
La seguente tabella è da usare nella Analisi dei requisiti per i requisiti: funzionali, di qualità e di vincolo; seguono le tabelle per il tracciamento dei requisiti e del registro delle modifiche.
\section*{Conteggio ore}
	\rowcolors{2}{gray!25}{gray!15}
	\begin{longtable} {
		>{\centering}m{42mm} 
		>{\centering}m{10mm}
		>{\centering}m{10mm}
		>{\centering}m{10mm}
		>{\centering}m{10mm}
		>{\centering}m{18mm}
		@{}m{0pt}@{}
		}
	\rowcolor{gray!50}
		\textbf{Re} & \textbf{Am} & \textbf{Pt} & \textbf{Pr} & \textbf{Ve} & \textbf{Totale} & \TBstrut \\
		Vittorio Corrizzato & - & - & - & - & - &\TBstrut \\ [2mm]
		Marco Dalla Libera & - & - & - & - & - &\TBstrut \\ [2mm]
		Marco Rampazzo & - & - & - & - & - &\TBstrut \\ [2mm]
		Vittorio Santagiuliana  & - & - & - & - & - &\TBstrut \\ [2mm]
		Rebecca Schiavon & - & - & - & - & - &\TBstrut \\ [2mm]
		Alessandro Spreafico & - & - & - & - & - &\TBstrut \\ [2mm]
		Massimo Toffoletto & - & - & - & - & - &\TBstrut \\ [2mm]
	\end{longtable}
	
	%\hfill \break
%\textbf{\Large{Diario delle modifiche}} \\

\addtocontents{toc}{\protect\setcounter{tocdepth}{0}} %Inserire questo per escludere una sezione dall'indice.

\section*{Registro delle modifiche} %Asterisco per fare sezione non numerata
\rowcolors{2}{gray!25}{gray!15}
\begin{longtable} {
		>{\centering}p{17mm} 
		>{\centering}p{19.5mm}
		>{\centering}p{24mm} 
		>{\centering}p{24mm} 
		>{}p{32mm}}
	\rowcolor{gray!50}
	\textbf{Versione} & \textbf{Data} & \textbf{Nominativo} & \textbf{Ruolo} & \textbf{Descrizione} \TBstrut \\
	1.0.0 & 2019-02-12 & Pinco Pallino & Analista e Verificatore & modificato \$3 \TBstrut \\ [2mm]
	1.0.0 & 2019-02-12 & Pinco Pallino & Analista e Verificatore & modificato \$3 \TBstrut \\ [2mm]
	1.0.0 & 2019-02-12 & Pinco Pallino & Analista e Verificatore & modificato \$3 \TBstrut \\ [2mm]
	
\end{longtable}

\addtocontents{toc}{\protect\setcounter{tocdepth}{4}} %Inserire questo per ripristinare il normale inserimento delle sezioni nell'indice. 4 significa fino al paragrah