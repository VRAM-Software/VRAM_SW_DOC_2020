\section{Qualità di Processo}
	\subsection{Processo di Fornitura}
		\subsubsection{Pianificazione}
			\paragraph{Metriche} \mbox{} \\ \\
				\textbf{BAC: Budget at Completion} Budget totale del progetto\glo.
				\begin{itemize}
					\item misurazione: numero intero;
					\item valore preferibile: pari a quanto preventivato;
					\item valore accettabile: preventivo-5\% $\le$ BAC $\le$ preventivo+5\%.
				\end{itemize}
				\textbf{PV: Planned Value} Valore dei lavoro pianificato fino a quel momento.
				\begin{itemize}
					\item misurazione: BAC $\cdot$ \%lavoro pianificato;
					\item valore preferibile: $\ge$ 0;
					\item valore accettabile: $\ge$0;
				\end{itemize}
				\textbf{EV: Earned Value} Valore del lavoro svolto fino a quel momento.
				\begin{itemize}
					\item misurazione: BAC $\cdot$ \%lavoro pianificato;
					\item valore preferibile: $\ge$ 0;
					\item valore accettabile: $\ge$ 0;
				\end{itemize}
				\textbf{AC: Actual cost} Denaro speso fino a quel momento.
				\begin{itemize}
					\item misurazione: numero intero;
					\item valore preferibile: 0 $\le$ AC $\le$ PV;
					\item valore accettabile: 0 $\le$ AC $\le$ budget totale.
				\end{itemize}
				\textbf{CPI: Cost Performance Index} Indice per i costi effettivi rispetto a quelli previsti.
				\begin{itemize}
					\item misurazione: $\frac{EV}{AC}$;
					\item valore preferibile: = 1;
					\item valore accettabile: 0.95 $\le$ CPI $\le$ 1.05.
				\end{itemize}
				\textbf{SPI: Schedule Performance Index} Indice per i tempi effettivi rispetto a quelli previsti.
				\begin{itemize}
					\item misurazione: $\frac{EV}{PV}$;
					\item valore preferibile: = 1;
					\item valore accettabile: 0.95 $\le$ SPI $\le$ 1.05.
				\end{itemize}
				\textbf{EAC: Estimated Cost at Completion} Budget totale stimato del progetto\glosp con il CPI del momento.
				\begin{itemize}
					\item misurazione: $AC+ \frac{BT-EV}{CPI}$ con BT=Budget Totale
					\item valore preferibile: pari a quanto preventivato;
					\item valore accettabile: preventivo-5\% $\le$ EAC $\le$ preventivo+5\%.
				\end{itemize}
				\textbf{SAC: Schedule at Completion} Tempo totale stimato del progetto\glosp con SPI del momento.
				\begin{itemize}
					\item misurazione: $\frac{TT}{SPI}$ con TT=Tempo Totale
					\item valore preferibile: pari a quanto preventivato;
					\item valore accettabile: pari a quanto preventivato.
				\end{itemize}
				
											
	\subsection{Processo di Sviluppo}
		\subsubsection{Analisi dei Requisiti} 
			\paragraph{Metriche} \mbox{} \\ \\
				\textbf{PROS: Percentuale dei Requisiti Obbligatori Soddisfatti} 
				\begin{itemize}
					\item misurazione: $\frac{requisiti \; obbligatori \; soddisfatti}{requisiti \; obbligatori \; totali}$;
					\item valore preferibile: 100\%;
					\item valore accettabile: 100\%.
				\end{itemize}
				\textbf{PRDS: Percentuale dei Requisiti Desiderabili Soddisfatti} 
				\begin{itemize}
					\item misurazione: $\frac{requisiti \; desiderabili \; soddisfatti}{requisiti \; desiderabili \; totali}$
					\item valore preferibile: 100\%;
					\item valore accettabile: 85\%.
				\end{itemize}
			
	\subsection{Processi di supporto}		
		\subsubsection{Documentazione}
			\paragraph{Metriche} \mbox{} \\ \\
			\textbf{I$_{G}$: Indice di Gulpease} Indica la leggibilità di un testo.
			\begin{itemize}
				\item misurazione: 89+$\frac{300\cdot numero \; di \; frasi-10\cdot numero \; di \; lettere}{numero \; di \; parole}$;
				\item valore preferibile: 60 $\le I_{G} \le$ 100;
				\item valore accettabile: 40 $\le I_{G} \le$ 100.
			\end{itemize}
			\textbf{Correttezza ortografica} Indica la presenza di errori ortografici nei documenti.
			\begin{itemize}
				\item misurazione: numero intero;
				\item valore preferibile: 0;
				\item valore accettabile: 0.
			\end{itemize}
				
		\subsubsection{Gestione della Qualità}
			\paragraph{Metriche} \mbox{} \\ \\
				\textbf{PMS: Percentuale di metriche soddisfatte} Indica la percentuale di metriche\glosp che raggiungono le soglie considerate accettabili e sono quindi soddisfatte sul numero di metriche\glosp totali.
				\begin{itemize}
					\item misurazione: $\frac{numero \; di \; metriche \; soddisfatte}{numero \; di \; metriche \; totali}$;
					\item valore preferibile: $\ge$ 80\%;
					\item valore accettabile: $\ge$ 60\%;
				\end{itemize}


		  
			
				

