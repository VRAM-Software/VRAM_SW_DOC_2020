\section{Introduzione}
	\subsection{Premessa}
	Il Piano di Qualifica è un documento che sarà ampliato incrementalmente con il proseguimento del progetto\glo, non è quindi da considerarsi completo. Questo modus operandi è supportato dall'adesione al modello a V\glo, secondo il quale nel periodo di analisi si può procedere alla stesura dei soli test di accettazione e di sistema.
	
	\subsection{Scopo del documento}
	Il compito del \textit{Piano di Qualifica} è fissare quantitativamente, tramite valori soglia o intervalli, gli obiettivi di qualità di prodotto\glosp e di processo\glosp assunti nel progetto\glo, utilizzando le metriche definite nelle \textit{Norme di Progetto} e specificare le modalità (tra esse i test) con le quali verificarne il raggiungimento.

	\subsection{Scopo del prodotto}
	Lo scopo del prodotto\glosp è creare un plug-in di Grafana\glosp che usi gli algoritmi di previsione SVM\glosp e RL\glosp per produrre dei valori che potranno essere aggiunti al flusso del monitoraggio come se fossero stati rilevati dal campo. Assieme al plug-in verrà sviluppato un programma per la gestione dei parametri degli algoritmi di previsione, che permetterà di allenare gli algoritmi con dei dati di test. Il fine del plug-in è monitorare la "liveliness" del sistema a supporto dei processi DevOps\glosp e di consigliare interventi nel sistema di produzione del software.
	
	\subsection{Glossario}
	I termini ambigui e bisognosi di spiegazione presenti in questo documento, contrassegnati da una 'G' a pedice, sono chiariti nel \textit{Glossario v. 1.1.1}.
	
	\subsection{Riferimenti}
		\subsubsection{Riferimenti Normativi}
		\begin{enumerate}
			\item \textbf{Norme di Progetto}: \textit{Norme di Progetto v. 1.1.1};	
		\end{enumerate}
	
		\subsubsection{Riferimenti Informativi}
		\begin{enumerate}
		    \item \textbf{Capitolato\glosp d'appalto C4 - Predire in Grafana}: \url{https://www.math.unipd.it/~tullio/IS-1/2019/Progetto/C4.pdf};
		    \item \textbf{Modello a V}: \url{https://www.math.unipd.it/~tullio/IS-1/2019/Dispense/L14.pdf};
		\end{enumerate}
	
	
	

	
	
	
	
