\subsubsection{Modello di qualità del prodotto} 
Per descrivere il modello di qualità del prodotto\glosp software il nostro gruppo ha deciso di utilizzare lo standard ISO/IEC 25010, che determina le caratteristiche della qualità di un prodotto\glosp software che verranno valutate. Il modello di qualità del prodotto\glo, definito nello standard, è suddiviso nelle otto seguenti caratteristiche:
	\paragraph{Idoneità funzionale} \mbox{}\\
	Per adeguatezza funzionale si intende la capacità di un prodotto\glosp software di fornire funzioni che soddisfano i requisiti impliciti ed espliciti, quando usati in un determinato contesto sotto specifiche condizioni.
	\begin{itemize}
		\item \textbf{Completezza}: la capacità del software di offrire un insieme di funzioni le cui funzionalità coprono tutte le attività e tutti gli obiettivi di un utente;
		\item \textbf{Correttezza}: la capacità del software di fornire agli utenti risultati corretti e precisi rispetto al grado di precisione richiesto;
		\item \textbf{Adeguatezza}: la capacità del software di fornire funzioni appropriate che facilitano l'utente a raggiungere i suoi obiettivi.
	\end{itemize}
	\paragraph{Efficienza prestazionale} \mbox{}\\
	L'efficienza prestazionale è la caratteristica relativa alle prestazioni rispetto al numero di risorse usate sotto condizioni precise.
	\begin{itemize}
		\item \textbf{Comportamento rispetto al tempo}: la capacità del software di eseguire le funzioni in un tempo di risposta, tempo processionale e un volume di produzione che soddisfa i requisiti;
		\item \textbf{Utilizzo delle risorse}: la capacità del software di eseguire le funzioni usando un numero di risorse disponibili che soddisfa i requisiti;
		\item \textbf{Capacità}: la capacità del software di avere limiti prestazionali massimi che soddisfano i requisiti.
	\end{itemize}
	\paragraph{Compatibilità} \mbox{}\\
	La compatibilità è la caratteristica di un software di scambiare informazioni con altri prodotti\glo, ed eseguire le sue funzioni mentre viene condiviso lo stesso ambiente hardware o software.
	\begin{itemize}
		\item \textbf{Coesistenza}: la capacità del software di eseguire le sue funzionalità senza impatti negativi mentre condivide lo stesso ambiente con un altro software;
		\item \textbf{Interoperabilità}: la capacità di due o più software di scambiare informazioni tra loro e usare tali dati per i loro scopi.
	\end{itemize}
	\paragraph{Usabilità} \mbox{}\\
	L'usabilità è la caratteristica di un software facilmente utilizzabile, comprensibile dall'utente ed efficacie a raggiungere gli obiettivi che un utente si prefissa.
	\begin{itemize}
		\item \textbf{Appropriatezza-Riconoscibilità}: la capacità del software di essere riconosciuto dall'utente come un prodotto\glosp adatto ai loro scopi;
		\item \textbf{Apprendibilità}: la capacità del software di ridurre l'impegno che un utente deve avere per capire le funzionalità del prodotto\glo ;
		\item \textbf{Operabilità}: la capacità del software di essere utilizzato dagli utenti in modo semplice;
		\item \textbf{Protezione errori dell'utente}: la capacità del software di proteggere l'utente dagli errori che lo stesso può causare;
		\item \textbf{Estetica dell'interfaccia utente}: la capacità del software di avere una interfaccia utente piacevole;
		\item \textbf{Accessibilità}: la capacità del software di essere usato da persone con caratteristiche che alterano in modo negativo l'operabilità del software.
	\end{itemize}
	\paragraph{Affidabilità} \mbox{}\\
	Un prodotto\glosp software è considerato affidabile quando mantiene un livello di prestazioni costanti per un determinato intervallo di tempo sotto precise condizioni che possono variare nel tempo.
	\begin{itemize} \mbox{}\\
		\item \textbf{Maturità}: la capacità del software di avere un comportamento affidabile quando usato in condizioni normali;
		\item \textbf{Disponibilità}: la capacità del software di essere disponibile quando ne è richiesto l'uso;
		\item \textbf{Tolleranza agli errori}: la capacità del software di tollerare situazioni critiche in caso di errori, in modo da permettere funzionalità costanti anche in presenza di malfunzionamenti;
		\item \textbf{Recuperabilità}: la capacità del software di ristabilire il sistema allo stato desiderato dopo malfunzionamenti.
	\end{itemize}
	\paragraph{Sicurezza} \mbox{}\\
	Per sicurezza in un prodotto\glosp software si intende il grado di protezione che il software ha sui dati per evitare che persone o servizi non autorizzati li ricevano.
	\begin{itemize}
		\item \textbf{Confidenzialità}: la capacità del software di assicurare la confidenzialità dei dati alle persone che hanno l'autorizzazione di accedere a tali informazioni;
		\item \textbf{Integrità}: la capacità del software di evitare modifiche o accessi non autorizzati;
		\item \textbf{Non ripudio}: la capacità del software di dimostrare che eventi o azioni sono avvenuti, in modo da non ripudiarli in un secondo momento;
		\item \textbf{Responsabilità}: la capacità del software di risalire a una sola entità a partire da una azione di una entità;
		\item \textbf{Autenticità}: la capacità del software di provare l'identità delle risorse utilizzate.
	\end{itemize}
	\paragraph{Manutenibilità} \mbox{}\\
	Per manutenibilità si intende la capacità di un prodotto\glosp software di essere modificato e aggiornato senza l'introduzione di problemi e anomalie in seguito a cambiamenti dell'ambiente e/o dei requisiti.
	\begin{itemize}
		\item \textbf{Modularità}: la capacità del software di essere composto da componenti in modo che un cambio di un componente non abbia un grande impatto sul resto del sistema;
		\item \textbf{Riutilizzabilità}: la capacità del software di riutilizzare risorse più di una volta;
		\item \textbf{Analizzabilità}: la capacità del software di essere facilmente analizzabile per individuare un errore, le causa di un fallimento o di identificare le parti da modificare;
		\item \textbf{Modificabilità}: la capacità del software di essere facilmente modificabile senza introdurre errori non gestibili dal programmatore;
		\item \textbf{Testabilità}: la capacità del software di essere testato facilmente così da verificare ogni aggiunta e/o modifica.
	\end{itemize}
	\paragraph{Portabilità} \mbox{}\\ 
	Per portabilità si intende la caratteristica del software di essere utilizzato in modo efficacie quando esso viene trasferito su diversi hardware o software.
	\begin{itemize}
		\item \textbf{Adattabilità}: la capacità del software di essere adattato per hardware o software in continuo cambiamento;
		\item \textbf{Installabilità}: la capacità del software di essere installato o disinstallato efficacemente in un ambiete specifico;
		\item \textbf{Sostituibilità}: la capacità del software di rimpiazzare un altro prodotto\glosp con lo stesso scopo nello stesso ambiente.
	\end{itemize}
