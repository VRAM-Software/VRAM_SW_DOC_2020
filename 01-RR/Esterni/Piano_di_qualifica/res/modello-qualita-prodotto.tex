\subsubsection{Modello di qualità del prodotto} 
Per descrivere il modello di qualità del prodotto software il nostro gruppo ha deciso di utilizzare lo standard ISO/IEC 25010, che classifica e definisce i parametri della qualità di un prodotto software. Il nostro gruppo ha considerato adatti per questo progetto i seguenti quattro parametri tralasciando l'efficienza e la portabilità dato che non vengono contemplati nel progetto.
	\paragraph{Funzionalità} \mbox{}\\
	Per funzionalità si intende la capacità di un prodotto software di fornire funzionalità che servono agli utenti per operare in un determinato contesto.
	\begin{itemize}
		\item \textbf{Adeguatezza}: la capacità del software di offrire un insieme di funzioni appropriate le cui funzionalità permettono all'utente di raggiungere determinati obbiettivi;
		\item \textbf{Accuratezza}: la capacità del software di fornire agli utenti risultati corretti e precisi e funzionalità concordanti;
		\item \textbf{Interoperabilità}: la capacità del software di comunicare e scambiare informazioni con servizi o altri sistemi senza l'introduzione di errori;
		\item \textbf{Sicurezza}: la capacità del software di proteggere i dati da persone o sistemi non autorizzati;
		\item \textbf{Conformità}: la capacità del software di essere conforme a standard, regole e leggi alla sua funzionalità;
	\end{itemize}
	\paragraph{Usabilità} \mbox{}\\
	L'usabilità è la capacità del prodotto software di essere facilmente utilizzabile e comprensibile dall'utente, se viene usato secondo alcune regole prestabilite dagli sviluppatori.
	\begin{itemize}
		\item \textbf{Comprensibilità}: un software è comprensibile se i suoi concetti sono facilmente compresibili dall'utente;
		\item \textbf{Apprendibilità}: è considerato apprendibile un software che riduce l'impegno che un utente deve avere per capire le funzionalità del prodotto;
		\item \textbf{Operabilità}: la capacità del software di essere utilizzato dagli utenti per raggiungere i loro scopi;
		\item \textbf{Attrattivà}: la caratteristica del software il cui utilizzo è piacevole all'utente, quindi l'interfaccia utente deve essere semplice, coerente e chiara;
		\item \textbf{Conformità}: la capacità del software di essere conforme a standard, regole e leggi alla sua usabilità.
	\end{itemize}
	\paragraph{Affidabilità} \mbox{}\\
	Un prodotto software è considerato affidabile quando mantiene un livello di prestazioni costanti per un determinato intervallo di tempo sotto precise condizioni che possono variare nel tempo.
	\begin{itemize}
		\item \textbf{Maturità}: un software può essere considerato maturo se riesce a evitare errori e/o risultati errati;
		\item \textbf{Tolleranza agli errori}: la capacità del software di tollerare situazioni critiche in caso di errori, in modo da permettere funzionalità costanti anche in presenza di malfunzionamenti;
		\item \textbf{Recuperabilità}: la recuperabilità è strettamente legata alla tolleranza agli errori e indica la capacità del software di ristabilire il suo funzionamento corretto in seguito a malfunzionamenti;
		\item \textbf{Conformità}: la capacità del software di essere conforme a standard, regole e leggi riguardo alla sua affidabilità.
	\end{itemize}
	\paragraph{Manutenibilità} \mbox{}\\
	Per manutenibilità si intende la capacità di un prodotto software di essere modificato e aggiornato senza l'introduzione di problemi e anomalie.
	\begin{itemize}
		\item \textbf{Analizzabilità}: per analizzabilità si intende la facilità con cui si può identificare un errore nel codice;
		\item \textbf{Modificabilità}: la capacità del software di essere facilmente modificabile senza introdurre errori non gestibili dal programmatore;
		\item \textbf{Stabilità}: un software è definito stabile se l'aggiornamento o modifica dello stesso non provoca errori o anomalie nel funzionamento;
		\item \textbf{Testabilità}: la capacità del software di essere testato facilmente così da verificare ogni aggiunta e/o modifica;
		\item \textbf{Conformità}: la capacità del software di essere conforme a standard, regole e leggi riguardo alla sua manutenibilità.
	\end{itemize}
