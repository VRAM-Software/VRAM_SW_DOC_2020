\section{Standard di qualità}
	\subsection{ISO/IEC 25010}
		Lo standard internazionale ISO/IEC 25010 è utilizzato per misurare la qualità del software, e per farlo utilizza una serie di parametri di qualità, con rispettive tecniche per la misurazione.
		%In particolare, le metriche per la misurazione della qualità sono definite nello standard 25053, che ne contiene circa 80.
		\subsubsection{Metriche di qualità interna}
			La qualità interna del software è quella relativa a progettazione e codifica, riguarda in particolare sviluppatori e progettisti, in quanto sono le figure che più percepiscono e possono trarre beneficio dalla qualità interna.
			La misurazione della qualità interna viene fatta principalmente attraverso l'analisi statica, non è prevista l'esecuzione del codice.
			La qualità esterna e in uso dovrebbero idealmente essere influenzate dalla qualità interna.
		\subsubsection{Metriche di qualità esterna}
			La qualità esterna del software è quella che viene valutata in fase di test, viene misurata attraverso l'analisi dinamica, ovvero tramite l'esecuzione del codice.
		\subsubsection{Metriche di qualità in uso}
			La qualità in uso è probabilmente la più difficile da misurare poiché è la qualità percepita dagli utenti effettivi del prodotto software, valutata in un ambito di utilizzo reale.
			Idealmente, dovrebbe essere il frutto di qualità interna e qualità esterna.
