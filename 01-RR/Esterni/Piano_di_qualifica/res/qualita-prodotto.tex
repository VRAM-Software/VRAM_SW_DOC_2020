\section{Qualità di prodotto}
    Per misurare la qualità di prodotto\glosp il gruppo ha deciso di prendere come riferimento lo standard ISO/IEC 25010 che definisce un modello di qualità del prodotto tramite otto caratteristiche e pone le basi per lo standard ISO/IEC 25023 per la misurazione di queste specifiche. Di seguito verranno elencate le voci che il gruppo ha ritenuto importanti in questo frangente del progetto\glo.
    \subsection{Documenti}
    	\subsubsection{Obiettivi}
    		\begin{itemize}
    			\item \textbf{Leggibilità}: il grado di facilità con cui un documento viene letto;
    			\item \textbf{Correttezza}: il grado di errori ortografici presenti nel documento.
    		\end{itemize}
	    \subsubsection{Metriche} \mbox{} \\ \\
	    \textbf{I$_{G}$: Indice di Gulpease} Indica la leggibilità di un testo.
	    \begin{itemize}
	    	\item misurazione: 89+$\frac{300\cdot numero \; di \; frasi-10\cdot numero \; di \; lettere}{numero \; di \; parole}$;
	    	\item valore preferibile: 60 $\le I_{G} \le$ 100;
	    	\item valore accettabile: 40 $\le I_{G} \le$ 100.
	    \end{itemize}
	    \textbf{Correttezza ortografica} Indica la quantità di errori ortografici nei documenti.
	    \begin{itemize}
	    	\item misurazione: numero intero;
	    	\item valore preferibile: 0;
	    	\item valore accettabile: 0.
	    \end{itemize}
    \subsection{Idoneità funzionale}
        Il grado con cui un prodotto\glosp o un sistema fornisce funzioni che soddisfano i requisti dichiarati e impliciti nell'\textit{Analisi dei Requisiti}.
        \subsubsection{Obiettivi}
            \begin{itemize}
                \item \textbf{Completezza}: il grado con cui l'insieme di funzioni copre tutte le specifiche attività e gli obiettivi dell'utente;
                \item \textbf{Correttezza}: il grado con cui un prodotto\glosp o un sistema fornisce con la giusta precisione il risultato corretto;
                \item \textbf{Adeguatezza}: il grado con cui le funzioni facilitano il compimento di attività e obiettivi specifici.
            \end{itemize}
        \subsubsection{Metriche}
            \paragraph{Copertura funzionale}
                La percentuale di requisiti soddisfatti su tutti i requisiti individuati durante l'analisi.
                \begin{itemize}
                    \item Misurazione: per calcolarla si usa la seguente formula: 
                    \[C_F=(1-N_{RNS}/N_{RI})*100\]
                    dove N$_{RNS}$ sono i requisiti non sviluppati e N$_{RI}$ i requisiti trovati durante l'\textit{Analisi dei Requisiti};
                    \item valore preferibile: 100\%;
                    \item valore accettabile: 100\%.
                \end{itemize}
    \subsection{Usabilità}
        Il grado con cui il prodotto\glosp o il sistema può essere usato da specifici utenti per completare particolari attività con efficacia, efficienza e soddisfazione. Nel caso del nostro prodotto\glosp ci si riferisce all'applicativo esterno adibito all'apprendimento degli algoritmi di previsione.
        \subsubsection{Obiettivi}
            \begin{itemize}
                \item \textbf{Apprendibilità}: il grado con cui il prodotto\glosp o il sistema può essere appreso con efficacia, efficienza e soddisfazione da uno specifico utente;
                \item \textbf{Appropriatezza-Riconoscibilità}: il grado con cui gli utenti possono riconoscere che un determinato prodotto\glosp o sistema è appropriato per i propri bisogni.
            \end{itemize}
        \subsubsection{Metriche}
            \paragraph{Completezza della documentazione}
                Percentuale delle funzioni che è descritta nella documentazione con un dettaglio tale da consentire all’utente di utilizzarle.
                \begin{itemize}
                    \item Misurazione: per calcolarla si usa la seguente formula:
                    \[C_{DOC}=(N_{FD}/N_{FI})*100\]
                    dove N$_{FD}$ sono le funzioni definite sulla documentazione e N$_{FI}$ sono le funzioni individuate nella documentazione;
                    \item valore preferibile: 100\%;
                    \item valore accettabile: 100\%.
                \end{itemize}
            \paragraph{Completezza di descrizione}
                Percentuale degli scenari d’uso descritta nella documentazione effettivamente presenti nel prodotto\glosp finale.
                \begin{itemize}
                    \item Misurazione: per calcolarla si usa la seguente formula:
                    \[C_{DESC}=(N_{UCI}\\N_{UCE})*100\]
                    dove N$_{UCI}$ è il numero di casi d'uso individuati e N$_{UCE}$ è il numero di casi d'uso effettivi del prodotto\glo;
                    \item valore preferibile: 100\%;
                    \item valore accettabile: 100\%.
                \end{itemize}
    \subsection{Affidabilità}
        Il grado con cui un sistema, un prodotto\glosp o un componente esegue delle funzioni sotto delle specifiche condizioni per uno specifico periodo di tempo.
        \subsubsection{Obiettivi}
            \begin{itemize}
                \item \textbf{Maturità}: il grado con cui un sistema, un prodotto\glosp o un componente è affidabile durante le normali condizioni di servizio;
                \item \textbf{Tolleranza agli errori}: il grado con cui un sistema, un prodotto\glosp o un componente riesce ad operare anche in presenza di errori hardware o software.
            \end{itemize}
        \subsubsection{Metriche}
            \paragraph{Robustezza agli errori}
                Percentuale di errori critici messa sotto controllo.
                \begin{itemize}
                    \item Misurazione: per calcolarla si usa la seguente formula:
                    \[R_E=N_{ER}/N_{T}*100\]
                    dove N$_ER$ sono gli errori rilevati e N$_T$ sono i test eseguiti;
                    \item valore preferibile: 0\%;
                    \item valore accettabile: 10\%.
                \end{itemize}
    \subsection{Manutenibilità}
        Il grado di efficacia ed efficienza con cui un prodotto\glosp o un sistema può essere modificato per miglioramenti, correzioni o adattamenti a cambiamenti ambientali e nei requisiti.
        \subsubsection{Obiettivi}
        \begin{itemize}
            \item \textbf{Analizzabilità}: il grado di efficacia ed efficienza con cui è possibile valutare l'impatto su un prodotto\glosp o un sistema di un eventuale cambiamento (in una o più parti);
            \item \textbf{Modificabilità}: il grado con cui un prodotto\glosp o un sistema può essere modificato efficacemente ed efficientemente, cioè senza introdurre difetti o degradando la qualità esistente.
        \end{itemize}
        \subsubsection{Metriche}
            \paragraph{Semplicità delle funzioni}
                La facilità di un metodo può essere indicata dal numero di parametri che richiede.
                \begin{itemize}
                    \item Misurazione: numero di parametri nei metodi;
                    \item valore preferibile: $\leq 3$;
                    \item valore accettabile: $\leq 6$;
                \end{itemize}
            \paragraph{Modularità del prodotto}
                Più viene perseguita la modularità più sarà facile aggiungere, correggere o adattare classi al prodotto\glo.
                \begin{itemize}
                    \item Misurazione: numero di metodi nelle classi;
                    \item valore preferibile: $\leq 8$;
                    \item valore accettabile: $\leq 15$;
                \end{itemize}
