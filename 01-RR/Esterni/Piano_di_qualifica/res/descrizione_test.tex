\section{Descrizione dei test}
Il nostro gruppo ha scelto di adottare il Modello a V\glosp per garantire la qualità del nostro prodotto. In particolare, questo modello prevede lo sviluppo dei test durante le attività di analisi dei requisiti, progettazione architetturale e progettazione di dettaglio oltre a validazione e collaudo.
In questo modo è possibile verificare la correttezza sia di tutti gli aspetti che compongono il progetto che delle singole parti sviluppate. Sono state individuate 4 tipologie di test:
\begin{itemize}
	\item Test di Accetazione;
	\item Test di Sistema;
	\item Test di Integrazione;
	\item Test di Unità.
\end{itemize}
Ogni volta che viene svolta un'attività viene definita una tabella con i test di una tipologia.
All'interno del documento \textit{Norme di Progetto v.1.1.1} vengono definite le caratteristiche dei test e i codici che identificano univocamente i singoli test.

\subsection{Test di Accettazione}
\addtocontents{toc}{\protect\setcounter{tocdepth}{0}} %Inserire questo per escludere una sezione dall'indice.

\rowcolors{2}{gray!25}{gray!15}
\setcounter{table}{0}
\begin{longtable} {
		>{}p{15mm} 
		>{}p{79.5mm}
		>{}p{15mm} 
		>{}p{15mm}
		>{}p{0mm}}
	\rowcolor{gray!50}
	\textbf{Codice} & \textbf{Descrizione} & \textbf{Stato} & \textbf{Esito} &\TBstrut \\
	TA1 & Verificare che l'utente possa addestrare gli algoritmi di previsione dei dati all'interno della piattaforma Grafana\glo & NI & NE  &\TBstrut \\ [2mm]
	TA1.1 & Verificare che l'utente possa selezionare e caricare dal proprio dispositivo un file JSON che contiene i dati di testing per l'addestramento& NI & NE  &\TBstrut \\ [2mm]
	TA1.2 & Verificare che l'utente possa scegliere il modello di predizione da utilizzare tra tutti quelli forniti & NI & NE  &\TBstrut \\ [2mm]
	TA1.3 & Verificare che l'utente possa avviare l'addestramento dell'algoritmo & NI & NE  &\TBstrut \\ [2mm]
	TA1.4 & Verificare che l'utente possa chiudere l'addestramento e visualizzare un messaggio di conferma se esso va a buon fine & NI & NE  &\TBstrut \\ [2mm]
	TA2 & Verificare che l'utente possa visualizzare l'indice della qualità delle previsioni & NI & NE  &\TBstrut \\ [2mm]
	TA3 & Verificare che, se l'utente inserisce un file JSON non valido, viene visualizzato un messaggio di errore & NI & NE  &\TBstrut \\ [2mm]
	TA4 & Verificare che l'utente possa addestrare gli algoritmi di previsione dei dati sull'applicazione esterna a Grafana\glo & NI & NE  &\TBstrut \\ [2mm]
	TA4.1 & Verificare che l'utente possa selezionare e caricare dal proprio dispositivo un file JSON che contiene i dati di testing per l'addestramento & NI & NE  &\TBstrut \\ [2mm]
	TA4.2 & Verificare che l'utente possa scegliere se addestrare il modello di predizione da utilizzare per l'addestramento tra tutti quelli forniti & NI & NE  &\TBstrut \\ [2mm]
	TA4.3 & Verificare che l'utente possa avviare l'addestramento dell'algoritmo & NI & NE  &\TBstrut \\ [2mm]
	TA4.4 & Verificare che l'utente possa chiudere l'addestramento dell'algoritmo e visualizzare il messaggio di conferma se esso è stato svolto correttamente & NI & NE  &\TBstrut \\ [2mm]
	TA4.5 & Verificare che al termine della procedura l'utente riceva dall'applicazione esterna un file JSON con i parametri per le previsioni & NI & NE  &\TBstrut \\ [2mm]
	TA5 & Verificare che l'utente possa visualizzare l'indice della qualità delle previsioni & NI & NE  &\TBstrut \\ [2mm]
	TA6 & Verificare che, se l'utente inserisce un file JSON non valido, viene visualizzato un messaggio di errore & NI & NE  &\TBstrut \\ [2mm]
	TA7 & Verificare che l'utente possa avviare il plugin & NI & NE  &\TBstrut \\ [2mm]
	TA8 & Verificare che l'utente possa caricare il file JSON ottenuto dall'addestramento all'interno del plugin & NI & NE  &\TBstrut \\ [2mm]
	TA9 & Verificare che l'utente possa associare i nodi letti dal file JSON al flusso dati & NI & NE  &\TBstrut \\ [2mm]
	TA9.1 & Verificare che l'utente possa inserire i nodi & NI & NE  &\TBstrut \\ [2mm]
	TA9.2 & Verificare che l'utente possa selezionare un flusso di dati statico su cui eseguire delle previsioni & NI & NE  &\TBstrut \\ [2mm]
	TA9.3 & Verificare che l'utente possa selezionare un flusso di dati continuo su cui eseguire delle previsioni & NI & NE  &\TBstrut \\ [2mm]
	TA9.4 & Verificare che l'utente possa collegare i nodi scelti al flusso di dati corrispondente & NI & NE  &\TBstrut \\ [2mm]
	TA9.5 & Verificare che l'utente possa visualizzare un messaggio che conferma il successo nel collegamento dei nodi al flusso dati & NI & NE  &\TBstrut \\ [2mm]
	TA10 & Verificare che, se il collegamento dei nodi al flusso dati non va a buon file, l'utente deve visualizzare un messaggio di errore & NI & NE  &\TBstrut \\ [2mm]
	TA11 & Verificare che l'utente possa visualizzare il grafico dei risultati della previsione all'interno di una dashboard\glosp precedentemente configurata & NI & NE  &\TBstrut \\ [2mm]
	TA12 & Verificare che l'utente possa fermare l'esecuzione del plugin rimuovendolo dalla dashboard\glo & NI & NE  &\TBstrut \\ [2mm]
	TA13 & Verificare che l'utente possa definire un alert\glosp all'interno del pannello della dashboard\glosp su cui si è applicato il plugin & NI & NE  &\TBstrut \\ [2mm]
	TA13.1 & Verificare che l'utente possa inserire un alert\glosp nel pannello della dashboard\glo & NI & NE  &\TBstrut \\ [2mm]
	TA13.2 & Verificare che l'utente possa definire le regole di funzionamento di un alert\glo & NI & NE  &\TBstrut \\ [2mm]
	TA13.3 & L'utente deve poter definire le condizioni di funzionamento di un alert\glo & NI & NE  &\TBstrut \\ [2mm]
	TA13.4 & L'utente deve poter definire il comportamento legato all'assenza di dati  & NI & NE  &\TBstrut \\ [2mm]
	TA14 & Verificare che l'utente visualizzi un messaggio di errore se viene inserito un input errato nella definizione di un alert\glo & NI & NE  &\TBstrut \\ [2mm]
	TA15 & Verificare che l'utente possa sospendere un alert\glo & NI & NE  &\TBstrut \\ [2mm]
	TA16 & Verificare che l'utente possa rimuovere un alert\glo & NI & NE  &\TBstrut \\ [2mm]
	\rowcolor{white}
	\caption{Test di Accettazione}
\end{longtable}

\addtocontents{toc}{\protect\setcounter{tocdepth}{4}} %Inserire questo per ripristinare il normale inserimento delle sezioni nell'indice. 4 significa fino al paragrah

\subsection{Test di Sistema}
\addtocontents{toc}{\protect\setcounter{tocdepth}{0}} %Inserire questo per escludere una sezione dall'indice.

\rowcolors{2}{gray!25}{gray!15}
\begin{longtable} {
		>{}p{15mm} 
		>{}p{79.5mm}
		>{}p{15mm} 
		>{}p{15mm}
		>{}p{0mm}}
	\rowcolor{gray!50}
	\textbf{Codice} & \textbf{Descrizione} & \textbf{Stato} & \textbf{Esito} &\TBstrut \\
	TS & Verificare che l'addestramento degli algoritmi produca un file JSON con i parametri per le previsioni & NI & NE  &\TBstrut \\ [2mm]
	TS & Verificare la corretta visualizzazione della bontà dei modelli di previsione a seguito dell'addestramento sui dati & NI & NE  &\TBstrut \\ [2mm]
	TS & Verificare che i nodi ricavati dal file JSON siano associati correttamente al flusso dati scelto in Grafana\glo & NI & NE  &\TBstrut \\ [2mm]
	TS & Applicare le previsioni su un flusso dati statico e visualizzare correttamente i dati ottenuti all'interno di un grafico contenuto nella dashboard\glo & NI & NE  &\TBstrut \\ [2mm]
	TS & Applicare le previsioni su un flusso dati continuo e visualizzare correttamente i dati ottenuti all'interno di un grafico contenuto nella dashboard\glo & NI & NE  &\TBstrut \\ [2mm]
	TS & Verificare che il sistema permetta all'utente inserire un alert\glo & NI & NE  &\TBstrut \\ [2mm]
	\rowcolor{white}
	\caption{Test di Sistema}
\end{longtable}

\addtocontents{toc}{\protect\setcounter{tocdepth}{4}} %Inserire questo per ripristinare il normale inserimento delle sezioni nell'indice. 4 significa fino al paragrah

\subsection{Test di Integrazione}
I test di integrazione verranno sviluppati in seguito alla progettazione architetturale.

\subsection{Test di Unità}
I test di integrazione verranno sviluppati in seguito alla progettazione di dettaglio e alla codifica.
