\section{Descrizione dei test}
Il nostro gruppo ha scelto di adottare il Modello a V\glosp per garantire la qualità del nostro prodotto. In particolare, questo modello prevede lo sviluppo dei test durante le attività di analisi dei requisiti, progettazione architetturale e progettazione di dettaglio oltre a validazione e collaudo.
In questo modo è possibile verificare la correttezza sia di tutti gli aspetti che compongono il progetto che delle singole parti sviluppate.

Sono state individuate 4 tipologie di test:
\begin{itemize}
	\item Test di Accetazione;
	\item Test di Sistema;
	\item Test di Integrazione;
	\item Test di unità.
\end{itemize}
Ogni volta che viene svolta un'attività viene definita una tabella con i test di una tipologia.
All'interno del documento \textit{Norme di Progetto v.1.1.1} vengono definite le caratteristiche dei test e i codici che identificano univocamente i singoli test.

\begin{itemize}
	\item STATO
	\item completata C;
	\item non completata NC;
	\item ESITO
	\item positivo P;
	\item negativo N;
\end{itemize}

\subsection{Test di Accettazione}
\addtocontents{toc}{\protect\setcounter{tocdepth}{0}} %Inserire questo per escludere una sezione dall'indice.

\rowcolors{2}{gray!25}{gray!15}
\begin{longtable} {
		>{\centering}p{15mm} 
		>{\centering}p{79.5mm}
		>{\centering}p{15mm} 
		>{\centering}p{15mm}
		>{}p{0mm}}
	\rowcolor{gray!50}
	\textbf{Codice} & \textbf{Descrizione} & \textbf{Stato} & \textbf{Esito} &\TBstrut \\
	TA1 & L'utente deve poter addestrare gli algoritmi di previsione dei dati all'interno della piattaforma Grafana & NC & N  &\TBstrut \\ [2mm]
	TA1.1 & L'utente deve poter selezionare dal proprio dispositivo un file JSON che contiene i dati di testing per l'addestramento & NC & N  &\TBstrut \\ [2mm]
	TA1.2 & L'utente deve poter scegliere se addestrare la SVM\glosp o la RL\glosp & NC & N  &\TBstrut \\ [2mm]
	TA1.3 & L'utente deve poter avviare l'addestramento dell'algoritmo & NC & N  &\TBstrut \\ [2mm]
	TA1.4 & L'utente deve poter visualizzare un messaggio di conferma se l'addestramento va a buon fine & NC & N  &\TBstrut \\ [2mm]
	TA2 & L'utente deve poter addestrare gli algoritmi di previsione dei dati su un applicativo esterno a Grafana & NC & N  &\TBstrut \\ [2mm]
	TA2.1 & L'utente deve poter selezionare dal proprio dispositivo un file JSON che contiene i dati di testing per l'addestramento & NC & N  &\TBstrut \\ [2mm]
	TA2.2 & L'utente deve poter scegliere se addestrare la SVM\glosp o la RL\glosp & NC & N  &\TBstrut \\ [2mm]
	TA2.3 & L'utente L'utente deve poter avviare l'addestramento dell'algoritmo & NC & N  &\TBstrut \\ [2mm]
	TA2.4 & L'utente deve poter visualizzare un messaggio di conferma se l'addestramento va a buon fine & NC & N  &\TBstrut \\ [2mm]
	TA2.5 & Verificare che al termine della procedura l'utente riceva in output dall'applicazione un file JSON contenente i parametri per le previsioni & NC & N  &\TBstrut \\ [2mm]
	TA3 & Verificare che, dato un JSON non valido, la fase di addestramento non vada a buon fine & NC & N  &\TBstrut \\ [2mm]
	TA3.1 & L'utente può scegliere se addestrare con regressione non lineare o con rete neurale & NC & N  &\TBstrut \\ [2mm]
	TA4 & L'utente può inserire una data source su Grafana da cui prelevare i dati & NC & N  &\TBstrut \\ [2mm]
	TA4.1 & L'utente può scegliere una data source tra quelle fornite da Grafana:
			\begin{enumerate}
				\item l'utente può selezionare una sorgente di lusso continua o statica;
			\end{enumerate} & NC & N  &\TBstrut \\ [2mm]
	TA4.2 & Verificare che l'utente possa inserire i dati nel form messo a disposizione:
			\begin{enumerate}
				\item inserire l'hostname;
				\item inserire il nome del database;
				\item inserire lo username;
				\item inserire la password;
			\end{enumerate} & NC & N  &\TBstrut \\ [2mm]
	TA4.3 & L'utente può salvare e testare la data source inserita & NC & N  &\TBstrut \\ [2mm]
	TA4.4 & Verificare che, con dei dati corretti, l'inserimento della data source vada a buon fine & NC & N  &\TBstrut \\ [2mm]
	TA5 & Verificare che con un hostname non corretto, l'inserimento non vada a buon fine & NC & N  &\TBstrut \\ [2mm]
	TA6 & Verificare che con un database, username e password non corretti, l'inserimento non vada a buon fine & NC & N  &\TBstrut \\ [2mm]
	TA10 & Verificare che in seguito all'addestramento l'utente visualizzi gli indici di bontà delle previsioni  & NC & N  &\TBstrut \\ [2mm]
	TA15 & L'utente deve poter inserire un alert nel pannello grafico della dashboard & NC & N  &\TBstrut \\ [2mm]
	TA15.1 & L'utente deve poter definire le regole di funzionamento di un alert & NC & N  &\TBstrut \\ [2mm]
	TA15.2 & L'utente deve poter definire le condizioni di funzionamento di un alert & NC & N  &\TBstrut \\ [2mm]
	TA15.3 & L'utente deve poter definire alcuni comportamenti speciali di un alert come l'assenza di dati (DEFINIRE MEGLIO QUESTI COMPOTAMENTI SPECIALI!) & NC & N  &\TBstrut \\ [2mm]
	TA16 & Verificare che il sistema fornisca un messaggio di errore se l'utente definisce in modo errato un alert & NC & N  &\TBstrut \\ [2mm]
	TA17 & Verificare che l'utente possa visualizzare l'elenco di tutti gli alert con i relativi stati & NC & N  &\TBstrut \\ [2mm]
	TA17.1 & Verificare l'esistenza della possibilità di rimuovere un alert & NC & N  &\TBstrut \\ [2mm]
	TA17.2 & L'utente deve poter sospendere un alert in ogni momento & NC & N  &\TBstrut \\ [2mm]
	TA &  & NC & N  &\TBstrut \\ [2mm]
	TA &  & NC & N  &\TBstrut \\ [2mm]
	TA &  & NC & N  &\TBstrut \\ [2mm]
	TA &  & NC & N  &\TBstrut \\ [2mm]
	
\end{longtable}

\addtocontents{toc}{\protect\setcounter{tocdepth}{4}} %Inserire questo per ripristinare il normale inserimento delle sezioni nell'indice. 4 significa fino al paragrah

\subsection{Test di Sistema}
\addtocontents{toc}{\protect\setcounter{tocdepth}{0}} %Inserire questo per escludere una sezione dall'indice.

\rowcolors{2}{gray!25}{gray!15}
\begin{longtable} {
		>{\centering}p{15mm} 
		>{\centering}p{79.5mm}
		>{\centering}p{15mm} 
		>{\centering}p{15mm}
		>{}p{0mm}}
	\rowcolor{gray!50}
	\textbf{Codice} & \textbf{Descrizione} & \textbf{Stato} & \textbf{Esito} &\TBstrut \\
	TS & Verificare che il sistema permetta all'utente di eseguire l'addestramento dei dati all'interno di Grafana & NC & N  &\TBstrut \\ [2mm]
	TS & Verificare che il software di addestramento interno al sistema fornisca un messaggio di conferma di avvenuto addestramento in caso questo vada a buon fine & NC & N  &\TBstrut \\ [2mm]
	TS & Verificare che l'addestramento eseguito all'interno di Grafana si interrompa e generi un errore se viene inserito un file JSON non valido & NC & N  &\TBstrut \\ [2mm]
	TS & Verificare che il sistema permetta all'utente di eseguire l'addestramento dei dati nell'applicazione esterna a Grafana & NC & N  &\TBstrut \\ [2mm]
	TS & Verificare che l'applicazione di addestramento esterno al sistema fornisca un messaggio di conferma di avvenuto addestramento in caso questo vada a buon fine & NC & N  &\TBstrut \\ [2mm]
	TS & Verificare che l'addestramento eseguito nell'applicazione esterna si interrompa e generi un errore se viene inserito un file JSON non valido & NC & N  &\TBstrut \\ [2mm]
	TS & Verificare il sistema permetta di impostare un nuovo alert & NC & N  &\TBstrut \\ [2mm]
	TS & Verificare che il sistema permetta di rimuovere un alert esistente & NC & N  &\TBstrut \\ [2mm]
	TS & Verificare che il sistema permetta di sospendere un alert in ogni momento & NC & N  &\TBstrut \\ [2mm]
	TS &  & NC & N  &\TBstrut \\ [2mm]
	TS &  & NC & N  &\TBstrut \\ [2mm]
	TS &  & NC & N  &\TBstrut \\ [2mm]
	TS &  & NC & N  &\TBstrut \\ [2mm]
	TS &  & NC & N  &\TBstrut \\ [2mm]
	TS &  & NC & N  &\TBstrut \\ [2mm]
	TS &  & NC & N  &\TBstrut \\ [2mm]
	TS &  & NC & N  &\TBstrut \\ [2mm]
	TS &  & NC & N  &\TBstrut \\ [2mm]	
\end{longtable}

\addtocontents{toc}{\protect\setcounter{tocdepth}{4}} %Inserire questo per ripristinare il normale inserimento delle sezioni nell'indice. 4 significa fino al paragrah

\subsection{Test di Integrazione}
I test di integrazione verranno sviluppati in seguito alla progettazione architetturale.

\subsection{Test di Unità}
I test di integrazione verranno sviluppati in seguito alla progettazione di dettaglio e alla codifica.
