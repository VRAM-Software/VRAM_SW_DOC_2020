\section{Informazioni generali}
    \subsection{Informazioni incontro}
        \begin{itemize}
            \item \textbf{Luogo}: \textit{Zucchetti} - Sede di Padova in Via Giovanni Cittadella, 7;
            \item \textbf{Data}: 2019-12-18;
            \item \textbf{Ora d'inizio}: 14.30;
            \item \textbf{Ora di fine}: 16.30;
            \item \textbf{Partecipanti}: \begin{itemize}
                \item Corrizzato Vittorio;
                \item Dalla Libera Marco;
                \item Santagiuliana Vittorio;
                \item Schiavon Rebecca;
                \item Spreafico Alessandro;
                \item Toffoletto Massimo;
                \item Piccoli Gregorio (proponente).
            \end{itemize}
        \end{itemize}
    \subsection{Argomenti trattati}
        In questo primo incontro con il proponente il gruppo ha presentato delle domande riguardanti il capitolato\glo, di seguito un riassunto delle tematiche trattate:
        \begin{enumerate}
            \item preferenze da parte dell'azienda sulla licenza del progetto\glo;
            \item conoscenze in ambito machine learning\glosp richieste per la realizzazione del progetto\glo ed eventuali librerie JavaScript consigliate per la               realizzazione della SVM\glosp e della RL\glo;
            \item funzionamento di Grafana\glosp e tecnologie a esso associate;
            \item applicazione di Java MX\glosp e JMeter\glosp ai fini del progetto\glosp ed eventuale accesso a uno storico dati per il training del modello;
            \item particolari esigenze di testing;
            \item chiarimenti sull'utilizzo di Docker.
        \end{enumerate}
\section{Verbale}
        \subsection{Punto 1}
            Come primo punto si è discussa l'eventuale preferenza da parte del proponente di una particolare licenza open source, essendoci in questo ambito una scelta molto ampia. Il proponente ha affermato che non ha particolari esigenze sotto questo aspetto ma ha consigliato le licenze Apache 2\glo, MIT\glosp o BSD\glosp che trovano applicazioni sia nell'open che nel commerciale. Il gruppo, dopo averne parlato, ha quindi deciso di utilizzare la licenza Apache 2\glo.
        \subsection{Punto 2}
            Successivamente il gruppo ha chiesto chiarimenti riguardanti la parte di machine learning\glosp (argomento che non viene affrontato nel corso di laurea) per capire, in caso, se fossero necessari incontri per approfondimenti su tale tematica. Il proponente si è mostrato disponibile per dei meeting in materia e ha inoltre indicato uno studente della magistrale d'Informatica come eventuale riferimento. Tale studente infatti, ha svolto lo stage curricolare in \textit{Zucchetti} e ha sviluppato una libreria JavaScript per la gestione delle SVM\glosp, argomento di centrale importanza per la realizzazione del progetto\glo. Questa e una libreria per la realizzazione della RL\glosp verranno fornite dall'azienda.
        \subsection{Punto 3}
            Sono state chieste delucidazioni sul funzionamento della piattaforma Grafana\glosp e delle varie tecnologie che ci comunicano. Il proponente ha mostrato al gruppo una dashboard\glosp tipica dell'applicazione, spiegando vari modi di manipolare i dati forniti dal monitoraggio. Dopo di che, ha disegnato su di un foglio uno schema base d'interazione tra Grafana\glosp e le tecnologie a esso associate (database e agenti esterni) descrivendo il percorso di un flusso di dati e dove il gruppo dovrebbe andare ad agire per lo sviluppo del progetto\glo. Ne è risultato che il gruppo dovrà sviluppare il plugin per la piattaforma Grafana\glosp che si occuperà in maniera autonoma di comunicare con il database.
        \subsection{Punto 4}
            Dopo si è trattato l'argomento del training dell'SVM\glosp e della RL\glo. Durante il secondo seminario con \textit{Zucchetti} si era parlato di Java MX\glosp e JMeter\glosp per tali fini, quindi sono stati chiesti chiarimenti aggiuntivi sul funzionamento di questi applicativi. Il proponente ha consigliato di usare questi due software in un sito Web di nostra proprietà (ad esempio quello sviluppato nell'ambito del progetto di Tecnologie Web). Difatti questi programmi riescono a simulare e monitorare il traffico per poi darlo in input ai plugin da noi sviluppati, i quali daranno a loro volta in output dei dati con cui dovremmo riuscire a valutare il funzionamento e l'accuratezza del nostro lavoro. Eventualmente, nel caso non si riuscisse ad applicare suddetta opzione, il training potrà essere fatto direttamente in Grafana\glo. Il gruppo ha deciso perciò di avvalersi di JavaMX\glosp e JMeter\glosp per il training dei plugin.
        \subsection{Punto 5}
            In seguito si è discusso sulle eventuali esigenze per il testing dei plugin. Oltre a quanto detto nel punto precedente, infatti, il gruppo aveva intenzione di applicare un sistema di analisi statica durante la scrittura del software. Il proponente ha affermato l'importanza della correttezza del codice, avvisando però che seguire certe regole potrebbe portare a problemi di sviluppo e a un calo delle prestazioni. Rimane comunque ai componenti la scelta se usare o meno queste tecnologie. Il gruppo, alla fine, ha deciso di utilizzare comunque SonarJS\glosp come strumento di analisi statica continuando in ogni caso a prioritizzare la funzionalità del prodotto\glo.
        \subsection{Punto 6}
            Come ultimo punto è stato chiesto come implementare l'utilizzo di Docker nel capitolato\glo. Il proponente ha attestato che, seppur i componenti sviluppati dall'azienda non usino tale programma, potrebbe essere funzionale nell'ambito di questo progetto\glosp l'ausilio di un container\glosp quale Docker. Il gruppo ha quindi deciso di utilizzare questa tecnologia nello sviluppo del prodotto\glo.
