\section{Capitolato 5 - Stalker}

\subsection{Informazioni}
\begin{itemize}
	\item \textbf{Nome}: Stalker;
	\item \textbf{Proponente}: Imola Informatica;
	\item \textbf{Committente}: Prof. Tullio Vardanega e Prof. Riccardo Cardin.
\end{itemize}

\subsection{Descrizione}
Il progetto\glosp Stalker nasce dalla necessità di tracciare il numero di persone presenti all'interno dei locali, richiesto dalle normative, utile in vari ambiti: dal controllare l'affluenza in luoghi di interesse o ad eventi, al monitorare le presenze del personale nel luogo di lavoro. La soluzione da loro proposta è un'applicazione mobile che invia la posizione dei dispositivi degli utenti a un server, il quale si occuperà dell'estrazione e memorizzazione delle informazioni utili.

\subsection{Obiettivi}
La soluzione software proposta da Imola Informatica prevede lo sviluppo di un'applicazione mobile e un'infrastruttura server a supporto, controllabile da un'interfaccia web di amministrazione. Lo scenario di utilizzo prevede che tutti gli utenti siano muniti di uno smartphone e abbiano la possibilità di installare l'apposita applicazione; i luoghi da monitorare possono essere corredati con access point o beacon Bluetooth per facilitare il tracciamento, in particolare nei luoghi chiusi. Lo scopo del progetto\glosp è quello di monitorare il numero di presenze, in forma anonima o autenticata, in uno o più luoghi circoscritti e fornire statistiche a riguardo.

\subsection{Requisiti di progetto}
Un'applicazione mobile con i seguenti requisiti:
\begin{itemize}
	\item recupero lista organizzazioni;
	\item possibilità di effettuare login tramite LDAP per organizzazioni che lo richiedono;
	\item storico accessi;
	\item visualizzazione real-time della propria presenza in un luogo monitorato e cronometro del tempo trascorso al suo interno.
\end{itemize}
Un'interfaccia web per l'amministrazione con le seguenti funzionalità:
\begin{itemize}
	\item funzionalità di login;
	\item creazione/modifica/eliminazione di organizzazioni;
	\item aggiunta/modifica/rimozione di luoghi, definiti da coordinate geografiche;
	\item configurazione collegamento server LDAP;
	\item invio di notifiche push alle applicazioni per segnalare l'aggiornamento delle liste di organizzazioni e luoghi;
	\item monitoraggio del numero di utenti presenti nei luoghi dell'organizzazione in tempo reale;
	\item ricerca sugli accessi dei dipendenti e creazione di report sulla frequentazione dei luoghi;
	\item gestione delle autorizzazioni per gli utenti dell'interfaccia web.
\end{itemize}
L'infrastruttura su cui si baserà l'applicazione dovrà sottostare ai seguenti vincoli:
\begin{itemize}
	\item comunicazioni solo all'entrata o uscita dall'area interessata;
	\item cifratura di tutte le comunicazioni tra app e server;
	\item architettura scalabile (verticalmente e orizzontalmente) e tollerante a picchi di traffico per il lato server;
	\item test di carico che dimostrino il funzionamento in varie situazioni.
\end{itemize}

\subsection{Tecnologie interessate}
\begin{itemize}
	\item \textbf{Java} o \textbf{Swift}: per lo sviluppo dell'applicazione mobile;
	\item \textbf{NodeJS} o \textbf{Python}: per la gestione del back-end;
	\item utilizzo di protocolli asincroni\glosp per le comunicazioni tra app e server;
	\item \textbf{HTML5}, \textbf{CSS3} e \textbf{JavaScript}: per la creazione dell'interfaccia web lato server;
	\item utilizzo del pattern di Publisher/Subscriber, ovvero mittenti e destinatari di messaggi dialogano attraverso un tramite (dispatcher\glo);
	\item \textbf{IAAS Kubernetes} o \textbf{Openshift} o \textbf{Rancher}: framework IAAS o PAAS per il rilascio delle componenti server;
	\item \textbf{API REST} o \textbf{gRPC}: funzionalità esposte dal server;
	\item \textbf{GPS} o \textbf{sistemi ibridi}: per la geolocalizzazione;
	\item \textbf{LDAP}: Lightweight Directory Access Protocol cioè un protocollo per il servizio di autenticazione;
	\item test unitari e d'integrazione per tutte le componenti applicative.
\end{itemize}

\subsection{Aspetti positivi}
\begin{itemize}
	\item Lo sviluppo di applicazioni mobili è una conoscenza richiesta;
	\item argomento interessante, anche grazie alla potenziale utilità nell'ambito della sicurezza;
\end{itemize}

\subsection{Criticità e fattori di rischio}
\begin{itemize}
	\item I requisiti sono numerosi e richiedono conoscenze non banali;
	\item lo sviluppo lato server potrebbe portare complicazioni, viste le numerose tecnologie coinvolte e i requisiti in scalabilità;
	\item la precisione ottenibile con le tecnologie attuali non è sufficiente per rendere l’applicazione usabile nella realtà;
	\item nella presentazione è stato discusso il funzionamento del sistema GPS, ma completamente tralasciata la funzionalità dei back-end di geolocalizzazione presenti nei moderni sistemi mobile (ad esempio i Google Mobile Services per Android).
\end{itemize}

\subsection{Conclusioni}
Nonostante il nobile obiettivo, la difficile fattibilità di un prodotto\glosp concreto e la possibilità di complicazioni scoraggia il gruppo nella decisione d'intraprendere questo capitolato\glo.
