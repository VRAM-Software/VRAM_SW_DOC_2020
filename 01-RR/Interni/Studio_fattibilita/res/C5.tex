\section{Capitolato 5 - Stalker}

\subsection{Informazioni}
\begin{itemize}
	\item \textbf{Nome}: Stalker
	\item \textbf{Proponente}: Imola Informatica
	\item \textbf{Committente}: Prof. Tullio Vardanega
\end{itemize}

\subsection{Descrizione}
Il progetto Stalker nasce dalla necessità di tracciare il numero di persone presenti all'interno dei locali, richiesto dalle normative, utile in vari ambiti: dal controllare l'affluenza in luoghi di interesse o ad eventi, al monitorare le presenze del personale nel luogo di lavoro. La soluzione da loro proposta è un un'applicazione mobile che invii la posizione dei dispositivi degli utenti ad un server, il quale si occuperà dell'estrazione e memorizzazione delle informazioni utili.

\subsection{Obiettivi}
La soluzione software proposta da Imola Informatica prevede lo sviluppo di un'applicazione mobile e un'infrastruttura server a supporto, controllabile da un'interfaccia web di amministrazione. Lo scenario di utilizzo prevede che tutti gli utenti siano muniti di uno smartphone e abbiano la possibilità di installare l'apposita applicazione; i luoghi da monitorare possono essere corredati con access point\glosp o beacon Bluetooth\glosp per facilitare il tracciamento, in particolare nei luoghi chiusi. Lo scopo del progetto è quello di monitorare il numero di presenze, in forma anonima o autenticata, in uno o più luoghi circoscritti e fornire statistiche a riguardo.

\subsection{Requisiti}
Un'applicazione mobile con i seguenti requisiti:
\begin{itemize}
	\item Recupero lista organizzazioni
	\item Possibilità di effettuare login tramite LDAP\glo per organizzazioni che lo richiedono
	\item Storico accessi
	\item Visualizzazione in tempo reale della propria presenza all'interno di un luogo monitorato e cronometro del tempo trascorso al suo interno
\end{itemize}
Un'interfaccia web per l'amministrazione con le seguenti funzionalità:
\begin{itemize}
	\item Funzionalità di login
	\item Creazione/modifica/eliminazione di organizzazioni
	\item Aggiunta/modifica/rimozione di luoghi, definiti da coordinate geografiche
	\item Configurazione collegamento server LDAP\glo
	\item Invio di notifiche push\glosp alle applicazioni per segnalare l'aggiornamento delle liste di organizzazioni e luoghi
	\item Monitoraggio del numero di utenti presenti nei luoghi dell'organizzazione in tempo reale
	\item Ricerca sugli accessi dei dipendenti e creazione di report sulla frequentazione dei luoghi
	\item Gestione delle autorizzazioni per gli utenti dell'interfaccia web
\end{itemize}
L'infrastruttura su cui si baserà l'applicazione dovrà sottostare ai seguenti vincoli:
\begin{itemize}
	\item Comunicazioni solo all'entrata od uscita dall'area interessata
	\item Cifratura di tutte le comunicazioni tra app e server
	\item Architettura scalabile\glosp (verticalmente e orizzontalmente) e tollerante a picchi di traffico per il lato server
	\item Test di carico che dimostrino il funzionamento in varie situazioni
\end{itemize}

\subsection{Tecnologie interessate}
\begin{itemize}
	\item Java\glo o Swift\glo (applicazione mobile)
	\item NodeJS\glo o Python\glo (back-end\glo)
	\item Utilizzo di protocolli asincroni\glosp per le comunicazioni tra app e server
	\item HTML5\glo, CSS3\glo e Javascript\glo (interfaccia web lato server)
	\item Utilizzo del pattern di Publisher/Subscriber\glo, ovvero mittenti e destinatari dimessaggi dialogano attraverso un tramite(dispatcher)
	\item Utilizzo dell’IAAS Kubernetes o di un PAAS, Openshift o Rancher, per il rilascio delle componenti server
	\item API REST\glosp esposte dal server, o gRPC\glo in alternativa
	\item GPS\glo/sistemi ibridi di geolocalizzazione
	\item LDAP\glo (Lightweighgt Directory Access Protocol)
	\item Test unitari\glosp e d'integrazione per tutte le componenti applicative
\end{itemize}

\subsection{Aspetti positivi}
\begin{itemize}
	\item Lo sviluppo di applicazioni mobili è una conoscenza richiesta
	\item Argomento interessante, anche grazie alla potenziale utilità nell'ambito della sicurezza
\end{itemize}

\subsection{Criticità e fattori di rischio}
\begin{itemize}
	\item I requisiti sono numerosi e richiedono conoscenze non banali
	\item Lo sviluppo lato server potrebbe portare complicazioni, viste le numerose tecnologie coinvolte e i requisiti in scalabilità
	\item La precisione ottenibile con le tecnologie attuali non è sufficiente per rendere l’applicazione usabile nella realtà
	\item Nella presentazione è stato discusso il funzionamento del sistema GPS ma completamente tralasciata la funzionalità dei backend di geolocalizzazione presenti nei moderni sistemi mobile (ad esempio i Google Mobile Services per Android)
\end{itemize}

\subsection{Conclusioni}
Nonostante il nobile obiettivo, la difficile fattibilità di un prodotto concreto e la possibilità di complicazioni scoraggia il gruppo nella decisione d'intraprendere questo capitolato.
