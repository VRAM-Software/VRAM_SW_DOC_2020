\section{Capitolato 3 - NaturalAPI}
\subsection{Informazioni generali}
\begin{itemize}
	\item Nome: NaturalAPI;
	\item Proponente: Teal Blue;
	\item Committente: Prof. Tullio Vardanega e Prof. Riccardo Cardin;
\end{itemize}
\subsection{Descrizione}
L'obbiettivo di questo capitolato è quello di far parlare a tutti gli stakeholders\glosp un linguaggio comune, in modo da velocizzare ed evitare confusione durante la progettazione. Teal Blue vuole utilizzare le specifiche e i requisiti di un software, scritti in linguaggio naturale (Inglese, Italiano, ecc.), per generare API\glo.
\subsection{Finalità del progetto}
Il prodotto da sviluppare è \textbf{NaturalAPI} un toolkit\glosp che dovrà generare API\glosp complete e test automatici a partire da un linguaggio naturale. Per la realizzazione di questo prodotto dovranno essere sviluppati tre PoC\glosp o feature\glo.
\begin{enumerate}
	\item \textbf{NaturalAPI Discover: }\textit{estrattore di BDL\glo}. \\*
	L'estrattore \textit{Discover} ha lo scopo di estrarre entità, processi e combinazioni tra essi da un documento testuale di business.
	\item \textbf{NaturalAPI Design: }\textit{parser\glosp di scenari e caratteristiche}. \\*
	Questa PoC\glosp dovrà creare una BAL\glosp API\glosp in tempo reale a partire dai documenti di Gherkin\glosp e da un BDL\glo.
	\item \textbf{NaturalAPI Develop: }\textit{esportatore di linguaggio}. \\* 
	Questa feature\glosp dovrà convertire un BAL\glosp in casi di test e API\glosp nel linguaggio di programmazione scelto, supportando la creazione e l'aggiornamento di nuove repository\glo.
\end{enumerate} 
Ogni feature\glosp sopraindicata dovrà essere accessibile attraverso almeno due dei seguenti modi: interfaccia da linea di comando, GUI\glosp minimale o un'interfaccia web REST\glo.
La parte logica del prodotto finale dovrà essere esportata in una delle seguenti modalità: come una libreria, come parte di un eseguibile o come un processo indipendente locale o remoto.  


\subsection{Tecnologie interessate}
\begin{itemize}
	\item NLP\glosp o Natural Language Processing cioè un trattamento informatico del linguaggio naturale, che si occupa della realizzazione di sistemi in grado di comprendere il linguaggio naturale	
	\item Dependency Parser\glosp cioè un parser che si occupa di analizzare la struttura grammaticale di una frase in linguaggio naturale per identificare le relazioni tra parole chiavi e parole che le modificano
	\item BDD\glosp Behaviour Driven Development che è un processo di sviluppo software agile che ha, alla sua base, una continua comunicazione tra tutti gli stakeholders\glosp di un progetto informatico
	\item Hiptest\glosp e Cucumber\glosp che sono degli strumenti software che supportano il processo BDD\glo; il primo serve per eseguire test automatici, mentre il secondo legge le specificazioni software in linguaggio naturale, scritte con alcune regole di sintassi (Gherkin\glo), e controlla che il software rispetti i requisiti
	\item Generazione API\glosp e DLS\glosp utilizzando:
	\begin{itemize}
		\item OpenAPI\glo: uno standard per descrivere API\glosp
		\item Swagger\glo: un framework\glosp che aiuta a sviluppare servizi web REST\glosp
		\item OWL\glo: un linguaggio web semantico
	\end{itemize}
	\item Un qualsiasi framework\glosp a scelta come Qt\glo, React\glo, ecc.
\end{itemize} 
\subsection{Aspetti positivi}
\begin{itemize} 
	\item Il proponente non impone vincoli sui linguaggi di programmazione da utilizzare, questo permette  al gruppo di sviluppare un prodotto con un linguaggio o un framework\glosp considerato più interessante. 
	\item I requisiti, le tecnologie e il modo di esportare il prodotto finale sono spiegati in modo preciso e esauriente.
	\item L'azienda Teal Blue si dimostra disponibile a incontri ed a una comunicazione aperta con il fornitore.
\end{itemize}
\subsection{Criticità e fattori di rischio}
\begin{itemize}
	\item Questo capitolato non ha suscitato molto interesse nel gruppo a causa dell'eccessiva astrattezza del capitolato e per il fatto che i concetti su cui prepararsi non erano, per il gruppo VRAM Software, stimolanti. 
	\item Le tecnologie da imparare sono molte ed è arduo quantificare il tempo necessario per raggiungere una preparazione sufficiente per gestire in modo produttivo le tecnologie elencate.
\end{itemize}
\subsection{Conclusione}
Il gruppo ha trovato interessante l'idea di Teal Blue di rendere la comunicazione tra tutti stakeholders\glosp chiara e veloce, tuttavia abbiamo deciso di orientarci verso un progetto meno astratto.