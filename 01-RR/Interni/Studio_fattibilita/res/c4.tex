\section{Capitolato 4 - Predire in Grafana}

\subsection{Informazioni generali}
\begin{itemize}
	\item Nome: Predire in Grafana\glo: Monitoraggio predittivo per DevOps\glo;
	\item Proponente: Zucchetti;
	\item Committente: Prof. Tullio Vardanega e Prof. Riccardo Cardin;
\end{itemize}

\subsection{Descrizione}
Con questo capitolato, Zucchetti mira a realizzre due plug-in\glosp per lo strumento di monitoraggio Grafana\glo che applichi gli algoritmi di Support Vector Machine(SVM)\glosp e Regressione Lineare(RL)\glosp al flusso di dati che sono stati ricevuti per allarmi o segnalazione tra la linea di produzione del software e le segnalazioni tra gli operatori del servizio Cloud\glo.

\subsection{Requisiti e finalità di progetto}
Il sistema richiede due plug-in\glosp di Grafana\glosp scritti in Javascript\glosp che eseguiranno i calcoli letti da un file json\glosp e produrranno valori tali da essere aggiunti al flusso di monitoraggio. In particolare viene richiesto di:
\begin{itemize}
	\item Produrre un file json\glosp dai dati di addestramento con i parametri per le previsioni attraverso SVM\glosp per le classificazioni o RL\glo
	\item Leggere la definizione del predittore\glosp dal file in formato json\glo
	\item Associare i predittori\glosp letti dal file json\glosp al flusso di dati presente in Grafana\glo
	\item Applicare la previsione e fornire i nuovi dati ottenuti dalla previsione al sistema di Grafana\glo
	\item Rendere disponibili i dati al sistema di creazione di grafici e dashboard\glosp per la loro visualizzazione
\end{itemize}
Opzionalemente è richiesto di implementare le seguenti funzionalità:
\begin{itemize}
	\item Offrire la possibilità di definire degli "alert\glo" in base a livelli di soglia raggiunti dai nodi collegati alle previsioni
	\item Fornire i dati che definiscono bontà dei modelli di previsione utilizzati
	\item Offrire la possibilità di applicare delle trasformazioni alle misure lette dal campo per ottenere delle regressioni\glosp esponenziali o logaritmiche
	\item Offrire la possiblità di addestrare SVM\glosp o RL\glosp direttamente in Grafana\glo
	\item Implementare dei meccanismi di apprendimento di flusso, per poter disporre di sistemi di previsione che si adattano costantemente ai dati rilevati sul campo
	\item Utilizzare altri metodi di previsione che possano dare benefici alla previsione dei dati, tra cui la versione delle SVM\glosp adattate alla regressione\glosp o piccole reti neurali\glosp per la classificazione
\end{itemize}

\subsection{Tecnologie interessate}
\begin{itemize}
	\item \textbf{Javascript}: linguaggio di programmazione richiesto per costruire i plug-in\glosp di Grafana\glosp ed eseguire i calcoli tramite SVM\glosp e RL\glosp per fare le previsioni
	\item \textbf{Grafana}: sistema di monitoraggio open-source\glosp estendibile con plug-in\glosp javascript\glo
	\item \textbf{Libreria Javascript per reti neurali}: librerie che permettono di implementare reti neurali\glosp per migliorare le previsioni, utili a soddisfare uno dei requisiti opzionali
\end{itemize}
 
\subsection{Aspetti positivi}
\begin{itemize} 
	\item Svolgere questo capitolato permette di acquisire competenze in ambito di analisi e previsione dei dati, molto richieste in ambito lavorativo
	\item La presentazione del problema è stata molto chiara e l'azienda è apparsa molto disponibile.
	\item Gli algoritmi di SVM\glosp e RL\glosp verranno forniti dall'azienda
\end{itemize}
\subsection{Criticità e fattori di rischio}
\begin{itemize}
	\item Gli algoritmi da utilizzare sono basati su concetti matematici e ciò ha demotivato il gruppo per lo svolgimento 
	\item Servono delle basi di machine learning\glosp non fornite nel nostro corso di laurea
\end{itemize}
\subsection{Conclusione}
Il gruppo ha trovato molto chiaro e definito il problema, ma i concetti matematici richiesti per affrontarlo hanno demotivato l'interesse del gruppo verso questa scelta.