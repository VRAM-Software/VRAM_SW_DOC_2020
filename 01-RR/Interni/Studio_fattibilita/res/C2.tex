\section{Capitolato 2 - Etherless}
\subsection{Informazioni generali}
    \begin{itemize}
        \item \textbf{Nome}: Etherless;
        \item \textbf{Proponente}: Red Babel;
        \item \textbf{Committente}: Prof. Tullio Vardanega e Prof. Riccardo Cardin.
    \end{itemize}
\subsection{Descrizione}
    Etherless è una Cloud Application Platform che permette agli sviluppatori che la utilizzano di caricare nel cloud\glosp delle funzioni
    JavaScript. Tali procedure possono poi essere acquistate da terzi tramite l'impiego della criptovaluta Ethereum implementata
    grazie alla tecnologia della blockchain.
\subsection{Obiettivi di progetto}
    Il progetto\glosp si propone di aiutare gli sviluppatori fornendo:
    \begin{itemize}
        \item un framework Serverless\glosp che gestisca il costo computazionale della funzione;
        \item un servizio di Smart Contracts\glosp che sovrintenda il processo di pagamento.
    \end{itemize}
    Quest'ultima tecnologia sarà di supporto anche per l'acquirente in quanto assicurerà il completamento della transazione solo a lavoro verificato e completato.
\subsection{Requisiti di progetto}
È richiesto di dividere lo sviluppo in tre fasi:
\begin{itemize}
   	\item \textbf{Local}: utilizzo dell'applicativo in ambiente locale, per cui può essere utilizzato Ethereum testrpc di Truffle per
   	l'emulazione\glosp della blockchain;
   	\item \textbf{Test}: utilizzo dell'applicativo in ambiente di testing, in cui può essere usata la soluzione proposta al punto precedente;
   	\item \textbf{Staging}: utilizzo dell'applicativo in un ambiente pubblico, in tal caso si potrà usufruire di Ropsten Ethereum come rete di testing prima del rilascio.
\end{itemize}
Inoltre è obbligatorio separare il software in tre parti:
\begin{itemize}
   	\item \textbf{etherless-cli}: modulo attraverso il quale lo sviluppatore interagisce con Etherless;
   	\item \textbf{etherless-smart}: modulo per l'interazione tra etherless-cli e la parte server;
   	\item \textbf{etherless-server}: modulo che ascolta gli eventi emessi da etherless-smart per attivare le rispettive funzioni lambda\glo. 
\end{itemize}
Ognuna di queste operazioni dovrà poi essere caricata e versionata\glosp tramite GitHub o GitLab.
\subsection{Tecnologie interessate}
    Per l'implementazione delle varie funzionalità vengono date dall'azienda delle linee guida sulle tecnologie da utilizzare:
    \begin{itemize}
        \item \textbf{TypeScript 3.6}: linguaggio di programmazione da impiegare, tramite l'approccio Promise\glosp o async-await\glo, nello sviluppo della piattaforma Etherless;
        \item \textbf{Solidity}: linguaggio per la creazione e la gestione degli Smart Contracts\glo;
        \item \textbf{AWS Lambda}: piattaforma computazione serverless\glosp fornita da Amazon per la coordinazione degli eventi;
        \item \textbf{Serverless Framework}: framework web per la creazione di applicazioni su AWS Lambda;
        \item \textbf{typescript-eslint}: strumento di analisi statica\glosp del codice TypeScript.
    \end{itemize}
    AWS Lambda può essere corredata con altri componenti quali: AWS API Gateway (per eventi HTTP), AWS DynamoDB (database non relazionale)
    o AWS S3 (servizio di memorizzazione). Tutto questo poi può essere gestito tramite AWS CloudFormation (piattaforma per l'organizzazione delle
    risorse AWS).
\subsection{Aspetti positivi}
\begin{itemize}
    \item Ethereum e la tecnologia della blockchain più in generale sono tematiche molto attuali e innovative che suscitano interesse tra i componenti del gruppo
    anche per le possibili applicazioni future;
    \item l'azienda ha presentato delle richieste chiare e precise, aspetto valutato positivamente dal gruppo.
\end{itemize}
\subsection{Criticità e fattori di rischio}
\begin{itemize}
    \item L'azienda ha sede in Olanda, il rapporto con i proponenti potrebbe essere quindi meno efficace;
    \item seppur siano proposte tematiche interessanti il gruppo ha presentato dei dubbi riguardanti la difficoltà e le tempistiche per l'approfondimento
    delle tecnologie presentate.
\end{itemize}
\subsection{Conclusioni}
Il capitolato\glosp è apparso stimolante per quanto riguarda le tecnologie utilizzate e convincente nella sua esposizione. Tuttavia la distanza geografica e la
mole di lavoro prevista sono risultate un ostacolo nell'effettiva realizzazione del software. Per questo motivi il gruppo ha deciso di orientare la sua scelta
su un altro progetto\glo.
