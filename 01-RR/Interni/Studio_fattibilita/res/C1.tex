\section{Capitolato 1 - Autonomous Highlights Platform}
\subsection{Informazioni generali}
\begin{itemize}
	\item \textbf{Nome}: Autonomous Highlights Platform;
	\item \textbf{Proponente}: Zero12;
	\item \textbf{Committente}: Prof. Tullio Vardanega e Prof. Riccardo Cardin;
\end{itemize}
\subsection{Descrizione}
Autonomous Highligths Platform prevede lo sviluppo di una piattaforma web capace di ricevere in input dei video di eventi sportivi e fornire in output un video contenente i momenti salienti.
\subsection{Obiettivi del progetto}
Il progetto\glosp prevede la creazione di un software che riceve in input dei video di eventi sportivi come una partita di calcio, Formula 1, Moto GP o altri sport e riesca a creare in autonomia un video di massimo 5 minuti con i soli suoi momenti salienti (highlights) da fornire in output. A questo fine la piattaforma dovrà essere dotata di un modello di machine learning\glosp in grado d'identificare ogni momento importante dell’evento.
In particolare il flusso di generazione del video sarà così organizzato:
\begin{itemize}
	\item Caricamento del video dell'evento desiderato;
	\item Individuazione degli highlights;
	\item Estrazione delle parti individuate dal video originario;
	\item Generazione del video dei momenti salienti.
\end{itemize}
\subsection{Requisiti di progetto}
Al fine del corretto svolgimento del progetto\glosp sarà necessario rispettare i seguenti vincoli forniti dall'azienda proponente:
\begin{itemize}
	\item Utilizzo di Sage Maker per la costruzione del modello d'intelligenza artificiale\glo;
	\item Strutturazione di un'architettura a multiservizi\glo;
	\item Permettere il caricamento dei video tramite riga di comando;
	\item Creazione di un'interfaccia web per l'analisi e il controllo dello stato di elaborazione del video.
\end{itemize}
\subsection{Tecnologie interessate}
Le tecnologie previste per la realizzazione del capitolato\glosp sono:
\begin{itemize}
	\item \textbf{Amazon Web Services}\glo, in particolare:
	\begin{itemize}
		\item \textbf{AWS Elastic Container Service} come servizio per la gestione dei contenitori\glosp ad alte prestazioni;
		\item \textbf{AWS Dynamo DB}, database non relazionale del tipo NoSQL per il supporto e la gestione dei tag;
		\item \textbf{AWS Elastic Transcoder} per la conversione e la rielaborazione dei video;
		\item \textbf{AWS Sage Maker} per la creazione di un modello per il riconoscimento degli eventi salienti se si sceglie d'implementare un addestramento supervisionato dell'intelligenza artificiale\glo;
		\item \textbf{AWS Rekognition} da usare nel caso si scelga, invece, l'apprendimento non supervisionato e quindi l'utilizzo di modelli già esistenti.
	\end{itemize}
	\item \textbf{NodeJS} per lo sviluppo di API Restful\glosp JSON al fine di garantire una scalabilità ottimale;
	\item \textbf{Python} per lo sviluppo delle componenti necessarie di machine learning\glo;
	\item \textbf{HTML5}, \textbf{CSS3} e \textbf{JavaScript} per lo sviluppo dell'interfaccia web. 
\end{itemize}
\subsection{Aspetti positivi}
\begin{itemize}
	\item Le tecnologie proposte risultano innovative e utili per la loro larga diffusione nel mondo del lavoro;
	\item La documentazione delle tecnologie reperibile è ben approfondita;
	\item Le specifiche del capitolato\glosp sono state fornite in modo chiaro e preciso. 
\end{itemize}
\subsection{Criticità e fattori di rischio}
\begin{itemize}
	\item La maggior parte delle tecnologie proposte, seppure molto interessanti, non è prevista dal nostro corso di laurea. Dunque lo svolgimento di questo capitolato\glosp richiederebbe quindi un numero di ore di apprendimento difficilmente quantificabile;
	\item Si è ritenuto non semplice e dispendioso in termini di tempo il reperimento e l'analisi dei video per l'apprendimento della macchina.
\end{itemize}
\subsection{Conclusioni}
Lo scopo del capitolato\glosp è risultato accattivante, tuttavia la quantità di nuove tecnologie da apprendere ha demotivato il gruppo a scegliere questo progetto\glo.