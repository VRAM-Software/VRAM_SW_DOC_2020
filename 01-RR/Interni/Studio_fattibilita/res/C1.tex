\subsection{Capitolato C1 - Autonomous Highlights Platform}
\subsubsection{Informazioni generali}
\begin{itemize}
	\item \textbf{Nome}: Autonomous Highlights Platform
	\item \textbf{Proponente}: Zero12
	\item \textbf{Committente}: Prof. Tullio Vardanega e Prof. Riccardo Cardin
\end{itemize}
\subsubsection{Descrizione}
Creazione di un software in grado di ricevere in input video di eventi sportivi e che crei un video contenente i momenti salienti dell'evento.
\subsubsection{Obiettivi di progetto}
Il progetto prevede la creazione di un software che riceve in input dei video di eventi sportivi come una partita di calcio, Formula1, Motogp o altro sport e riesca a creare in autonomia un video di massimo 5 minuti con i soli suoi momenti salienti (highlights). A questo fine la piattaforma dovrà essere dotata di un modello di machine learning\glosp in grado di identificare ogni momento importante dell’evento.
In particolare il flusso di generazione del video sarà così organizzato:
\begin{itemize}
	\item Caricamento del video dell'evento desiderato;
	\item Individuazione degli highlights;
	\item Estrazione delle parti individuate dal video originario;
	\item Generazione del video dei momenti salienti.
\end{itemize}
\subsubsection{Requisiti di progetto}
Al fine del corretto svolgimento del progetto sarà necessario rispettare i seguenti vincoli forniti dall'azienda proponente:
\begin{itemize}
	\item Utilizzo di Sage Maker\glosp per la costruzione del modello di intelligenza artificiale\glo;
	\item Strutturazione di un'architettura a multiservizi\glo;
	\item Permettere il caricamento dei video tramite riga di comando;
	\item Creazione di un'interfaccia web per l'analisi e il controllo dello stato di elaborazione del video.
\end{itemize}
\subsubsection{Tecnologie interessate}
Le tecnologie previste per la realizzazione del capitolato sono:
\begin{itemize}
	\item \textbf{Amazon web services\glo}, in particolare:
	\begin{itemize}
		\item \textbf{AWS Elastic Container Service\glo} come servizio per la gestione dei contenitori;
		\item \textbf{AWS Dynamo DB\glo}, database non relazionale in NoSQL\glosp per il supporto e la gestione dei tag
		\item \textbf{AWS Elastic Transcoder\glo} per la conversione e la rielaborazione dei video
		\item \textbf{AWS Sage Maker\glo} per la creazione di un modello per il riconoscimento degli eventi salienti se si sceglie di implementare un addestramento supervisionato dell'intelligenza artificiale;
		\item \textbf{AWS Rekognition\glo} da usare nel caso si scelga, invece, l'apprendimento non supervisionato e quindi l'utilizzo di modelli già esistenti.
	\end{itemize}
	\item \textbf{NodeJS\glo} per lo sviluppo di API Restful JSON\glosp al fine di garantire una scalabilità ottimale;
	\item \textbf{Python\glo} per lo sviluppo delle componenti necessarie di machine learning;
	\item \textbf{HTML5\glo}, \textbf{CSS3\glo} e \textbf{Javascript\glo} per lo sviluppo dell'interfaccia web. 
\end{itemize}
\subsubsection{Aspetti positivi}
\begin{itemize}
	\item Il progetto è stato ritenuto interessante dal gruppo per la sua tematica accattivante e vicina agli interessi personali di ogni membro;
	\item Le tecnologie proposte risultano innovative e utili per la loro larga diffusione nel mondo del lavoro;
	\item La documentazione delle tecnologie reperibile è ben approfondita;
	\item Le specifiche del capitolato sono state fornite in modo chiaro e preciso. 
\end{itemize}
\subsubsection{Criticità e fattori di rischio}
\begin{itemize}
	\item La maggior parte delle tecnologie proposte non è prevista dal nostro corso di laurea, lo svolgimento di questo capitolato richiederebbe quindi un numero di ore di apprendimento difficilmente quantificabile.
\end{itemize}
\subsubsection{Conclusioni}
Lo scopo del capitolato è risultato accattivante, tuttavia la quantità di nuove tecnologie da apprendere ha demotivato il gruppo a scegliere questo progetto.