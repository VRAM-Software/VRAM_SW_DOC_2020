\section{Capitolato 3 - NaturalAPI}
\subsection{Informazioni generali}
\begin{itemize}
	\item \textbf{Nome}: NaturalAPI;
	\item \textbf{Proponente}: Teal Blue;
	\item \textbf{Committente}: Prof. Tullio Vardanega e Prof. Riccardo Cardin.
\end{itemize}
\subsection{Descrizione}
L'obiettivo di questo capitolato\glosp è far parlare a tutti gli stakeholder\glosp un linguaggio comune, in modo da velocizzare ed evitare confusione durante la progettazione\glo. Teal Blue vuole utilizzare le specifiche e i requisiti di un software, scritti in linguaggio naturale (inglese, italiano, ecc.), per generare API\glo.

\subsection{Obiettivi di progetto}
Il prodotto\glosp da sviluppare è NaturalAPI un toolkit\glosp che dovrà generare API\glosp complete e test automatici a partire da un linguaggio naturale. Il workflow\glosp di questo prodotto\glosp consiste nel ricevere in input del testo in linguaggio naturale. Da questo testo verrà generato un BDL (Business Domain Language). Dal BDL dovrà essere sviluppato un BAL (Business Application Language) che, in seguito, verrà convertito in una API\glo.

\subsection{Requisiti di progetto}
Per la realizzazione di questo prodotto\glosp è richiesto lo sviluppo di tre PoC\glosp o feature\glo.
\begin{itemize}
	\item \textbf{NaturalAPI Discover}: estrattore di BDL. \\*
	L'estrattore Discover ha lo scopo di estrarre entità, azioni e combinazioni tra esse da un documento testuale contenente termini specifici di un dato business;
	\item \textbf{NaturalAPI Design}: parser di scenari e caratteristiche. \\*
	Questa PoC\glosp dovrà creare una BAL API\glosp in tempo reale a partire dai documenti di Gherkin e da un BDL;
	\item \textbf{NaturalAPI Develop}: esportatore di linguaggio. \\*
	Questa feature\glosp dovrà convertire un BAL in casi di test e API\glosp nel linguaggio di programmazione scelto, supportando la creazione e l'aggiornamento di nuove repository\glo.
\end{itemize}
Ogni feature\glosp sopraindicata dovrà essere accessibile attraverso almeno due dei seguenti modi: interfaccia da linea di comando, GUI minimale o un'interfaccia web REST\glo. La parte logica del prodotto\glosp finale dovrà invece essere esportata in una delle seguenti modalità: come una libreria, come parte di un eseguibile o come un'applicazione indipendente locale o remota. 
Per quanto riguarda l'input e l'output bisognerà sottostare allo standard UTF-8 e all'indentatura UNIX. 

\subsection{Tecnologie interessate}
\begin{itemize}
	\item \textbf{NLP}: Natural Language Processing cioè un trattamento informatico del linguaggio naturale, che si occupa della realizzazione di sistemi in grado di comprendere il linguaggio naturale;	
	\item \textbf{Dependency Parser}: cioè un parser che si occupa di analizzare la struttura grammaticale di una frase in linguaggio naturale per identificare le relazioni tra parole chiave e parole che le modificano;
	\item \textbf{BDD}: Behaviour Driven Development cioè un processo\glosp di sviluppo software agile che ha, alla sua base, una continua comunicazione tra tutti gli stakeholder\glosp di un progetto\glosp informatico;
	\item \textbf{Hiptest} e \textbf{Cucumber}: sono degli strumenti software che supportano il BDD; il primo serve per eseguire test automatici, mentre il secondo legge le specifiche software in linguaggio naturale, scritte con alcune regole di sintassi (Gherkin), e controlla che il software rispetti i requisiti;
	\item Generazione API\glosp e DLS utilizzando:
	\begin{itemize}
		\item \textbf{OpenAPI}: uno standard per descrivere API\glo;
		\item \textbf{Swagger}: un framework che aiuta a sviluppare servizi web REST\glo;
		\item \textbf{OWL}: un linguaggio web semantico.
	\end{itemize}
	\item Un qualsiasi framework a scelta come Qt, React, ecc.
\end{itemize} 
\subsection{Aspetti positivi}
\begin{itemize} 
	\item Il proponente non impone vincoli sui linguaggi di programmazione da utilizzare, questo permette al gruppo di sviluppare un prodotto\glosp con un linguaggio o un framework considerato più interessante;
	\item i requisiti, le tecnologie e il modo di esportare il prodotto\glosp finale sono spiegati in modo preciso ed esauriente;
	\item l'azienda Teal Blue si dimostra disponibile a incontri e a una comunicazione aperta con il fornitore.
\end{itemize}
\subsection{Criticità e fattori di rischio}
\begin{itemize}
	\item Questo capitolato\glosp non ha suscitato molto interesse a causa dell'eccessiva astrattezza e per il fatto che i concetti su cui prepararsi non erano, per il gruppo \textit{VRAM Software}, stimolanti;
	\item le tecnologie da imparare sono molte ed è arduo quantificare il tempo necessario per raggiungere una preparazione sufficiente per gestire in modo produttivo le suddette.
\end{itemize}
\subsection{Conclusione}
Il gruppo ha trovato interessante l'idea di Teal Blue di rendere la comunicazione tra tutti stakeholder\glosp chiara e veloce, tuttavia abbiamo deciso di orientarci verso un capitolato\glosp meno astratto.
