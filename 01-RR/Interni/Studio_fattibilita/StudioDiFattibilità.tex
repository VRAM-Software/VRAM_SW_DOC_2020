\documentclass[12pt,a4paper]{article} %formato del documento e grandezza caratteri

% Tutti i pacchetti usati, da inserire nel preambolo prima delle configurazioni

\usepackage[T1]{fontenc} %serve per andare a capo correttamente in italiano
\usepackage[utf8]{inputenc} %serve per le lettere accentate
\usepackage[english,italian]{babel} %serve per far apparire "indice" al posto di "contents"; imposta italiano lingua principale, inglese secondaria.

\usepackage{graphicx} %serve per includere le immagini

\usepackage{hyperref} %serve per rendere clickabili le sezioni dell'indice

\usepackage{float} %per poter usare l'opzione [H] nelle figure; [H] permette di tenere fissate le immagini che altrimenti LaTeX si sposta a piacere.

\usepackage{listings} %serve per poter mettere snippets di codice nel testo

\usepackage{lastpage} %serve per poter fare " \rfoot{\thepage\ di \pageref{LastPage}} "

\usepackage{fancyhdr} %Per header personalizzati %include il file package.tex
\input{config/config_latex.tex} %inclusione file che permette di modificare i vari comandi custom
%Preambolo: la parte prima del \begin{document}

\documentclass[12pt,a4paper]{article} %formato del documento e grandezza caratteri

% Tutti i pacchetti usati, da inserire nel preambolo prima delle configurazioni

\usepackage[T1]{fontenc} %Permette la sillabazione su qualsiasi testo contenente caratteri
\usepackage[utf8]{inputenc} %Serve per usare la codifica utf-8
\usepackage[english,italian]{babel} %Imposta italiano lingua principale, inglese secondaria. Es. serve per far apparire "indice" al posto di "contents"

\usepackage{graphicx} %Serve per includere le immagini

\usepackage[hypertexnames=false]{hyperref} %Gestisce i riferimenti/link. Es. Serve per rendere clickabili le sezioni dell'indice

\usepackage{float} %Serve per migliore la definizione di oggetti fluttuanti come figure e tabelle. Es. poter usare l'opzione [H] nelle figure ovvero tenere fissate le immagini che altrimenti LaTeX si sposta a piacere.

\usepackage{listings} %Serve per poter mettere snippets di codice nel testo

\usepackage{lastpage} %Serve per poter introdurre un'etichetta a cui si può fare riferimento Es. piè di pagina; poter fare " \rfoot{\thepage\ di \pageref{LastPage}} "

\usepackage{fancyhdr} %Per header e piè di pagina personalizzati

%Sono alcuni package che potranno esserci utili in futuro
%\usepackage{charter}
%\usepackage{eurosym}
\usepackage{subcaption}
%\usepackage{wrapfig}
%\usepackage{background}
\usepackage{longtable} % tabella che può continuare per più di una pagina
\usepackage[table]{xcolor} % ho dovuto aggiungere table in modo da poter colorare le row della tabella, dava: undefined control sequences
%\usepackage{colortbl}

\usepackage{dirtree} % usato per creare strutte tree-view in stile filesystem
\usepackage{xspace} % usato per inserire caratteri spazio
\usepackage[official]{eurosym}
\usepackage{pdflscape} %include il file package.tex

% Configurazioni varie, da inserire nel preambolo dopo i pacchetti

\hypersetup{hidelinks} %serve per nascondere riquadri rossi che circondano i link 

\lstset{literate= {à}{{\`a}}1 } %Permette di usare lettere accentate nei listings

\pagestyle{fancy} %Imposto stile pagina
\fancyhf{} %Reset, se lo tolgo LaTex mette impostazioni di default (p.es numerazione pagine di default)

\lhead{\includegraphics[scale=0.25]{img/Logo_header.png}} %Left header che compare in ogni pagina
\rhead{\leftmark} %Nome della top-level structure (p.es. Section in article o Chapter in book) in ogni pagina

%Setto il colore dei link
\hypersetup{
	colorlinks,
	linkcolor=[HTML]{404040},
	citecolor={purple!50!black},
	urlcolor={blue!50!black}
}

%Tabelle e tabulazione (può tornare utile)
%\setlength{\tablcolsep}{10pt}
%\renewcommand{\arraystretch}{1.4}

%Comando per aggiungere le pagine di ogni sezione
\newcommand{\newSection}[1]{%
	\newpage
	\input{res/sections/#1}
}

% Comandi per aggiungere padding a parole contenute nella tabella; è una specie di strut (un carattere invisibile)
\newcommand\Tstrut{\rule{0pt}{2.6ex}} % top padding
\newcommand\Bstrut{\rule[-0.9ex]{0pt}{0pt}} % bottom padding
\newcommand{\TBstrut}{\Tstrut\Bstrut} % top & bottom padding

\newcommand{\glo}{$_G$} %Comando per aggiungere il pedice G
\newcommand{\glosp}{$_G$ } %Comando per aggiungere il pedice G con spazio

\begin{document}

% #### FRONTESPIZIO (frontmatter) ####
\setlength{\headheight}{39pt} %distanzia l'header
\pagenumbering{gobble} %Toglie il numero di pagina
\begin{titlepage}
	\begin{center}
		\includegraphics[scale=0.6]{img/logo.png} \\ %Logo
		\vspace{0.5cm} %Aggiunge uno spazio verticale di 0.5 cm
		
		{\LARGE Progetto "Predire in Grafana"} \\ %Autore, prende variabile definita nel config_latex.tex
		\vspace{0.5cm} %Attenzione a mettere il punto e NON la virgola
		
		{\Huge \textbf{\DocTitle}} \\ %Titolo, prende variabile definita nel config_latex.tex
		\vspace{0.5cm}
		
		\DocDate \\ %Data, prende variabile definita nel config_latex.tex
		\vspace{1cm}
		
		%Allineamento colonne: l=left r=right c=center, 
		%va specificato per ogni colonna
		%Se si vuole la riga tra colonne mettere "|"
		
		\begin{tabular}{r | l} %Elementi colonne separate da "&", le righe finiscono con "\\"
			Versione & \ver \\
			Approvazione & \app \\ %
			Redazione & \red \\
			Verifica & \test \\
			Stato & \stat \\
			Uso & \use \\
		    Destinato a & Azienda \\
						& Prof. Tullio Vardanega \\
						& Prof. Riccardo Cardin \\
			Email di riferimento& vram.software@gmail.com
		\end{tabular}
		\vfill
		\textbf{Descrizione} \\
		\DocDesc
	\end{center}
\end{titlepage}
\clearpage

% #### INDICE (tableofcontents )####
%\pagenumbering{Roman} %Pagine con i  numeri romani + reset a I
%\renewcommand{\footrulewidth}{1pt} %Di default footrulewidth==0 e quindi è invisibile
%\tableofcontents %Provoca la stampa dell'indice
%\label{LastPageRoman} %Indica l'ultima pagina dell'indice
%\rfoot{\thepage\ di \pageref{LastPageRoman}} %Pagina n di m, con numeri Romani
%\clearpage

% #### PARTE PRINCIPALE (mainmatter) ####

%\pagenumbering{arabic} %Pagine con i numeri arabi + reset a 1
%\rfoot{\thepage\ di \pageref{LastPage}} %Pagina n di m, con numeri Arabi; usa il pacchetto "lastpage", in caso non sia possibile usare tale pacchetto mettere al fondo dell'ultima pagina "\label{LastPage}"
%\clearpage %inclusione file principale con la pagina del titolo
%\rhead{\DocTitle}
\pagebreak
\textbf{\Large{Diario delle modifiche}} \\
\rowcolors{2}{gray!25}{gray!15}
\begin{longtable} {
		>{\centering}p{17mm} 
		>{\centering}p{19.5mm}
		>{\centering}p{24mm} 
		>{\centering}p{24mm} 
		>{}p{32mm}}
	\rowcolor{gray!50}
	\textbf{Versione} & \textbf{Data} & \textbf{Nominativo} & \textbf{Ruolo} & \textbf{Descrizione} \TBstrut \\
	1.0.0 & 2019-02-12 & Pinco Pallino & Analista e Verificatore & modificato \$3 \TBstrut \\ [2mm]
	1.0.0 & 2019-02-12 & Pinco Pallino & Analista e Verificatore & modificato \$3 \TBstrut \\ [2mm]
	1.0.0 & 2019-02-12 & Pinco Pallino & Analista e Verificatore & modificato \$3 \TBstrut \\ [2mm]
	1.0.0 & 2019-02-12 & Pinco Pallino & Analista e Verificatore & modificato \$3 \TBstrut \\ [2mm]
	1.0.0 & 2019-02-12 & Pinco Pallino & Analista e Verificatore & modificato \$3 \TBstrut \\ [2mm]
	1.0.0 & 2019-02-12 & Pinco Pallino & Analista e Verificatore & modificato \$3 \TBstrut \\ [2mm]
	1.0.0 & 2019-02-12 & Pinco Pallino & Analista e Verificatore & modificato \$3 \TBstrut \\ [2mm]
	1.0.0 & 2019-02-12 & Pinco Pallino & Analista e Verificatore & modificato \$3 \TBstrut \\ [2mm]
	1.0.0 & 2019-02-12 & Pinco Pallino & Analista e Verificatore & modificato \$3 \TBstrut \\ [2mm]
	1.0.0 & 2019-02-12 & Pinco Pallino & Analista e Verificatore & modificato \$3 \TBstrut \\ [2mm]
	1.0.0 & 2019-02-12 & Pinco Pallino & Analista e Verificatore & modificato \$3 \TBstrut \\ [2mm]
	1.0.0 & 2019-02-12 & Pinco Pallino & Analista e Verificatore & modificato \$3 \TBstrut \\ [2mm]
	1.0.0 & 2019-02-12 & Pinco Pallino & Analista e Verificatore & modificato \$3 \TBstrut \\ [2mm]
	1.0.0 & 2019-02-12 & Pinco Pallino & Analista e Verificatore & modificato \$3 \TBstrut \\ [2mm]
	1.0.0 & 2019-02-12 & Pinco Pallino & Analista e Verificatore & modificato \$3 \TBstrut \\ [2mm]
	1.0.0 & 2019-02-12 & Pinco Pallino & Analista e Verificatore & modificato \$3 \TBstrut \\ [2mm]
	1.0.0 & 2019-02-12 & Pinco Pallino & Analista e Verificatore & modificato \$3 \TBstrut \\ [2mm]
	1.0.0 & 2019-02-12 & Pinco Pallino & Analista e Verificatore & modificato \$3 \TBstrut \\ [2mm]
	1.0.0 & 2019-02-12 & Pinco Pallino & Analista e Verificatore & modificato \$3 \TBstrut \\ [2mm]
	1.0.0 & 2019-02-12 & Pinco Pallino & Analista e Verificatore & modificato \$3 \TBstrut \\ [2mm]
	1.0.0 & 2019-02-12 & Pinco Pallino & Analista e Verificatore & modificato \$3 \TBstrut \\ [2mm]
	1.0.0 & 2019-02-12 & Pinco Pallino & Analista e Verificatore & modificato \$3 \TBstrut \\ [2mm]
	1.0.0 & 2019-02-12 & Pinco Pallino & Analista e Verificatore & modificato \$3 \TBstrut \\ [2mm]
	1.0.0 & 2019-02-12 & Pinco Pallino & Analista e Verificatore & modificato \$3 \TBstrut \\ [2mm]
	1.0.0 & 2019-02-12 & Pinco Pallino & Analista e Verificatore & modificato \$3 \TBstrut \\ [2mm]
	1.0.0 & 2019-02-12 & Pinco Pallino & Analista e Verificatore & modificato \$3 \TBstrut \\ [2mm]
	1.0.0 & 2019-02-12 & Pinco Pallino & Analista e Verificatore & modificato \$3 \TBstrut \\ [2mm]
	1.0.0 & 2019-02-12 & Pinco Pallino & Analista e Verificatore & modificato \$3 \TBstrut \\ [2mm]
	1.0.0 & 2019-02-12 & Pinco Pallino & Analista e Verificatore & modificato \$3 \TBstrut \\ [2mm]
	1.0.0 & 2019-02-12 & Pinco Pallino & Analista e Verificatore & modificato \$3 \TBstrut \\ [2mm]
	1.0.0 & 2019-02-12 & Pinco Pallino & Analista e Verificatore & modificato \$3 \TBstrut \\ [2mm]
	1.0.0 & 2019-02-12 & Pinco Pallino & Analista e Verificatore & modificato \$3 \TBstrut \\ [2mm]
	1.0.0 & 2019-02-12 & Pinco Pallino & Analista e Verificatore & modificato \$3 \TBstrut \\ [2mm]
	1.0.0 & 2019-02-12 & Pinco Pallino & Analista e Verificatore & modificato \$3 \TBstrut \\ [2mm]
	1.0.0 & 2019-02-12 & Pinco Pallino & Analista e Verificatore & modificato \$3 \TBstrut \\ [2mm]
\end{longtable} %inclusione file con tabella registro delle modifiche
\pagebreak

% #### INDICE (tableofcontents )####
\pagenumbering{Roman} %Pagine con i  numeri romani + reset a I
\renewcommand{\footrulewidth}{1pt} %Di default footrulewidth==0 e quindi è invisibile
\tableofcontents %Provoca la stampa dell'indice
\label{LastPageRoman} %Indica l'ultima pagina dell'indice
\rfoot{\thepage\ di \pageref{LastPageRoman}} %Pagina n di m, con numeri Romani
\clearpage

% #### PARTE PRINCIPALE (mainmatter) ####
\pagenumbering{arabic} %Pagine con i numeri arabi + reset a 1
\rfoot{\thepage\ di \pageref{LastPage}} %Pagina n di m, con numeri Arabi; usa il pacchetto "lastpage", in caso non sia possibile usare tale pacchetto mettere al fondo dell'ultima pagina "\label{LastPage}"
\clearpage

\setcounter{secnumdepth}{4} %Permette di andare fino alla profondità del paragraph con la numerazione delle sezioni

\pagebreak
\section{Introduzione}
    \subsection{Scopo del documento}
        Nel seguente documento verranno analizzati i capitolati\glo presentati nell'ambito del progetto di Ingegneria del Software. Saranno
        esposte le ragioni che hanno portato il gruppo alla scelta di C6 (\textit{ThiReMa - Things Relationship Management}) e
        alla conseguente esclusione delle altre proposte.
    \subsection{Glossario}
        Per evitare eventuali ambiguità relative a termini tecnici o altamente specifici questo documento ed i successivi verranno
        corredati da un \textit{Glossario} dove verranno illustrate tali parole. Le voci interessate saranno indicate da una 'G' a
        pedice.
    \subsection{Riferimenti}
        \subsection{Normativi}
            \begin{itemize}
                \item \textbf{Norme di Progetto}: \textit{Norme di Progetto}.
            \end{itemize}
        \subsection{Informativi}
            \begin{itemize}
                \item \textbf{Capitolato\glo d'appalto C1 - Autonomous Highlights Platform}: \url{https://www.math.unipd.it/~tullio/IS-1/2019/Progetto/C1.pdf}
                \item \textbf{Capitolato\glo d'appalto C2 - Etherless}: \url{https://www.math.unipd.it/~tullio/IS-1/2019/Progetto/C2.pdf}
                \item \textbf{Capitolato\glo d'appalto C3 - NaturalAPI}: \url{https://www.math.unipd.it/~tullio/IS-1/2019/Progetto/C3.pdf}
                \item \textbf{Capitolato\glo d'appalto C4 - Predire in Grafana}: \url{https://www.math.unipd.it/~tullio/IS-1/2019/Progetto/C4.pdf}
                \item \textbf{Capitolato\glo d'appalto C5 - Stalker}: \url{https://www.math.unipd.it/~tullio/IS-1/2019/Progetto/C5.pdf}
                \item \textbf{Capitolato\glo d'appalto C6 - ThiReMa - Things Relationship Management}: \url{https://www.math.unipd.it/~tullio/IS-1/2019/Progetto/C6.pdf}
            \end{itemize}

\pagebreak
\section{Capitolato 1 - Autonomous Highlights Platform}
\subsection{Informazioni generali}
\begin{itemize}
	\item \textbf{Nome}: Autonomous Highlights Platform;
	\item \textbf{Proponente}: Zero12;
	\item \textbf{Committente}: Prof. Tullio Vardanega e Prof. Riccardo Cardin;
\end{itemize}
\subsection{Descrizione}
Autonomous Highligths Platform prevede lo sviluppo di una piattaforma web capace di ricevere in input dei video di eventi sportivi e fornire in output un video contenente i momenti salienti.
\subsection{Obiettivi del progetto}
Il progetto prevede la creazione di un software che riceve in input dei video di eventi sportivi come una partita di calcio, Formula1, Motogp o altri sport e riesca a creare in autonomia un video di massimo 5 minuti con i soli suoi momenti salienti (highlights) da fornire in output. A questo fine la piattaforma dovrà essere dotata di un modello di machine learning\glosp in grado di identificare ogni momento importante dell’evento.
In particolare il flusso di generazione del video sarà così organizzato:
\begin{itemize}
	\item Caricamento del video dell'evento desiderato;
	\item Individuazione degli highlights;
	\item Estrazione delle parti individuate dal video originario;
	\item Generazione del video dei momenti salienti.
\end{itemize}
\subsection{Requisiti di progetto}
Al fine del corretto svolgimento del progetto sarà necessario rispettare i seguenti vincoli forniti dall'azienda proponente:
\begin{itemize}
	\item Utilizzo di Sage Maker\glosp per la costruzione del modello di intelligenza artificiale\glo;
	\item Strutturazione di un'architettura a multiservizi\glo;
	\item Permettere il caricamento dei video tramite riga di comando;
	\item Creazione di un'interfaccia web per l'analisi e il controllo dello stato di elaborazione del video.
\end{itemize}
\subsection{Tecnologie interessate}
Le tecnologie previste per la realizzazione del capitolato sono:
\begin{itemize}
	\item \textbf{Amazon web services\glo}, in particolare:
	\begin{itemize}
		\item \textbf{AWS Elastic Container Service\glo} come servizio per la gestione dei contenitori ad alte prestazioni;
		\item \textbf{AWS Dynamo DB\glo}, database non relazionale in NoSQL\glosp per il supporto e la gestione dei tag;
		\item \textbf{AWS Elastic Transcoder\glo} per la conversione e la rielaborazione dei video;
		\item \textbf{AWS Sage Maker\glo} per la creazione di un modello per il riconoscimento degli eventi salienti se si sceglie di implementare un addestramento supervisionato dell'intelligenza artificiale;
		\item \textbf{AWS Rekognition\glo} da usare nel caso si scelga, invece, l'apprendimento non supervisionato e quindi l'utilizzo di modelli già esistenti.
	\end{itemize}
	\item \textbf{NodeJS\glo} per lo sviluppo di API Restful JSON\glosp al fine di garantire una scalabilità ottimale;
	\item \textbf{Python\glo} per lo sviluppo delle componenti necessarie di machine learning;
	\item \textbf{HTML5\glo}, \textbf{CSS3\glo} e \textbf{Javascript\glo} per lo sviluppo dell'interfaccia web. 
\end{itemize}
\subsection{Aspetti positivi}
\begin{itemize}
	\item Le tecnologie proposte risultano innovative e utili per la loro larga diffusione nel mondo del lavoro;
	\item La documentazione delle tecnologie reperibile è ben approfondita;
	\item Le specifiche del capitolato sono state fornite in modo chiaro e preciso. 
\end{itemize}
\subsection{Criticità e fattori di rischio}
\begin{itemize}
	\item La maggior parte delle tecnologie proposte, seppure molto interessanti, non è prevista dal nostro corso di laurea. Dunque lo svolgimento di questo capitolato richiederebbe quindi un numero di ore di apprendimento difficilmente quantificabile.
	\item Si è ritenuto non semplice e dispendioso in termini di tempo il reperimento e l'analisi dei video per l'apprendimento della macchina
\end{itemize}
\subsection{Conclusioni}
Lo scopo del capitolato è risultato accattivante, tuttavia la quantità di nuove tecnologie da apprendere ha demotivato il gruppo a scegliere questo progetto.
\pagebreak
\section{Capitolato 2 - \textit{Etherless}}
\subsection{Informazioni generali}
    \begin{itemize}
        \item \textbf{Nome}: Etherless;
        \item \textbf{Proponente}: Red Babel;
        \item \textbf{Committente}: Prof. Tullio Vardanega e Prof. Riccardo Cardin.
    \end{itemize}
\subsection{Descrizione}
    Etherless è una \textit{Cloud Application Platform}\glosp che permette agli sviluppatori che la utilizzano di caricare nel cloud\glosp delle funzioni
    \textit{JavaScript}\glo. Tali procedure possono poi essere acquistate da terzi tramite l'impiego della criptovaluta\glosp \textit{Ethereum}\glosp implementata
    grazie alla tecnologia della \textit{blockchain}\glo.
\subsection{Obiettivi di progetto}
    Il progetto si propone di aiutare gli sviluppatori fornendo:
    \begin{itemize}
        \item un framework\glosp \textit{Serverless}\glosp che gestisca il costo computazionale della funzione;
        \item un servizio di \textit{Smart Contracts}\glosp che sovrintenda il processo di pagamento.
    \end{itemize}
    Quest'ultima tecnologia sarà di supporto anche per l'acquirente in quanto assicurerà il completamento della transazione solo a lavoro verificato e completato.
\subsection{Requisiti di progetto}
È richiesto di dividere lo sviluppo in tre fasi:
\begin{itemize}
   	\item \textbf{Local}: utilizzo dell'applicativo in ambiente locale, in cui può essere utilizzato \textit{Ethereum testrpc}\glosp di \textit{Truffle} per
   	l'emulazione\glosp della blockchain\glo;
   	\item \textbf{Test}: utilizzo dell'applicativo in ambiente di testing, in cui può essere usata la soluzione proposta al punto precedente;
   	\item \textbf{Staging}: utilizzo dell'applicativo in un ambiente pubblico, in tal caso si potrà usufruire di \textit{Ropsten Ethereum}\glosp come rete di testing.
\end{itemize}
Inoltre è obbligatorio separare il software in tre parti:
\begin{itemize}
   	\item \textbf{etherless-cli}: modulo attraverso il quale lo sviluppatore interagisce con Etherless;
   	\item \textbf{etherless-smart}: modulo per l'interazione tra \textit{etherless-cli} e la parte server;
   	\item \textbf{etherless-server}: modulo che ascolta gli eventi emessi da \textit{etherless-smart} per attivare le rispettive funzioni lambda\glo. 
\end{itemize}
Ognuna di queste operazioni dovrà poi essere caricata e versionata tramite \textit{GitHub}\glosp o \textit{GitLab}\glo.
\subsection{Tecnologie interessate}
    Per l'implementazione delle varie funzionalità vengono date dall'azienda delle linee guida sulle tecnologie da utilizzare:
    \begin{itemize}
        \item \textbf{Typescript 3.6}\glo: linguaggio di programmazione da impiegare, tramite l'approccio \textit{Promise}\glosp o \textit{async-await}\glosp, nello sviluppo della piattaforma \textit{Etherless};
        \item \textbf{Solidity}\glo: linguaggio per la creazione e la gestione degli \textit{Smart Contracts}\glo;
        \item \textbf{AWS Lambda}\glo: piattaforma computazione serverless fornita da \textit{Amazon} per la coordinazione degli eventi;
        \item \textbf{Serverless Framework}\glo: framework\glosp Web per la creazione di applicazioni su \textit{AWS Lambda}\glo;
        \item \textbf{typescript-eslint}\glo: strumento di analisi statica\glosp del codice \textit{Typescript}\glo.
    \end{itemize}
    \textit{AWS Lambda}\glosp può essere corredata con altri componenti quali: \textit{AWS API Gateway}\glosp (per eventi HTTP\glo), \textit{AWS DynamoDB}\glosp (database non relazionale\glo)
    o \textit{AWS S3}\glosp (servizio di memorizzazione). Tutto questo poi può essere gestito tramite \textit{AWS CloudFormation}\glosp (piattaforma per l'organizzazione delle
    risorse AWS).
\subsection{Aspetti positivi}
\begin{itemize}
    \item Ethereum\glosp e la tecnologia della blockchain\glosp più in generale sono tematiche molto attuali e innovative che suscitano interesse tra i componenti del gruppo
    anche per le possibili applicazioni future;
    \item L'azienda ha avvallato delle richieste chiare e precise, aspetto valutato positivamente dal gruppo.
\end{itemize}
\subsection{Criticità e fattori di rischio}
\begin{itemize}
    \item L'azienda ha sede in Olanda, il rapporto con i proponenti potrebbe essere quindi meno efficace;
    \item Seppur siano proposte tematiche interessanti il gruppo ha presentato dei dubbi riguardanti la difficoltà e le tempistiche per l'approfondimento
    delle tecnologie presentate.
\end{itemize}
\subsection{Conclusioni}
Il capitolato è apparso stimolante per quanto riguarda le tecnologie utilizzate e convincente nella sua esposizione. Tuttavia la distanza geografica e la
mole di lavoro prevista sono risultate un ostacolo nell'effettiva realizzazione del software. Per questo motivi il gruppo ha deciso di vertere la sua scelta
su un altro progetto.

\pagebreak
\section{Capitolato 3 - NaturalAPI}
\subsection{Informazioni generali}
\begin{itemize}
	\item \textbf{Nome}: NaturalAPI;
	\item \textbf{Proponente}: Teal Blue;
	\item \textbf{Committente}: Prof. Tullio Vardanega e Prof. Riccardo Cardin.
\end{itemize}
\subsection{Descrizione}
L'obiettivo di questo capitolato\glosp è quello di far parlare a tutti gli stakeholder\glosp un linguaggio comune, in modo da velocizzare ed evitare confusione durante la progettazione\glo. Teal Blue vuole utilizzare le specifiche e i requisiti di un software, scritti in linguaggio naturale (inglese, italiano, ecc.), per generare API\glo.

\subsection{Obiettivi di progetto}
Il prodotto\glosp da sviluppare è NaturalAPI un toolkit\glosp che dovrà generare API\glosp complete e test automatici a partire da un linguaggio naturale. Il workflow\glosp di questo prodotto\glosp consiste nel ricevere in input del testo in linguaggio naturale. Da questo testo verrà generato un BDL (Business Domain Language). Dal BDL dovrà essere sviluppato un BAL (Business Application Language) che, in seguito, verrà convertito in una API\glo.

\subsection{Requisiti di progetto}
Per la realizzazione di questo prodotto\glosp è richiesto lo sviluppo di tre PoC\glosp o feature\glo.
\begin{itemize}
	\item \textbf{NaturalAPI Discover}: estrattore di BDL. \\*
	L'estrattore Discover ha lo scopo di estrarre entità, azioni e combinazioni tra esse da un documento testuale contenente termini specifici di un dato business;
	\item \textbf{NaturalAPI Design}: parser di scenari e caratteristiche. \\*
	Questa PoC\glosp dovrà creare una BAL API\glosp in tempo reale a partire dai documenti di Gherkin e da un BDL;
	\item \textbf{NaturalAPI Develop}: esportatore di linguaggio. \\*
	Questa feature\glosp dovrà convertire un BAL in casi di test e API\glosp nel linguaggio di programmazione scelto, supportando la creazione e l'aggiornamento di nuove repository\glo.
\end{itemize}
Ogni feature\glosp sopraindicata dovrà essere accessibile attraverso almeno due dei seguenti modi: interfaccia da linea di comando, GUI minimale o un'interfaccia web REST\glo. La parte logica del prodotto\glosp finale dovrà invece essere esportata in una delle seguenti modalità: come una libreria, come parte di un eseguibile o come un'applicazione indipendente locale o remota. 
Per quanto riguarda l'input e l'output bisognerà sottostare allo standard UTF-8 e all'indentatura UNIX. 

\subsection{Tecnologie interessate}
\begin{itemize}
	\item \textbf{NLP}: Natural Language Processing cioè un trattamento informatico del linguaggio naturale, che si occupa della realizzazione di sistemi in grado di comprendere il linguaggio naturale;	
	\item \textbf{Dependency Parser}: cioè un parser che si occupa di analizzare la struttura grammaticale di una frase in linguaggio naturale per identificare le relazioni tra parole chiave e parole che le modificano;
	\item \textbf{BDD}: Behaviour Driven Development cioè un processo\glosp di sviluppo software agile che ha, alla sua base, una continua comunicazione tra tutti gli stakeholder\glosp di un progetto\glosp informatico;
	\item \textbf{Hiptest} e \textbf{Cucumber}: sono degli strumenti software che supportano il BDD; il primo serve per eseguire test automatici, mentre il secondo legge le specifiche software in linguaggio naturale, scritte con alcune regole di sintassi (Gherkin), e controlla che il software rispetti i requisiti;
	\item Generazione API\glosp e DLS utilizzando:
	\begin{itemize}
		\item \textbf{OpenAPI}: uno standard per descrivere API\glo;
		\item \textbf{Swagger}: un framework che aiuta a sviluppare servizi web REST\glo;
		\item \textbf{OWL}: un linguaggio web semantico.
	\end{itemize}
	\item Un qualsiasi framework a scelta come Qt, React, ecc.
\end{itemize} 
\subsection{Aspetti positivi}
\begin{itemize} 
	\item Il proponente non impone vincoli sui linguaggi di programmazione da utilizzare, questo permette al gruppo di sviluppare un prodotto\glosp con un linguaggio o un framework considerato più interessante;
	\item i requisiti, le tecnologie e il modo di esportare il prodotto\glosp finale sono spiegati in modo preciso ed esauriente;
	\item l'azienda Teal Blue si dimostra disponibile a incontri e a una comunicazione aperta con il fornitore.
\end{itemize}
\subsection{Criticità e fattori di rischio}
\begin{itemize}
	\item Questo capitolato\glosp non ha suscitato molto interesse a causa dell'eccessiva astrattezza e per il fatto che i concetti su cui prepararsi non erano, per il gruppo \textit{VRAM Software}, stimolanti;
	\item le tecnologie da imparare sono molte ed è arduo quantificare il tempo necessario per raggiungere una preparazione sufficiente per gestire in modo produttivo le suddette.
\end{itemize}
\subsection{Conclusione}
Il gruppo ha trovato interessante l'idea di Teal Blue di rendere la comunicazione tra tutti stakeholder\glosp chiara e veloce, tuttavia abbiamo deciso di orientarci verso un capitolato\glosp meno astratto.

\pagebreak
\section{Capitolato 4 - Predire in Grafana}

\subsection{Informazioni generali}
\begin{itemize}
	\item \textbf{Nome}: Predire in Grafana\glo: monitoraggio predittivo per DevOps\glo;
	\item \textbf{Proponente}: \textit{Zucchetti};
	\item \textbf{Committente}: Prof. Tullio Vardanega e Prof. Riccardo Cardin;
\end{itemize}

\subsection{Descrizione}
Con questo capitolato\glo, \textit{Zucchetti} mira a realizzare due plug-in per lo strumento di monitoraggio Grafana\glo. Essi devono applicare gli algoritmi di Support Vector Machine\glosp (SVM) e Regressione Lineare\glosp (RL) al flusso di dati al fine di prevedere livelli al di sopra di una certa soglia e quindi generare un allarme, oppure permettere agli operatori del servizio cloud\glosp di generare segnalazioni dei punti critici che l’erogazione mette in evidenza. In tal modo la linea di produzione del software può intervenire con cognizione di causa sul sistema.

\subsection{Obiettivi di progetto}
Il sistema richiede di implementare due plug-in di Grafana\glosp scritti in linguaggio JavaScript che eseguiranno i calcoli (Support Vector Machine\glosp o Regressione Lineare\glo) letti da un file JSON e produrranno valori tali da essere aggiunti al flusso di monitoraggio.

\subsection{Requisiti di progetto}
\begin{itemize}
	\item Produrre un file JSON dai dati di addestramento con i parametri per le previsioni attraverso SVM\glosp per le classificazioni o RL\glo;
	\item leggere la definizione del predittore\glosp dal file in formato JSON;
	\item associare i predittori\glosp letti dal file JSON al flusso di dati presente in Grafana\glo;
	\item applicare la previsione e fornire i nuovi dati ottenuti al sistema di Grafana\glo;
	\item rendere disponibili i dati al sistema di creazione di grafici e dashboard\glosp per la loro visualizzazione.
\end{itemize}
Opzionalmente è richiesto d'implementare le seguenti funzionalità:
\begin{itemize}
	\item offrire la possibilità di definire degli "alert"\glosp in base ai livelli di soglia raggiunti dai nodi collegati alle previsioni;
	\item fornire i dati che definiscono bontà dei modelli di previsione utilizzati;
	\item offrire la possibilità di applicare delle trasformazioni alle misure lette dal campo per ottenere delle regressioni\glosp esponenziali o logaritmiche;
	\item offrire la possibilità di addestrare SVM\glosp o RL\glosp direttamente in Grafana\glo;
	\item implementare dei meccanismi di apprendimento di flusso, per poter disporre di sistemi di previsione che si adattano costantemente ai dati rilevati sul campo;
	\item utilizzare altri metodi di previsione che possano dare benefici al calcolo dei dati, tra cui la versione delle SVM\glosp adattate alla regressione\glosp o piccole reti neurali\glosp per la classificazione.
\end{itemize}

\subsection{Tecnologie interessate}
\begin{itemize}
	\item \textbf{JavaScript}: linguaggio di programmazione richiesto per costruire i plug-in di Grafana\glosp ed eseguire i calcoli tramite SVM\glosp e RL\glosp per fare le previsioni;
	\item \textbf{Grafana}: sistema di monitoraggio open source estendibile con plug-in JavaScript;
	\item \textbf{Librerie JavaScript per reti neurali}\glo: librerie che permettono di implementare reti neurali\glosp per migliorare le previsioni, utili a soddisfare uno dei requisiti opzionali.
\end{itemize}

\subsection{Aspetti positivi}
\begin{itemize} 
	\item Svolgere questo capitolato\glosp permette di acquisire competenze in ambito di analisi e previsione dei dati, molto richieste in ambito lavorativo;
	\item la presentazione del problema è stata molto chiara e l'azienda è apparsa molto disponibile;
	\item gli algoritmi di SVM\glosp e RL\glosp verranno forniti dall'azienda;
    \item l'azienda si rende disponibile a impartire delle basi di machine learning\glosp che non sono fornite nel nostro corso di laurea.
\end{itemize}
\subsection{Criticità e fattori di rischio}
\begin{itemize}
	\item Gli algoritmi da utilizzare sono basati su concetti matematici valutati di non facile apprendimento e che quindi potrebbero porterare ad una fase di studio non indifferente. 
\end{itemize}
\subsection{Conclusione}
Il gruppo ha trovato molto chiaro e definito il problema, inoltre l'azienda, forse anche perché di grandi dimensioni, si è mostrata molto disponibile a dedicare risorse per i gruppi che sceglieranno il suo capitolato\glo.

\pagebreak
\section{Capitolato 5 - Stalker}

\subsection{Informazioni}
\begin{itemize}
	\item \textbf{Nome}: Stalker
	\item \textbf{Proponente}: Imola Informatica
	\item \textbf{Committente}: Prof. Tullio Vardanega
\end{itemize}

\subsection{Descrizione}
Il progetto Stalker nasce dalla necessità di tracciare il numero di persone presenti all'interno dei locali, richiesto dalle normative, utile in vari ambiti: dal controllare l'affluenza in luoghi di interesse o ad eventi, al monitorare le presenze del personale nel luogo di lavoro. La soluzione da loro proposta è un un'applicazione mobile che invii la posizione dei dispositivi degli utenti ad un server, il quale si occuperà dell'estrazione e memorizzazione delle informazioni utili.

\subsection{Obiettivi}
La soluzione software proposta da Imola Informatica prevede lo sviluppo di un'applicazione mobile e un'infrastruttura server a supporto, controllabile da un'interfaccia web di amministrazione. Lo scenario di utilizzo prevede che tutti gli utenti siano muniti di uno smartphone e abbiano la possibilità di installare l'apposita applicazione; i luoghi da monitorare possono essere corredati con access point\glosp o beacon Bluetooth\glosp per facilitare il tracciamento, in particolare nei luoghi chiusi. Lo scopo del progetto è quello di monitorare il numero di presenze, in forma anonima o autenticata, in uno o più luoghi circoscritti e fornire statistiche a riguardo.

\subsection{Requisiti}
Un'applicazione mobile con i seguenti requisiti:
\begin{itemize}
	\item Recupero lista organizzazioni;
	\item Possibilità di effettuare login tramite LDAP\glo per organizzazioni che lo richiedono;
	\item Storico accessi;
	\item Visualizzazione in tempo reale della propria presenza all'interno di un luogo monitorato e cronometro del tempo trascorso al suo interno.
\end{itemize}
Un'interfaccia web per l'amministrazione con le seguenti funzionalità:
\begin{itemize}
	\item Funzionalità di login;
	\item Creazione/modifica/eliminazione di organizzazioni;
	\item Aggiunta/modifica/rimozione di luoghi, definiti da coordinate geografiche;
	\item Configurazione collegamento server LDAP\glo;
	\item Invio di notifiche push\glosp alle applicazioni per segnalare l'aggiornamento delle liste di organizzazioni e luoghi;
	\item Monitoraggio del numero di utenti presenti nei luoghi dell'organizzazione in tempo reale;
	\item Ricerca sugli accessi dei dipendenti e creazione di report sulla frequentazione dei luoghi;
	\item Gestione delle autorizzazioni per gli utenti dell'interfaccia web.
\end{itemize}
L'infrastruttura su cui si baserà l'applicazione dovrà sottostare ai seguenti vincoli:
\begin{itemize}
	\item Comunicazioni solo all'entrata od uscita dall'area interessata;
	\item Cifratura di tutte le comunicazioni tra app e server;
	\item Architettura scalabile\glosp (verticalmente e orizzontalmente) e tollerante a picchi di traffico per il lato server;
	\item Test di carico che dimostrino il funzionamento in varie situazioni.
\end{itemize}

\subsection{Tecnologie interessate}
\begin{itemize}
	\item Java\glo o Swift\glo (applicazione mobile);
	\item NodeJS\glo o Python\glo (back-end\glo);
	\item Utilizzo di protocolli asincroni\glosp per le comunicazioni tra app e server;
	\item HTML5\glo, CSS3\glo e Javascript\glo (interfaccia web lato server);
	\item Utilizzo del pattern di Publisher/Subscriber\glo, ovvero mittenti e destinatari dimessaggi dialogano attraverso un tramite(dispatcher);
	\item Utilizzo dell’IAAS Kubernetes o di un PAAS, Openshift o Rancher, per il rilascio delle componenti server;
	\item API REST\glosp esposte dal server, o gRPC\glo in alternativa;
	\item GPS\glo/sistemi ibridi di geolocalizzazione;
	\item LDAP\glo (Lightweighgt Directory Access Protocol);
	\item Test unitari\glosp e d'integrazione per tutte le componenti applicative.
\end{itemize}

\subsection{Aspetti positivi}
\begin{itemize}
	\item Lo sviluppo di applicazioni mobili è una conoscenza richiesta;
	\item Argomento interessante, anche grazie alla potenziale utilità nell'ambito della sicurezza;
\end{itemize}

\subsection{Criticità e fattori di rischio}
\begin{itemize}
	\item I requisiti sono numerosi e richiedono conoscenze non banali;
	\item Lo sviluppo lato server potrebbe portare complicazioni, viste le numerose tecnologie coinvolte e i requisiti in scalabilità;
	\item La precisione ottenibile con le tecnologie attuali non è sufficiente per rendere l’applicazione usabile nella realtà;
	\item Nella presentazione è stato discusso il funzionamento del sistema GPS ma completamente tralasciata la funzionalità dei backend di geolocalizzazione presenti nei moderni sistemi mobile (ad esempio i Google Mobile Services per Android).
\end{itemize}

\subsection{Conclusioni}
Nonostante il nobile obiettivo, la difficile fattibilità di un prodotto concreto e la possibilità di complicazioni scoraggia il gruppo nella decisione d'intraprendere questo capitolato.

\pagebreak
\section{Capitolato 6 - ThiReMa}
\subsection{Informazioni generali}
\begin{itemize}
	\item \textbf{Nome}: ThiReMa;
	\item \textbf{Proponente}: Sanmarco Informatica;
	\item \textbf{Committente}: Prof. Tullio Vardanega e Prof. Riccardo Cardin;
\end{itemize}
\subsection{Descrizione}
Lo scopo di questo progetto\glosp è creare un'applicazione web che permetta di valutare la correlazione fra dati operativi\glosp e fattori influenzanti\glosp in ambito IoT. I dati saranno ricavati da dispositivi IoT eterogenei tramite piattaforma Apache Kafka, quindi il software si occuperà di raccolta, storicizzazione, monitoraggio, analisi e presentazione dei dati.
L'obiettivo finale del progetto\glosp è rendere i processi\glosp aziendali intelligenti trasformando i dati in informazioni, così da permettere ad esempio la pianificazione di manutenzioni predittive calcolate su informazioni reali, in automatico.

\subsection{Obiettivi del progetto}
Realizzare il prodotto ThiReMa, una web application che raccoglierà dati da dispositivi IoT eterogenei tramite piattaforma Apache Kafka, li monitorerà e storicizzerà in un time series database e li presenterà elaborati all'utente tramite interfaccia web. Inoltre dovrà inviare notifiche tempestive agli operatori tramite un servizio basato su Telegram e permettere, sempre tramite tale servizio, di comandare i dispositivi IoT stessi.

\subsection{Requisiti di progetto}
\begin{itemize}
	\item Lettura dati da gateway e sensori IoT tramite Apache Kafka;
	\item salvataggio dei dati su time series database come TimescaleDB, ClickHouse tramite Apache Kafka;
	\item creazione di una web application per presentare i dati raccolti;
	\item gestione di utenti, ruoli e gruppi tramite web application;
	\item possibilità di configurare da quali dispositivi i dati devono essere raccolti;
	\item possibilità di mettere in correlazione i dati tra loro, scegliendo quali visualizzare;
	\item possibilità di inviare comandi di configurazione o azione ai gateway tramite web application o Telegram;
	\item deve prevedere dei meccanismi di avviso e notifica per ente.
\end{itemize}
Opzionalmente è richiesto d'implementare le seguenti funzionalità:
\begin{itemize}
	\item possibilità di impostare una frequenza di interrogazione dei dispositivi ed invio dati;
	\item possibilità di specificare delle funzioni di accumulo sui dati;
	\item rappresentare i dati raccolti e le correlazioni tramite grafici;
	\item possibilità di scegliere fra più algoritmi di correlazione;
	\item sviluppare ed istanziare le feature\glosp tramite uso della tecnologia di containerizzazione\glosp Docker.
\end{itemize}

\subsection{Tecnologie interessate}
\begin{itemize}
	\item \textbf{Apache Kafka}: piattaforma di stream-processing distribuito, permette di raccogliere e monitorare flussi di record di dati come se fossero una coda di messaggi. Permette anche la storicizzazione e l'elaborazione di questi stream di dati;
	\item \textbf{Time Series Database}: sono database molto efficienti nella storicizzazione di dati IoT, in quanto occupano poca memoria pur mantenendo le informazioni basilari necessarie. A questi database è possibile affiancare altri database per contenere i metadati dei dispositivi IoT ed i dati degli utenti. Alcuni possibili database sono PostgreSQL, TimescaleDB, ClickHouse;
	\item \textbf{Java}: popolare linguaggio di programmazione interpretato ed orientato agli oggetti. È consigliato il suo uso per realizzare la business logic\glosp del programma tramite piattaforma Kafka;
	\item \textbf{Bootstrap}: popolare framework CSS utilizzato per realizzare l'interfaccia grafica di siti ed applicazioni web. È consigliato il suo uso per la realizzazione dell'interfaccia web dell'applicazione;
	\item \textbf{Docker}: è una tecnologia di containerizzazione\glosp che permette di eseguire in modo efficiente più applicativi software in ambienti dedicati ed isolati risiedenti in un'unica macchina fisica. Tali ambienti sono detti container\glosp e permettono di eseguire più servizi in un'unica macchina fisica rendendoli indipendenti gli uni dagli altri, è quindi possibile fermare un container\glosp lasciando gli altri regolarmente attivi.
\end{itemize} 
\subsection{Aspetti positivi}
\begin{itemize} 
	\item Sviluppo di competenze nell'ambito IoT, molto richieste dal mondo del lavoro ed interessanti per i membri del gruppo;
	\item sviluppo di conoscenze in ambito Big Data ed analisi dei dati, molto richieste attualmente dal mondo del lavoro e probabilmente anche in futuro;
	\item utilizzo di Java, un linguaggio di programmazione molto popolare e previsto da un insegnamento del nostro corso di studi;
	\item architettura del sistema ben definita e componenti ben chiare;
	\item l'ambito del capitolato\glosp è interessante per tutti i membri del gruppo.
\end{itemize}
\subsection{Criticità e fattori di rischio}
\begin{itemize}
	\item Numero elevato di componenti da realizzare ed integrare che comportano di conseguenza molte tecnologie da apprendere; 
	\item probabile necessità di effettuare test in sede aziendale per avere accesso ai dispositivi fisici.
\end{itemize}
\subsection{Conclusioni}
Il gruppo ha trovato l'ambito di applicazione del problema interessante ed ha apprezzato la chiarezza con cui sono definite le varie parti del sistema. Proprio l'elevato numero di componenti da realizzare ha però fatto calare la volontà dei membri del gruppo a svolgere questo capitolato\glo.

\pagebreak

\end{document}
