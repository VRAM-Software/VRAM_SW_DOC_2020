\section{Informazioni generali}
    \subsection{Informazioni incontro}
        \begin{itemize}
            \item \textbf{Luogo}: Dipartimento di Matematica "Tullio Levi-Civita";
            \item \textbf{Data}: 06-12-2019;
            \item \textbf{Ora d'inizio}: 14.00;
            \item \textbf{Ora di fine}: 16.30;
            \item \textbf{Partecipanti}: \begin{itemize}
                \item Corrizzato Vittorio;
                \item Dalla Libera Marco;
                \item Rampazzo Marco;
                \item Santagiuliana Vittorio;
                \item Schiavon Rebecca;
                \item Spreafico Alessandro;
                \item Toffoletto Massimo.
            \end{itemize}
        \end{itemize}
    \subsection{Argomenti trattati}
        Durante la prima riunione di gruppo, i componenti hanno discusso e deliberato la scelta di un nome e un logo rappresentativo.
        Successivamente ci si è confrontati sulla scelta del capitolato\glo e sugli strumenti di supporto da utilizzare nella realizzazione
        di quest'ultimo.
\section{Verbale}
    Inizialmente il gruppo ha dibattuto per decidere un nome e un logo a esso associato. Ne è seguita una votazione in cui i componenti hanno
    concordato per "\textit{VRAM Software}" e per l'immagine a fronte di questo verbale (che verrà poi utilizzata nei successivi documenti). Dopo di che, si è discusso
    sugli strumenti da utilizzare per una gestione ottimale del materiale di lavoro. La scelta è ricaduta sulle seguenti tecnologie:
    \begin{itemize}
        \item \textbf{Git}\glo: come software di controllo versione;
        \item \textbf{GitHub}\glo: come servizio di hosting;
        \item \textbf{Slack}\glo e \textbf{Telegram}\glo: come canali di comunicazione;
        \item \textbf{Google Calendar}\glo: per organizzare riunioni e appuntare scadenze;
        \item \textbf{Google Drive}\glo: per la condivisione di documenti o applicativi utili alla redazione o allo sviluppo.
    \end{itemize}
    Infine, i partecipanti hanno condiviso le loro opinioni riguardanti i vari capitolati\glo, e dopo aver esposto vantaggi e svantaggi di ognuno, hanno concordato sulla
    scelta di C6 (\textit{ThiReMa - Things Relationship Management}).

        
    


    