\section{Informazioni generali}
    \subsection{Informazioni incontro}
        \begin{itemize}
            \item \textbf{Luogo}: Dipartimento di Matematica "Tullio Levi-Civita";
            \item \textbf{Data}: 2019-12-13;
            \item \textbf{Ora d'inizio}: 14.00;
            \item \textbf{Ora di fine}: 16.30;
            \item \textbf{Partecipanti}: \begin{itemize}
                \item Corrizzato Vittorio;
                \item Dalla Libera Marco;
                \item Rampazzo Marco;
                \item Santagiuliana Vittorio;
                \item Schiavon Rebecca;
                \item Spreafico Alessandro;
                \item Toffoletto Massimo.
            \end{itemize}
        \end{itemize}
    \subsection{Argomenti trattati}
        Nel secondo incontro, i componenti del gruppo hanno discusso delle seguenti tematiche:
        \begin{enumerate}
            \item divisione dei ruoli per l'inizio dei lavori sulla documentazione;
            \item studio preliminare per la stesura delle \textit{Norme di Progetto v. 1.1.1};
            \item scrittura di una serie di domande da porre all'incontro con \textit{Zucchetti}.
        \end{enumerate}
\section{Verbale}
    \subsection{Punto 1}
        Come primo argomento è stata discussa la divisione dei ruoli per cominciare la stesura dei documenti. In questo frangente del progetto\glosp sono attivi tre ruoli: responsabile, analista e verificatore. Suddette cariche sono state suddivise in questo modo:
        \begin{itemize}
            \item \textbf{Responsabile}: Toffoletto Massimo;
            \item \textbf{Analisti}: Corrizzato Vittorio, Santagiuliana Vittorio e Schiavon Rebecca;
            \item \textbf{Verficatori}: Dalla Libera Marco, Rampazzo Marco e Spreafico Alessandro. 
        \end{itemize}
    \subsection{Punto 2}
        Successivamente si è passati a un'analisi preliminare delle \textit{Norme di Progetto v. 1.1.1}. È stato deciso di cominciare da questo documento perché necessario per la stesura dei documenti successivi in quanto fornirà le norme a cui attenersi per la produzione di elaborati corretti e tra loro coerenti. Dopo aver individuato i principali processi\glosp da normare, quindi, si è proceduto alla divisione dei carichi di lavoro, dalla quale sono risultate le seguenti mansioni:
        \begin{itemize}
            \item Schiavon Rebecca: analisi dei processi primari;
            \item Corrizzato Vittorio: analisi dei processi di supporto;
            \item Santagiuliana Vittorio: analisi dei processi organizzativi;
            \item Rampazzo Marco: verifica dei processi primari;
            \item Dalla Libera Marco: verifica dei processi di supporto;
            \item Spreafico Alessandro: verifica dei processi organizzativi;
            \item Toffoletto Massimo: approvazione della versione finale del documento.
        \end{itemize}
        Inoltre, Spreafico Alessandro si è proposto per la creazione di un template \LaTeX per uniformare tra loro i documenti.
    \subsection{Punto 3}
    Infine, in vista di un incontro con l'azienda proponente \textit{Zucchetti} il gruppo ha stilato delle domande per il chiarimento di alcuni dubbi inerenti alla progettazione\glosp del capitolato\glo. Tali temi verranno in seguito trattati su un verbale esterno.