\section{Informazioni generali}
    \subsection{Informazioni incontro}
        \begin{itemize}
            \item \textbf{Luogo}: Videochiamata tramite Skype;
            \item \textbf{Data}: 2020-01-05;
            \item \textbf{Ora d'inizio}: 9.00;
            \item \textbf{Ora di fine}: 12.00;
            \item \textbf{Partecipanti}: \begin{itemize}
                \item Corrizzato Vittorio;
                \item Dalla Libera Marco;
                \item Rampazzo Marco;
                \item Santagiuliana Vittorio;
                \item Schiavon Rebecca;
                \item Spreafico Alessandro;
                \item Toffoletto Massimo.
            \end{itemize}
        \end{itemize}
    \subsection{Argomenti trattati}
        Nel corso della quinta riunione, il gruppo ha discusso riguardo i seguenti argomenti:
        \begin{enumerate}
            \item revisione dell'avanzamento della stesura dell'\textit{Analisi dei Requisiti v. 1.1.1}; 
            \item scrittura della e-mail per chiedere un incontro e delle domande da sottoporre al suddetto incontro al proponente \textit{Zucchetti};
            \item scrittura delle domande da sottoporre al proponente \textit{Prof. Tullio Vardanega} al colloquio precedentemente fissato;
            \item analisi preliminare per la redazione del \textit{Piano di Progetto v. 1.1.1}.
        \end{enumerate}
\section{Verbale}
    \subsection{Punto 1}
        Inizialmente il gruppo ha esaminato lo stato di avanzamento dei lavori dell'\textit{Analisi dei Requisiti v. 1.1.1}. È sorta la necessità di ottenere delucidazioni su specifici casi d'uso da parte di alcuni membri del gruppo ed è quindi nato un dibattito fra i membri del gruppo che ha portato al chiarimento di alcuni dei dubbi sorti e alla stesura di alcune domande per i proponenti per altri;
    \subsection{Punto 2}
        Si è passati quindi alla scrittura della e-mail per richiedere al proponente \textit{Zucchetti} un incontro e alla conseguente stesura delle domande da porre al fine di chiarire i dubbi sorti nella stesura dell'\textit{Analisi dei Requisiti v. 1.1.1}. Verrà redatto un verbale esterno in cui saranno trattati tali temi.
    \subsection{Punto 3}
        Il gruppo ha continuato con la scrittura delle domande da porre all'incontro con il proponente \textit{Prof. Tullio Vardanega}. Tali domande vertono principalmente sulla correttezza di ciò che è stato fatto finora nella stesura dell'\textit{Analisi dei Requisiti v. 1.1.1} e verranno discusse in un successivo verbale.
     \subsection{Punto 4}
     	Il gruppo ha infine svolto un'analisi preliminare necessaria per la redazione del \textit{Piano di Progetto v. 1.1.1}. È stato stilato un indice per la realizzazione di questo documento ed è stato deciso che gli analisti se ne occuperanno dopo aver terminato la scrittura dell'\textit{Analisi dei Requisiti v. 1.1.1}.
     	I ruoli rimangono quindi invariati rispetto a quanto scritto nel precedente verbale.