\section{Informazioni generali}
    \subsection{Informazioni incontro}
        \begin{itemize}
            \item \textbf{Luogo}: Videochiamata tramite Skype;
            \item \textbf{Data}: 2020-01-05;
            \item \textbf{Ora d'inizio}: 9.00;
            \item \textbf{Ora di fine}: 12.00;
            \item \textbf{Partecipanti}: \begin{itemize}
                \item Corrizzato Vittorio;
                \item Dalla Libera Marco;
                \item Rampazzo Marco;
                \item Santagiuliana Vittorio;
                \item Schiavon Rebecca;
                \item Spreafico Alessandro;
                \item Toffoletto Massimo.
            \end{itemize}
        \end{itemize}
    \subsection{Argomenti trattati}
        Nel corso della quinta riunione, il gruppo ha discusso riguardo i seguenti argomenti:
        \begin{enumerate}
            \item revisione dell'avanzamento della stesura del \textit{Piano di Qualifica v. 1.1.1}; 
            \item stesura domande da porre al proponente \textit{Zucchetti} e conseguente scrittura della e-mail per chiedere un incontro;
            \item scrittura delle domande da sottoporre al proponente \textit{Prof. Tullio Vardanega} al colloquio precedentemente fissato;
            \item discussione sulla redazione del \textit{Piano di Progetto v. 1.1.1} e conseguente rotazione dei ruoli;
            \item aggiornamento e revisione delle \textit{Norme di Progetto};
            \item discussione generale dei problemi incontrati finora dal gruppo.
        \end{enumerate}
\section{Verbale}
    \subsection{Punto 1}
        Inizialmente il gruppo ha esaminato lo stato di avanzamento dei lavori del \textit{Piano di Qualifica v. 1.1.1}. È sorta la necessità di ottenere delucidazioni su specifiche metriche\glosp da parte di alcuni membri del gruppo ed è quindi nato un dibattito fra i membri del gruppo che ha portato al chiarimento di alcuni dei dubbi sorti.
    \subsection{Punto 2}
        Si è passati quindi alla scrittura della e-mail per richiedere al proponente \textit{Zucchetti} un incontro e alla conseguente stesura delle domande da porre al fine di chiarire gli ultimi dubbi sorti nella redazione dell'\textit{Analisi dei Requisiti v. 1.1.1}. Verrà redatto un verbale esterno in cui saranno trattati tali temi.
    \subsection{Punto 3}
        Il gruppo ha continuato con la scrittura delle domande da porre all'incontro con il committente \textit{Prof. Tullio Vardanega}. Tali domande vertono principalmente sulla correttezza di ciò che è stato fatto finora nella stesura dell'\textit{Analisi dei Requisiti v. 1.1.1}.
    \subsection{Punto 4}
     	Il gruppo ha svolto un'analisi dei rischi e redatto delle linee guida necessarie per la stesura del \textit{Piano di Progetto v. 1.1.1}. 
        Di conseguenza viene aggiunto il nuovo ruolo di progettista e le mansioni vengono ruotate nel seguente modo:
        \begin{itemize}
            \item \textbf{Responsabile}: Corrizzato Vittorio;
            \item \textbf{Amministratore}: Santagiuliana Vittorio;
            \item \textbf{Progettista}: Spreafico Alessandro;
            \item \textbf{Analisti}: Dalla Libera Marco, Rampazzo Marco e Toffoletto Massimo;
            \item \textbf{Verificatori}: Schiavon Rebecca. 
        \end{itemize}
    \subsection{Punto 5}
        Dato il proseguimento del \textit{Piano di Qualifica} e l'individuazione dei rischi per il \textit{Piano di Progetto} il gruppo ha registrato le modifiche e gli aggiornamenti proposti per continuare la normazione di queste attività nelle \textit{Norme di Progetto}.
    
