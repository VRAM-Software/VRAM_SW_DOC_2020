\section{Informazioni generali}
    \subsection{Informazioni incontro}
        \begin{itemize}
            \item \textbf{Luogo}: Videochiamata tramite Skype;
            \item \textbf{Data}: 2019-12-24;
            \item \textbf{Ora d'inizio}: 14.00;
            \item \textbf{Ora di fine}: 16.30;
            \item \textbf{Partecipanti}: 
            \begin{itemize}
                \item Corrizzato Vittorio;
                \item Dalla Libera Marco;
                \item Rampazzo Marco;
                \item Santagiuliana Vittorio;
                \item Schiavon Rebecca;
                \item Spreafico Alessandro;
                \item Toffoletto Massimo.
            \end{itemize}
        \end{itemize}
    \subsection{Argomenti trattati}
        Nel quarto incontro, i componenti del gruppo hanno discusso delle seguenti tematiche:
        \begin{enumerate}
        	\item divisione dei ruoli 
        	\item studio preliminare per la stesura dell'\textit{Analisi dei Requisiti v. 1.1.1}
            \item elenco dei casi d'uso del prodotto\glosp derivato dallo studio dei requisiti del capitolato\glosp e degli attori coinvolti;
        \end{enumerate}
\section{Verbale}
    \subsection{Punto 1}
        Come primo argomento è stata discussa la divisione dei ruoli per la stesura dell'\textit{Analisi dei Requisiti v. 1.1.1}. Si è deciso di suddividere i ruoli nel seguente modo:
        \begin{itemize}
            \item \textbf{Responsabile}: Corrizzato Vittorio;
            \item \textbf{Analisti}: Dalla Libera Marco, Rampazzo Marco, Schiavon Rebecca, Spreafico Alessandro;
            \item \textbf{Verficatori}: Santagiuliana Vittorio e Toffoletto Massimo. 
        \end{itemize}
    \subsection{Punto 2}
        Si è quindi svolto uno studio preliminare per la stesura dell'\textit{Analisi dei Requisiti v. 1.1.1} da cui è scaturita la necessità di revisionare insieme e più in profondità i requisiti del prodotto esposti nel capitolato. In seguito è stato deciso l'indice per la scrittura del documento e sono state quindi definite le seguenti mansioni:
        \begin{itemize}
        	\item Rampazzo Marco: analisi di "Introduzione" e "Descrizione Generale";
        	\item Dalla Libera Marco, Schiavon Rebecca: analisi di "Casi d'Uso";
        	\item Spreafico Alessandro: analisi di "Requisiti";
        	\item Santagiuliana Vittorio: verifica di "Introduzione", "Descrizione Generale" e "Requisiti";
        	\item Toffoletto Massimo: verifica di "Casi d'Uso";
        	\item Corrizzato Vittorio: approvazione della versione finale del documento.
        \end{itemize}
    \subsection{Punto 3}
    Infine, è stato stilato un elenco sintetico dei casi d'uso e degli attori coinvolti per facilitare il compito degli analisti e permettere una divisione più specifica delle mansioni nella stesura del documento.