\section{Informazioni generali}
    \subsection{Informazioni incontro}
        \begin{itemize}
            \item \textbf{Luogo}: Videochiamata tramite Skype;
            \item \textbf{Data}: 2019-12-29;
            \item \textbf{Ora d'inizio}: 14.00;
            \item \textbf{Ora di fine}: 16.30;
            \item \textbf{Partecipanti}: 
            \begin{itemize}
                \item Corrizzato Vittorio;
                \item Dalla Libera Marco;
                \item Rampazzo Marco;
                \item Santagiuliana Vittorio;
                \item Schiavon Rebecca;
                \item Spreafico Alessandro;
                \item Toffoletto Massimo.
            \end{itemize}
        \end{itemize}
    \subsection{Argomenti trattati}
        Nel quarto incontro, i componenti del gruppo hanno discusso delle seguenti tematiche:
        \begin{enumerate}
        	\item divisione dei ruoli;
        	\item revisione e modifiche per la stesura dell'\textit{Analisi dei Requisiti v. 1.1.1};
            \item discussione preliminare del \textit{Piano di Qualifica};
            \item aggiornamento e revisione delle \textit{Norme di Progetto};
            \item Discussione generale dei problemi incontrati finora dal gruppo.
        \end{enumerate}
\section{Verbale}
    \subsection{Punto 1}
        Come primo argomento è stata discussa la divisione dei ruoli per continuare la stesura dei documenti. Si è deciso di suddividere i ruoli nel seguente modo:
        \begin{itemize}
            \item \textbf{Responsabile}: Toffoletto Massimo;
            \item \textbf{Amministratore}: Rampazzo Marco;
            \item \textbf{Analisti}: Corrizzato Vittorio, Dalla Libera Marco e Schiavon Rebecca;
            \item \textbf{Verficatori}: Spreafico Alessandro e Santagiuliana Vittorio. 
        \end{itemize}
    \subsection{Punto 2}
        Si è quindi svolta una revisione che ha portato a delle modifiche nella stesura dell'\textit{Analisi dei Requisiti v. 1.1.1}. Più in particolare sono stati scelti in modo definitivo i casi d'uso che sono stati successivamente indicizzati per facilitare la divisione della redazione di essi tra i componenti del gruppo. Dopo l'esplicitazione di questa parte saranno raffinati, di conseguenza, anche i requisiti richiesti dal proponente.
    \subsection{Punto 3}
        Infine, è stata effettuata una discussione preliminare riguardante il \textit{Piano di Qualifica}. In questo frangente è quindi cominciata la ricerca, da parte dei componenti del gruppo, di alcune metriche per garantire una consistenza e un'efficienza degli standard e della qualità dei prodotti\glosp presenti e futuri.
    \subsection{Punto 4}
        Dato il proseguimento dell'\textit{Analisi dei Requisiti} e l'inizio del \textit{Piano di Qualifica} il gruppo ha registrato le modifiche e gli aggiornamenti proposti per continuare la normazione di queste attività nelle \textit{Norme di Progetto}.