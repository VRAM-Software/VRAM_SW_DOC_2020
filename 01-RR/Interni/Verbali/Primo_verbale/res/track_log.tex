% \textbf = grassetto; \Large = font più grande
% \rowcolors{quanti colori alternare}{colore numero riga pari}{colore numero riga dispari}: colori alternati per riga
% \rowcolor{color}: cambia colore di una riga
% p{larghezza colonna}: p è un tipo di colonna di testo verticalmente allineata sopra, ci sarebbe anche m che è centrata a metà ma non è precisa per questo utilizzo TBStrut; la sintassi >{\centering} indica che il contenuto della colonna dovrà essere centrato
% \TBstrut fa parte di alcuni comandi che ho inserito in config.tex che permetto di aggiungere un po' di padding al testo
% \\ [2mm] : questra scrittura indica che lo spazio dopo una break line deve essere di 2mm
% 
\setcounter{secnumdepth}{0}
%\hfill \break
%\textbf{\Large{Diario delle modifiche}} \\
\section{Riepilogo tracciamenti}
\rowcolors{2}{gray!25}{gray!15}
\begin{longtable} {
		>{\centering}p{17mm} 
		%>{\centering}p{19.5mm}
		%>{\centering}p{24mm} 
		%>{\centering}p{24mm} 
		>{}p{120mm}}
	\rowcolor{gray!50}
	\textbf{Codice} & \multicolumn{1}{c}{\textbf{Decisione}} \\%\textbf{Decisione} \\ %\TBstrut \\
	VI\_1.1 & Scelto \textit{VRAM Software} come nome del gruppo. \TBstrut \\ [2mm]
	VI\_1.2 & Scelta immagine a fronte del documento come logo del gruppo. \TBstrut \\ [2mm]
	VI\_1.3 & Scelto \textit{Git} come sistema di versionamento. \TBstrut \\ [2mm]
	VI\_1.4 & Scelto \textit{GitHub} come servizio di hosting. \TBstrut \\ [2mm]
	VI\_1.5 & Scelti \textit{Slack} e \textit{Telegram} come canali di comunicazione. \TBstrut \\ [2mm]
	VI\_1.6 & Scelto \textit{Google Calendar} per organizzare riunioni e appuntare scadenze. \TBstrut \\ [2mm]
	VI\_1.7 & Scelto \textit{Google Drive} per la condivisione di strumenti utili. \TBstrut \\ [2mm]
	VI\_1.8 & Scelto il capitolato C4. \TBstrut \\ [2mm]
\end{longtable}