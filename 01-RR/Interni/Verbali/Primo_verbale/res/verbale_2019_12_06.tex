\section{Informazioni generali}
    \subsection{Informazioni incontro}
        \begin{itemize}
            \item \textbf{Luogo}: Dipartimento di Matematica "Tullio Levi-Civita";
            \item \textbf{Data}: 2019-12-06;
            \item \textbf{Ora d'inizio}: 14.00;
            \item \textbf{Ora di fine}: 16.30;
            \item \textbf{Partecipanti}: \begin{itemize}
                \item Corrizzato Vittorio;
                \item Dalla Libera Marco;
                \item Rampazzo Marco;
                \item Santagiuliana Vittorio;
                \item Schiavon Rebecca;
                \item Spreafico Alessandro;
                \item Toffoletto Massimo.
            \end{itemize}
        \end{itemize}
    \subsection{Argomenti trattati}
        Durante la prima riunione, i componenti hanno discusso e deliberato le seguenti scelte:
        \begin{enumerate}
            \item nome e logo del gruppo;
            \item strumenti di supporto da utilizzare;
            \item discussione e scelta del capitolato\glo.
        \end{enumerate}
\section{Verbale}
    \subsection{Punto 1}
        Inizialmente il gruppo ha dibattuto per decidere un nome e un logo a esso associato. Ne è seguita una votazione in cui i componenti hanno
        concordato per \textit{VRAM Software} e per l'immagine a fronte di questo verbale (che verrà poi utilizzata nei successivi documenti).
    \subsection{Punto 2}
        In seguito, si è discusso sugli strumenti da utilizzare per una gestione ottimale del materiale di lavoro. La scelta è ricaduta sulle seguenti tecnologie:
        \begin{itemize}
            \item \textbf{Git}: come software di controllo versione;
            \item \textbf{GitHub}: come servizio di hosting;
            \item \textbf{Slack} e \textbf{Telegram}: come canali di comunicazione;
            \item \textbf{Google Calendar}: per organizzare riunioni e appuntare scadenze;
            \item \textbf{Google Drive}: per la condivisione di documenti o applicativi utili alla redazione o allo sviluppo.
        \end{itemize}
    \subsection{Punto 3}
        Infine, i partecipanti hanno condiviso le loro opinioni riguardanti i vari capitolati\glo, e dopo aver esposto vantaggi e svantaggi di ognuno, hanno concordato sulla scelta di C4 (\textit{Predire in Grafana}\glo) proposto dall'azienda \textit{Zucchetti}.