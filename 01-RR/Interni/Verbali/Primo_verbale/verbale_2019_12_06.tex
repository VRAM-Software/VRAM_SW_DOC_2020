%documento di prova per testare il template, può essere usato come base per la creazione dei documenti finali

\documentclass[12pt,a4paper]{article} %formato del documento e grandezza caratteri

% Tutti i pacchetti usati, da inserire nel preambolo prima delle configurazioni

\usepackage[T1]{fontenc} %serve per andare a capo correttamente in italiano
\usepackage[utf8]{inputenc} %serve per le lettere accentate
\usepackage[english,italian]{babel} %serve per far apparire "indice" al posto di "contents"; imposta italiano lingua principale, inglese secondaria.

\usepackage{graphicx} %serve per includere le immagini

\usepackage{hyperref} %serve per rendere clickabili le sezioni dell'indice

\usepackage{float} %per poter usare l'opzione [H] nelle figure; [H] permette di tenere fissate le immagini che altrimenti LaTeX si sposta a piacere.

\usepackage{listings} %serve per poter mettere snippets di codice nel testo

\usepackage{lastpage} %serve per poter fare " \rfoot{\thepage\ di \pageref{LastPage}} "

\usepackage{fancyhdr} %Per header personalizzati %include il file package.tex
\input{config/config_latex.tex} %inclusione file che permette di modificare i vari comandi custom
%Preambolo: la parte prima del \begin{document}

\documentclass[12pt,a4paper]{article} %formato del documento e grandezza caratteri

% Tutti i pacchetti usati, da inserire nel preambolo prima delle configurazioni

\usepackage[T1]{fontenc} %Permette la sillabazione su qualsiasi testo contenente caratteri
\usepackage[utf8]{inputenc} %Serve per usare la codifica utf-8
\usepackage[english,italian]{babel} %Imposta italiano lingua principale, inglese secondaria. Es. serve per far apparire "indice" al posto di "contents"

\usepackage{graphicx} %Serve per includere le immagini

\usepackage[hypertexnames=false]{hyperref} %Gestisce i riferimenti/link. Es. Serve per rendere clickabili le sezioni dell'indice

\usepackage{float} %Serve per migliore la definizione di oggetti fluttuanti come figure e tabelle. Es. poter usare l'opzione [H] nelle figure ovvero tenere fissate le immagini che altrimenti LaTeX si sposta a piacere.

\usepackage{listings} %Serve per poter mettere snippets di codice nel testo

\usepackage{lastpage} %Serve per poter introdurre un'etichetta a cui si può fare riferimento Es. piè di pagina; poter fare " \rfoot{\thepage\ di \pageref{LastPage}} "

\usepackage{fancyhdr} %Per header e piè di pagina personalizzati

%Sono alcuni package che potranno esserci utili in futuro
%\usepackage{charter}
%\usepackage{eurosym}
\usepackage{subcaption}
%\usepackage{wrapfig}
%\usepackage{background}
\usepackage{longtable} % tabella che può continuare per più di una pagina
\usepackage[table]{xcolor} % ho dovuto aggiungere table in modo da poter colorare le row della tabella, dava: undefined control sequences
%\usepackage{colortbl}

\usepackage{dirtree} % usato per creare strutte tree-view in stile filesystem
\usepackage{xspace} % usato per inserire caratteri spazio
\usepackage[official]{eurosym}
\usepackage{pdflscape} %include il file package.tex

% Configurazioni varie, da inserire nel preambolo dopo i pacchetti

\hypersetup{hidelinks} %serve per nascondere riquadri rossi che circondano i link 

\lstset{literate= {à}{{\`a}}1 } %Permette di usare lettere accentate nei listings

\pagestyle{fancy} %Imposto stile pagina
\fancyhf{} %Reset, se lo tolgo LaTex mette impostazioni di default (p.es numerazione pagine di default)

\lhead{\includegraphics[scale=0.25]{img/Logo_header.png}} %Left header che compare in ogni pagina
\rhead{\leftmark} %Nome della top-level structure (p.es. Section in article o Chapter in book) in ogni pagina

%Setto il colore dei link
\hypersetup{
	colorlinks,
	linkcolor=[HTML]{404040},
	citecolor={purple!50!black},
	urlcolor={blue!50!black}
}

%Tabelle e tabulazione (può tornare utile)
%\setlength{\tablcolsep}{10pt}
%\renewcommand{\arraystretch}{1.4}

%Comando per aggiungere le pagine di ogni sezione
\newcommand{\newSection}[1]{%
	\newpage
	\input{res/sections/#1}
}

% Comandi per aggiungere padding a parole contenute nella tabella; è una specie di strut (un carattere invisibile)
\newcommand\Tstrut{\rule{0pt}{2.6ex}} % top padding
\newcommand\Bstrut{\rule[-0.9ex]{0pt}{0pt}} % bottom padding
\newcommand{\TBstrut}{\Tstrut\Bstrut} % top & bottom padding

\newcommand{\glo}{$_G$} %Comando per aggiungere il pedice G
\newcommand{\glosp}{$_G$ } %Comando per aggiungere il pedice G con spazio

\begin{document}

% #### FRONTESPIZIO (frontmatter) ####
\setlength{\headheight}{39pt} %distanzia l'header
\pagenumbering{gobble} %Toglie il numero di pagina
\begin{titlepage}
	\begin{center}
		\includegraphics[scale=0.6]{img/logo.png} \\ %Logo
		\vspace{0.5cm} %Aggiunge uno spazio verticale di 0.5 cm
		
		{\LARGE Progetto "Predire in Grafana"} \\ %Autore, prende variabile definita nel config_latex.tex
		\vspace{0.5cm} %Attenzione a mettere il punto e NON la virgola
		
		{\Huge \textbf{\DocTitle}} \\ %Titolo, prende variabile definita nel config_latex.tex
		\vspace{0.5cm}
		
		\DocDate \\ %Data, prende variabile definita nel config_latex.tex
		\vspace{1cm}
		
		%Allineamento colonne: l=left r=right c=center, 
		%va specificato per ogni colonna
		%Se si vuole la riga tra colonne mettere "|"
		
		\begin{tabular}{r | l} %Elementi colonne separate da "&", le righe finiscono con "\\"
			Versione & \ver \\
			Approvazione & \app \\ %
			Redazione & \red \\
			Verifica & \test \\
			Stato & \stat \\
			Uso & \use \\
		    Destinato a & Azienda \\
						& Prof. Tullio Vardanega \\
						& Prof. Riccardo Cardin \\
			Email di riferimento& vram.software@gmail.com
		\end{tabular}
		\vfill
		\textbf{Descrizione} \\
		\DocDesc
	\end{center}
\end{titlepage}
\clearpage

% #### INDICE (tableofcontents )####
%\pagenumbering{Roman} %Pagine con i  numeri romani + reset a I
%\renewcommand{\footrulewidth}{1pt} %Di default footrulewidth==0 e quindi è invisibile
%\tableofcontents %Provoca la stampa dell'indice
%\label{LastPageRoman} %Indica l'ultima pagina dell'indice
%\rfoot{\thepage\ di \pageref{LastPageRoman}} %Pagina n di m, con numeri Romani
%\clearpage

% #### PARTE PRINCIPALE (mainmatter) ####

%\pagenumbering{arabic} %Pagine con i numeri arabi + reset a 1
%\rfoot{\thepage\ di \pageref{LastPage}} %Pagina n di m, con numeri Arabi; usa il pacchetto "lastpage", in caso non sia possibile usare tale pacchetto mettere al fondo dell'ultima pagina "\label{LastPage}"
%\clearpage %inclusione file principale con la pagina del titolo
\textbf{\Large{Diario delle modifiche}} \\
\rowcolors{2}{gray!25}{gray!15}
\begin{longtable} {
		>{\centering}p{17mm} 
		>{\centering}p{19.5mm}
		>{\centering}p{24mm} 
		>{\centering}p{24mm} 
		>{}p{32mm}}
	\rowcolor{gray!50}
	\textbf{Versione} & \textbf{Data} & \textbf{Nominativo} & \textbf{Ruolo} & \textbf{Descrizione} \TBstrut \\
	1.0.0 & 2019-02-12 & Pinco Pallino & Analista e Verificatore & modificato \$3 \TBstrut \\ [2mm]
	1.0.0 & 2019-02-12 & Pinco Pallino & Analista e Verificatore & modificato \$3 \TBstrut \\ [2mm]
	1.0.0 & 2019-02-12 & Pinco Pallino & Analista e Verificatore & modificato \$3 \TBstrut \\ [2mm]
	1.0.0 & 2019-02-12 & Pinco Pallino & Analista e Verificatore & modificato \$3 \TBstrut \\ [2mm]
	1.0.0 & 2019-02-12 & Pinco Pallino & Analista e Verificatore & modificato \$3 \TBstrut \\ [2mm]
	1.0.0 & 2019-02-12 & Pinco Pallino & Analista e Verificatore & modificato \$3 \TBstrut \\ [2mm]
	1.0.0 & 2019-02-12 & Pinco Pallino & Analista e Verificatore & modificato \$3 \TBstrut \\ [2mm]
	1.0.0 & 2019-02-12 & Pinco Pallino & Analista e Verificatore & modificato \$3 \TBstrut \\ [2mm]
	1.0.0 & 2019-02-12 & Pinco Pallino & Analista e Verificatore & modificato \$3 \TBstrut \\ [2mm]
	1.0.0 & 2019-02-12 & Pinco Pallino & Analista e Verificatore & modificato \$3 \TBstrut \\ [2mm]
	1.0.0 & 2019-02-12 & Pinco Pallino & Analista e Verificatore & modificato \$3 \TBstrut \\ [2mm]
	1.0.0 & 2019-02-12 & Pinco Pallino & Analista e Verificatore & modificato \$3 \TBstrut \\ [2mm]
	1.0.0 & 2019-02-12 & Pinco Pallino & Analista e Verificatore & modificato \$3 \TBstrut \\ [2mm]
	1.0.0 & 2019-02-12 & Pinco Pallino & Analista e Verificatore & modificato \$3 \TBstrut \\ [2mm]
	1.0.0 & 2019-02-12 & Pinco Pallino & Analista e Verificatore & modificato \$3 \TBstrut \\ [2mm]
	1.0.0 & 2019-02-12 & Pinco Pallino & Analista e Verificatore & modificato \$3 \TBstrut \\ [2mm]
	1.0.0 & 2019-02-12 & Pinco Pallino & Analista e Verificatore & modificato \$3 \TBstrut \\ [2mm]
	1.0.0 & 2019-02-12 & Pinco Pallino & Analista e Verificatore & modificato \$3 \TBstrut \\ [2mm]
	1.0.0 & 2019-02-12 & Pinco Pallino & Analista e Verificatore & modificato \$3 \TBstrut \\ [2mm]
	1.0.0 & 2019-02-12 & Pinco Pallino & Analista e Verificatore & modificato \$3 \TBstrut \\ [2mm]
	1.0.0 & 2019-02-12 & Pinco Pallino & Analista e Verificatore & modificato \$3 \TBstrut \\ [2mm]
	1.0.0 & 2019-02-12 & Pinco Pallino & Analista e Verificatore & modificato \$3 \TBstrut \\ [2mm]
	1.0.0 & 2019-02-12 & Pinco Pallino & Analista e Verificatore & modificato \$3 \TBstrut \\ [2mm]
	1.0.0 & 2019-02-12 & Pinco Pallino & Analista e Verificatore & modificato \$3 \TBstrut \\ [2mm]
	1.0.0 & 2019-02-12 & Pinco Pallino & Analista e Verificatore & modificato \$3 \TBstrut \\ [2mm]
	1.0.0 & 2019-02-12 & Pinco Pallino & Analista e Verificatore & modificato \$3 \TBstrut \\ [2mm]
	1.0.0 & 2019-02-12 & Pinco Pallino & Analista e Verificatore & modificato \$3 \TBstrut \\ [2mm]
	1.0.0 & 2019-02-12 & Pinco Pallino & Analista e Verificatore & modificato \$3 \TBstrut \\ [2mm]
	1.0.0 & 2019-02-12 & Pinco Pallino & Analista e Verificatore & modificato \$3 \TBstrut \\ [2mm]
	1.0.0 & 2019-02-12 & Pinco Pallino & Analista e Verificatore & modificato \$3 \TBstrut \\ [2mm]
	1.0.0 & 2019-02-12 & Pinco Pallino & Analista e Verificatore & modificato \$3 \TBstrut \\ [2mm]
	1.0.0 & 2019-02-12 & Pinco Pallino & Analista e Verificatore & modificato \$3 \TBstrut \\ [2mm]
	1.0.0 & 2019-02-12 & Pinco Pallino & Analista e Verificatore & modificato \$3 \TBstrut \\ [2mm]
	1.0.0 & 2019-02-12 & Pinco Pallino & Analista e Verificatore & modificato \$3 \TBstrut \\ [2mm]
	1.0.0 & 2019-02-12 & Pinco Pallino & Analista e Verificatore & modificato \$3 \TBstrut \\ [2mm]
\end{longtable} %inclusione file con tabella diario delle modifiche
\pagebreak

% #### INDICE (tableofcontents )####
\pagenumbering{Roman} %Pagine con i  numeri romani + reset a I
\renewcommand{\footrulewidth}{1pt} %Di default footrulewidth==0 e quindi è invisibile
\tableofcontents %Provoca la stampa dell'indice
\label{LastPageRoman} %Indica l'ultima pagina dell'indice
\rfoot{\thepage\ di \pageref{LastPageRoman}} %Pagina n di m, con numeri Romani
\clearpage

% #### PARTE PRINCIPALE (mainmatter) ####
\pagenumbering{arabic} %Pagine con i numeri arabi + reset a 1
\rfoot{\thepage\ di \pageref{LastPage}} %Pagina n di m, con numeri Arabi; usa il pacchetto "lastpage", in caso non sia possibile usare tale pacchetto mettere al fondo dell'ultima pagina "\label{LastPage}"
\clearpage

\setcounter{secnumdepth}{4} %Permette di andare fino alla profondità del paragraph con la numerazione delle sezioni

\pagebreak
\section{Informazioni generali}
    \subsection{Informazioni incontro}
        \begin{itemize}
            \item \textbf{Luogo}: Dipartimento di Matematica "Tullio Levi-Civita";
            \item \textbf{Data}: 06-12-2019;
            \item \textbf{Ora d'inizio}: 14.00;
            \item \textbf{Ora di fine}: 16.30;
            \item \textbf{Partecipanti}: \begin{itemize}
                \item Corrizzato Vittorio;
                \item Dalla Libera Marco;
                \item Rampazzo Marco;
                \item Santagiuliana Vittorio;
                \item Schiavon Rebecca;
                \item Spreafico Alessandro;
                \item Toffoletto Massimo.
            \end{itemize}
        \end{itemize}
    \subsection{Argomenti trattati}
        Durante la prima riunione di gruppo, i componenti hanno discusso e deliberato la scelta di un nome e un logo rappresentativo.
        Successivamente ci si è confrontati sulla scelta del capitolato\glo e sugli strumenti di supporto da utilizzare nella realizzazione
        di quest'ultimo.
\section{Verbale}
    Inizialmente il gruppo ha dibattuto per decidere un nome e un logo a esso associato. Ne è seguita una votazione in cui i componenti hanno
    concordato per "\textit{VRAM Software}" e per l'immagine a fronte di questo verbale (che verrà poi utilizzata nei successivi documenti). Dopo di che, si è discusso
    sugli strumenti da utilizzare per una gestione ottimale del materiale di lavoro. La scelta è ricaduta sulle seguenti tecnologie:
    \begin{itemize}
        \item \textbf{Git}\glo: come software di controllo versione;
        \item \textbf{GitHub}\glo: come servizio di hosting;
        \item \textbf{Slack}\glo e \textbf{Telegram}\glo: come canali di comunicazione;
        \item \textbf{Google Calendar}\glo: per organizzare riunioni e appuntare scadenze;
        \item \textbf{Google Drive}\glo: per la condivisione di documenti o applicativi utili alla redazione o allo sviluppo.
    \end{itemize}
    Infine, i partecipanti hanno condiviso le loro opinioni riguardanti i vari capitolati\glo, e dopo aver esposto vantaggi e svantaggi di ognuno, hanno concordato sulla
    scelta di C6 (\textit{ThiReMa - Things Relationship Management}).

        
    


    
\pagebreak

% \textbf = grassetto; \Large = font più grande
% \rowcolors{quanti colori alternare}{colore numero riga pari}{colore numero riga dispari}: colori alternati per riga
% \rowcolor{color}: cambia colore di una riga
% p{larghezza colonna}: p è un tipo di colonna di testo verticalmente allineata sopra, ci sarebbe anche m che è centrata a metà ma non è precisa per questo utilizzo TBStrut; la sintassi >{\centering} indica che il contenuto della colonna dovrà essere centrato
% \TBstrut fa parte di alcuni comandi che ho inserito in config.tex che permetto di aggiungere un po' di padding al testo
% \\ [2mm] : questra scrittura indica che lo spazio dopo una break line deve essere di 2mm
% 
\setcounter{secnumdepth}{0}
%\hfill \break
%\textbf{\Large{Diario delle modifiche}} \\
\section{Riepilogo tracciamenti}
\rowcolors{2}{gray!25}{gray!15}
\begin{longtable} {
		>{\centering}p{17mm} 
		%>{\centering}p{19.5mm}
		%>{\centering}p{24mm} 
		%>{\centering}p{24mm} 
		>{}p{120mm}}
	\rowcolor{gray!50}
	\textbf{Codice} & \multicolumn{1}{c}{\textbf{Decisione}} \\%\textbf{Decisione} \\ %\TBstrut \\
	VI\_1.1 & Scelto \textit{VRAM Software} come nome del gruppo. \TBstrut \\ [2mm]
	VI\_1.2 & Scelta immagine a fronte del documento come logo del gruppo. \TBstrut \\ [2mm]
	VI\_1.3 & Scelto \textit{Git} come sistema di versionamento. \TBstrut \\ [2mm]
	VI\_1.4 & Scelto \textit{GitHub} come servizio di hosting. \TBstrut \\ [2mm]
	VI\_1.5 & Scelti \textit{Slack} e \textit{Telegram} come canali di comunicazione. \TBstrut \\ [2mm]
	VI\_1.6 & Scelto \textit{Google Calendar} per organizzare riunioni e appuntare scadenze. \TBstrut \\ [2mm]
	VI\_1.7 & Scelto \textit{Google Drive} per la condivisione di strumenti utili. \TBstrut \\ [2mm]
	VI\_1.8 & Scelto il capitolato C4. \TBstrut \\ [2mm]
\end{longtable}

\end{document}