\section{Processi Primari}
\subsection{Fornitura}
\subsubsection{Scopo}
Lo scopo del processo\glosp di fornitura è stabilire norme, tempi e risorse necessarie a svolgere l'intero progetto\glo.
A seguito dell'analisi sui capitolati\glo, verrà redatto il documento \textit{Studio di Fattibilità v. 1.1.1} nel quale verrà manifestata la decisione del capitolato\glosp scelto; sarà quindi necessario formalizzare un contratto, insieme all'azienda proponente, per la consegna del prodotto\glo. 
Per fare quanto sopra descritto si devono svolgere le seguenti attività:
\begin{itemize}
	\item avvio;
	\item contrattazione con il proponente;
	\item pianificazione del lavoro;
	\item esecuzione di quanto pianificato con controllo della qualità;
	\item revisione e valutazione del prodotto\glo;
	\item completamento e consegna al proponente.
\end{itemize}
\subsubsection{Aspettative}
Il gruppo si aspetta di mantenere un dialogo costante con l'azienda proponente instaurando un rapporto collaborativo al fine di comprendere meglio le esigenze. In particolare ci si focalizza sui seguenti termini:
\begin{itemize}
	\item definizione degli elementi fondamentali su cui focalizzarsi per soddisfare le necessità del proponente;
	\item definizione di vincoli e requisiti sui processi\glo;
	\item stima delle tempistiche e della pianificazione del lavoro;
	\item verifica continua e chiarimento di eventuali dubbi;
	\item accordo sulla qualifica del prodotto\glo.
\end{itemize} 
\subsubsection{Descrizione}
La sezione dei processi\glosp di fornitura specifica le regole che devono essere rispettate affinché il nostro gruppo possa diventare fornitore del proponente \textit{Zucchetti}.
\subsubsection{Attività}
\paragraph{Studio di Fattibilità}\mbox{}\\ [1mm]
Lo \textit{Studio di Fattibilità} è un'attività che ha lo scopo di stilare le motivazioni per cui il gruppo ha preferito un capitolato\glosp rispetto agli altri.
Risulta quindi necessario individuare costi e benefici di ogni capitolato\glosp al fine di determinare quello che meglio si bilancia con le aspettative del gruppo e le risorse a disposizione.
Dunque gli analisti redigono il documento \textit{Studio di Fattibilità} dove per ogni capitolato\glosp vengono indicati:
\begin{itemize}
	\item \textbf{Informazioni Generali}: elenco che presenta le seguenti informazioni: 
	\begin{itemize}
		\item nome del progetto\glo;
		\item proponente;
		\item committente.
	\end{itemize} 
	\item \textbf{Descrizione}: breve esposizione dell'argomento del capitolato\glo;
	\item \textbf{Obiettivi di Progetto}: descrizione dettagliata centrata sugli obiettivi da raggiungere;
	\item \textbf{Vincoli}: elenco dei vincoli imposti dal proponente per la realizzazione del progetto\glo;
	\item \textbf{Tecnologie Interessate}: elenco e descrizione sintetica delle tecnologie che dovranno essere impiegate nello svolgimento;
	\item \textbf{Aspetti positivi}: elenco delle principali motivazioni che porterebbero il gruppo a scegliere il capitolato\glo;
	\item \textbf{Criticità e fattori di rischio}: elenco delle principali motivazioni che porterebbero il gruppo a non scegliere il capitolato\glo;
	\item \textbf{Conclusioni}: sintesi delle motivazioni per cui il capitolato\glosp è stato scelto o è stato escluso.
\end{itemize}
\paragraph{Piano di Progetto}\mbox{}\\ [1mm]
Il \textit{Piano di Progetto} è un documento redatto dal responsabile in collaborazione con gli amministratori ed ha lo scopo di pianificare le attività per la realizzazione del progetto\glosp e individuare le risorse disponibili da assegnare a ciascuna di esse.
Questo documento è composto da:
\begin{itemize}
	\item \textbf{Introduzione}: descrizione generale del documento che specifica anche lo scopo del documento e del prodotto\glosp e il calendario delle attività;
	\item \textbf{Analisi dei rischi}: analisi dettagliata dei rischi che potrebbero sorgere durante il ciclo di sviluppo del progetto\glo. In allegato vengono forniti stime probabilistiche, livelli di rischio e modalità previste per fare prevenzione o per affrontarli qualora non sia stato possibile evitarli;
	\item \textbf{Modello di sviluppo}: descrizione del modello di sviluppo selezionato per lo svolgimento del progetto\glo;
	\item \textbf{Pianificazione}: specifica dei tempi e dei ruoli per ogni attività individuata al fine di affrontare al meglio le scadenze del capitolato\glo. Essa viene limitata dalla quantità di risorse;
	\item \textbf{Preventivo e Consuntivo}: stima dei costi necessari per lo svolgimento di ogni scadenza prevista e conseguente costruzione del preventivo per il progetto\glo. Stesura di un \textit{Verbale} interno su cui riportare l'andamento effettivo rispetto a quanto previsto;
	\item \textbf{Organigramma}: presentazione del nostro organigramma.
\end{itemize}
\paragraph{Piano di Qualifica}\mbox{}\\ [1mm]
Questa attività prevede la stesura del documento \textit{Piano di Qualifica} che viene redatto dai verificatori. Esso ha lo scopo di descrivere strategie e tecniche di verifica e validazione\glosp che devono essere applicate dai verificatori nelle loro attività.
Questo documento è composto da:
\begin{itemize}
	\item \textbf{Introduzione}: descrizione generale del documento che specifica anche lo scopo del documento e del prodotto\glo;
	\item \textbf{Qualità di Processo}\glo: individuazione dei processi\glosp da adottare in riferimento ad uno standard e delle metriche per la misurazione e il monitoraggio della loro qualità;
	\item \textbf{Qualità di Prodotto}\glo: individuazione degli obiettivi posti sul prodotto\glosp che sono necessari per raggiungere una buona qualità e delle metriche per misurarla;
	\item \textbf{Descrizione dei test}: individuazione dei test a cui sottoporre il prodotto\glosp per garantire che i requisiti vengano soddisfatti;
	\item \textbf{Standard di Qualità}: elenco degli standard di qualità scelti;
	\item \textbf{Resoconto delle attività di verifica}: resoconti dei risultati dati dalle metriche di qualità calcolate per ogni attività;
	\item \textbf{Valutazioni per il miglioramento}: esposizione dei problemi rilevati e delle soluzioni da attuare per migliorare.
\end{itemize}
\subsubsection{Strumenti}
Durante il processo\glosp di fornitura sono utilizzati i seguenti strumenti della suite di Microsoft.
\paragraph{Microsoft Excel}\mbox{}\\ [1mm]
Per svolgere semplici calcoli, creazione di grafici, diagrammi e tabelle si è scelto di utilizzare il software Microsoft Excel.
\paragraph{Microsoft Project}\mbox{}\\ [1mm]
Per realizzare il diagramma di Gantt necessario per la pianificazione, in particolare di tempi, risorse e analisi dei carichi di lavoro, si è scelto di utilizzare Microsoft Project.

\subsection{Sviluppo}
\subsubsection{Scopo}
Lo scopo del processo\glosp di sviluppo è svolgere le attività necessarie per la realizzazione del prodotto\glosp richiesto.
\subsubsection{Aspettative}
Il gruppo si è posto le seguenti aspettative:
\begin{itemize}
	\item stabilire gli obiettivi di sviluppo;
	\item stabilire i vincoli tecnologici e di progettazione\glo;
	\item realizzare un prodotto\glosp che soddisfi i requisiti imposti dai proponenti e superi i test per avere una buona qualità.
\end{itemize}
\subsubsection{Descrizione}
Il processo\glosp di sviluppo è suddiviso nelle seguenti attività:
\begin{itemize}
	\item \textbf{Analisi dei Requisiti};
	\item \textbf{Progettazione}\glo;
	\item \textbf{Codifica}.
\end{itemize}
\subsubsection{Attività}
\paragraph{Analisi dei Requisiti}
\paragraph*{Scopo}\mbox{}\\ [1mm] 
Lo scopo dell' \textit{Analisi dei Requisiti} è individuare ed analizzare i requisiti del progetto\glosp al fine di:
\begin{itemize}
	\item individuare lo scopo del lavoro;
	\item definire un'analisi dettagliata e precisa del dominio dell'applicazione e dei requisiti che i progettisti dovranno seguire;
	\item definire dei riferimenti per le attività di controllo dei test da fornire ai verificatori;
	\item fornire una base solida da cui poter effettuare raffinamenti successivi e fornire un miglioramento continuo del progetto\glo.
	%\item creare dei riferimenti per la stima dei costi.
\end{itemize}
\paragraph*{Aspettative}\mbox{}\\ [1mm]
L'obiettivo è stilare una documentazione formale sui requisiti di progetto\glosp richiesti dal proponente.
\paragraph*{Descrizione}\mbox{}\\ [1mm]
L'analisi dei requisiti viene redatta eseguendo le seguenti attività:
\begin{itemize}
	\item \textbf{Comprensione del capitolato}\glo: eseguita attraverso la lettura e l'analisi della documentazione fornita;
	\item \textbf{Comunicazioni con il proponente}: eseguita al fine di migliorare l'analisi del progetto\glo;
	\item \textbf{Comunicazioni interne al gruppo};
	\item \textbf{Analisi dei possibili casi d'uso}\glo.
\end{itemize}
\paragraph*{Casi d'uso}\mbox{}\\ [1mm]
Un caso d'uso\glosp definisce un insieme di scenari con un obiettivo finale in comune per un attore\glo. Sono ottenuti attraverso la valutazione di ogni requisito e descrivono l'insieme delle funzionalità fornite dal sistema dal punto di vista degli utenti.
Per descrivere ogni caso d'uso viene utilizzata la seguente struttura:
\begin{itemize}
	\item \textbf{Codice Identificativo};
	\item \textbf{Titolo};
	\item \textbf{Attori}\glosp \textbf{primari};
	\item \textbf{Attori}\glosp \textbf{secondari};
	\item \textbf{Descrizione};
	\item \textbf{Precondizione};
	\item \textbf{Post-condizione};
	\item \textbf{Scenario principale};
	\item \textbf{Scenario alternativo};
	\item \textbf{Inclusioni} (ove presenti);
	\item \textbf{Estensioni} (ove presenti);
	\item \textbf{Generalizzazioni} (ove presenti);
	\item \textbf{Diagramma UML}.	
\end{itemize}
Il codice identificativo sarà scritto in questo formato: \\
\textbf{UC[codice\_padre].[codice\_figlio]} \\
%Da decidere
Dove:
\begin{itemize}
	\item \textbf{codice\_padre:} numero che identifica univocamente i casi d'uso\glo;
	\item \textbf{codice\_figlio:} numero progressivo che identifica i sotto-casi;
\end{itemize}
\textbf{Requisiti}\\%Da decidere
Per descrivere un requisito viene utilizzata la seguente struttura:
\begin{itemize}
	\item codice identificativo;
	\item classificazione;
	\item descrizione;
	\item fonti.
\end{itemize} 
Il \textbf{Codice Identificativo} sarà scritto in questo formato: \\
\textbf{R[Importanza][Tipologia][Codice]} \\
Dove:
\begin{itemize}
	\item \textbf{Importanza} può assumere i seguenti valori:
	\begin{itemize}
		\item 1: requisito obbligatorio;
		\item 2: requisito desiderabile;
		\item 3: requisito opzionale.
	\end{itemize}
	\item \textbf{Tipologia} può assumere i seguenti valori:
	\begin{itemize}
		\item F: Funzionale;
		\item Q: Prestazionale;
		\item P: Qualitativo;
		\item V: Vincolo.
	\end{itemize}
	\item\textbf{Codice}: numero progressivo identificativo strutturato nel formato: [codice\_padre].[codice\_figlio]
\end{itemize}
Le \textbf{Fonti} possono essere:
\begin{itemize}
	\item capitolato\glo: il requisito è stato quindi individuato dalla lettura del capitolato\glo;
	\item interno: il requisito è stato individuato ed aggiunto in seguito ad un'analisi interna;
	\item caso d'uso\glo: il requisito è stato individuato dallo studio di un caso d'uso\glo;
	\item proponente: il requisito è stato individuato in seguito ad un colloquio con il proponente.
\end{itemize}
I diagrammi UML\glosp vengono realizzati usando la versione 2.0 del linguaggio.\pagebreak
\paragraph{Progettazione}
\paragraph*{Scopo}\mbox{}\\ [1mm]
Lo scopo della progettazione\glosp è descrivere una soluzione soddisfacente per gli stakeholder\glo. Questa attività definisce l'architettura logica del prodotto\glosp software richiesto, a partire dall'\textit{Analisi dei Requisiti}
Viene quindi definita l'architettura del prodotto\glosp finale che dovrà:
\begin{itemize}
	\item soddisfare i requisiti individuati nell'\textit{Analisi dei Requisiti};
	\item essere comprensibile;
	\item essere modulare;
	\item definire le proprie parti con specifiche chiare e coese;
	\item garantire robustezza al fine di riuscire a gestire errori improvvisi;
	\item essere organizzata per facilitare cambiamenti futuri.
\end{itemize}
\paragraph*{Aspettative}\mbox{}\\ [1mm]
La progettazione\glosp ha come risultato finale la realizzazione dell'architettura del sistema.
\paragraph*{Descrizione}\mbox{}\\ [1mm]
La progettazione\glosp è composta da due sezioni:
\begin{itemize}
	\item technology baseline\glo;
	\item product baseline\glo.	
\end{itemize}
\paragraph*{Technology Baseline}\mbox{}\\ [1mm]
Nella technology baseline\glosp si trovano:
\begin{itemize}
	\item \textbf{Tecnologie utilizzate}: descrizione con l'indicazione di vantaggi e svantaggi e l'utilizzo;
	\item \textbf{Design pattern}: descrizione del design pattern utilizzati all'interno dell'architettura. Per ciascuno di essi deve essere fornito un diagramma che ne definisce la struttura e una descrizione;
	\item \textbf{Diagrammi UML}:
	\begin{itemize}
		\item \textbf{Diagrammi delle classi};
		\item \textbf{Diagrammi dei package};
		\item \textbf{Diagrammi di attività};
		\item \textbf{Diagrammi di sequenza}.
	\end{itemize}
	\item \textbf{Tracciamento delle componenti}: riferimento di ogni requisito al componente che lo soddisfa;
	\item \textbf{Test di integrazione}: necessari per verificare che l'unione delle parti funzioni correttamente.
\end{itemize}
\paragraph*{Product Baseline}\mbox{}\\ [1mm]
Nella product baseline\glosp sono presenti:
\begin{itemize}
	\item \textbf{Descrizione delle classi}: descrizione degli obiettivi e delle funzionalità di ogni classe;
	\item \textbf{Tracciamento delle classi}: riferimento di ogni requisito alla classe che lo soddisfa;
	\item \textbf{Test di unità}: necessari per verificare il corretto funzionamento di ogni singola parte del software.
\end{itemize}
\paragraph{Codifica}
\paragraph*{Scopo}\mbox{}\\ [1mm]
Lo scopo di questa sezione è dare una norma per definire tutte le regole di codifica per garantire la leggibilità e la manutenibilità del codice.
\paragraph*{Aspettative}\mbox{}\\ [1mm]
L'obiettivo della codifica è creare un prodotto\glosp coerente con i requisiti e le aspettative del proponente, il cui codice sia leggibile, uniforme e permetta di eseguire più facilmente le attività di manutenzione e verifica. Dunque i programmatori sono tenuti a seguire tali regole durante l'attività di implementazione.
\paragraph*{Descrizione}\mbox{}\\ [1mm]
La codifica consiste nella scrittura del codice per realizzare il prodotto\glo. La sua struttura dovrà rispettare le regole definite nel \textit{Piano di Qualifica} al fine di garantire una buona qualità.
\paragraph*{Stile di Codifica}\mbox{}\\ [1mm]
In seguito vengono elencate le norme da rispettare nel processo\glosp di codifica:
\begin{itemize}
	\item \textbf{Lingua}: la lingua con cui vengono scritti codice e commenti deve essere l'inglese;
	\item \textbf{Indentazione}: i blocchi annidati devono essere indentati con una tabulazione di quattro spazi;
	\item \textbf{Nomenclatura}: i nomi di classi, metodi e variabili devono essere univoci ed esplicativi;
	\item \textbf{Classi}: il nome di ciascuna classe deve iniziare con la lettera maiuscola; 
	\item \textbf{Costanti}: i nomi delle costanti devono essere scritti usando solo lettere maiuscole;
	\item \textbf{Metodi}: il nome di ciascun metodo deve iniziare con una lettera minuscola. Nei metodi con nome composto, le iniziali di ogni singola parola che li compone deve essere una lettera maiuscola seguendo la pratica CamelCase;
	\item \textbf{Ricorsione}: l'uso della ricorsione è da evitare ove possibile;
	\item \textbf{Condizioni}: l'uso di condizioni multiple è da evitare ove possibile per limitare la complessità delle singole espressioni;
	\item \textbf{Parentesizzazione}: le parentesi di delimitazione dei costrutti devono essere inserite in linea;
	\item \textbf{Struttura dei metodi}: ogni metodo deve avere uno ed un solo compito e il suo contenuto deve essere il più possibile breve in termini di righe di codice.
\end{itemize}
%\subsubsection{Metriche di qualità}
%\paragraph*{Scopo}\mbox{}\\ [1mm]
%Lo scopo di questa sezione è quello di definire delle metriche\glosp per valutare in modo quantitativo la qualità delle attività che si svolgono durante il processo\glosp di sviluppo.
%\paragraph*{Analisi dei Requisiti}\mbox{}\\ [1mm]
%I parametri utilizzati per valutare la qualità dell'analisi dei requisiti sono:
%La percentuale di requisiti obbligatori soddisfatti (\textbf{PROS}), calcolata con la seguente formula: $\frac{requisiti \; obbligatori \; soddisfatti}{requisiti \; obbligatori \; totali}$; \\
%La percentuale di requisiti desiderabili soddisfatti (\textbf{PRDS}), calcolata con la seguente formula: $\frac{requisiti \; desiderabili \; soddisfatti}{requisiti \; desiderabili \; totali}$.
\subsubsection{Strumenti}
Nella valutazione dei costi da affrontare per lo sviluppo del software e per la formazione del personale, abbiamo già identificato le principali tecnologie e strumenti che verranno utilizzare durante il progetto\glo.
\paragraph{TeXstudio}\mbox{}\\ [1mm]
Per la stesura dell'\textit{Analisi dei Requisiti} viene utilizzato TeXStudio come ambiente di sviluppo.
\paragraph{SonarJS}\mbox{}\\ [1mm]
Per l'analisi statica\glosp del codice viene utilizzato il code analyzer SonarJS\glosp su piattaforma di SonarCloud.
\paragraph{WebStorm}\mbox{}\\ [1mm]
Per la codifica viene utilizzato WebStorm, un ambiente di sviluppo integrato per JavaScript che offre piena compatibilità con Windows, Linux e SonarJS\glo.
\paragraph{LucidChart}\mbox{}\\ [1mm]
Per la realizzazione dei diagrammi UML\glo.	