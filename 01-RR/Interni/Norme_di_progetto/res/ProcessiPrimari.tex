\section{Processi Primari}

\subsection{Fornitura}
\subsubsection{Descrizione}
La sezione dei processi di fornitura specifica le regole che devono essere rispettate affinchè il nostro gruppo possa diventare fornitore del proponente Zucchetti e dei proponenti Prof. Tullio Vardanega e Prof. Riccardo Cardin.
\subsubsection{Scopo}
Lo scopo del processo di fornitura è di stabilire le norme, i tempi e le risorse necessarie nell'intero svolgimento del progetto.
Al seguito di un'analisi sui capitolati, verrà redatto uno Studio di Fattibilità al termine del quale sarà manifestata la decisione del capitolato\glosp scelto; sarà quindi necessario formalizzare un contratto, insieme all'azienda proponente, per la consegna del prodotto. 
Per svolgere quanto sopra descritto sarà necessario seguire le seguenti attività:
\begin{itemize}
	\item Avvio;
	\item Contrattazione con il proponente;
	\item Pianificazione del lavoro;
	\item Esecuzione di quanto pianificato e continuo controllo della qualità;
	\item Revisione e valutazione di quanto prodotto;
	\item Consegna al proponente.
\end{itemize}
\subsubsection{Aspettative}
Il gruppo si aspetta di mantenere un dialogo costante con l'azienda proponente instaurando un rapporto collaborativo al fine di comprenderne al meglio le esigenze. In particolare ci si focalizza sui seguenti termini:
\begin{itemize}
	\item definire gli elementi fondamentali su cui focalizzarsi per soddisfare al meglio le necessità del proponente
	\item definire vincoli e requisiti sui processi; ??
	\item stimare le tempistiche per il lavoro;
	\item verifica continua con chiarimento di eventuali dubbi;
	\item accordarsi sulla qualifica del prodotto??
\end{itemize}
\subsubsection{Attività}
\paragraph{Studio di Fattibilità}
Lo Studio di Fattibilità viene redatto dagli analisti con lo scopo di stilare le motivazioni per cui il gruppo ha preferito un progetto rispetto agli altri.
Risulta quindi necessario individuare i costi e i benefici di ogni capitolato\glosp al fine di determinare quello che meglio si bilancia con le aspettative del gruppo tenendo sempre conto delle risorse a disposizione.
Per ogni capitolato\glosp andranno indicati:
\begin{itemize}
	\item \textbf{Informazioni Generali:} elenco che presenta le informazioni basilari: 
	\begin{itemize}
		\item Nome del progetto;
		\item Proponente;
		\item Committente.
	\end{itemize} 
	\item \textbf{Descrizione:} breve inquadramento dell'argomento del capitolato\glosp;
	\item \textbf{Obiettivi di Progetto:} descrizione dettagliata focalizzata sugli obiettivi da raggiungere;
	\item \textbf{Vincoli:} elenco dei vincoli imposti dal proponente per la realizzazione del progetto;
	\item \textbf{Tecnologie Interessate:} elenco e descrizione sintetica delle tecnologie che dovranno essere impiegate durante lo svolgimento del capitolato\glo;
	\item \textbf{Aspetti positivi:} elenco delle principali motivazioni che porterebbero il gruppo a scegliere il capitolato\glo;
	\item \textbf{Criticità e fattori di rischio:} elenco delle principali motivazioni che potrerebbero il gruppo a non scegliere il capitolato\glo;
	\item \textbf{Conclusioni:} sintesi delle motivazioni per cui il capitolato\glosp è stato scelto o è stato escluso.
\end{itemize}
\paragraph{Piano di Progetto}
Il Piano di Progetto viene redatto dal responsabile in collaborazione con gli amministratori e ha lo scopo di pianificare le attività per la realizzazione del progetto e individuare le risorse disponibili da assegnare a ciascuna di esse.
Questo documento è composto da:
\begin{itemize}
	\item \textbf{Analisi dei rischi:} analisi dei rischi che potrebbero sorgere nel ciclo di sviluppo del progetto con allegati probabilità e livelli di rischio e delle modalità previste per prevenirli o affrontarli nel caso non sia stato possibile evitarli;
	\item \textbf{Modello di sviluppo\glo:} descrizione del modello di sviluppo selezionato per lo svolgimento del progetto;
	\item \textbf{Pianificazione:} specifica dei tempi e dei ruoli per ogni attività individuata al fine di affrontare al meglio le scadenze del capitolato\glo. Essa viene limitata dalla quantità di risorse;
	\item \textbf{Preventivo e Consultivo:} stima dei costi necessari per lo svolgimento di ogni scadenza prevista e conseguente costruzione del preventivo per la totalità del progetto. Stesura di un verbale interno su cui riportare l'andamento effettivo rispetto a quanto previsto.
\end{itemize}
\paragraph{Piano di Qualifica}
Il Piano di Qualifica viene redatto dai verificatori e ha lo scopo di descrivere strategie e tecniche di verifica e validazione che i verificatori useranno nelle loro attività.
Questo documento presenterà:
\begin{itemize}
	\item \textbf{Qualità di Processo:} individuazione dei processi da adottare dagli standard di riferimento e delle metriche per misurarne e monitorarne la qualità;
	\item \textbf{Qualità di Prodotto:} individuazione delle caratteristiche del prodotto necessarie per una buona qualità e delle metriche per misurarne la qualità;
	\item \textbf{Specifiche dei test:} individuazione dei test a cui sottoporre il prodotto per garantire la soddisfazione dei suoi requisiti;
	\item \textbf{Standard di Qualità:} elenco degli standard di qualità selezionati;
	\item \textbf{Valutazioni per il miglioramento:} esposizione dei problemi rilevati e delle soluzioni da attuare;
	\item \textbf{Resoconto delle attività di verifica:} resoconti dei risultati dati dalle metriche di qualità effettuate per ogni attività.
\end{itemize}

\subsubsection{Strumenti??}
Durante il Processo di Fornitura sono utilizzati i seguenti strumenti della suite di Microsoft
\paragraph{Microsoft Excel}
Per svolgere semplici calcoli, creazione di grafici, diagrammi e tabelle si è scelto di utilizzare il software Microsoft Excel.
\paragraph{Microsoft Project}
Per realizzare il diagramma di Gantt\glosp necessario per la pianificazione, in particolare di tempi, risorse e analisi dei carichi di lavoro, si è scelto di utilizzare Microsoft Project.

//eventuali screen di Excel e/o Project

\subsection{Sviluppo}
	\subsubsection{Scopo}
Lo scopo dello sviluppo è quello di svolgere le attività necessarie per la realizzazione del prodotto richiesto.
\subsubsection{Aspettative}
Il gruppo si è posto come aspettativa di fissare gli obiettivi e i vincoli tecnologici e di progettazione per realizzare un prodotto completo e di buona qualità che sia ritenuto idoneo dal proponente.
\subsubsection{Descrizione}
Lo sviluppo di divide nelle seguenti attività:
\begin{itemize}
	\item Analisi dei Requisiti
	\item Progettazione
	\item Codifica
\end{itemize}
\subsubsection{Attività}
\paragraph{Analisi dei Requisiti} \mbox{}
\\L'Analisi dei Requisiti viene redatta dagli analisti.
\\ \textbf{Scopo}\\ 
Lo scopo di questo documento è quello di individuare ed analizzare i requisiti del progetto al fine di:
\begin{itemize}
	\item individuare lo scopo del lavoro;
	\item delineare dei riferimenti precisi da fornire ai progettisti che saranno da seguire per l'intero processo di sviluppo e ai verificatori per le attività di controllo della qualità;
	\item creare dei riferimenti per la stima dei costi.
\end{itemize}
\textbf{Descrizione}\\
L'analisi dei requisiti viene redatta eseguendo le seguenti attività:
\begin{itemize}
	\item \textbf{comprensione del capitolato\glo} attraverso la lettura e l'analisi della documentazione fornita;
	\item \textbf{comunicazioni con il proponente} al fine di migliorare l'analisi del progetto;
	\item \textbf{comunicazioni interne al gruppo}
	\item \textbf{analisi dei possibili casi d'uso}
\end{itemize}
\textbf{Casi d'uso}\\
Un caso d'uso definisce l'insieme di utenti, all'interno di uno scenario, con un obiettivo finale comune. Questo insieme è ottenuto attraverso la valutazione di ogni requisito.
Per descrivere ogni caso d'uso viene utilizzata la seguente struttura:
\begin{itemize}
	\item Codice Identificativo;
	\item Titolo;
	\item Diagramma UML;
	\item Attori primari;
	\item Attori secondari;
	\item Descrizione;
	\item Scenario;
	\item Scenario alternativo (ove presente);
	\item Inclusioni (ove presenti)
	\item Estensioni (ove presenti)
	\item Specializzazioni (ove presenti);
	\item Precondizione;
	\item Postcondizione;	
\end{itemize}
Il codice identificativo sarà scritto in questo formato: \\
\textbf{UC[codice\_padre].[codice\_figlio]} \\
Dove:
\begin{itemize}
	\item \textbf{codice\_padre:} numero che identifica univocamente i casi d'uso;
	\item \textbf{codice\_figlio:} numero progressivo che identifica i sottocasi;
\end{itemize}
\textbf{Requisiti}\\
Per descrivere un requisito viene utilizzata la seguente struttura:
\begin{itemize}
	\item Codice Identificativo;
	\item Classificazione;
	\item Descrizione;
	\item Fonti.
\end{itemize} 
Il \textbf{Codice Identificativo} sarà scritto in questo formato: \\
\textbf{R[Importanza][Tipologia][Codice]} \\
Dove:
\begin{itemize}
	\item \textbf{Importanza} può assumere i seguenti valori:
	\begin{itemize}
		\item 1: requisito obbligatorio;
		\item 2: requisito desiderabile;
		\item 3: requisito opzionale.
	\end{itemize}
	\item \textbf{Tipologia} può assumere i seguenti valori:
	\begin{itemize}
		\item F: Funzionale;
		\item Q: Prestazionale;
		\item P: Qualitativo;
		\item V: Vincolo.
	\end{itemize}
	\item\textbf{Codice}: numero progressivo identificativo strutturato nel formato: [codice\_padre].[codice\_figlio]
\end{itemize}
Le \textbf{Fonti} possono essere:
\begin{itemize}
	\item capitolato\glo: il requisito è stato quindi individuato dalla lettura del capitolato\glo;
	\item interno: il requisito è stato individuato ed aggiunto in seguito ad un'analisi interna;
	\item caso d'uso: il requisito è stato individuato dallo studio di un caso d'uso;
	\item proponente: il requisito è stato individuato in seguito ad un colloquio con il proponente.
\end{itemize}
\textbf{UML} \\
I diagrammi UML vengono realizzati usando la versione 2.0 del linguaggio.
\paragraph{Progettazione} \mbox{} \\
L'attività di progettazione è compito dei Progettisti.
\\ \textbf{Scopo}\\
Lo scopo della progettazione è di descrivere una soluzione soddisfacente per gli stakeholder. 
Viene quindi definita l'architettura del prodotto finale che dovrà:
\begin{itemize}
	\item Soddisfare i requisiti individuati nell'Analisi dei Requisiti;
	\item Essere comprensibile;
	\item Essere modulare;
	\item Garantire robustezza al fine di riuscire a gestire errori improvvisi.
\end{itemize}
\textbf{Descrizione}\\
La progettazione è composta da due sezioni:
\begin{itemize}
	\item Technology baseline;
	\item Product baseline.	
\end{itemize}
\textbf{Tecnology Baseline}\\
Nella Tecnology Baseline si trovano:
\begin{itemize}
	\item Descrizione delle \textbf{Tecnologie Utilizzate} con l'indicazione di vantaggi e svantaggi;
	\item Descrizione dei \textbf{Design pattern} utilizzati con corrispondente diagramma esplicativo;
	\item Elenco dei \textbf{Diagrammi UML}, in particolare:
	\begin{itemize}
		\item Diagrammi delle Classi;
		\item Diagrammi dei Package;
		\item Diagrammi di Attività;
		\item Diagrammi di Sequenza.
	\end{itemize}
	\item \textbf{Tracciamento delle Componenti:} riferimento per ogni requisito al componente atto a soddisfarlo;
	\item Descrizione dei \textbf{Test di Integrazione}, necessari per verificare che l'unione delle parti funzioni nel modo previsto.
\end{itemize}
\textbf{Product Baseline}\\
Nella Product Baseline si trovano:
\begin{itemize}
	\item \textbf{Descrizione delle Classi};
	\item \textbf{Tracciamento delle Classi:} riferimento per ogni requisito alla classe che lo soddisfa;
	\item \textbf{Test di Unità:} necessari per verificare che ogni parte funzioni individualmente nel modo previsto.
\end{itemize}
\paragraph{Codifica} \mbox{}
\\\textbf{Scopo}\\
Lo scopo di questa sezione è quello di normare tutte le regole di codifica al fine di garantire leggibilità e manutenibilità del codice. I Programmatori dovranno quindi seguire le seguenti regole.
\\\textbf{Stile di Codifica}\\
In seguito vengono elencate le norme da rispettare nel processo di codifica.
\begin{itemize}
	\item \textbf{Lingua:} la lingua con cui andranno scritti codice e commenti dovrà essere l'inglese;
	\item \textbf{Indentazione:} i blocchi annidati dovranno essere indentati con una tabulazione e si dovranno utilizzare per ogni livello 4 spazi;
	\item \textbf{Nomi:} i nomi di classi, metodi e variabili dovranno essere univoci ed esplicativi;
	\item \textbf{Classi:} il nome di ciascuna classe dovrà iniziare con la lettera maiuscola; 
	\item \textbf{Costanti:} il nome di ciascuna costante dovrà essere scritto usando solo lettere maiuscole;
	\item \textbf{Metodi:} il nome di ciascun metodo deve iniziare con una lettera minuscola ad eccezione dei nomi composti che saranno scritti con le iniziali di ogni singola parola che li compone maiuscola seguendo la pratica CamelCase;
	\item \textbf{Ricorsione:} l'uso della ricorsione è da evitare ove possibile;
	\item \textbf{Condizioni:} l'uso di condizioni multiple è da evitare ove possibile per limitare la complessità delle singole espressioni\glo.
\end{itemize}
\subsubsection{Strumenti}
\paragraph{TeXstudio}
Per la stesura dell'Analisi dei Requisiti viene utilizzato TeXstudio come ambiente di sviluppo.
\paragraph{SonarJS}
Per l'analisi statica del codice viene utilizzato il code analizer\glosp SonarJS su piattaforma di SonarCloud.
\paragraph{WebStorm}
Per la codifica viene utilizzato WebStorm, un ambiente di sviluppo integrato per JavaScript che offre piena compatibilità con Windows, Linux e SonarJS.
\paragraph{LucidChart}
Per la realizzazione dei diagrammi UML ??	