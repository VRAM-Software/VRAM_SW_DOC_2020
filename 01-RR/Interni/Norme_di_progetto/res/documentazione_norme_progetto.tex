\section{Processi di supporto}
    \subsection{Documentazione}
        \subsubsection{Descrizione}
            In questo capitolo verrà illustrato come il gruppo ha gestito il processo\glosp di redazione della documentazione utile ai fini del progetto\glo.
            Si andranno perciò a chiarire le fasi di costruzione del documento, la sua struttura, gli standard tipografici utilizzati e gli strumenti
            usati per la stesura.
        \subsubsection{Previsioni}
            Tale sezione si prefigge due obiettivi:
            \begin{itemize}
                \item standardizzare la scrittura della documentazione al fine di offrire degli elaborati consistenti e coerenti tra di loro;
                \item dare delle linee guida per la produzione di documenti corretti e validi.
            \end{itemize}
        \subsubsection{Obiettivi}
            Questo documento stabilisce delle norme per la produzione e la verifica dei prossimi elaborati. Tali indicazioni saranno condivise e seguite
            da tutti gli elementi del gruppo.
        \subsubsection{Ciclo di vita dei documenti}
            Ogni documento, durante il suo ciclo di vita, attraversa tre fasi principali:
            \begin{itemize}
                \item \textbf{Stesura}: in questo stadio viene effettuata la redazione del documento seguendo le norme indicate e i template disponibili.
                                        Il responsabile, a seguito di un'analisi preliminare, divide (se necessario) il lavoro tra vari redattori che procederanno
                                        alla compilazione.
                \item \textbf{Verifica}: terminata la fase di stesura i redattori sottometteranno l'elaborato al controllo dei verificatori. I verificatori, 
                                         nominati in precedenza dal responsabile, analizzeranno il documento. Nel caso in cui si evidenzino delle criticità
                                         segnaleranno le dovute modifiche ai redattori, in alternativa comunicheranno al responsabile la conformità del materiale.
                \item \textbf{Approvazione}: alla fine il documento viene vagliato dal responsabile che ne approva ufficialmente il rilascio.
            \end{itemize}
        \subsubsection{Struttura dei documenti}
            \paragraph{Template \LaTeX}\mbox{}\\ [1mm]
                Per la produzione di documenti consistenti e coerenti tra loro il gruppo ha deciso di creare un template \LaTeX che permette, al momento della
                stesura, di concentrarsi unicamente sul contenuto e di non impegnare troppo tempo nella strutturazione della pagina. Il template è formato da
                più parti, tra cui:
                \begin{itemize}
                    \item un file per la corretta formattazione della prima pagina;
                    \item un file contenente i pacchetti \LaTeX utilizzati per una migliore rappresentazione dei contenuti;
                    \item un file per modificare i comandi personalizzati, i quali garantiscono una maggiore modularità del template;
                    \item due file per la gestione delle tabelle.
                \end{itemize}
            \paragraph{Prima pagina}\mbox{}\\ [1mm]
                Il frontespizio è strutturato in questo modo:
                \begin{itemize}
                    \item \textbf{Logo del gruppo}: corredato dal nome, il tutto centrato orizzontalmente;
                    \item \textbf{Capitolato}\glo: nome del progetto\glosp scelto dal gruppo (\textit{Predire in Grafana}), centrato orizzontalmente;
                    \item \textbf{Titolo del documento}: nome esplicativo del contenuto del documento, centrato orizzontalmente e in grassetto;
                    \item \textbf{Data}: data di approvazione finale dell'elaborato, centrata orizzontalmente;
                    \item \textbf{Tabella informativa}: contenente informazioni quali:
                    \begin{itemize}
                        \item versione del documento;
                        \item nominativo dell'approvatore (e quindi responsabile) del documento;
                        \item nominativo/i del/i redattore/i del documento;
                        \item nominativo/i del/i verificatore/i del documento;
                        \item stato del documento (approvato o non approvato);
                        \item tipo d'uso del documento (interno o esterno);
                        \item nominativi dei destinatari del documento;
                        \item email di riferimento del gruppo.
                    \end{itemize} 
                    \item \textbf{Descrizione}: breve riassunto del contenuto del documento.
                \end{itemize}
            \paragraph{Registro delle modifiche}\mbox{}\\ [1mm]
                La pagina successiva mostra una tabella contenente una raccolta cronologica delle modifiche avvenute sul documento.
                Le colonne espongono i seguenti contenuti:
                \begin{itemize}
                    \item versione del documento dopo la modifica;
                    \item data della modifica;
                    \item nominativo del componente responsabile della modifica;
                    \item ruolo del componente responsabile della modifica;
                    \item breve descrizione della modifica effettuata;
                \end{itemize}
            \paragraph{Indice}\mbox{}\\ [1mm]
                L'indice riassume in modo schematico i contenuti del documento mostrando la loro suddivisione in sezioni, sottosezioni e paragrafi.
                È presente all'inizio di ogni elaborato dopo il registro delle modifiche. I suoi elementi sono cliccabili e rimandano alla pagina
                corrispondente. In alcuni documenti viene corredato di due liste:
                \begin{itemize}
                    \item \textbf{Lista delle immagini}: riportante le immagini utilizzate tramite rispettiva didascalia;
                    \item \textbf{Lista delle tabelle}: riportante le tabelle utilizzate tramite rispettiva didascalia (sono esclusi il registro delle modifiche
                                                        e il riepilogo tracciamenti)
                \end{itemize}
            \paragraph{Contenuto}\mbox{}\\ [1mm]
                Il corpo principale del documento è così strutturato:
                \begin{itemize}
                    \item intestazione in alto con logo del gruppo riadattato a sinistra e nome del documento a destra;
                    \item una riga divisoria;
                    \item il contenuto della pagina formattato con una dimensione di 12 punti e la visualizzazione in A4 formato articolo
                          (\verb|\documentclass{article}|);
                    \item una riga divisoria opzionale in caso di appunti;
                    \item eventuali note sul testo riportate con un numero progressivo a esponente;
                    \item una riga divisoria;
                    \item il contenuto a piè di pagina con a destra il numero progressivo della pagina visualizzata sul totale (nell'indice invece dei numeri
                          arabi vengono utilizzati quelli romani).
                \end{itemize}
        \subsubsection{Norme tipografiche}
                \paragraph{Nomi dei file}\mbox{}\\ [1mm]
                    Per organizzare i file in modo più ordinato e corretto il gruppo ha scelto di seguire la pratica dello "snake\_case"\glo.
                    Tale prassi, nell'ambito di questo progetto\glo, è applicata seguendo tre regole:
                    \begin{itemize}
                        \item i file con nome composto da più parole useranno il carattere underscore come separatore;
                        \item i nomi dei file saranno scritti interamente in minuscolo;
                        \item non vengono tralasciate preposizioni di alcun tipo dal nome del file.
                    \end{itemize}
                \paragraph{Glossario}
                    \begin{itemize}
                        \item Ogni parola da inserire nel \textit{Glossario} è corredata da una "G" a pedice per tutte le sue occorrenze;
                        \item sono escluse dalla regola precedente i titoli e il nome delle sezioni del documento.
                    \end{itemize}
                \paragraph{Stili di testo}
                \begin{itemize}
                    \item \textbf{Grassetto}: il grassetto viene usato solamente per evidenziare le parole a cui si vuole dare risalto e per la prima parola
                                              di un elenco puntato se quest'ultimo è del tipo: \textbf{Termine}: definizione/spiegazione.
                    \item \textbf{Corsivo}: il corsivo viene usato per evidenziare i nomi dei documenti oltre che per il nome dell'azienda
                                            proponente (\textit{Zucchetti}) e del gruppo (\textit{VRAM Software}).
                    \item \textbf{Maiuscolo}: in aggiunta alle normali regole dell'italiano le lettere maiuscole verranno utilizzate anche per riportare i nomi dei
                                              documenti e di certi capitoli.
                \end{itemize}
                \paragraph{Elenchi puntati}\mbox{}\\ [1mm]
                    Ogni voce dell'elenco puntato sarà preceduta da un "•" (punto elenco) per quanto riguarda il primo livello e da un "-" per il secondo.
                    Alla fine di ogni elemento, invece, si aggiunge il carattere ";" a meno che non sia l'ultimo dove va messo il segno di punteggiatura ".".
                    Se la voce dell'elenco è del tipo termine-spiegazione/definizione il termine va in grassetto.
                \paragraph{Convenzioni}\mbox{}\\ [1mm]
                    Vengono usate tre convenzioni per la data, l'orario (i quali prendono come riferimento ISO 8601) e la valuta. Le date saranno espresse nella forma: \newline \newline
                    \centerline{\textbf{YYYY-MM-DD}} \newline \newline
                    dove:
                    \begin{itemize}
                        \item \textbf{YYYY}: indica l'anno;
                        \item \textbf{MM}: indica il mese e deve sempre avere due cifre;
                        \item \textbf{DD}: indica il giorno e deve sempre avere due cifre.
                    \end{itemize}
                    Gli orari saranno invece rappresentati nella forma: \newline \newline
                    \centerline{\textbf{HH:MM}} \newline \newline
                    dove:
                    \begin{itemize}
                        \item \textbf{HH}: indica le ore (in formato 24 ore) e deve sempre avere due cifre;
                        \item \textbf{MM}: indica i minuti e deve sempre avere due cifre.
                    \end{itemize}
                    La valuta infine sarà scritta nella forma: \newline \newline
                    \centerline{\textbf{X.XXX,YY}} \newline \newline
                    dove:
                    \begin{itemize}
                        \item \textbf{XXXX}: rappresenta la parte intera della somma;
                        \item \textbf{.}: viene usato come separatore ogni tre cifre intere;
                        \item \textbf{YY}: rappresenta la parte decimale della somma;
                    \end{itemize}
            \subsubsection{Produzione}
                    Il progetto\glosp prevede la stesura di documenti che si dividono in interni ed esterni a seconda dei destinatari. Questi elaborati verranno poi valutati
                    in quattro revisioni. Tali elementi progettuali saranno elencati e spiegati di seguito con le rispettive sigle.
                \paragraph{Documenti esterni}
                    \begin{itemize}
                        \item \textbf{Analisi dei Requisiti - AdR}: contiene la descrizione degli attori\glosp del sistema e la loro interazione con i casi d'uso\glo,
                                                                    fornendo una visione chiara ai progettisti sul problema da trattare;
                        \item \textbf{Manuale Utente - MU}: serve per guidare l'utente nell'utilizzo del software;
                        \item \textbf{Manuale Sviluppatore - MS}: ha lo scopo di spiegare a eventuali sviluppatori esterni il funzionamento
                                                                  del sistema;
                        \item \textbf{Piano di Progetto - PdP}: dà indicazioni sulle tempistiche e i costi che il gruppo prevede d'impiegare
                                                                per la realizzazione del progetto\glo;
                        \item \textbf{Piano di Qualifica - PdQ}: fornisce le specifiche riguardanti il controllo qualità dei processi\glosp e dei
                                                                 prodotti\glosp tramite delle metriche\glosp misurabili.
                    \end{itemize}
                \paragraph{Documenti interni}
                    \begin{itemize}
                        \item \textbf{Glossario - G}: documento dove sono riportate i termini che necessitano di spiegazione o che potrebbero causare confusione o
                                                      fraintendimenti;
                        \item \textbf{Norme di Progetto - NdP}: linee guida per la gestione delle attività di progetto\glo;
                        \item \textbf{Studio di Fattibilità - SdF}: breve analisi di tutti i capitolati\glosp proposti.
                    \end{itemize}
                \paragraph{Verbale}\mbox{}\\ [1mm]
                    Il verbale della riunione è un documento particolare in quanto può essere sia interno che esterno. Sarà interno quando
                    i partecipanti saranno solo i componenti del gruppo; esterno quando saranno presenti anche referenti dell'azienda.
                    È inoltre presente la tabella "Riepilogo dei tracciamenti" che tiene segno delle modifiche discusse e
                    approvate in sede d'incontro. Questo prospetto presenta una colonna con un codice nel formato: \newline \newline
                    \centerline{\textbf{V(I/E)\_X.Y}} \newline \newline
                    dove:
                    \begin{itemize}
                        \item \textbf{V(I/E)}: indica se il verbale è interno o esterno;
                        \item \textbf{X}: indica il numero dell'incontro che si sta organizzando;
                        \item \textbf{Y}: indica il numero della scelta.
                    \end{itemize}
                \paragraph{Revisioni}
                \begin{itemize}
                    \item \textbf{Revisione dei Requisiti - RR}: analisi iniziale del capitolato per concordare i requisiti col proponente;
                    \item \textbf{Revisione di Progettazione - RP}: revisione intermedia per accertare la fattibilità del progetto\glo;
                    \item \textbf{Revisione di Qualifica - RQ}: revisione intermedia per l'approvazione delle verifiche e l'attivazione del processo\glosp di validazione\glo;
                    \item \textbf{Revisione di Accettazione - RA}: revisione finale per il collaudo del software e l'accertamento del soddisfacimento dei requisiti presentati in RR.
                \end{itemize}
                \paragraph{Altre sigle}
                    \begin{itemize}
                        \item \textbf{Responsabile - Re};
                        \item \textbf{Amministratore - Am};
                        \item \textbf{Analista - An};
                        \item \textbf{Progettista - Pr};
                        \item \textbf{Programmatore - Pt};
                        \item \textbf{Verificatore - Ve};
                        \item \textbf{Proof of Concept - PoC\glo}.
                    \end{itemize}
            \subsubsection{Elementi grafici}
                    \paragraph{Immagini}\mbox{}\\ [1mm]
                        Le immagini sono sempre centrate e presentano una didascalia, tranne il logo nella prima pagina.
                    \paragraph{Tabelle}\mbox{}\\ [1mm]
                        Le tabelle sono sempre centrate e presentano una didascalia, tranne il "Registro delle modifiche" e il "Riepilogo tracciamenti".
                        Inoltre le tabelle presentano un'alternanza di colori per favorirne la lettura.
                    \paragraph{Didascalie}\mbox{}\\ [1mm]
                        Le didascalie vengono indicate con un numero progressivo del tipo X.Y, dove:
                        \begin{itemize}
                            \item \textbf{X}: indica la sezione a cui appartiene l'elemento;
                            \item \textbf{Y}: indica che tabella o immagine è della sezione.
                        \end{itemize}
            \subsubsection{Strumenti}
                    \paragraph{\LaTeX}\mbox{}\\ [1mm]
                        Per stilare i documenti è stato usato il linguaggio di markup \LaTeX. Questo linguaggio permette di scrivere dei testi mantenendo il focus
                        sul contenuto e non sulla forma. La struttura infatti è stata dichiarata all'inizio del progetto\glosp e poi riutilizzata permettendo una gestione degli elaborati scalabile, modulare e ordinata.
                    \paragraph{TeXStudio}\mbox{}\\ [1mm]
                        Per scrivere in \LaTeX viene usata l'IDE \textit{TeXStudio} creata appositamente per questo fine e che quindi offre delle scorciatoie
                        per i vari comandi, un correttore automatico per la lingua italiana e un servizio di compilazione e visualizzazione dei PDF integrato;
                    \paragraph{Lucidchart}\mbox{}\\ [1mm]
                        Per la creazione dei diagrammi UML\glosp e di altro tipo viene usata la piattaforma online Lucidchart che consente agli utenti di collaborare alla stesura, alla revisione e alla condivisione di grafici. \newline \newline
                        \centerline{\url{https://www.lucidchart.com}}