\subsection{Verifica}
	\subsubsection{Scopo}
		La verifica ha lo scopo di dare prove e sicurezze che i risultati ottenuti da tutte le attività attuate in un periodo preso in considerazione raggiungano tutti i requisiti richiesti senza aver causato l'introduzione di errori. La verifica deve essere svolta per tutto il ciclo di vita del software, più precisamente al passare di ogni milestone\glo.
	\subsubsection{Aspettative}
		La verifica viene effettuata per raggiungere i seguenti obbiettivi:
		\begin{itemize}
			\item Cercare consistenza, completezza e correttezza nelle singole attività.
			\item Ottenere supporto per una successiva validazione del prodotto.
		\end{itemize}
	\subsubsection{Descrizione}
		Il processo di verifica prende in esame una attività e, attraverso processi definiti successivamente, cerca di ottenere le aspettative per avere la conferma sui risultati che vengono ottenuti.
	\subsubsection{Attività}
		La verifica è realizzata principalmente attraverso l'analisi ,che ha lo scopo di accertare la correttezza e l'assenza di errori, essa deve essere svolta in modo critico per non tralasciare nulla.
	\subsubsection{Analisi}
		 Le attività che si effettuano durante la verifica sono due tipi di analisi, complementari tra loro. Queste attività sono: l'\textbf{Analisi statica} e l'\textbf{Analisi dinamica}.
		\paragraph{Analisi statica}
			L'analisi statica è lo studio della documentazione e del codice sorgente, essa non richiede l'esecuzione del prodotto che si sta sviluppando quindi viene applicata ad ogni prodotto di processo. L'analisi statica può essere effettuata manualmente in due modi:
			\begin{itemize}
				\item \textbf{Walkthrough} \\*
					Tutti i componenti del gruppo analizzano tutti i documenti e codice realizzato, senza tralasciare nulla, in modo da rilevare errori. Questa ricerca è ad ampio spettro infatti viene effettuata anche se non vengono trovate anomalie;
				\item \textbf{Inspection} \\*
					L'inspection è un metodo di lettura o desk check per cercare errori specifici utilizzando liste di controllo o check list.
			\end{itemize}
			% bisogna inserire una tabella con le liste di controllo?
			% 8lab l'ha fatto ma mi sembra di copiare e basta sinceramente	
			Il nostro gruppo, per l'analisi statica del codice ha scelto di utilizzare SonarJS\glo, un analizzatore statico di codice JavaScript. 
		\paragraph{Analisi dinamica}
			L'analisi dinamica viene effettuata mediante test sul codice, quindi richiede l'esecuzione del prodotto SW. Questi test verificano e certificano il corretto funzionamento del prodotto. 
			\subparagraph{Test}
				Per la corretta verifica del codice, i test devono seguire regole ben precise che definiscono i parametri da definire e le proprietà che devono avere test per essere considerati efficaci:
				\textbf{Parametri:} \\*
				\begin{itemize}
					\item \textbf{}
					\item \textbf{}
					\item \textbf{}
					\item \textbf{}
					\item \textbf{}
				\end{itemize}
				
				\textbf{Proprietà}
				\begin{itemize}
					\item
					\item
					\item
					\item
					\item
				\end{itemize}
			
			
			
			
			
			
			
			
			
			
			
			
			
			
			
			
			
			
			
			
			