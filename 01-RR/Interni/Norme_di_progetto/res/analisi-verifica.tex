\subsection{Verifica}
	\subsubsection{Scopo}
		La verifica ha lo scopo di dare prove e sicurezze che i risultati ottenuti da tutte le attività attuate in un periodo preso in considerazione raggiungano tutti i requisiti richiesti senza aver causato l'introduzione di errori. La verifica deve essere svolta per tutto il ciclo di vita del software, più precisamente al passare di ogni milestone\glo.
	\subsubsection{Aspettative}
		La verifica viene effettuata per raggiungere i seguenti obbiettivi:
		\begin{itemize}
			\item Cercare consistenza, completezza e correttezza nelle singole attività.
			\item Ottenere supporto per una successiva validazione del prodotto.
		\end{itemize}
	\subsubsection{Descrizione}
		Il processo di verifica prende in esame una attività e, attraverso processi definiti successivamente, cerca di ottenere le aspettative per avere la conferma sui risultati che vengono ottenuti.
	\subsubsection{Attività}
		La verifica è realizzata principalmente attraverso l'analisi ,che ha lo scopo di accertare la correttezza e l'assenza di errori, essa deve essere svolta in modo critico per non tralasciare nulla.
	\subsubsection{Analisi}
		 Le attività che si effettuano durante la verifica sono due tipi di analisi, complementari tra loro. Queste attività sono: l'\textbf{Analisi statica} e l'\textbf{Analisi dinamica}.
		\paragraph{Analisi statica}
			L'analisi statica è lo studio della documentazione e del codice sorgente, essa non richiede l'esecuzione del prodotto che si sta sviluppando quindi viene applicata ad ogni prodotto di processo. L'analisi statica può essere effettuata manualmente in due modi:
			\begin{itemize}
				\item \textbf{Walkthrough} \\*
					Tutti i componenti del gruppo analizzano tutti i documenti e codice realizzato, senza tralasciare nulla, in modo da rilevare errori. Questa ricerca è ad ampio spettro infatti viene effettuata anche se non vengono trovate anomalie;
				\item \textbf{Inspection} \\*
					L'inspection è un metodo di lettura o desk check per cercare errori specifici utilizzando liste di controllo o check list.
			\end{itemize}
			% bisogna inserire una tabella con le liste di controllo?
			% 8lab l'ha fatto ma mi sembra di copiare e basta sinceramente	
			Il nostro gruppo, per l'analisi statica del codice ha scelto di utilizzare SonarJS\glo, un analizzatore statico di codice JavaScript. 
		\paragraph{Analisi dinamica}
			L'analisi dinamica viene effettuata mediante test sul codice, quindi richiede l'esecuzione del prodotto software. Questi test verificano e certificano il corretto funzionamento del prodotto, producendo una misura della qualità del prodotto. 
			\subparagraph{Test}
				Per la corretta verifica del codice, i test devono seguire regole ben precise che definiscono i parametri da definire e le proprietà che devono avere test per essere considerati efficaci:
				\textbf{Parametri da definire in un test:} \\*
				\begin{itemize}
					\item L'input preso in considerazione
					\item L'output che si aspetta di ricevere
					\item L'ambiente, quindi hardware e sistema operativo, in cui viene eseguito il test
					\item Lo stato iniziale del prodotto, precedente all'esecuzione del test
					\item In caso di utilizzo di istruzioni opzionali aggiuntive esse devono essere note
				\end{itemize} \\*
				\textbf{Un test può essere definito efficace se:}
				\begin{itemize}
					\item I parametri citati sopra sono correttamente indicati
					\item Il nome di un test deve essere autoesplicativo per capire subito che cosa si sta testando
					\item Esso è ripetibile, quindi ci si aspetta un risultato costante per tutte le esecuzioni di quel test
					\item In caso di fallimento del test, esso devo fornire informazioni utili a risolvere il problema identificato
					\item In caso di possibili effetti indesiderati ci si aspetta un avvertimento
					\item Esso è esaustivo e accurato, infatti devono poter identificare una parte del progetto che potrebbe essere causa di errori
				\end{itemize}
				Il cosiddetto software testing è un procedimento importante per misurare la qualità di un prodotto software, per questo motivo sono definiti cinque livelli che dividono i tipi di test da effettuare secondo una gerarchia. Essi sono: 
				\begin{itemize}
					\item \textbf{Test d'unità}
					\item \textbf{Test d'integrazione}
					\item \textbf{Test di sistema}
					\item \textbf{Test di regressione}
					\item \textbf{Test di accettazione}
				\end{itemize} \\*
				
				\textbf{Test d'unità} \\*
				I test d'unità si definiscono tali perché controllano il corretto funzionamento di una sezione di codice specifica. Usualmente questi tipi di test vengono eseguiti su funzioni o, se si parla di programmazione ad oggetti, su un metodo di una classe, quindi su una parte minimale del codice. È buona norma rendere questi test automatici in modo da minimizzare il tempo di esecuzione e non devono richiedere l'interazione dello sviluppatore. Questo tipo di test viene scritto dal programmatore che sviluppa le singole unità, per verificare l'assenza di errori e documentare il funzionamento dell'unità. \\*
				
				
				\textbf{Test d'integrazione} \\*
				Eseguire test d'integrazione è il passo successivo ai test d'unità, infatti in questo livello del software testing si cerca di trovare anomalie ed errori tra le interfacce dei singoli componenti, previa conferma del corretto funzionamento delle singole unità. Usualmente questi tipi di test vengono effettuati testando componenti in modo iterativo per evitare di individuare più facilmente e velocemente i problemi. \\*
				
				\textbf{Test di sistema} \\*
				I test di sistema controllano il funzionamento corretto del prodotto e verificano che il sistema soddisfi tutti i requisiti definiti nel documento: \textit{Analisi dei Requisiti}. Con l'esecuzione di questi test si riesce a definire e documentare tutte le funzionalità del sistema e possono individuare criticità come il problema del black-box\glo.
				\\*
				
				\textbf{Test di regressione} \\*
				I test di regressione sono dei particolari tipi di test che vanno a controllare l'integrazione tra il sistema e correzioni o estensioni del codice aggiunte successivamente. Questi test vengono eseguiti singolarmente per ogni aggiunta al sistema; per ogni estensione vengono eseguiti nel seguente ordine: il test d'unità, il test d'integrazione con il sistema e infine il test di sistema per confermare il corretto funzionamento del prodotto. \\*
				
				\textbf{Test di accettazione} \\*
				I test di accettazione, o di collaudo, vengono effettuati ad ogni pre-rilascio, quindi dopo l'esecuzione dei test di sistema, essi servono a confermare il soddisfacimento dei requisiti da parte del committente infatti questi tipi di test ne richiedono la presenza. Se questi test vengono superati significa che il prodotto software è pronto per il rilascio. \\*