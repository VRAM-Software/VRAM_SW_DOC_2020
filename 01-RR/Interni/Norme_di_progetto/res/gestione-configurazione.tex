\subsection{Gestione della configurazione}
	\subsubsection{Descrizione}
		In questo capitolo verrà illustrato come il gruppo ha gestito la configurazione degli strumenti e delle risorse utilizzate per svolgere il progetto.
		Sarà quindi descritta la configurazione del repository\glosp su Github, del sistema di versionamento\glosp Git e dei servizi Github.
	\subsubsection{Previsioni}
		Questa sezione ha lo scopo supportare i processi di documentazione e di sviluppo e manutenzione del software rendendoli definiti e ripetibili.  
	\subsubsection{Obiettivi} 
		L'obiettivo principale è quindi di rendere documenti e codice sorgente univocamente identificati e facilmente identificabili, evidenziandone versioni e modifiche. Vuole anche agevolare l'identificazione delle relazioni esistenti fra gli elementi e fa da supporto alla fase di verifica.
	\subsubsection{Versionamento}
		Le modifiche effettuate ai documenti ed ai file contenenti codice sorgente sono identificate da un numero di versione presente all'interno dei file stessi. Per quanto riguarda i documenti, ogni numero di versione sarà presente nel nome del file ed avrà una corrispondente riga nella tabella delle modifiche così da avere uno storico delle modifiche effettuate al documento. Per il codice sorgente del software le modifiche effettuate in una versione saranno consultabili sulla sezione rilasci di Github e saranno identificate da opportuni tag.
		\newline
		Per identificare univocamente le versioni dei documenti seguiamo le indicazioni dell'ente \textit{ETSI}, dunque la numerazione delle versioni è composta da 3 cifre nel formato X.Y.Z e la prima bozza ha versione 0.0.1. Le modifiche al documento aggiorneranno le cifre nel seguente modo:
		\begin{itemize}
			\item \textbf{Z} : Questa cifra viene incrementata ogni volta che vengono effettuate delle modifiche editoriali\glosp al documento, es. X.Y.1, X.Y.2 etc.;
			\item \textbf{Y} : Questa cifra viene incrementata ogni volta che vengono effettuate delle modifiche tecniche\glosp al documento. Se sono state eseguite modifiche sia editoriali che tecniche allora entrambe le cifre Z e Y saranno incrementate, es. X.1.1, X.2.2 etc;
			\item \textbf{X} : Questa cifra viene incrementata ogni volta che viene rilasciata una nuova versione finale del documento, ovvero una versione di rilascio.			
		\end{itemize}
		La prima versione finale ha versione 1.1.1.
		\newline
		Durante la modifica e la revisione del documento le bozze vengono incrementate le cifre Z ed Y, es. 1.2.0, 1.2.1, 1.30, etc.
		\newline
		Quando il documento viene approvato come finale allora viene incrementata la cifra X, ad esempio la bozza 1.5.3 diventa la versione finale 2.1.1.
		\newline
		Ogni incremento sulle singole cifre è rigorosamente un +1, non possono essere saltati numeri.
		
	\subsubsection{Gestione delle modifiche}
		Al fine di monitorare e limitare le modifiche al ramo principale del repository\glo, \textit{master}, è utilizzato il meccanismo di pull request fornito da Github. Ogni membro del gruppo può creare branch secondari su cui effettuare modifiche, tuttavia per unirle al branch \textit{master} è necessario aprire una pull request che dovrà essere revisionata dai verificatori tramite i servizi di revisione integrati in Github. Una volta revisionata positivamente è compito del responsabile del documento approvare la pull request ed effettuare quindi l'effettiva unione delle modifiche nel branch master.
		\newline
		In sintesi, per effettuare modifiche ai documenti sono previsti i seguenti passaggi:
		\begin{itemize}
			\item Contattare il responsabile del documento affinché autorizzi la modifica del documento;
			\item Creare un branch secondario ed effettuare le modifiche al documento;
			\item Aprire una pull request per unire le modifiche al ramo \textit{master};
			\item I verificatori revisionano la pull request ed eventualmente richiedono aggiornamenti;
			\item Completata la revisione il responsabile approva la pull request.
		\end{itemize}
		
	\subsubsection{Repository\glosp}
		Per tenere traccia di versioni e modifiche fatte a documenti e codice è utilizzato il sistema di versionamento\glosp distribuito Git, che può essere utilizzato tramite riga di comando o utilità grafiche come Github Desktop o GitKraken.
		Il repository\glosp dei documenti è ospitato sul sito Github all'indirizzo: 
		\begin{center}
			\url{https://github.com/marcoDallas/VRAM_SW_DOC_2020/}
		\end{center}
		Al fine di fornire un'agevole e standardizzata navigazione, il contenuto del repository\glosp è organizzato in modo gerarchico tramite directory secondo il seguente schema:
		\newline
		\dirtree{%
			.1 root.
				.2 RR.
					.3 Esterni.
					.3 Interni.
				.2 RP.
					.3 Esterni.
					.3 Interni.
				.2 RQ.
					.3 Esterni.
					.3 Interni.
				.2 RA.
					.3 Esterni.
					.3 Interni.
				.2 Guide.
				.2 Template.
					.3 Images.
					.3 config.
					.3 img.
					.3 res.
		}
		\mbox{}\\ % forza un newline
		Nel dettaglio:
		\begin{itemize}
			\item \textbf{RR} : contiene i sorgenti \LaTeX dei documenti relativi alla revisione dei requisiti;
			\item \textbf{RP} : contiene i sorgenti \LaTeX dei documenti relativi alla revisione di progettazione;
			\item \textbf{RQ} : contiene i sorgenti \LaTeX dei documenti relativi alla revisione di qualifica;
			\item \textbf{RA} : contiene i sorgenti \LaTeX dei documenti relativi alla revisione di accettazione; 
			\item \textbf{Guide} : contiene brevi indicazioni interne su come usare i comandi \LaTeX usati nei documenti e su come riutilizzare il template per generare nuovi documenti;
			\item \textbf{Template} : contiene i sorgenti \LaTeX usati per generare la base comune di tutti i documenti.
		\end{itemize}
		
	\subsubsection{Tipologie di file non accettate}
		Tramite un apposito file \textit{.gitignore} presente nella root directory della gerarchia vengono definiti i tipi di file non accettati all'interno del repository\glo. Vengono esclusi tutti i file di compilazione, compilati o temporanei in quanto il repository\glosp dovrebbe contenere solamente i seguenti formati di file:
		\begin{itemize}
			\item File sorgente \LaTeX con estensione .tex;
			\item File immagine, preferibilmente in formato .png;
			\item File testuali in formato .md o .txt.			
		\end{itemize}
	
	\subsubsection{Ulteriori Strumenti Github utilizzati}
		In aggiunta ai servizi già elencati, al fine di migliorare efficacia ed efficienza, vengono utilizzate le funzionalità di \textit{Issue Tracking System},
		% milestone qui è usato come nome di funzione github quindi non va a glossario  
		\textit{Milestone} e \textit{Project Board} integrate in Github. Ognuna di queste funzionalità viene usata solo da chi autorizzato, ad esempio rilasci di versioni e chiusure di milestone\glosp sono concesse solo al responsabile di progetto.
		\newline
		Per il repository\glosp contenente il codice sorgente del software saranno usate anche le Github Actions al fine di implementare la pratica della continuous integration.
		