\subsection{Gestione della configurazione}
	\subsubsection{Descrizione}
		In questo capitolo verrà illustrato come il gruppo ha gestito la configurazione degli strumenti e delle risorse utilizzate per svolgere il progetto.
		Sarà quindi descritta la configurazione del repository/glosp su Github, del sistema di versionamento Git e dei servizi Github.
	\subsubsection{Previsioni}
		Questa sezione ha lo scopo supportare i processi di documentazione e di sviluppo e manutenzione del software rendendoli definiti e ripetibili.  
	\subsubsection{Obiettivi} 
		L'obiettivo principale è quindi di rendere documenti e codice sorgente univocamente identificati e facilmente identificabili, evidenziandone versioni e modifiche. Vuole anche agevolare l'identificazione delle relazioni esistenti fra gli elementi e fa da supporto alla fase di verifica.
	\subsubsection{Versionamento}
		Le modifiche effettuate ai documenti ed ai file contenenti codice sorgente sono identificate da un numero di versione presente all'interno dei file stessi. Per quanto riguarda i documenti, ogni numero di versione avrà una corrispondente riga nella tabella delle modifiche così da avere uno storico delle modifiche effettuate al documento. Per il codice sorgente le modifiche effettuate in una versione saranno consultabili sul repository Git e saranno identificate da opportuni tag.
		
		* formato scelto per le versioni dei documenti
		
		* formato scelto per le versioni del software
		
	\subsubsection{Relazioni fra le parti del sistema}
		* Assemblaggio delle varie componenti software / documenti in un elemento unico
		
	\subsubsection{Gestione delle modifiche}
		* meccanismi di pull request previsti da Github
	
	\subsubsection{Gestione dei rilasci}
		* gestione del rilascio dei pdf su drive - dei compilati del codice come pacchetti su github
	
	\subsubsection{Repository}
		Per tenere traccia di versioni e modifiche fatte a documenti e codice è utilizzato il sistema di versionamento distribuito Git, che può essere utilizzato tramite riga di comando o utilità come Github Desktop o GitKraken.
		Il repository dei documenti è ospitato sul sito Github all'indirizzo: \url{https://github.com/marcoDallas/VRAM_SW_DOC_2020/}

		* descrizione della struttura in directory del repo
		
		* gitignore
		
		* che sezione?
		* Issue
		* Project Board
		* Milestone
		