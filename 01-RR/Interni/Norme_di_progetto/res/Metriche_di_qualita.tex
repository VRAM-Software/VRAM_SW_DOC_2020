\subsection{Metriche di qualità}
Misureremo la qualità tramite le seguenti metriche\glo, così da non basarci su condizioni empiriche ma bensì concrete.

\subsubsection{Documentazione}
Un documento dovrebbe essere efficace nel comunicare il suo contenuto ed efficiente nel farlo, cioè non essere prolisso.
Questi metriche\glosp servono per misurare la sintassi e la semantica.
\begin{itemize}
	\item Numero di errori ortografici;
	\item Index fog di Guinning (indica il numero di anni di istruzione necessari per comprendere il testo in esame).
\end{itemize} 

\subsubsection{Pianificazione}
La misurazione della performance della pianificazione sarà utile per migliorare l'efficienza futura.
\begin{itemize}
	\item (Tempo di lavoro reale - Tempo di lavoro previsto)/Tempo di lavoro previsto;
	\item Numero di scadenze in cui il lavoro è stato terminato puntualmente.
\end{itemize}

\subsubsection{Verifica}
La misurazione effettuate nel processo\glosp di verifica sono utili per determinare lo stato dell'efficacia del prodotto\glosp
\begin{itemize}
	\item Numero di requisiti di soddisfatti/Numero di requisiti totali;
	\item Numero di test che hanno avuto successo/Numero di test totali.
\end{itemize}
