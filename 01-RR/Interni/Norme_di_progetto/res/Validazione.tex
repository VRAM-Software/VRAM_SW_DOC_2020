\subsection{Validazione}

\subsubsection{Scopo}
Lo scopo del processo\glosp di validazione è accertare che il prodotto\glo, per uno specifico uso, corrisponda alle attese.

\subsubsection{Descrizione}
Le attese sono quelle del committente e corrispondono, se stilata correttamente, all'analisi dei requisiti. 
Per eseguire una validazione si deve avere un prodotto\glosp che è a un sufficiente livello di avanzamento, quindi il numero di validazioni svolte è molto minore rispetto al numero delle verifiche, che iniziano a venire eseguite prima.

\subsubsection{Attività}
Ci sono due modi di eseguire una validazione: internamente o esternamente. Una validazione interna ha successo se i requisiti in precedenza scritti sono soddisfatti, una esterna invece se il committente dà parere positivo. Più nel particolare la validazione consiste nel:

\begin{itemize}
	\item Selezionare i requisiti da soddisfare;
	\item Individuare dei test specifici per misurare il soddisfacimento dei requisiti selezionati;
	\item Eseguire i test;
	\item Analizzare i risultati dei test.
\end{itemize}

