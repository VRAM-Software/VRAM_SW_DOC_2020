\subsection{Gestione della qualità}
\subsubsection{Scopo}
Lo scopo della gestione della qualità è quello di dare una valutazione oggettiva riguardo ai servizi che il prodotto deve fornire in modo mantenere uno standard di qualità prestabilito e da soddisfare i requisiti impliciti ed espliciti del proponente.
\subsubsection{Aspettative}
Le aspettative della gestione della qualità possono essere riassunti dai seguenti punti, che rappresentano gli obbiettivi che vogliono essere raggiunti dal gruppo. Si vuole ottenere:
\begin{itemize}
	\item qualità per tutta la durata del ciclo di vita del software;
	\item una copertura totale per tutti i requisiti impliciti ed espliciti richiesti dal proponente;
	\item la soddisfazione del cliente con il prodotto sviluppato;
	\item oggettività nella ricerca della qualità in modo da ottenerla in tutti i processi e i prodotti;
\end{itemize}
\subsubsection{Descrizione}
Per una descrizione approfondita sul piano della qualità si faccia riferimento al seguente documento: \textit{Piano di Qualità 1.0.0}.
Nel documento sopracitato sono descritte le metodologie e regole utilizzate dal gruppo \textbf{VRAM Software} per cercare di raggiungere le aspettative definite nella sezione precedente (aggiungere numero).
Il documento \textit{Piano di Qualità 1.0.0} contiene:
\begin{itemize}
	\item la presentazione delle caratteristiche e gli attributi considerati più importanti del prodotto
	\item la descrizione degli standard di qualità utilizzati per la gestione del processo di vita del software
	\item l'individuazione dei processi definiti dagli standard descritti
\end{itemize}
Per ogni processo individuato vengono individuati:
\begin{itemize}
	\item gli obbiettivi da raggiungere
	\item le strategie per raggiungere
	\item le metriche per misurare l'avanzamento
\end{itemize}
L'obbiettivo che si vuole raggiungere, utilizzando le regole proposte nel documento \textit{Piano di Qualità}, è quello di ottenere un livello di qualità soddisfacente per software e documentazione.
\subsubsection{Attività}
Per ogni processo abbiamo deciso di pianificare lo sviluppo in tre attività:
\begin{itemize}
	\item \textbf{Pianificazione} \\*
		\begin{itemize}
			\item ci siamo posti degli obbiettivi per perseguire la qualità
			\item abbiamo definito delle strategie per raggiungere tali obbiettivi
			\item abbiamo disposto le persone e le risorse da utilizzare per la realizzazione del processo
		\end{itemize}
	\item \textbf{Valutazione} \\*
		Abbiamo messo in atto le strategie proposte nella \textbf{Pianificazione} valutandone i risultati
	\item \textbf{Reazione} \\*
		Per ogni risultato ottenuto abbiamo modificato opportunamente le nostre strategie
\end{itemize}
\subsubsection{Strumenti}
Gli strumenti utilizzati per la gestione della qualità sono:
\begin{itemize}
	\item strumenti forniti dallo standard \textbf{ISO-12207}
	\item le metriche
\end{itemize}
