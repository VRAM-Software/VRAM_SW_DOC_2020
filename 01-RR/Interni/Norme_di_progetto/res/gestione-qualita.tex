\subsection{Gestione della qualità}
\subsubsection{Scopo}
Lo scopo della gestione della qualità è dare una valutazione oggettiva riguardo ai servizi che il prodotto\glosp deve fornire in modo da mantenere uno standard di qualità prestabilito e da soddisfare i requisiti impliciti ed espliciti del proponente.
\subsubsection{Aspettative}
Le aspettative della gestione della qualità possono essere riassunti dai seguenti punti, che rappresentano gli obiettivi che vogliono essere raggiunti dal gruppo. Si vuole ottenere:
\begin{itemize}
	\item qualità per tutta la durata del ciclo di vita del software;
	\item una copertura totale di tutti i requisiti impliciti ed espliciti richiesti dal proponente;
	\item la soddisfazione del cliente con il prodotto\glosp sviluppato;
	\item oggettività nella ricerca della qualità in modo da ottenerla in tutti i processi\glosp e in tutti i prodotti\glo.
\end{itemize}
\subsubsection{Descrizione}
Per la gestione della qualità dello sviluppo di questo progetto\glosp si è cercato di applicare lo standard ISO 9000, il quale afferma che il sistema di qualità, cioè l'insieme dei metodi e regole per gestire la qualità di un prodotto\glo, è costituito da tre passaggi: lo sviluppo, il controllo e il miglioramento continuo. Questi tre passaggi sono ripetuti per ogni processo\glosp in modo di cercare di prevenire possibili errori e anomalie.
Per una descrizione approfondita sul piano della qualità, che è uno dei costituenti del sistema di qualità, si faccia riferimento al seguente documento: \textit{Piano di Qualifica v. 1.1.1}.
Nel documento sopracitato sono descritte le metodologie e regole utilizzate dal gruppo \textit{VRAM Software} per cercare di raggiungere le aspettative definite nella sezione precedente.
Il contenuto sintetico del documento \textit{Piano di Qualifica v. 1.1.1} contiene:
\begin{itemize}
	\item la presentazione delle caratteristiche e gli attributi considerati più importanti del prodotto\glo;
	\item la descrizione degli standard di qualità utilizzati per la gestione del ciclo di vita del software;
	\item l'individuazione dei processi\glosp definiti dagli standard descritti, nei quali sono successivamente stati individuati gli obiettivi, le strategie per raggiungerli e le metriche\glosp per misurarne l'avanzamento.
\end{itemize}
Con il controllo e il miglioramento dei processi\glosp a ogni prodotto\glosp vengono poi associati: gli obiettivi da raggiungere e le metriche\glosp utilizzate.
L'obiettivo che si vuole raggiungere, utilizzando le regole proposte nel documento \textit{Piano di Qualifica v. 1.1.1}, è ottenere un livello di qualità soddisfacente per software e documentazione, perché se si riesce a perseguire la qualità per tutta il ciclo di vita del prodotto\glo, allora le possibilità di ottenere un prodotto\glosp stabile, fonte di miglioramento continuo, sarà più alta.
\subsubsection{Attività}
Per ogni processo\glosp abbiamo deciso di pianificare lo sviluppo in tre attività:
\begin{itemize}
	\item \textbf{Pianificazione}:
		\begin{itemize}
			\item ci siamo posti degli obiettivi per perseguire la qualità;
			\item abbiamo definito delle strategie per raggiungere tali obiettivi;
			\item abbiamo disposto le persone e le risorse da utilizzare per la realizzazione del processo\glo.
		\end{itemize}
	\item \textbf{Valutazione}: abbiamo messo in atto le strategie proposte nella pianificazione valutandone i risultati;
	\item \textbf{Reazione}: per ogni risultato ottenuto abbiamo modificato opportunamente le nostre strategie.
\end{itemize}
\subsubsection{Metriche}
Il parametro che abbiamo utilizzato è il seguente:
	\begin{itemize}
		\item \textbf{PMS}: Percentuale di metriche\glosp soddisfatte rispetto alla totalità di metriche\glosp utilizzare all'interno del progetto\glo.
	\end{itemize}
\subsubsection{Strumenti}
Gli strumenti utilizzati per la gestione della qualità sono:
\begin{itemize}
	\item strumenti forniti dallo standard ISO-12207, che contiene la descrizione del processo\glo: "Gestione della Qualità";
	\item le metriche\glosp utilizzate per misurare il perseguimento della qualità.
\end{itemize}
