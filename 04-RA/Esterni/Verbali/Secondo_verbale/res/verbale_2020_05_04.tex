\section{Informazioni generali}
    \subsection{Informazioni incontro}
        \begin{itemize}
            \item \textbf{Luogo}: Videochiamata tramite Skype;
            \item \textbf{Data}: 2020-05-04;
            \item \textbf{Ora d'inizio}: 17.00;
            \item \textbf{Ora di fine}: 17.20;
            \item \textbf{Partecipanti}: 
            \begin{itemize}
            	\item Corrizzato Vittorio;
                \item Dalla Libera Marco;
                \item Rampazzo Marco;
                \item Santagiuliana Vittorio;
                \item Schiavon Rebecca;
                \item Spreafico Alessandro;
                \item Toffoletto Massimo;
                \item Piccoli Gregorio (proponente).
            \end{itemize}
        \end{itemize}
    \subsection{Argomenti trattati}
        In questo incontro con il proponente \textit{Zucchetti} i membri del gruppo hanno presentato al dott. Piccoli lo stato di avanzamento del prodotto\glosp e ricevuto dei feedback anche in ottica della ormai prossima revisione di validazione\glosp e collaudo. Gli argomenti trattati sono:
        \begin{enumerate}
            \item revisione sullo stato di avanzamento del prodotto\glo;
            \item migliorie dell'interfaccia utente.
        \end{enumerate}
\section{Verbale}
        \subsection{Punto 1}
            L'incontro è cominciato con una breve demo per informare il proponente riguardo le nuove funzionalità del prodotto\glo. Tali aggiornamenti portano quest'ultimo ad un livello di maturità utile per il collaudo. Il dott. Piccoli è stato soddisfatto dei risultati raggiunti. Il gruppo quindi procederà quindi al raffinamento delle funzionalità già presenti e alla correzione di alcune minuzie estetiche segnalate del proponente.
        \subsection{Punto 2}
            Durante la demo è stata segnalata la presenza di un form di troppo per la selezione dell'algoritmo di predizione. Quest'ultimo infatti viene scelto dall'utente, prima, quando inserisce i dati da addestrare e poi anche dopo il loro inserimento. La seconda possibilità di scelta risulta eccessiva perché la decisione dell'algoritmo è influenzata dai dati che si analizzano, quindi una volta selezionato SVM\glosp o RL\glosp durante l'immissione dei dati non è necessario chiedere nuovamente la selezione dell'algoritmo. Questa rimozione permette anche una migliore organizzazione dei restanti elementi dell'interfaccia utente. Inoltre, è stato consigliato di modificare il nome dell'applicazione, infatti durante la demo era segnalata come "VRAM Software Applicativo Esterno", nomenclatura utile dal nostro punto di vista ma poco informativa per l'utente finale.