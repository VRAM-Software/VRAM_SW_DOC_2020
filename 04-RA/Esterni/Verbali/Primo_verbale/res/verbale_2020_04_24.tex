\section{Informazioni generali}
    \subsection{Informazioni incontro}
        \begin{itemize}
            \item \textbf{Luogo}: Corrispondenza tramite posta elettronica;
            \item \textbf{Data}: 2020-04-24;
            \item \textbf{Partecipanti}: 
            \begin{itemize}
            	\item Corrizzato Vittorio;
                \item Dalla Libera Marco;
                \item Rampazzo Marco;
                \item Santagiuliana Vittorio;
                \item Schiavon Rebecca;
                \item Spreafico Alessandro;
                \item Toffoletto Massimo;
                \item Piccoli Gregorio (proponente).
            \end{itemize}
        \end{itemize}
    \subsection{Argomenti trattati}
        In questo confronto con il proponente \textit{Zucchetti} il gruppo ha posto alcune domande riguardanti la visualizzazione dei dati durante la fase di addestramento degli algoritmi di predizione.
\section{Verbale}
    \subsection{Punto 1}
        Il gruppo ha chiesto al proponente se sia corretto visualizzare nel grafico la totalità dei dati nonostante solo i due terzi siano impiegati nell'addestramento, infatti il terzo rimanente è impiegato per il calcolo dell'indice di qualità.
        La visualizzazione di tutti i dati permetterebbe di mitigare eventuali discordanze con l'addestramento.
        Il proponente ha suggerito di visualizzare sia i dati di addestramento che quelli utilizzati per il calcolo dell'indice di qualità, distinguendoli tramite simboli differenti all'interno del grafico.
