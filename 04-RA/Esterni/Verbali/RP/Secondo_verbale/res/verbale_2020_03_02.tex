\section{Informazioni generali}
    \subsection{Informazioni incontro}
        \begin{itemize}
            \item \textbf{Luogo}: Videochiamata tramite Skype;
            \item \textbf{Data}: 2020-03-02;
            \item \textbf{Ora d'inizio}: 11.00;
            \item \textbf{Ora di fine}: 11.20;
            \item \textbf{Partecipanti}: 
            \begin{itemize}
                \item Corrizzato Vittorio;
                \item Dalla Libera Marco;
                \item Rampazzo Marco;
                \item Santagiuliana Vittorio;
                \item Schiavon Rebecca;
                \item Spreafico Alessandro;
                \item Toffoletto Massimo;
                \item Piccoli Gregorio (proponente).
            \end{itemize}
        \end{itemize}
    \subsection{Argomenti trattati}
        Durante la videochiamata il gruppo ha esposto al proponente \textit{Zucchetti} un PoC\glosp e ha ricevuto riscontri a riguardo al fine di perfezionarlo in tempo per la presentazione della technology baseline al prof. Cardin.
        Il proponente ha dato un giudizio generalmente positivo, ma nel contempo ha individuato alcune cose da migliorare.
        \begin{enumerate}
            \item data, ora, e identificativo algoritmo nel file JSON contenente il predittore;
            \item salvataggio dei file;
            \item descrizione addestramento in corso;
            \item visualizzazione grafico al posto di una serie numerica nella creazione del pannello.  
        \end{enumerate}
\section{Verbale}
    \subsection{Punto 1}
        Il proponente ha espresso la necessità di avere nel file JSON contenente i predittori anche la data, l'ora di creazione e l'identificativo dell'algoritmo usato per crearlo.
    \subsection{Punto 2}
        Il proponente ha fatto notare che deve essere l'utente a decidere quando salvare i file e che è pratica comune dare la possibilità di dare un nome a piacere ai file.
    \subsection{Punto 3}
        Il proponente ha suggerito di dare la possibilità agli utenti di inserire delle annotazioni sull'addestramento che stanno svolgendo.
    \subsection{Punto 4}
        Il proponente ha detto che è preferibile visualizzare già il grafico al posto di una serie numerica nella creazione del pannello nel plug-in. 
     
        