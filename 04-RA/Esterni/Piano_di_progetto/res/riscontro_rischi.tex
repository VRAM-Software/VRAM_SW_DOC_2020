\appendix
\section{Riscontro dei rischi}
    \subsection{Attualizzazione dei rischi 2020-01-14}
        \rowcolors{2}{gray!25}{gray!15}
	    \begin{longtable} {
		    >{}p{10mm} 
		    >{}p{24mm}
		    >{}p{32mm} 
            >{}p{32mm}
		    }
	    \rowcolor{gray!50}
        \textbf{ID} & \textbf{Tempistiche} & \textbf{Descrizione} & \textbf{Manutenzione migliorativa}	\TBstrut \\
        RT1 & Creazione del template \LaTeX \xspace e della repository\glo & Nella prima fase del progetto\glosp ci sono state difficoltà dovute all'apprendimento di \LaTeX, il cui funzionamento era sconosciuto a tutti i membri del gruppo, e alla gestione di GitHub, noto a tutti i componenti, ma mai utilizzato al pieno delle sue possibilità & Per quanto riguarda \LaTeX \xspace è stato necessario un breve periodo di autoapprendimento da parte di alcuni componenti del gruppo, mentre per GitHub i membri del gruppo che hanno partecipato al corso di "Tecnologie Open Source" hanno contribuito a proporre delle soluzioni studiate a lezione  \TBstrut \\ [2mm]
        RG1 & \textit{Glossario} & È stato oggetto di una lunga discussione il metodo di stesura del \textit{Glossario}, soprattutto la scelta dei termini da inserire in esso & Alla fine di una condivisione di opinioni il responsabile ha definito delle linee guida per la scelta dei lemmi da inserire nel \textit{Glossario} che hanno messo d'accordo i componenti del gruppo \TBstrut \\ [2mm]
        RG2 & Rischio permanente & I membri del gruppo sono dovuti andare incontro ad inevitabili sovrapposizioni di impegni durante la stesura dei documenti, tale rischio ha influenzato l'incidenza di RS1 & Ogni componente, per quanto possibile, ha cercato di ritagliarsi del tempo per rispettare le milestone e, eventualmente, ha chiesto assistenza ad altri componenti più disponibili \TBstrut \\ [2mm]
        RO2 & Gestione della repository\glo & Collegandosi a RT1 la scarsa esperienza nell'utilizzo di GitHub ha generato lacune nella corretta gestione dell'issue tracking system integrato nella piattaforma & I membri del gruppo che hanno seguito il corso di "Tecnologie Open Source", e quindi più pratici nei sistemi di versionamento\glo, hanno risolto le inconsistenze individuate in GitHub \TBstrut \\ [2mm]
        RO3 & Stesura primi documenti & Con la stesura dei primi documenti alcuni componenti hanno trovato difficoltà ad adattarsi ai vari ruoli assegnati & I membri del gruppo si sono assistiti a vicenda cercando di colmare le debolezze evidenziate sotto certi aspetti dando delle linee guida sui processi\glosp di analisi e di verifica dei documenti \TBstrut \\ [2mm]
        RR1 & \textit{Studio di Fattibilità} e \textit{Analisi dei Requisiti} & Non avendo esperienza sugli strumenti da utilizzare certe specifiche richieste non sono state chiare o sono state mal interpretate & Per ovviare a questo problema sono stati indetti due incontri con il proponente che hanno contribuito a risolvere i dubbi sorti durante l'analisi dei documenti \TBstrut \\ [2mm]
        RS1 & Rischio permanente & Data la scarsa esperienza nella gestione di una tale mole di lavoro e gli impegni personali dei componenti certe milestone sono risultate imprecise e difficilmente rispettabili & Pur essendo stato un problema persistente man mano si è diminuita l'imprecisione delle scadenze inoltre, il proseguimento del lavoro in gruppo dovrebbe portare ad avere maggiore consapevolezza dei tempi e quindi il calo di incidenza del rischio \TBstrut \\ [2mm]
        \rowcolor{white}
        \caption{Attualizzazione dei rischi 2020-01-14}
        \end{longtable}
    \pagebreak
    \subsection{Attualizzazione dei rischi 2020-03-02}
        \rowcolors{2}{gray!25}{gray!15}
	    \begin{longtable} {
		    >{}p{10mm} 
		    >{}p{24mm}
		    >{}p{32mm} 
            >{}p{32mm}
		    }
	    \rowcolor{gray!50}
        \textbf{ID} & \textbf{Tempistiche} & \textbf{Descrizione} & \textbf{Manutenzione migliorativa}	\TBstrut \\
        RS1 & Rischio permanente & Come già successo nel periodo precedente, anche se in misura ridotta, l'inesperienza e gli impegni accademici hanno causato un lieve ritardo nel completamento delle milestone pianificate. Questo è dovuto anche a cause esterne all'ambito universitario che hanno impedito alcuni incontri tra i componenti del gruppo & Rispetto al periodo precedente, sono diminuiti i problemi sotto questo punto di vista; tuttavia per cause non prevedibili e pianificabili, si è accumulato un leggero ritardo nel completamento delle milestone. Il gruppo si propone di seguire ancor più rigorosamente i propositi del periodo precedente che hanno comunque migliorato la pianificazione del lavoro \TBstrut \\ [2mm]
        RT1 & Sviluppo del PoC\glo & Con il primo approccio all'attività di codifica sono sorti dei problemi causati dall'inesperienza dei membri del gruppo nei confronti delle tecnologie richieste per svolgere il progetto\glo & Alcuni componenti che avevano già utilizzato alcune tecnologie, hanno inizialmente configurato l'ambiente di sviluppo. Successivamente, dopo un ulteriore attività di apprendimento autonomo, hanno istruito il resto del gruppo su quanto appreso per un utilizzo corretto dei nuovi strumenti e delle nuove tecnologie \TBstrut \\ [2mm]
        RG3 & Inizio RP & A causa dell'inesperienza del gruppo nello svolgimento di progetti di questa mole, alcune correzioni indicate dai committenti nella documentazione prodotta non sono risultate chiare & Questo problema è stato risolto attraverso degli incontri con i committenti ed il conseguente chiarimento delle correzioni fatte \TBstrut \\ [2mm]
        RR1 & Sviluppo del PoC\glo & Durante lo sviluppo del PoC\glosp sono stati individuati nuovi strumenti utili ai fini del progetto\glo & Il gruppo ha risolto questo problema analizzando insieme la nuova tecnologia e, nel caso in cui risultasse migliore dei precedenti, proseguendo alla normazione e all'utilizzo di quest'ultimo \TBstrut \\ [2mm]
        \rowcolor{white}
        \caption{Attualizzazione dei rischi 2020-03-02}
        \end{longtable}
    \pagebreak
    \subsection{Attualizzazione dei rischi 2020-04-05}
    \rowcolors{2}{gray!25}{gray!15}
	    \begin{longtable} {
		    >{}p{10mm} 
		    >{}p{24mm}
		    >{}p{32mm} 
            >{}p{32mm}
		    }
	    \rowcolor{gray!50}
        \textbf{ID} & \textbf{Tempistiche} & \textbf{Descrizione} & \textbf{Manutenzione migliorativa}	\TBstrut \\
        RG3 & Inizio RQ & Nella correzione relativa alla RP sono stati individuati dal committente degli errori nel sistema di versionamento\glosp da noi adottato; la risoluzione di questi problemi è risultata più ostica del previsto & Per risolvere questo problema ci siamo confrontati con il committente (tramite e-mail e videoconferenza) e successivamente abbiamo discusso internamente per arrivare ad una soluzione coerente con quanto spiegato dal prof. Vardanega \TBstrut \\ [2mm]
        RT1 & Attività di codifica & Durante l'attività di codifica un toolkit\glosp utilizzato per lo sviluppo efficiente di plug-in Grafana\glosp (grafana-toolkit), ha aggiornato automaticamente le sue dipendenze (comportamento sconosciuto dal gruppo) impedendo la normale compilazione del programma & Inizialmente non siamo riusciti a trovare una soluzione agli errori presentati al momento della compilazione. Successivamente però, dopo uno studio più approfondito del toolkit\glo, abbiamo capito il problema, installato gli aggiornamenti fatti in background dal toolkit\glosp e ripristinato il funzionamento del plug-in \TBstrut \\ [2mm]
        RT1 & Attività di codifica & Durante l'attività di codifica abbiamo riscontrato delle difficoltà nell'apprendimento dei linguaggi di programmazione utilizzati, anche dopo gli insegnamenti forniti dai membri più esperti del gruppo & I problemi individuati sono dovuti prevalentemente al fatto che sono state sviluppate delle funzionalità altamente specifiche del software ed è iniziata l'esecuzione degli unit test. Generalmente, quindi, le istruzioni date dai componenti del gruppo più esperti sono state utili, ma sono stati necessari dei confronti ulteriori per definire gli aspetti più avanzati dei linguaggi \TBstrut \\ [2mm]
	\end{longtable}
