\section{Modello di sviluppo}
Dall'analisi del progetto\glosp è stato identificato come modello di sviluppo più appropriato un modello di sviluppo incrementale. Tuttavia si è anche notato che, dato che sarà sviluppato un plug-in, all'occorrenza sarà necessario utilizzare un modello di sviluppo a componenti in quanto, quando applicabile, faciliterà e velocizzerà notevolmente lo sviluppo tramite l'applicazione di:
\begin{itemize}
	\item \textbf{Component reuse}: riuso di componenti di un'applicazione esistente, siano essi sottosistemi o singoli oggetti che potranno essere utilizzati tramite opportune API\glo;
	\item \textbf{Object and function reuse}: riuso di componenti software che realizzano una singola funzione o classe oggetto che potranno essere utilizzati come librerie durante lo sviluppo di nuovo codice sorgente.
\end{itemize}
Si è quindi ritenuto che il modello incrementale con utilizzo, dove opportuno, di modello a componenti sia la soluzione migliore per realizzare il prodotto\glosp richiesto, in quanto permetterà di agevolare e strutturare in modo efficace ed efficiente lo sviluppo, portando continuamente valore aggiunto al prodotto\glosp e velocizzando lo sviluppo tramite il riuso di componenti esistenti.

\subsection{Modello incrementale}
Il modello di sviluppo incrementale rappresenta il principale modello di sviluppo utilizzato per sviluppare il plug-in, in quanto la sua natura ad incrementi porta i seguenti vantaggi:
\begin{itemize}
	\item agevolazione dell'organizzazione dello sviluppo, attraverso una corrispondenza fra requisiti ed incrementi;
	\item agevolazione della verifica che viene effettuata ad ogni incremento, senza dover necessariamente bloccare lo sviluppo degli altri incrementi;
	\item agevolazione del monitoraggio dei progressi, in quanto ogni incremento porta valore aggiunto al prodotto\glo.
\end{itemize}
La prima attività del modello di sviluppo incrementale consiste nell'individuazione dei requisiti da implementare, che vanno classificati in base alla loro rilevanza per gli stakeholder\glo. È così possibile sviluppare per primi i requisiti di importanza critica che rappresentano le funzionalità cardine del prodotto\glo, lasciando in secondo piano quelli non prioritari e facendo corrispondere ad ogni incremento uno o più requisiti.
Il metodo di lavoro incrementale è quindi così riassunto:
\begin{itemize}
	\item individuare i requisiti del prodotto\glo;
	\item definire degli incrementi facendo loro corrispondere uno o più requisiti;
	\item analizzare se negli incrementi è possibile riutilizzare componenti secondo il modello di sviluppo a componenti;
	\item i seguenti passaggi saranno eseguiti per ogni incremento:
	\begin{itemize}
		\item suddivisione dello sviluppo del singolo incremento fra gli sviluppatori;
		\item sviluppo del singolo incremento;
		\item verifica dello sviluppo da parte dei verificatori;
		\item validazione\glosp dell'incremento tramite riunione di gruppo, in cui si verifica se l'incremento soddisfa effettivamente i requisiti a lui assegnati;
		\item discussione ed eventuale attuazione di miglioramenti per i prossimi incrementi.
	\end{itemize}
\end{itemize}

\subsubsection{Incrementi individuati}
La seguente tabella riporta i principali incrementi individuati e definiti inizialmente. Essi sono identificati da un numero intero progressivo maggiore di zero, la loro priorità è data agli incrementi che implementano i requisiti obbligatori, seguiti dagli incrementi riguardanti i requisiti desiderabili. Infine sono riportati degli incrementi che implementano i requisiti opzionali e che verrano svolti qualora fosse possibile o necessario.
\rowcolors{2}{gray!25}{gray!15}
\begin{longtable} {
		>{\raggedright\arraybackslash}p{85mm}
		>{\raggedleft\arraybackslash}p{40mm}
	}
	\rowcolor{gray!50} 
	\textbf{Incremento} & 
	\textbf{Requisiti} 	\TBstrut \\
	
	Incremento 1: creazione struttura base applicativo addestramento &
	R1F4  \TBstrut \\ [2mm]		
	
	Incremento 2: importazione file CSV e file JSON su applicativo esterno & 
	R1F4.1, R1F4.1.1, R1F4.1.2, R1F4.8, R1F4.9, R1F4.9.1, R1F4.9.2  \TBstrut \\ [2mm]
	
	Incremento 3: sviluppo algoritmo addestramento SVM\glosp e sviluppo algoritmo predizione SVM\glosp nel plug-in& 
	R1F4.2, R1F4.10, R2F4.6  \TBstrut \\ [2mm]
	
	Incremento 4: sviluppo algoritmo addestramento RL\glosp e sviluppo algoritmo predizione RL\glosp nel plug-in & 
	R1F4.11, R1F11  \TBstrut \\ [2mm]
	
	Incremento 6: selezione modello e avvio addestramento & 
	R1F4.4, R1F4.11, R1F4.12 \TBstrut \\ [2mm]
	
	Incremento 7: creazione ed esportazione file JSON con i parametri addestramento &
	R1F4.5, R1F4.6, R1F4.7 \TBstrut \\ [2mm]
	
	Incremento 8: visualizzazione qualità previsioni &
	R1F5, R1F5.1, R1F5.2, R1F5.3 \TBstrut \\ [2mm]
	
	Incremento 9: realizzazione struttura base plug-in Grafana\glosp &
	R1F7, R1F19, R1F20 \TBstrut \\ [2mm]
	
	Incremento 12: importazione file JSON nel plug-in &
	R1F8 \TBstrut \\ [2mm]
	
	Incremento 13: lettura file JSON e configurazione algoritmi plug-in &
	R1F8, R1F11 \TBstrut \\ [2mm]
	
	Incremento 14: associazione nodi al flusso dati, elaborazione e visualizzazione dei dati &
	R1F9, R1F9.1, R1F9.2, R1F9.4, R1F9.5, R1F11 \TBstrut \\ [2mm]
	
	Incremento 16: arresto del plug-in &
	R1F12, R1F20 \TBstrut \\ [2mm]
	
	Incremento 17: errore JSON e CSV non validi ed errore collegamento nodi &
	R2F6, R2F18, R2F10 \TBstrut \\ [2mm]
	
	Incremento 19: predisposizione addestramento interno &
	R3F1, R3F1.1, R3F1.1.1, R3F1.1.2, R3F1.5, R3F1.6, R3F1.6.1, R3F1.6.2 \TBstrut \\ [2mm]
	
	Incremento 20: addestramento interno &
	R3F1.2, R3F1.3, R3F1.4, R3F1.7 \TBstrut \\ [2mm]
	
	Incremento 21: chiusura addestramento interno &
	R3F1.4, R3F1.8 \TBstrut \\ [2mm]
	
	Incremento 22: indice previsioni e errore &
	R3F2, R3F2.1, R3F2.2, R3F2.3 \TBstrut \\ [2mm]
	
	Incremento 23: sviluppo algoritmo regressione non lineare &
	R3F1.11, R3F4.3 \TBstrut \\ [2mm]
	
	Incremento 24: sviluppo algoritmo reti neurali\glo &
	R3F1.11, R3F4.3 \TBstrut \\ [2mm]
	
	Incremento 25: flusso dati continuo &
	R3F9.3 \TBstrut \\ [2mm]
	\rowcolor{white}
	\caption{Tabella degli incrementi}
\end{longtable}

\subsection{Modello a componenti}
Per la realizzazione del prodotto\glosp sarà anche utilizzato, quando possibile, il modello di sviluppo a componenti, così da velocizzare e standardizzare lo sviluppo dei requisti.
I componenti evidenziati dall'analisi sono principalmente gli elementi esistenti di Grafana\glosp e gli algoritmi di predittività forniti da \textit{Zucchetti}.
In particolare tali elementi vengono inquadrati nelle seguenti classi di componenti:
\begin{itemize}
	\item \textbf{Elementi Grafana}\glo: Sono le librerie e le funzionalità fornite da Grafana\glo, il loro uso viene quindi inquadrato come component reuse;
	\item \textbf{Algoritmi di predittività}: Sono gli algoritmi che ci sono stati forniti da \textit{Zucchetti} e saranno riutilizzabili come librerie durante lo sviluppo del plug-in. Il loro uso viene quindi inquadrato come object and function reuse.
\end{itemize}
