\section{Glossario}

%\subsection*{C}
%\subsubsection*{CSV}
%Il comma-separated values (abbreviato in CSV) è un formato di file basato su file di testo utilizzato per l'importazione ed esportazione di una tabella di dati. 

\subsection*{D}
\subsubsection*{Datasource}
Una datasource è una sorgente di dati. Solitamente si tratta di un database.

\subsubsection*{Dashboard}
In italiano cruscotto; interfaccia che permette all'utente di tenere sotto controllo gli indicatori più importanti dell'ambiente in cui sta lavorando. È caratteristica fondamentale l'aggiornamento automatico dei dati, senza che vi debba essere un'interazione con l'utente.

\subsection*{G}
\subsubsection*{Grafana}
Software ad uso generico per la produzione di cruscotti informativi (dashboard in inglese) e composizione di grafici. Viene utilizzato come un'applicazione web.

\subsection*{I}
\subsubsection*{InfluxDB}
InfluxDB è un database basato sul concetto di serie temporale. InfluxDB è specializzato e
ottimizzato per il salvataggio e la lettura di serie temporali: record salvati in ordine temporale
e caratterizzati dal timestamp, ovvero un campo che indica una data. InfluxDB è utilizzato
per lo più in ambiti in cui è necessario salvare valori generati da sensori oppure analytics in
tempo reale.

%\subsection*{J}
%\subsubsection*{JSON}
%Acronimo di JavaScript Object Notation, è un formato adatto all'interscambio di dati fra applicazioni client/server. È basato sul linguaggio JavaScript Standard ma ne è indipendente.

\subsection*{M}
\subsubsection*{Machine learning}
Il machine learning (in italiano apprendimento automatico) è una branca dell'intelligenza artificiale che utilizza metodi statistici per migliorare progressivamente la performance di un algoritmo nell'identificare pattern nei dati.

\subsection*{P}
\subsubsection*{Predittore}
Dati o variabili su cui applico le tecniche di regressione o di classificazione per ottenere un dato la cui diretta rilevazione sarebbe impossibile o troppo onerosa.

\subsubsection*{Prodotto}
Si definisce prodotto qualsiasi bene scambiabile sul mercato che può rispondere alle esigenze di un compratore. Un esempio di prodotto informatico è il software che è composto dal codice e dalla documentazione.	

\subsubsection*{RL}
Acronimo di Regressione lineare, algoritmo di machine learning\glosp che ha la funzione di prevedere un valore di una variabile dipendente (y) in base a una determinata variabile indipendente (x) secondo una relazione di tipo lineare.
\subsection*{S}
\subsubsection*{SVM}
Acronimo di Support Vector Machine; algoritmo di apprendimento automatico supervisionato che può essere utilizzato sia per scopi di classificazione che di regressione.

