\section{Specifica e progresso dei test}
Il nostro gruppo ha scelto di adottare il Modello a V\glosp per garantire la qualità del nostro prodotto\glo. In particolare, questo modello prevede lo sviluppo dei test durante le attività di analisi dei requisiti, progettazione\glosp architetturale e progettazione\glosp di dettaglio oltre a validazione\glosp e collaudo.
In questo modo è possibile verificare la correttezza sia di tutti gli aspetti che compongono il progetto\glosp che delle singole parti sviluppate. Sono state individuate quattro tipologie di test:
\begin{itemize}
	\item test di accettazione;
	\item test di sistema;
	\item test di integrazione;
	\item test di unità.
\end{itemize}
Ogni volta che viene svolta un'attività viene definita una tabella con i test di una tipologia.

\paragraph*{Codifica dei test} \mbox{} \\ [1mm]
Per ogni test vengono definiti il codice identificativo, la descrizione, lo stato, l'esito e l'importanza.
\begin{itemize}
	\item il codice identificativo è scritto in questo formato:\\
	\\ \textbf{tipologia[codice\_padre].[codice\_figlio]} \\
	\\ Dove:
	\begin{itemize}
		\item \textbf{tipologia}:
		\begin{itemize}
			\item \textbf{TS}: test di sistema;
			\item \textbf{TI}: test di integrazione;
			\item \textbf{TU}: test di unità.
		\end{itemize}
		\item \textbf{codice\_padre}: numero intero progressivo maggiore di zero che identifica univocamente il test;
		\item \textbf{codice\_figlio}: numero intero progressivo maggiore di zero che indica i sotto-test che compongono il padre.
	\end{itemize}
	\item descrizione: spiegazione concisa e completa di ciò che verifica il test;
	\item stato:
	\begin{itemize}
		\item \textbf{I}: implementato;
		\item \textbf{NI}: non implementato.
	\end{itemize}
	\item Esito:
	\begin{itemize}
		\item \textbf{P}: positivo;
		\item \textbf{N}: negativo;
		\item \textbf{NE}: non eseguito.
	\end{itemize}
	\item Importanza:
	\begin{itemize}
		\item \textbf{1}: test di una funzionalità primaria nel nostro prodotto;
		\item \textbf{2}: test di una funzionalità accessoria e quindi non indispensabile.
	\end{itemize}
\end{itemize}

\subsection{Test di accettazione}
\addtocontents{toc}{\protect\setcounter{tocdepth}{0}} %Inserire questo per escludere una sezione dall'indice.

\rowcolors{2}{gray!25}{gray!15}
\setcounter{table}{0}
\begin{longtable} {
	>{}p{12mm}
	>{}p{79.5mm}
	>{}p{9mm}
	>{}p{8mm}
	>{}p{14mm}
	>{}p{0mm}}
	\rowcolor{gray!50}
	\textbf{Codice} & \textbf{Descrizione} & \textbf{Stato} & \textbf{Esito} & \textbf{Priorità} & \TBstrut \\
	TA1 & Verificare che l'utente possa addestrare gli algoritmi di previsione sull'applicazione esterna & I & P & 1 & \TBstrut \\ [2mm]
	TA1.1 & Verificare che l'utente possa selezionare e caricare, dal suo dispositivo, un file CSV contenente i dati su cui effettuare l'addestramento & I & P & 1 & \TBstrut \\ [2mm]
	TA1.2 & Verificare che l'utente possa selezionare e caricare, dal suo dispositivo, un file JSON contenente la configurazione di un addestramento precedentemente eseguito & I & P & 1 & \TBstrut \\ [2mm]
	TA1.1.1 & Verificare che l'inserimento di un file CSV non valido venga visualizzato un messaggio d'errore & I & P & 2 & \TBstrut \\ [2mm]
	TA1.2.1 & Verificare che l'inserimento di un file JSON non valido venga visualizzato un messaggio d'errore & I & P & 2 & \TBstrut \\ [2mm]
	TA2 & Verificare che l'utente possa visualizzare un grafico a dispersione che rappresenti i dati utilizzati per l'addestramento nell'applicazione esterna & I & P & 1 & \TBstrut \\ [2mm]
	TA3 & Verificare che l'utente possa inserire delle note che verranno scritte nel file JSON contenente il risultato dell'addestramento & I & P & 1 & \TBstrut \\ [2mm]
	TA4 & Verificare che l'utente possa scegliere quale algoritmo utilizzare per effettuare l'addestramento dei dati & I & P & 1 & \TBstrut \\ [2mm]
	TA5 & Verificare che l'utente posso avviare l'addestramento dell'algoritmo di predizione scelto utilizzando i dati inseriti & I & P & 1 & \TBstrut \\ [2mm]
	TA6 & Verificare che, alla fine del processo di addestramento, venga visualizzato un messaggio di operazione completata con successo & I & P & 2 & \TBstrut \\ [2mm]
	TA7 & Verificare che, alla fine del processo di addestramento, vengano visualizzati gli indici di qualità delle previsioni eseguite sull'app esterna & I & P & 1 & \TBstrut \\ [2mm]
	TA8 & Verificare che l'utente, alla fine del processo di addestramento, riceva un file JSON contenente il risultato dell'addestramento & I & P & 1 & \TBstrut \\ [2mm]
	TA9	& Verificare che l'utente possa avviare il plug-in & I & P & 1 & \TBstrut \\ [2mm]
	TA10 & Verificare che l'utente possa caricare il file JSON ottenuto dall'addestramento effettuato dall'applicazione esterna & I & P & 1 & \TBstrut \\ [2mm]
	TA11 & Verificare che l'utente possa associare i predittori letti dal file JSON al flusso dati & I & P & 1 & \TBstrut \\ [2mm]
	TA11.1 & Verificare che l'utente possa selezionare un flusso di dati statico su cui eseguire delle previsioni & I & P & 1 & \TBstrut \\ [2mm]
	TA11.2 & Verificare che l'utente possa visualizzare un messaggio che conferma il successo nel collegamento dei nodi al flusso dati & I & P & 2 & \TBstrut \\ [2mm]
	TA12 & Verificare che, se il collegamento dei nodi al flusso dati non va a buon file, l'utente visualizzi un messaggio di errore & I & P & 2 & \TBstrut \\ [2mm]
	TA13 & Verificare che l'utente possa visualizzare il grafico dei risultati della previsione all'interno di una dashboard\glosp precedentemente configurata & I & P & 1 & \TBstrut \\ [2mm]
	TA14 & Verificare che l'utente possa fermare l'esecuzione del plug-in rimuovendo il relativo pannello dalla dashboard\glo & I & P & 1 & \TBstrut \\ [2mm]
	TA15 & Verificare che l'utente possa definire un alert\glosp all'interno del pannello della dashboard\glosp su cui si è applicato il plug-in & I & P & 2 & \TBstrut \\ [2mm]
	TA15.1 & Verificare che l'utente possa inserire un alert\glosp nel pannello della dashboard\glo & I & P & 2 & \TBstrut \\ [2mm]
	TA15.2 & Verificare che l'utente possa definire le regole di funzionamento di un alert\glo & I & P & 2 & \TBstrut \\ [2mm]
	TA15.3 & L'utente deve poter definire le condizioni di funzionamento di un alert\glo & I & P & 2 & \TBstrut \\ [2mm]
	TA15.4 & L'utente deve poter definire il comportamento legato all'assenza di dati in un determinato intervallo & I & P & 2 & \TBstrut \\ [2mm]
	TA16 & Verificare che l'utente visualizzi un messaggio di errore se viene inserito un input errato nella definizione di un alert\glo & I & P & 2 & \TBstrut \\ [2mm]
	TA17 & Verificare che l'utente possa sospendere un alert\glo & I & P & 2 & \TBstrut \\ [2mm]
	TA18 & Verificare che l'utente possa rimuovere un alert\glo & I & P & 2 & \TBstrut \\ [2mm]
	\rowcolor{white}
	\caption{Test di accettazione}
\end{longtable}

\addtocontents{toc}{\protect\setcounter{tocdepth}{4}} %Inserire questo per ripristinare il normale inserimento delle sezioni nell'indice. 4 significa fino al paragrah

\subsection{Test di sistema}
\addtocontents{toc}{\protect\setcounter{tocdepth}{0}} %Inserire questo per escludere una sezione dall'indice.

\rowcolors{2}{gray!25}{gray!15}
\begin{longtable} {
	>{}p{12mm}
	>{}p{79.5mm}
	>{}p{9mm}
	>{}p{8mm}
	>{}p{14mm}
	>{}p{0mm}}
	\rowcolor{gray!50}
	\textbf{Codice} & \textbf{Descrizione} & \textbf{Stato} & \textbf{Esito} & \textbf{Priorità} & \TBstrut \\
	TS1 & Verificare che l'addestramento degli algoritmi produca un file JSON con i parametri per le previsioni & I & P & 1 & \TBstrut \\ [2mm]
	TS2 & Verificare che la previsione possa essere fermata dall'utente & I & P & 2 & \TBstrut \\ [2mm]
	TS3 & Verificare che successivamente al caricamento del file CSV venga effettutata la lettura e il caricamento dei dati nello stato dell'applicazione & I & P & 1 & \TBstrut \\ [2mm]
	TS4 & Verificare che successivamente al caricamento del file JSON venga effettutata la lettura e il caricamento della configurazione di un addestramento precedente nello stato dell'applicazione & I & P & 1 & \TBstrut \\ [2mm]
	TS5 & Verificare che le note scritte dall'utente vengano inserite nel file JSON di output & I & P & 1 & \TBstrut \\ [2mm]
	TS6 & Verificare la corretta visualizzazione della bontà dei modelli di previsione a seguito dell'addestramento sui dati & I & P & 1 & \TBstrut \\ [2mm]
	TS7 & Verificare la corretta visualizzazione del grafico relativo ai dati e alla modifica dello stesso in seguito all'addestramento dei dati & I & P & 1 & \TBstrut \\ [2mm]
	TS8 & Verificare che i nodi ricavati dal file JSON siano associati correttamente al flusso dati scelto in Grafana\glo & I & P & 1 & \TBstrut \\ [2mm]
	TS9 & Applicare le previsioni su un flusso dati statico e visualizzare correttamente i dati ottenuti all'interno di un grafico contenuto nella dashboard\glo & I & P & 1 & \TBstrut \\ [2mm]
	TS10 & Verificare che l'utente possa definire il comportamento di un alert & I & P & 2 & \TBstrut \\ [2mm]
	TS11 & Verificare che l'utente possa aggiungere un alert al nostro pannello per monitorare i dati della previsione & I & P & 2 & \TBstrut \\ [2mm]
	TS12 & Verificare che l'utente possa rimuovere un alert dal nostro pannello & I & P & 2 & \TBstrut \\ [2mm]
	TS13 & Verificare che la rimozione del nostro pannello implichi l'arresto del plug-in & I & P & 1 & \TBstrut \\ [2mm]
	TS14 & Verificare che i messaggi d'errore relativi agli alert vengano visualizzati & I & P & 2 & \TBstrut \\ [2mm]
	TS15 & Verificare che i messaggi d'errore relativi ad un caricamento di un file JSON non valido vengano visualizzati & I & P & 2 & \TBstrut \\ [2mm]
	TS16 & Verificare che i messaggi d'errore relativi all'associazione dei predittori vengano visualizzati & I & P & 2 & \TBstrut \\ [2mm]
	\rowcolor{white}
	\caption{Test di sistema}
\end{longtable}

\addtocontents{toc}{\protect\setcounter{tocdepth}{4}} %Inserire questo per ripristinare il normale inserimento delle sezioni nell'indice. 4 significa fino al paragrah

\subsection{Test di integrazione}
\addtocontents{toc}{\protect\setcounter{tocdepth}{0}} %Inserire questo per escludere una sezione dall'indice.

\rowcolors{2}{gray!25}{gray!15}
\begin{longtable} {
		>{}p{12mm}
		>{}p{79.5mm}
		>{}p{9mm}
		>{}p{8mm}
		>{}p{14mm}
		>{}p{0mm}}
	\rowcolor{gray!50}
	\textbf{Codice} & \textbf{Descrizione} & \textbf{Stato} & \textbf{Esito} &\textbf{Priorità} & \TBstrut \\
	%------->Test app esterna
	%TI      &  & I & P  & \TBstrut \\ [2mm]
    TI1      &  Verificare che vengano renderizzati i nomi dei file selezionati mediante il componente FileInput. & I & P & 2 & \TBstrut \\ [2mm]
    TI2      &  Verificare che venga aggiornato lo stato del componente App e che venga fatto partire l'addestramento dopo aver cliccato il button 'Inizia addestramento RL' / 'Inizia addestramento SVM' per verificare la correttezza del metodo startTraining() del componente App e l'integrazione con il componente Button. & I & P  & 1 & \TBstrut \\ [2mm]
    TI3      &  Verificare che venga aperta la finestra per scegliere il nome del file json dopo che l’utente ha schiacciato il pulsante 'salva json' per dimostrare la correttezza del metodo handleOpenModal(event) del componente App e del metodo render del componente SaveFileModal. & I & P  & 1 & \TBstrut \\ [2mm]
    TI4      &  Verificare che venga chiusa la finestra per scegliere il nome del file json dopo che l’utente ha schiacciato il pulsante chiudi nella medesima finestra e quindi dimostrare la correttezza del metodo handleCloseModal(event) del componente App e del metodo render del componente SaveFileModal. & I & P  & 1 & \TBstrut \\ [2mm]
    TI5      &  Verificare che venga chiusa la finestra per scegliere il nome del file json dopo che l’utente ha schiacciato lo sfondo della stessa finestra per dimostrare la correttezza del handleCloseModal(event) del componente App e del metodo render del componente SaveFileModal. & I & P  & 1 & \TBstrut \\ [2mm]
    TI6      &  Verificare che venga modificato lo stato del componente App dopo che l'utente ha cliccato il button salva json nel componente SaveFileModal. & I & P  & 1 & \TBstrut \\ [2mm]
    TI7      &  Verificare che venga modificato lo stato del componente App quando l'utente modifica l'elemento input nel componente SaveFileModal. & I & P  & 1 & \TBstrut \\ [2mm]
    TI8      &  Verificare che venga modificato lo stato del componente App quando l'utente modifica l'elemento textarea nel componente ControlPanel. & I & P  & 1 & \TBstrut \\ [2mm]
    TI9      &  Verificare che la funzione onChange di App gestisca in modo corretto i file in formato json scelti mediante l'elemento input del componenente FileInput. & I & P  & 1 & \TBstrut \\ [2mm]
    TI10     &  Verificare che la funzione onChange di App gestisca in modo corretto i file in formato csv scelti mediante l'elemento input del componenente FileInput. & I & P  & 1 & \TBstrut \\ [2mm]
    TI11     &  Verificare che la funzione onChange di App gestisca in modo corretto i file nulli scelti mediante l'elemento input del componenente FileInput. & I & P  & 1 & \TBstrut \\ [2mm]
    TI12     &  Verificare che il componente ChangeParamModal venga chiuso e che i dati inseriti dall'utente vengano resettati se la funzione handleCloseParamModal viene chiamata. & I & P  & 1 & \TBstrut \\ [2mm]
    TI13     &  Verificare che la funzione onChange di App gestisca in modo corretto i file con un formato non accettato scelti mediante l'elemento input del componenente FileInput. & I & P  & 2 & \TBstrut \\ [2mm]
    TI14     &  Verificare che il pulsante 'seleziona parametri' cambi lo stato del componente App per renderizzare il componente ChangeParamModal. & I & P  & 1 & \TBstrut \\ [2mm]
    TI15     &  Verificare che, cliccando i checkbox del componente SetDataCheckboxes, venga cambiato lo stato del componente App. & I & P  & 2 & \TBstrut \\ [2mm]
    TI16     &  Verificare che cliccando l'input del componente FolderInput venga chiamata la funzione di App per aprire una finestra dove l'utente potrà scegliere la cartella dove salvare il file json risultante. & I & P  & 1 & \TBstrut \\ [2mm]
    TI17     &  Verificare che settare nuovi parametri mediante la finestra che corrisponde al componente ChangeParamModal resetti i vecchi parametri salvati all'interno del componente App. & I & P  & 1 & \TBstrut \\ [2mm]
    TI18     &  Verificare che scrivere una stringa con un carattere speciale nell'input di SaveFileModal porti ad un errore per stringa non accettabile. & I & P  & 2 & \TBstrut \\ [2mm]

	%------->Test plugin
	TI19	& Verificare che la classe processData quando viene inizializzata con un json di configurazione con proprietà 'pluginAim' uguale a 'svm', chiami la funzione PerformPrediction della ProcessSvm che a sua volta chiami il metodo predict di SvmPrediction che ritornerà il risulato della predizione sui dati inseriti con i metodi setter di processData. & I & P & 1 & \TBstrut \\ [2mm]
	TI20	& Verificare che la classe processData quando viene inizializzata con un json di configurazione con proprietà 'pluginAim' uguale a 'rl', chiami la funzione PerformPrediction della ProcessRl che a sua volta chiami il metodo predict di RlPrediction che ritornerà il risulato della predizione sui dati inseriti con i metodi setter di processData. & I & P & 1 & \TBstrut \\ [2mm]
	TI21	& Verificare che la classe WriteInflux venga istanziata correttamente tramite costruttore, creando un'istanza InfluxDB collegata ad un database contenuto nell'immagine docker InfluxDB scaricata da dockerhub. & I & P & 1 & \TBstrut \\ [2mm]
	TI22	& Verificare che i metodi writeArrayToInflux e writePointToInflux scrivano correttamente i dati in un istanza docker influxDB scaricata da dockerhub. & I & P & 1 & \TBstrut \\ [2mm]

	\rowcolor{white}
	\caption{Test di integrazione}
\end{longtable}

\addtocontents{toc}{\protect\setcounter{tocdepth}{4}} %Inserire questo per ripristinare il normale inserimento delle sezioni nell'indice. 4 significa fino al paragrah


\subsection{Test di unità}
\addtocontents{toc}{\protect\setcounter{tocdepth}{0}} %Inserire questo per escludere una sezione dall'indice.

\rowcolors{2}{gray!25}{gray!15}
\begin{longtable} {
		>{}p{12mm}
		>{}p{79.5mm}
		>{}p{9mm}
		>{}p{8mm}
		>{}p{14mm}
		>{}p{0mm}}
	\rowcolor{gray!50}
	\textbf{Codice} & \textbf{Descrizione} & \textbf{Stato} & \textbf{Esito} & \textbf{Priorità}  & \TBstrut \\
	%TU		& & I & P & \TBstrut \\ [2mm]
	%------->VIEW: Button
	TU1			& Verificare che il componente venga renderizzato correttamente controllando che venga effettuata la renderizzazione degli elementi contenuti al suo interno, per dimostrare la correttezza del metodo Render() di  Button. & I & P & 1 & \TBstrut \\ [2mm]
	TU2			& Verificare che le funzioni passate come proprietà al componente Button vengano correttamente chiamate dopo un evento onClick. & I & P & 1 & \TBstrut \\ [2mm]
	TU3			& Verificare che il componente Button renderizzi un testo differente rispetto a quello normale se viene effettuata una operazione asincrona mediante il Button per dimostrare la correttezza del metodo Render() di  Button. & I & P & 2 & \TBstrut \\ [2mm]
	TU4			& Verificare che il componente Button renderizza un testo differente rispetto a quello normale se viene settato come disabilitato per dimostrare la correttezza del metodo Render() di Button. & I & P & 2 & \TBstrut \\ [2mm]
	TU5			& Verificare che il componente Button renderizza un testo aggiuntivo se viene settata la proprietà showMessage a true per dimostrare la correttezza del metodo Render() di  Button. & I & P & 2 & \TBstrut \\ [2mm]
	TU6			& Verificare che il componente Button non renderizzi un testo aggiuntivo se viene settata la proprietà showMessage a false, per dimostrare la correttezza del metodo Render() di  Button. & I & P & 2 & \TBstrut \\ [2mm]
	%------>ChangeParamModal
	TU7			& Verificare che il componente venga renderizzato correttamente controllando che venga effettuata la renderizzazione degli elementi contenuti al suo interno, per dimostrare la correttezza del metodo Render() di  ChangeParamModal. & I & P & 1 & \TBstrut \\ [2mm]
	TU8			& Verificare che le funzioni passate come proprietà al componente ChangeParamModal vengano correttamente chiamate dopo un evento onClick. & I & P & 1 & \TBstrut \\ [2mm]
	TU9			& Verificare che gli elementi selezioni dall'utente vengano salvati nello stato del componente ChangeParamModal con il medesimo ordine per verificare la correttezza del metodo setValues(). & I & P & 1 & \TBstrut \\ [2mm]
	TU10		& Verificare che se l'utente seleziona l'opzione 'seleziona un algoritmo' venga aggiunto, nella posizione corretta, un elemento null. & I & P & 1 & \TBstrut \\ [2mm]
	%------>ControlPanel
	TU11		& Verificare che il componente venga renderizzato correttamente controllando che venga effettuata la renderizzazione degli elementi contenuti al suo interno, per dimostrare la correttezza del metodo Render() di  ControlPanel . & I & P & 1 & \TBstrut \\ [2mm]
	%------->FileInput
	TU12		& Verificare che il componente venga renderizzato correttamente controllando che venga effettuata la renderizzazione degli elementi contenuti al suo interno, per dimostrare la correttezza del metodo Render() di FileInput . & I & P & 1 & \TBstrut \\ [2mm]
	TU13		& Verificare che le funzioni passate come proprietà al componente FileInput vengano correttamente chiamate dopo un evento onChange. & I & P & 1 & \TBstrut \\ [2mm]
	TU14		& Verificare che se l'utente seleziona un file valido, l'elemento dell'elemento input del componente FileInput cambi colore per dimostrare la correttezza del metodo render(). & I & P & 2 & \TBstrut \\ [2mm]
	%------->FolderInput
	TU15		& Verificare che il componente venga renderizzato correttamente controllando che venga effettuata la renderizzazione degli elementi contenuti al suo interno, per dimostrare la correttezza del metodo Render() di  FolderInput. & I & P & 1 & \TBstrut \\ [2mm]
	%------->Graph
	TU16		& Verificare che il componente venga renderizzato correttamente controllando che venga effettuata la renderizzazione degli elementi contenuti al suo interno, per dimostrare la correttezza del metodo Render() di  Graph. & I & P & 1 & \TBstrut \\ [2mm]
	%------->Input
	TU17		& Verificare che il componente venga renderizzato correttamente controllando che venga effettuata la renderizzazione degli elementi contenuti al suo interno, per dimostrare la correttezza del metodo Render() di  Input. & I & P & 1 & \TBstrut \\ [2mm]
	TU18		& Verificare che le funzioni passate come proprietà al componente Input vengano correttamente chiamate dopo un evento onChange. & I & P & 1 & \TBstrut \\ [2mm]
	%------->ResultPanel
	TU19		& Verificare che il componente venga renderizzato correttamente controllando che venga effettuata la renderizzazione degli elementi contenuti al suo interno in caso sia utilizzato l'algoritmo svm, per dimostrare la correttezza del metodo Render() di ResultPanel. & I & P & 1 & \TBstrut \\ [2mm]
	TU20		& Verificare che il componente venga renderizzato correttamente controllando che venga effettuata la renderizzazione degli elementi contenuti al suo interno in caso sia utilizzato l'algoritmo rl, per dimostrare la correttezza del metodo Render() di ResultPanel. & I & P & 1 & \TBstrut \\ [2mm]
	%------->SaveFileModal
	TU21		& Verificare che il componente venga renderizzato correttamente controllando che venga effettuata la renderizzazione degli elementi contenuti al suo interno, per dimostrare la correttezza del metodo Render() di SaveFileModal. & I & P & 1 & \TBstrut \\ [2mm]
	TU22		& Verificare che la funzione corretta, passata come proprietà, venga chiamata dopo aver cliccato il pulsante 'Chiudi'. & I & P & 1 & \TBstrut \\ [2mm]
	TU23		& Verificare che la funzione corretta, passata come proprietà, venga chiamata dopo aver cliccato il pulsante 'Salva Json'. & I & P & 1 & \TBstrut \\ [2mm]
	TU24		& Verificare che la funzione corretta, passata come proprietà, venga chiamata dopo aver cliccato lo sfondo. & I & P & 2 & \TBstrut \\ [2mm]
	%------->SetDataCheckBox
	TU25		& Verificare che il componente venga renderizzato correttamente controllando che venga effettuata la renderizzazione degli elementi contenuti al suo interno, per dimostrare la correttezza del metodo Render() di SetDataCheckBox. & I & P & 1 & \TBstrut \\ [2mm]
	TU26		& Verificare che le funzioni passate come proprietà al componente SetDataCheckBox vengano correttamente chiamate dopo un evento onClick. & I & P & 1 & \TBstrut \\ [2mm]
	%------->QualityIndex
	TU27		& Verificare che venga renderizzato l'elemento del componente QualityIndex del colore rosso se l'indice passato come proprietà è inferiore a 40\%. & I & P & 2 & \TBstrut \\ [2mm]
	TU28		& Verificare che venga renderizzato l'elemento del componente QualityIndex del colore giallo se l'indice passato come proprietà è compreso tra 40\% e 60\%. & I & P & 2 & \TBstrut \\ [2mm]
	TU29		& Verificare che venga renderizzato l'elemento del componente QualityIndex del colore verde se l'indice passato come proprietà è superiore a 60\% & I & P & 2 & \TBstrut \\ [2mm]
	%------->TextArea
	TU30		& Verificare che il componente venga renderizzato correttamente controllando che venga effettuata la renderizzazione degli elementi contenuti al suo interno, per dimostrare la correttezza del metodo Render() di TextArea. & I & P & 1 & \TBstrut \\ [2mm]
	TU31		& Verificare che le funzioni passate come proprietà al componente TextArea vengano correttamente chiamate dopo un evento onChange. & I & P & 1 & \TBstrut \\ [2mm]
	%------->Grid
	TU32		& Verificare che il componente venga renderizzato correttamente controllando che venga effettuata la renderizzazione degli elementi contenuti al suo interno, per dimostrare la correttezza del metodo Render() di Grid. & I & P & 1 & \TBstrut \\ [2mm]
	%------->RenderDataSvm
	TU33		& Verificare che il componente RenderDataSvm esegua correttamente il rendering dei dati usati per l'addestramento per dimostrare la correttezza del metodo Render() di RenderDataSvm. & I & P & 1 & \TBstrut \\ [2mm]
	TU34		& Verificare che il componente RenderDataRl esegua correttamente il rendering dei dati usati per il calcolo della qualità per dimostrare la correttezza del metodo Render() di RenderDataRl . & I & P & 1 & \TBstrut \\ [2mm]
	%------->RenderDataRl
	TU35		& Verificare che il componente RenderDataRl esegua correttamente il rendering dei dati usati per l'addestramento per dimostrare la correttezza del metodo Render() di RenderDataRl . & I & P & 1 & \TBstrut \\ [2mm]
	TU36		& Verificare che il componente RenderDataSvm esegua correttamente il rendering dei dati usati per per il calcolo della qualità per dimostrare la correttezza del metodo Render() di RenderDataSvm. & I & P & 1 & \TBstrut \\ [2mm]
	%------->ScatterPlot
	TU37		& Verificare che il componente venga renderizzato correttamente controllando che venga effettuata la renderizzazione degli elementi contenuti al suo interno, per dimostrare la correttezza del metodo Render() di ScatterPlot. & I & P & 1 & \TBstrut \\ [2mm]
	%------->TrendLine
	TU38		& Verificare che il componente venga renderizzato correttamente controllando che venga effettuata la renderizzazione degli elementi contenuti al suo interno, per dimostrare la correttezza del metodo Render() di TrendLine. & I & P & 1 & \TBstrut \\ [2mm]
	TU39		& Verificare che le funzioni passate come proprietà al componente TrendLine vengano correttamente chiamate nel metodo Render() del componente. & I & P & 1 & \TBstrut \\ [2mm]
	%------->App
	TU40		& Verificare che il componente venga renderizzato correttamente controllando che venga effettuata la renderizzazione degli elementi contenuti al suo interno, per dimostrare la correttezza del metodo Render() di App . & I & P & 1 & \TBstrut \\ [2mm]

	%------->VIEWMODEL: PerformReading
	TU41		& Verificare che venga lanciata un'eccezione in caso venga inizializzata la classe astratta PerformReading. & I & P & 2 & \TBstrut \\ [2mm]
	TU42		& Verificare che venga lanciata un'eccezione in caso venga chiamata la funzione callRead se non è stata implementata in una classe che estende la classe PerformReading. & I & P & 2 & \TBstrut \\ [2mm]
	TU43		& Verificare che venga lanciata un'eccezione in caso venga chiamata la funzione getReader se non è stata implementata in una classe che estende la classe PerformReading. & I & P & 2 & \TBstrut \\ [2mm]
	%------->PerformReadingCsv
	TU44		& Verificare che venga creato l'oggetto concreto ReadCsv. & I & P & 1 & \TBstrut \\ [2mm]
	TU45		& Verificare che venga chiamata la funzione readFile della classe concreta ReadCsv. & I & P & 1 & \TBstrut \\ [2mm]
	%------->PerformReadingJson
	TU46		& Verificare che venga creato l'oggetto concreto ReadJson. & I & P & 1 & \TBstrut \\ [2mm]
	TU47		& Verificare che venga chiamata la funzione readFile della classe concreta ReadJson. & I & P & 1 & \TBstrut \\ [2mm]
	%------->PerformTraining
	TU48		& Verificare che venga lanciata un'eccezione in caso venga inizializzata la classe astratta PerformTraining. & I & P & 2 & \TBstrut \\ [2mm]
	TU49		& Verificare che venga lanciata un'eccezione in caso venga chiamata la funzione callTrain se non è stata implementata in una classe che estende la classe PerformTraining. & I & P & 2 & \TBstrut \\ [2mm]
	TU50		& Verificare che venga lanciata un'eccezione in caso venga chiamata la funzione getTrainer se non è stata implementata in una classe che estende la classe PerformTraining. & I & P & 2 & \TBstrut \\ [2mm]
	%------->PerformTrainingRl
	TU51		& Verificare che venga creato l'oggetto concreto RlTrainer. & I & P & 1 & \TBstrut \\ [2mm]
	TU52		& Verificare che le funzioni della classe concreta RlTrainer vengano chiamate con i parametri corretti. & I & P & 1 & \TBstrut \\ [2mm]
	%------->PerformTrainingSvm
	TU53		& Verificare che venga creato l'oggetto concreto SvmTrainer. & I & P & 1 & \TBstrut \\ [2mm]
	TU54		& Verificare che le funzioni della classe concreta SvmTrainer vengano chiamate con i parametri corretti. & I & P & 1 & \TBstrut \\ [2mm]
	%------->PerformWriting
	TU55		& Verificare che venga lanciata un'eccezione in caso venga inizializzata la classe astratta PerformWriting. & I & P & 2 & \TBstrut \\ [2mm]
	TU56		& Verificare che venga lanciata un'eccezione in caso venga chiamata la funzione callWrite se non è stata implementata in una classe che estende la classe PerformWriting. & I & P & 2 & \TBstrut \\ [2mm]
	TU57		& Verificare che venga lanciata un'eccezione in caso venga chiamata la funzione getWriter se non è stata implementata in una classe che estende la classe PerformWriting. & I & P & 2 & \TBstrut \\ [2mm]
	%------->PerformWritingJson
	TU58		& Verificare che venga creato l'oggetto concreto WriteJson. & I & P & 1 & \TBstrut \\ [2mm]
	TU59		& Verificare che venga chiamata con i parametri corretti la funzione writeToDisk della classe WriteJson. & I & P & 1 & \TBstrut \\ [2mm]
	%------->ProcessReading
	TU60		& Verificare che venga creato l'oggetto concreto corretto a seconda dell'estensione del file che si sta leggendo. & I & P & 1 & \TBstrut \\ [2mm]
	TU61		& Verificare che la funzione getPath restituisca il path del file corretto all'interno della classe ProcessReading. & I & P & 1 & \TBstrut \\ [2mm]
	TU62		& Verificare che la funzione callRead dell'oggetto creato a seconda della estensione del file venga chiamata. & I & P & 1 & \TBstrut \\ [2mm]
	%------->ProcessTraining
	TU63		& Verificare che venga creato l'oggetto concreto corretto a seconda dell'algoritmo che si vuole addestrare. & I & P & 1 & \TBstrut \\ [2mm]
	TU64		& Verificare che la funzione getParams ritorni tutti i parametri utilizzati per l'addestramento. & I & P & 1 & \TBstrut \\ [2mm]
	TU65		& Verificare che venga chiamata la funzione callTrain della strategia concreta. & I & P & 1 & \TBstrut \\ [2mm]
	%------->ProcessWriting
	TU66		& Verificare che venga creato l'oggetto concreto corretto a seconda della estensione del file che si sta scrivendo. & I & P & 1 & \TBstrut \\ [2mm]
	TU67		& Verificare che la funzione getFileInfo ritorni tutte i dati necessari per la scrittura di un file su disco. & I & P & 1 & \TBstrut \\ [2mm]
	TU68		& Verificare che venga chiamata la funzione callWrite della strategia concreta. & I & P & 1 & \TBstrut \\ [2mm]
	%------->MODEL: Read
	TU69		& Verificare che venga lanciata un'eccezione in caso venga inizializzata la classe astratta Read. & I & P & 2 & \TBstrut \\ [2mm]
	TU70		& Verificare che venga lanciata un'eccezione in caso venga chiamata la funzione parser se non è stata implementata in una classe che estende la classe Read. & I & P & 2 & \TBstrut \\ [2mm]
	TU71		& Verificare che il metodo readFile legga correttamente il file selezionato dall'utente. & I & P & 1 & \TBstrut \\ [2mm]
	TU72		& Verificare che il metodo readFile lanci un'eccezione nel caso in cui ci fosse un errore nella lettura del file caricato dall'utente. & I & P & 2 & \TBstrut \\ [2mm]
	%------->Write
	TU73		& Verificare che venga lanciata un'eccezione in caso venga inizializzata la classe astratta Write. & I & P & 2 & \TBstrut \\ [2mm]
	TU74		& Verificare che venga lanciata un'eccezione in caso venga chiamata la funzione parser se non è stata implementata in una classe che estende la classe Write. & I & P & 2 & \TBstrut \\ [2mm]
	TU75		& Verificare che il metodo buildTrainedFile ritorni un file javascript che contiene tutte le informazioni riguardanti l'addestramento e il suo risultato. & I & P & 1 & \TBstrut \\ [2mm]
	TU76		& Verificare che il metodo writeToDisk scriva il file correttamente. & I & P & 1 & \TBstrut \\ [2mm]
	%------->WriteJson
	TU77		& Verificare che la funzione parser ritorni una stringa da un oggetto javascript. & I & P & 1 & \TBstrut \\ [2mm]
	%------->ReadCsv
	TU78		& Verificare che la funzione parser ritorni un array di oggetti javascript dato un file csv. & I & P & 1 & \TBstrut \\ [2mm]
	%------->ReadJson
	TU79		& Verificare che la funzione parser ritorni un array di oggetti javascript dato una stringa che rappresenta i contenuti di un file Json. & I & P & 1 & \TBstrut \\ [2mm]
	%------->RlTrainer
	TU80		& Verificare che la funzione insertData inserisca nel campo 'data' l'array dei dati inseriti dall'utente se il numero dei parametri è >= 3. & I & P & 1 & \TBstrut \\ [2mm]
	TU81		& Verificare che la funzione insertData inserisca nel campo 'data' l'array dei dati inseriti dall'utente se il numero dei parametri è < 3. & I & P & 1 & \TBstrut \\ [2mm]
	TU82		& Verificare che il metodo train chiami il metodo train dal modulo esterno regression.module. & I & P & 1 & \TBstrut \\ [2mm]
	TU83		& Verificare che il metodo buildTrainedObject ritorni un oggetto javascript che contiene i parametri corretti. & I & P & 1 & \TBstrut \\ [2mm]
	TU84		& Verificare che la funzione setParams imposti tutti i parametri corretti nella classe RlTrainer. & I & P & 1 & \TBstrut \\ [2mm]
	TU85		& Verificare che la funzione setOptions imposti tutte le opzioni corrette nella classe RlTrainer. & I & P & 1 & \TBstrut \\ [2mm]
	TU86		& Verificare che il metodo getQualityIndex ritorni il valore corretto. & I & P & 1 & \TBstrut \\ [2mm]
	%------->SvmTrainer
	TU87		& Verificare che la funzione translateData ritorni un array di dati con un numero di parametri < 4 se il csv in input ha meno di 4 predittori. & I & P & 2 & \TBstrut \\ [2mm]
	TU88		& Verificare che la funzione translateData ritorni un array di dati con un numero di parametri >= 4 se il csv in input ha 4 o più predittori. & I & P & 2 & \TBstrut \\ [2mm]
	TU89		& Verificare che il metodo train chiami il metodo train del modulo esterno ml-modules. & I & P & 1 & \TBstrut \\ [2mm]
	TU90		& Verificare che il metodo buildTrainerObject ritorni un oggetto js con i parametri corretti. & I & P & 1 & \TBstrut \\ [2mm]
	TU91		& Verificare che il metodo setParams imposti i parametri corretti nella classe SvmTrainer . & I & P & 1 & \TBstrut \\ [2mm]
	TU92		& Verificare che il metodo getQualityIndex ritorni il valore corretto. & I & P & 1 & \TBstrut \\ [2mm]

	%------->ATTENZIONE INIZIO TEST PLUGIN
	%TU		& & I & P & \TBstrut \\ [2mm]
	%------->panelCtrl
	TU93		& Verificare che il costruttore imposti correttamente la gestione di eventi per il pannello. & I & P & 1 & \TBstrut \\ [2mm]
	TU94		& Verificare il funzionamento del metodo 'link' che tramite una chiamata jQuery andrà ad ottenere l'elemento html in cui andrà successivamente inserito il grafico. & I & P & 1 & \TBstrut \\ [2mm]
	TU95		& Verificare che il metodo getter per il contenitore del grafico ritorni il contenuto della proprietà '\_graphDiv' della classe. & I & P & 2 & \TBstrut \\ [2mm]
	TU96		& Verificare che in seguito all'inizializzazione del pannello sia assegnato un numero di versione tra le proprietà del pannello. & I & P & 2 & \TBstrut \\ [2mm]
	TU97		& Verificare che la versione del pannello salvata all'interno della configurazione non venga aggiornata se già superiore o uguale a quella del grafico. & I & P & 2 & \TBstrut \\ [2mm]
	TU98		& Verificare che la funzione 'deleteJsonClick' cancelli la configurazione di predizione attualmente caricata e visualizzi il messaggio di completamento dell'operazione. & I & P & 1 & \TBstrut \\ [2mm]
	TU99		& Verificare che la funzione 'uploadJsonClick' carichi una nuova configurazione di predizione da un file JSON e visualizzi il messaggio di completamento dell'operazione. & I & P & 1 & \TBstrut \\ [2mm]
	TU100		& Verificare che venga visualizzato un errore se il file JSON caricato dall'utente è firmato da un autore diverso da 'VRAMSoftware'. & I & P & 2 & \TBstrut \\ [2mm]
	TU101		& Verificare che il metodo 'confirmQueries' crei una nuova mappa contenente una lista di query e la passi al controller, chiamando infine il metodo 'onChange'. & I & P & 1 & \TBstrut \\ [2mm]
	TU102		& Verificare che venga visualizzato un errore se l'utente seleziona la stessa fonte dati a più predittori nelle impostazioni del pannello. & I & P & 2 & \TBstrut \\ [2mm]
	TU103		& Verificare che il metodo 'confirmDatabaseSettings' crei un oggetto contente tutte le informazioni necessarie all'accesso al database e lo passi al controller, chiamando infine il metodo 'onChange'. & I & P & 1 & \TBstrut \\ [2mm]
	TU104		& Verificare che il metodo 'onInitEditMode' effettui tutte le chiamate necessarie ad avviare la modalità di modifica del pannello. & I & P & 1 & \TBstrut \\ [2mm]
	TU105		& Verificare che in caso il grafico sia già in modalità modifica, non possa essere attivato il pannello di impostazioni del plug-in. & I & P & 2 & \TBstrut \\ [2mm]
	TU106		& Verificare il caricamento di dati tramite flussi di dati o snapshot. & I & P & 1 & \TBstrut \\ [2mm]
	TU107		& Verificare che il metodo per l'aggiornamento delle fonti dati costruisca una nuova lista delle fonti dati, la compari con quella salvata nella configurazione del plug-in, e la sostituisca solo nel caso vi siano delle differenze. & I & P & 1 & \TBstrut \\ [2mm]
	TU108		& Verificare che il metodo 'onDataError', collegato all'evento di errore nell'aggiornamento della dashboard, chiami il metodo 'plotlyOnDataError' del grafico e il metodo 'render' del pannello. & I & P & 2 & \TBstrut \\ [2mm]
	TU109		& Verificare che il metodo 'onResize', collegato all'evento che segnala il cambiamento delle dimensioni del pannello, chiami il metodo apposito per aggiornare le dimensioni del grafico. & I & P & 2 & \TBstrut \\ [2mm]
	TU110		& Verificare che il metodo 'onRender', collegato all'evento di render del pannello, chiami il metodo di rendering del grafico. & I & P & 1 & \TBstrut \\ [2mm]
	TU111		& Verificare che il metodo 'onRefresh', collegato all'evento di aggiornamento del pannello, chiami il metodo di aggiornamento del grafico. & I & P & 2 & \TBstrut \\ [2mm]
	TU112		& Verificare che in caso non sia presente l'elemento contenitore per il grafico, non venga chiamato il metodo per il ridimensionamento. & I & P & 2 & \TBstrut \\ [2mm]
	TU113		& Verificare che in caso non sia presente l'elemento contenitore per il grafico, non venga chiamato il metodo per l'aggiornamento. & I & P & 2 & \TBstrut \\ [2mm]
	TU114		& Verificare che il metodo 'onRender' non avvii il render del grafico se un altro pannello è in modalità schermo intero. & I & P & 2 & \TBstrut \\ [2mm]
	TU115		& Verificare che il metodo 'onRefresh' non avvii il refresh del grafico se un altro pannello è in modalità schermo intero. & I & P & 2 & \TBstrut \\ [2mm]
	TU116		& Verificare che un nuovo pannello, in assenza di configurazioni salvate all'interno di Grafana, carichi le configurazioni di default. & I & P & 1 & \TBstrut \\ [2mm]
	%------->config
	TU117		& Verificare che il costruttore della classe 'GrafanaPredictionControl' inizializzi tutti i campi dell'oggetto. & I & P & 1 & \TBstrut \\ [2mm]
	TU118		& Verificare che il metodo 'postUpdate' possa aggiornare lo stato della classe 'GrafanaPredictionControl' in 'disabilitato'. & I & P & 1 & \TBstrut \\ [2mm]
	TU119		& Verificare che il metodo 'postUpdate' possa aggiornare lo stato della classe 'GrafanaPredictionControl' in 'abilitato'. & I & P & 1 & \TBstrut \\ [2mm]
	TU120		& Verificare che il costruttore non sovrascriva la configurazione già salvata in Grafana\glo. & I & P & 1 & \TBstrut \\ [2mm]
	%------->datasource
	TU121		& Verificare il funzionamento del metodo statico di duplicazione di una fonte dati. & I & P & 1 & \TBstrut \\ [2mm]
	TU122		& Verificare il funzionamento del metodo per ottenere l'indirizzo della fonte dati. & I & P & 2 & \TBstrut \\ [2mm]
	TU123		& Verificare il funzionamento del metodo per ottenere la porta della fonte dati. & I & P & 2 & \TBstrut \\ [2mm]
	TU124		& Verificare il funzionamento del metodo per ottenere il nome del database della fonte dati. & I & P & 2 & \TBstrut \\ [2mm]
	TU125		& Verificare il funzionamento del metodo per ottenere il nome utente per l'accesso al database della fonte dati. & I & P & 2 & \TBstrut \\ [2mm]
	TU126		& Verificare il funzionamento del metodo 'getPassword' per ottenere la password del database della fonte dati. & I & P & 2 & \TBstrut \\ [2mm]
	TU127		& Verificare il funzionamento del metodo 'getUrl' per ottenere l'url del database della fonte dati. & I & P & 2 & \TBstrut \\ [2mm]
	TU128		& Verificare il corretto funzionamento del metodo che restituisce il tipo di una fonte dati. & I & P & 2 & \TBstrut \\ [2mm]
	TU129		& Verificare il corretto funzionamento del metodo che restituisce il nome di una fonte dati. & I & P & 2 & \TBstrut \\ [2mm]
	TU130		& Verificare il corretto funzionamento del metodo per ottenere l'id che Grafana\glosp ha assegnato alla fonte dati. & I & P & 1 & \TBstrut \\ [2mm]
	TU131		& Verificare il funzionamento del metodo di clonazione di una fonte dati. & I & P & 1 & \TBstrut \\ [2mm]
	TU132		& Verificare il corretto funzionamento del metodo di clonazione di una fonte dati con database alternativo. & I & P & 2 & \TBstrut \\ [2mm]
	TU133		& Verificare che il metodo per la clonazione di una fonte dati con un nome alternativo lanci un'eccezione se non viene passato un parametro contenente il nuovo nome. & I & P & 2 & \TBstrut \\ [2mm]
	TU134		& Verificare il corretto funzionamento del metodo di comparazione tra fonti dati in base all'indirizzo dell'host. & I & P & 1 & \TBstrut \\ [2mm]
	TU135		& Verificare che alla costruzione della fonte dati sia lanciata un'eccezione se non è presente un parametro 'URL' valido. & I & P & 2 & \TBstrut \\ [2mm]
	TU136		& Verificare che alla costruzione della fonte dati sia lanciata un'eccezione se non è presente un parametro 'type' valido. & I & P & 2 & \TBstrut \\ [2mm]
	TU137		& Verificare che alla costruzione della fonte dati sia lanciata un'eccezione se non è presente un nome. & I & P & 2 & \TBstrut \\ [2mm]
	TU138		& Verificare che il metodo statico per la clonazione lanci un'eccezione se la fonte dati passata non ha un formato valido. & I & P & 2 & \TBstrut \\ [2mm]
	%------->processdata
	TU139		& Verificare che il metodo 'getStrategy' della classe 'ProcessData' ritorni la strategia correntemente inizializzata nel controller. & I & P & 1 & \TBstrut \\ [2mm]
	TU140		& Verificare che l'inizializzazione della classe 'ProcessData' con la strategia SVM\glosp avvenga con successo. & I & P & 1 & \TBstrut \\ [2mm]
	TU141		& Verificare che l'inizializzazione della classe 'ProcessData' con la strategia di regressione lineare\glosp avvenga con successo. & I & P & 1 & \TBstrut \\ [2mm]
	TU142		& Verificare che se viene avviato il controller con un parametro 'datalist' non valido venga segnalato un errore. & I & P & 2 & \TBstrut \\ [2mm]
	TU143		& Verificare che se viene avviato il controller con un parametro mancante venga segnalato un errore. & I & P & 2 & \TBstrut \\ [2mm]
	%------->selectinfluxdbtab
	TU144		& Verificare il funzionamento del costruttore del componente 'SelectInfluxDBCtrl'. & I & P & 1 & \TBstrut \\ [2mm]
	TU145		& Verificare il funzionamento del metodo 'getDatasources' chiamato dal costruttore del componente 'SelectInfluxDBCtrl'. & I & P & 1 & \TBstrut \\ [2mm]
	TU146		& Verificare il funzionamento del metodo che si occupa dell'aggiornamento dei parametri per l'accesso al database in seguito alla selezione da parte dell'utente. & I & P & 1 & \TBstrut \\ [2mm]
	TU147		& Verificare che venga visualizzato un messaggio di errore nel caso sia stato inserito un nome del database errato. & I & P & 2 & \TBstrut \\ [2mm]
	TU148		& Verificare che venga visualizzato un messaggio di errore nel caso sia stata selezionata una fonte di dati non inizializzata. & I & P & 2 & \TBstrut \\ [2mm]
	TU149		& Verificare che venga visualizzato un messaggio di errore nel caso non sia stato inserito un nome per il campo chiave nel database. & I & P & 2 & \TBstrut \\ [2mm]
	TU150		& Verificare che venga visualizzato un messaggio di errore nel caso non sia stato inserito un measurement dove scrivere i dati nel database. & I & P & 2 & \TBstrut \\ [2mm]
	TU151		& Verificare il funzionamento del metodo 'SelectInfluxDBDirective' utilizzato da Grafana\glosp per inizializzare il componente 'SelectInfluxDBCtrl'. & I & P & 1 & \TBstrut \\ [2mm]
	TU152		& Verificare che il metodo 'getDataSources' del componente 'SelectInfluxDBCtrl' ignori i database che non sono di tipo InfluxDB. & I & P & 1 & \TBstrut \\ [2mm]
	TU153		& Verificare che in seguito ad una risposta http incompleta non venga effettuato l'aggiornamento della lista di database. & I & P & 1 & \TBstrut \\ [2mm]
	TU154		& Verificare che il costruttore inizializzi una lista vuota di fonti dati nel caso questa non sia già presente. & I & P & 1 & \TBstrut \\ [2mm]
	%------->writeinflux
	TU155		& Verificare che se nel costruttore il parametro "host" non è valorizzato venga lanciato un errore. & I & P & 2 & \TBstrut \\ [2mm]
	TU156		& Verificare che se nel costruttore il parametro "port" non è valorizzato venga lanciato un errore. & I & P & 2 & \TBstrut \\ [2mm]
	TU157		& Verificare che se nel costruttore il parametro "database" non è valorizzato venga lanciato un errore. & I & P & 2 & \TBstrut \\ [2mm]
	TU158		& Verificare che se il database esiste già non venga creato. & I & P & 1 & \TBstrut \\ [2mm]
	TU159		& Verificare che se nel costruttore il parametro "measurement" non è valorizzato venga lanciato un errore. & I & P & 2 & \TBstrut \\ [2mm]
	TU160		& Verificare che se nel costruttore il parametro "fieldKey" non è valorizzato venga lanciato un errore. & I & P & 2 & \TBstrut \\ [2mm]
	TU161		& Verificare che il costruttore funzioni senza errori se hai i corretti parametri in input. & I & P & 1 & \TBstrut \\ [2mm]

	\rowcolor{white}
	\caption{Test di unità}
\end{longtable}
\addtocontents{toc}{\protect\setcounter{tocdepth}{4}}


\subsection{Tabella riassuntiva dei test}
\begin{longtable} {
		>{}p{42.85mm}
		>{}p{47mm}
		>{}p{47mm}
		>{}p{0mm}}
	\rowcolor{gray!50}

	\textbf{Tipologia test}	& \textbf{Numero test priorità 1}	& \textbf{Numero test priorità 2} 	  \TBstrut \\ [2mm]
	 Test di accettazione   & 15                                 & 13						\TBstrut \\ [2mm]
	 Test di sistema	    & 9                                 & 7					\TBstrut \\ [2mm]
	 Test di integrazione   & 18                                 & 4						\TBstrut \\ [2mm]
	 Test di unità		    & 98                                & 63						\TBstrut \\ [2mm]

	\rowcolor{white}
	\caption{Tabella riassuntiva dei test}
\end{longtable}


