\section{Qualità di processo}
Per misurare la qualità di processo\glosp il gruppo ha definito degli obiettivi di qualità e delle metriche\glosp che li rendano quantificabili prendendo come riferimento le norme della serie ISO 9000. Tali norme definiscono i requisiti per la realizzazione di un sistema di gestione della qualità, al fine di ottenere dei processi\glosp efficaci ed efficienti.
	\subsection{PRC-Q2 Processo di sviluppo}
		\subsubsection{OP-1 Individuazione completa dei requisiti} 
			Ci prefiggiamo individuare in modo corretto e completo i requisiti sin da subito per evitare modifiche che comportano un grosso dispendio di tempo.
			\paragraph{Metriche di qualità} \mbox{}
			\rowcolors{2}{gray!25}{gray!15}
			\begin{longtable} {
					>{}p{80mm} 
					>{}p{25mm}
					>{}p{25mm}
				}
				\rowcolor{gray!50}
				\textbf{Metrica} & \textbf{Preferibile} & \textbf{Accettabile} \TBstrut \TBstrut \\
				M-PROC01 Scostamento dei requisiti individuati & 0 & $\le 10$ \TBstrut \\ [2mm]
			\end{longtable}
			
		\subsubsection{OP-2 Sviluppo di codice comprensibile e manutenibile}
			Ci prefiggiamo di scrivere codice che segua le norme di codifica indicate nelle \textit{Norme di Progetto} e sia manutenibile nel tempo, al fine di garantire comprensibilità e manutenibilità del codice.
			\paragraph{Metriche di qualità} \mbox{}
			\begin{longtable} {
					>{}p{80mm} 
					>{}p{25mm}
					>{}p{25mm}
				}
				\rowcolor{gray!50}
				\textbf{Metrica} & \textbf{Preferibile} & \textbf{Accettabile} \TBstrut \TBstrut \\
				M-PROC02 Numero di parametri per metodo & $ \le 3$ & $ \le 5$ \TBstrut \\ [2mm]
				M-PROC03 Numero di metodi per classe & $ \le 8$ & $ \le 15$ \TBstrut \\ [2mm]
				M-PROC20 Livello di annidamento & $1 \le x \le 3$ & $1 \le x \le 7$ \TBstrut \\ [2mm]
			\end{longtable}
	
		\subsubsection{OP-9 Forte disaccoppiamento tra le componenti architetturali}
		Ci prefiggiamo di progettare l'architettura del prodotto\glosp software con un forte disaccoppiamento tra le componenti in modo che sia efficace, efficiente e facilmente manutenibile;
		\paragraph{Metriche di qualità} \mbox{}
		\rowcolors{2}{gray!25}{gray!15}
		\begin{longtable} {
				>{}p{80mm} 
				>{}p{25mm}
				>{}p{25mm}
			}
			\rowcolor{gray!50}
			\textbf{Metrica} & \textbf{Preferibile} & \textbf{Accettabile} \TBstrut \TBstrut \\
			M-PROC21 Profondità della gerarchia & $\le 4$ & $\le 7$ \TBstrut \\ [2mm]
			M-PROC22 Numero di design pattern & $5 \le x \le 6$ & $2 \le x \le 15$ \TBstrut \\ [2mm]
		\end{longtable}
			
	\subsection{PRC-Q5 Processo di garanzia della qualità}
		\subsubsection{OP-3 Monitoraggio della qualità}
			Ci prefiggiamo di monitorare la qualità dei processi\glosp e dei prodotti\glosp al fine di ottenere un controllo e un miglioramento continuo. 
			\paragraph{Metriche di qualità} \mbox{} 
			\begin{longtable} {
					>{}p{80mm} 
					>{}p{25mm}
					>{}p{25mm}
				}
				\rowcolor{gray!50}
				\textbf{Metrica} & \textbf{Preferibile} & \textbf{Accettabile} \TBstrut \TBstrut \\
				M-PROC04 Percentuale di metriche soddisfatte & $100\%$ & $\ge 60\%$ \TBstrut \\ [2mm]
			\end{longtable}

	\subsection{PRC-Q6 Processo di verifica}
		\subsubsection{OP-4 Efficacia dei test}
			Ci prefiggiamo di svolgere una verifica efficace su ogni parte del nostro prodotto\glo.
			\paragraph{Metriche di qualità} \mbox{}
			\begin{longtable} {
					>{}p{80mm} 
					>{}p{25mm}
					>{}p{25mm}
				}
				\rowcolor{gray!50}
				\textbf{Metrica} & \textbf{Preferibile} & \textbf{Accettabile} \TBstrut \TBstrut \\
				M-PROC05 Percentuale bug sistemati & $100\%$ & $100\%$ \TBstrut \\ [2mm]
			\end{longtable}
		\subsubsection{OP-5 Completezza dei test}
		Ci prefiggiamo di svolgere una verifica completa su ogni parte del nostro prodotto\glo.
		\paragraph{Metriche di qualità} \mbox{}
		\begin{longtable} {
				>{}p{80mm} 
				>{}p{25mm}
				>{}p{25mm}
			}
			\rowcolor{gray!50}
			\textbf{Metrica} & \textbf{Preferibile} & \textbf{Accettabile} \TBstrut \TBstrut \\
			M-PROC23 Code coverage & $100\%$ & $\ge 80\%$ \TBstrut \\ [2mm]
			M-PROC24 Branch coverage & $100\%$ & $\ge 80\%$ \TBstrut \\ [2mm]
			M-PROC25 Copertura dei test eseguiti & $\ge 90\%$ & $100\%$ \TBstrut \\ [2mm]
		\end{longtable}	
			
	\subsection{PRC-Q8 Processo di gestione dei cambiamenti}
		\subsubsection{OP-6 Risoluzione efficace dei problemi}
			Ci prefiggiamo di gestire ogni cambiamento necessario in tempi ragionevolmente brevi.
			\paragraph{Metriche di qualità} \mbox{}			
			\begin{longtable} {
					>{}p{80mm} 
					>{}p{25mm}
					>{}p{25mm}
				}
				\rowcolor{gray!50}
				\textbf{Metrica} & \textbf{Preferibile} & \textbf{Accettabile} \TBstrut \TBstrut \\
				M-PROC06 Tempo medio risoluzione errori & $\le 10$ minuti & $\le 120$ minuti \TBstrut \\ [2mm]
			\end{longtable}					

	\subsection{PRC-Q9 Processo di gestione organizzativa}
		\subsubsection{OP-7 Pianificazione efficace delle risorse}
			Ci prefiggiamo di rispettare le tempistiche e i costi indicati nel preventivo nel documento \textit{Piano di Progetto}.
			\paragraph{Metriche di qualità} \mbox{} 
			\begin{longtable} {
					>{}p{60mm} 
					>{}p{35mm}
					>{}p{50mm}
				}
				\rowcolor{gray!50}
				\textbf{Metrica} & \textbf{Preferibile} & \textbf{Accettabile} \TBstrut \TBstrut \\
				M-PROC07 Planned Value & $\ge$ 0 & $\ge$ 0 \TBstrut \\ [2mm]
				M-PROC08 Earned Value & = PV & $\ge$ 0 \TBstrut \\ [2mm]
				M-PROC09 Actual cost & 0 $\le$ AC $\le$ PV &0 $\le$ AC $\le$ budget totale \TBstrut \\ [2mm]				
				M-PROC10 Cost Performance Index & 1 & 0.95 $\le$ CPI $\le$ 1.05 \TBstrut \\ [2mm]				
				M-PROC11 Schedule Performance Index & 1 & 0.95 $\le$ SPI $\le$ 1.05 \TBstrut \\ [2mm]				
				M-PROC12 Estimated Cost at Completion & quanto preventivato & preventivo-5\% $\le$ EAC $\le$ preventivo+5\% \TBstrut \\ [2mm]
				M-PROC13 Schedule at Completion & quanto preventivato & quanto preventivato \TBstrut \\ [2mm]				
			\end{longtable}

		\subsubsection{OP-8 Prevenzione dei rischi} 
			Ci prefiggiamo di individuare fin da subito i rischi in modo completo per evitare il cambiamento degli stessi nel tempo e prevenirli.
			\paragraph{Metriche di qualità} \mbox{} 
			\begin{longtable} {
					>{}p{80mm} 
					>{}p{25mm}
					>{}p{25mm}
				}
				\rowcolor{gray!50}
				\textbf{Metrica} & \textbf{Preferibile} & \textbf{Accettabile} \TBstrut \TBstrut \\
				M-PROC14 Rischi non preventivati & 0 & $ \le 5$ \TBstrut \\ [2mm]
			\end{longtable}
			

	\subsection{Tabella riassuntiva delle metriche adottate}
	\rowcolors{2}{gray!25}{gray!15}
	\begin{longtable} {
		>{}p{50mm}  
		>{}p{80mm}
		}

		\rowcolor{gray!50}
		\multicolumn{2}{c}{\textbf{PRC-Q2 Processo di sviluppo}}\\
	\rowcolor{gray!50}
	\textbf{Obiettivi} & \textbf{Metriche} \TBstrut \\ [2mm]

		OP-1 Individuazione completa dei requisiti &
		M-PROC01 Scostamento dei requisiti individuati \TBstrut \\ [2mm]

		OP-2 Sviluppo di codice comprensibile e manutenibile &
		M-PROC02 Numero di parametri per metodo \newline
		M-PROC03 Numero di metodi per classe \newline
		M-PROC20 Livello di annidamento \TBstrut \\ [2mm]
		
		OP-9 Forte disaccoppiamento tra le componenti architetturali &
		M-PROC21 Profondità della gerarchia \newline
		M-PROC22 Numero di design pattern \TBstrut \\ [2mm]

		\rowcolor{gray!50}
		\multicolumn{2}{c}{\textbf{PRC-Q5 Processo di garanzia della qualità}}\\
	\rowcolor{gray!50}
	\textbf{Obiettivi} & \textbf{Metriche} \TBstrut \\ [2mm]

		OP-3 Monitoraggio della qualità &
		M-PROC04 Percentuale di metriche soddisfatte \TBstrut \\ [2mm]
		
	\rowcolor{gray!50}
	\multicolumn{2}{c}{\textbf{PRC-Q6 Processo di verifica}}\\
	\rowcolor{gray!50}
	\textbf{Obiettivi} & \textbf{Metriche} \TBstrut \\ [2mm]

		OP-4 Efficacia dei test &
		M-PROC05 Percentuale bug sistemati \TBstrut \\ [2mm]
		OP-5 Completezza dei test & 
		M-PROC23 Code coverage \newline
		M-PROC24 Branch coverage \newline
		M-PROC25 Copertura dei test eseguiti \TBstrut \\ [2mm]

	\rowcolor{gray!50}
	\multicolumn{2}{c}{\textbf{PRC-Q8 Processo di gestione dei cambiamenti}}\\
	\rowcolor{gray!50}
	\textbf{Obiettivi} & \textbf{Metriche} \TBstrut \\ [2mm]

		OP-6 Risoluzione efficace dei problemi &
		M-PROC06 Tempo medio risoluzione errori \TBstrut \\ [2mm]

	\rowcolor{gray!50}
		\multicolumn{2}{c}{\textbf{PRC-Q9 Processo di gestione organizzativa}}\\
	\rowcolor{gray!50}
		\textbf{Obiettivi} & \textbf{Metriche} \TBstrut \\ [2mm]

		OP-7 Pianificazione efficace delle risorse & 
		M-PROC07 Planned Value \newline
		M-PROC08 Earned Value \newline 
		M-PROC09 Actual cost \newline
		M-PROC10 Cost Performance Index \newline
		M-PROC11 Schedule Performance Index \newline
		M-PROC12 Estimated Cost at Completion \newline
		M-PROC13 Schedule at Completion \TBstrut \\ [2mm]

		OP-8 Prevenzione dei rischi & 
		M-PROC14 Rischi non preventivati \TBstrut \\ [2mm]

		\rowcolor{white}
		\caption{Tabella riassuntiva metriche\glosp adottate per la qualità di processo\glo}
	\end{longtable}