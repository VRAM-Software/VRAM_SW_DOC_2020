\section{Test}
	\subsection{Esecuzione dei test}
		\subsubsection{Applicativo di addestramento}
			Una volta avviato il server NodeJS, i test di unità con Enzyme e Jest vengono avviati con il comando \textbf{npm test} scegliendo l'opzione \textbf{a}. I loro risultati vengono visualizzati direttamente nel terminale. Inoltre i test vengono rieseguiti automaticamente ad ogni modifica del loro codice.
		\subsubsection{Plug-in}
			I test di unità con Jest vengono avviati con il comando \textbf{npm run test} e il loro esito viene visualizzato direttamente nel terminale. In alternativa, possono essere eseguiti tramite il comando \textbf{npm run ci-test} il quale, oltre all'esecuzione dei test, effettua il calcolo della copertura sul codice totale visualizzandola tramite una tabella nel terminale.
	\subsection{Scrittura dei test}
		Per ogni componente viene creato un file di test, con nome \textbf{<nome componente>.test.js} per l'applicativo esterno e \textbf{<nome componente>.test.ts} per il plug-in, all'interno del quale vengono definiti. L'obiettivo dei test di unità è verificare il funzionamento e la correttezza dei singoli componenti, 
		analizzandone la costruzione, il caricamento, l'elaborazione dei dati e le chiamate ad altri metodi.
	\subsection{SonarCloud}
		SonarCloud misura la qualità del codice. Infatti ad ogni commit analizza i sorgenti ed estrae indici su manutenibilità, complessità, affidabilità, duplicazione del codice, numero di bug e vulnerabilità. L'obiettivo è quello di avere codice sempre in buono stato e pronto al rilascio.
	\subsection{Copertura dei test}
		La copertura dei test viene controllata da Coveralls, che ad ogni nuova build eseguita dalla continuous integration esegue il calcolo e ne mantiene uno storico. Per accedere alle statistiche sulla copertura dei test bisogna premere sull'apposito badge nelle repository\glosp che apre la rispettiva pagina sul sito di Coveralls. Nel caso del plug-in, è possibile misurare la copertura anche tramite l'apposito comando di test.
