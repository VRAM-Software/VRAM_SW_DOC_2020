\subsection{Configurazione ambiente applicazione di addestramento}
\subsubsection{Contenuto file: package.json}
Il file \verb|package.json | contiene tutte le informazioni e le dipendenze necessarie della nostra applicazione.
\begin{itemize}
    \item \textbf{"main"}: questo attributo ha come valore il percorso del file che si occupa di gestire il processo principale dell'applicazione Electron;
    \item \textbf{"dependencies"}: questo attributo contiene la seguente lista di pacchetti che sono necessari per il corretto funzionamento dell'applicazione:
        \begin{itemize}
            \item \textbf{csvtojson}: il pacchetto npm csvtojson è utilizzato per gestire la conversione da un file csv ad un oggetto JavaScript;
            \item \textbf{d3}: la libreria d3 è utilizzata per gestire la renderizzazione del grafico, in particolare la gestione delle unità di misura;
            \item \textbf{electron-is-dev}: il pacchetto npm electron-is-dev è utilizzato per verificare se l'istanza dell'applicazione di addestramento è stata inizializzata in modalità sviluppo o produzione;
            \item \textbf{ml-modules}: la libreria ml-modules è utilizzata per l'implementazione dell'algoritmo di predizione SVM\glo;
            \item \textbf{precision-recall}: il pacchetto npm precision-recall è utilizzato per calcolare il risultato di un addestramento SVM\glo;
            \item \textbf{react}: il framework react viene utilizzato per gestire tutti gli aspetti relativi alla vista;
            \item \textbf{react-dom}: pacchetto da utilizzare insieme al framework react per gestire le modifiche del DOM per react;
            \item \textbf{react-scripts}: pacchetto da utilizzare insieme al framework react per gestire gli script necessari per lo sviluppo dell'applicazione come l'inizializzazione del server Node.Js di sviluppo;
            \item \textbf{@testing-library/jest-dom}: libreria per scrivere test con jest per il DOM;
            \item \textbf{@testing-library/react}: libreria per scrivere test con jest per react;
            \item \textbf{@testing-library/user-event}: libreria per scrivere test con jest simulando eventi che possono essere eseguiti da utenti.
        \end{itemize}
    \item \textbf{"devDependencies"}: questo attributo contiene la seguente lista di pacchetti necessari per il corretto funzionamento dell'applicazione durante lo sviluppo:
        \begin{itemize}
            \item \textbf{electron}: il framework electron è utilizzato per gestire il funzionamento di una applicazione desktop sviluppata con tecnologie web;
            \item \textbf{electron-packager}: il pacchetto npm electron-packager è utilizzato per creare un eseguibile partendo dal codice sorgente di una applicazione electron;
            \item \textbf{coveralls}: utilizzata per il calcolo del code coverage dei test automatici;
            \item \textbf{prettier}: pacchetto npm per effettuare una correzione automatica dell'indentazione del codice;
            \item \textbf{enzyme}: pacchetto npm utilizzato che fornisce delle funzioni per testare componenti React;
            \item \textbf{enzyme-adapter-react-16}: pacchetto npm da utilizzare con enzyme per scrivere test per componenti React;
            \item \textbf{spectron}: framework utilizzato per scrivere test non automatici per verificare il funzionamento corretto di una applicazione electron.
        \end{itemize} 
    \item \textbf{"scripts"}: questo attributo contiene una lista di tutti i comandi, utili per uno sviluppatore, che possono essere eseguiti da linea di comando:
        \begin{itemize}
            \item \textbf{electron}: il comando seguente avvia un'istanza di Electron;
            \begin{verbatim}
            	npm run electron
            \end{verbatim}
            \item \textbf{start}: il comando seguente avvia un server Node.js alla porta 3000;
            \begin{verbatim}
	            npm start
            \end{verbatim}
            \item \textbf{build}: il comando seguente genera una cartella: \verb|build| che contiene i file di produzione di React;
            \begin{verbatim}
            	npm build
            \end{verbatim}
            \item \textbf{test}: il comando seguente esegue tutti i test dell'applicazione;\\
            \begin{verbatim}
            	npm test
            \end{verbatim}
            \item \textbf{beautify}: il comando seguente esegue un controllo e correzione dell'indentazione del codice;\\
            \begin{verbatim}
            	npm run beautify
            \end{verbatim}
            \item \textbf{package-mac}: il comando seguente genera una release di produzione\glosp per il sistema operativo: MacOS;
            \begin{verbatim}
            	npm run package-mac
            \end{verbatim}
            \item \textbf{package-win}: il comando seguente genera una release di produzione\glosp per il sistema operativo: Windows;
            \begin{verbatim}
            	npm run package-win
            \end{verbatim}
            \item \textbf{package-linux}: il comando seguente genera una release di produzione\glosp per una qualsiasi distribuzione di Linux.
            \begin{verbatim}
            	npm run package-linux
            \end{verbatim}
        \end{itemize}
\end{itemize}
