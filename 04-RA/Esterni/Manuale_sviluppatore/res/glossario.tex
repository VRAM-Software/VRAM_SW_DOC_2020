\section{Glossario}

\subsection*{A}
\subsubsection*{Analisi statica}
Valutazione di un sistema o di un suo componente basato sulla sua forma, sulla sua struttura, sul suo contenuto e sulla documentazione di riferimento. Ciò significa che la valutazione avviene senza l'esecuzione del sistema o dell'oggetto dell'analisi.

\subsection*{B}
\subsubsection*{Binding}
Collegamento fra una entità di un software.
\subsubsection*{Business logic}
Logica di elaborazione che rende operativa un'applicazione, in altre parole implementa gli specifici algoritmi di manipolazione dei dati che caratterizzano l'applicazione.

\subsection*{C}
\subsubsection*{Callback}
Meccanismo tramite il quale un blocco di codice viene passato come parametro a dell'altro codice, quest'ultimo eseguirà il codice passato.

\subsection*{D}
\subsubsection*{Design pattern}
Soluzione progettuale generale e riusabile ad un problema ricorrente nell'ambito dell'ingegneria del software.
\subsubsection*{Dipendenze}
Lista di pacchetti che sono necessari a un software per funzionare.

\subsection*{G}
\subsubsection*{Grafana}
Software ad uso generico per la produzione di cruscotti informativi (dashboard in inglese) e composizione di grafici. Viene utilizzato come un'applicazione web.

\subsection*{M}
\subsubsection*{Machine learning}
Il machine learning (in italiano apprendimento automatico) è una branca dell'intelligenza artificiale che utilizza metodi statistici per migliorare progressivamente la performance di un algoritmo nell'identificare pattern nei dati.

\subsection*{P}
\subsubsection*{Path di sistema}
Percorso che parte dalla root directory\glosp e arriva fino al file desiderato.
\subsubsection*{Prodotto}
Si definisce prodotto qualsiasi bene scambiabile sul mercato che può rispondere alle esigenze di un compratore. Un esempio di prodotto informatico è il software che è composto dal codice e dalla documentazione.	
	
\subsection*{R}
\subsubsection*{Release di debug}
Prodotto compilato destinato agli sviluppatori che facilita l'individuazione di errori.

\subsubsection*{Release di produzione}
Prodotto compilato e pronto per essere usato dall'utente finale.


\subsubsection*{Repository}
Repository significa archivio. In un repository sono raccolti dati e informazioni in formato digitale, valorizzati e archiviati sulla base di metadati che ne permettono la rapida individuazione, anche grazie alla creazione di tabelle relazionali. Grazie alla sua peculiare architettura, un repository consente di gestire in modo ottimale anche grandi volumi di dati.
\subsubsection*{Root directory}
La root directory di una qualsiasi partizione, è la cartella più "alta" nella gerarchia delle cartelle.
\subsubsection*{RL}
Acronimo di Regressione lineare, algoritmo di machine learning\glosp che ha la funzione di prevedere un valore di una variabile dipendente (y) in base a una determinata variabile indipendente (x) secondo una relazione di tipo lineare.
\subsection*{S}
\subsubsection*{SVM}
Acronimo di Support Vector Machine; algoritmo di apprendimento automatico supervisionato che può essere utilizzato sia per scopi di classificazione che di regressione.

\subsection*{V}
\subsubsection*{Versionamento}
Il controllo di versione è un sistema che registra, nel tempo, le modifiche effettuate ad un file o ad una serie di file, permettendo così di recuperare una specifica versione dei file stessi in un secondo momento. Permette inoltre ad un team di collaborare in modo efficiente facilitando l'individuazione e la risoluzione di conflitti.

