% \textbf = grassetto; \Large = font più grande
% \rowcolors{quanti colori alternare}{colore numero riga pari}{colore numero riga dispari}: colori alternati per riga
% \rowcolor{color}: cambia colore di una riga
% p{larghezza colonna}: p è un tipo di colonna di testo verticalmente allineata sopra, ci sarebbe anche m che è centrata a metà ma non è precisa per questo utilizzo TBStrut; la sintassi >{\centering} indica che il contenuto della colonna dovrà essere centrato
% \TBstrut fa parte di alcuni comandi che ho inserito in config.tex che permetto di aggiungere un po' di padding al testo
% \\ [2mm] : questra scrittura indica che lo spazio dopo una break line deve essere di 2mm
% 
%\setcounter{secnumdepth}{0}
%\hfill \break
%\textbf{\Large{Diario delle modifiche}} \\
\section{Riepilogo delle decisioni}
\rowcolors{2}{gray!25}{gray!15}
\begin{longtable} {
		>{\centering}p{17mm} 
		%>{\centering}p{19.5mm}
		%>{\centering}p{24mm} 
		%>{\centering}p{24mm} 
		>{}p{120mm}}
	\rowcolor{gray!50}
	\textbf{Codice} & \multicolumn{1}{c}{\textbf{Decisione}} \\
	% \\ %\TBstrut \\
	VI\_14.1 & Revisione e completamento dell'incremento 13. \TBstrut \\ [2mm]
	VI\_14.2 & Scelta di MVC come architettura per il plug-in e di MVVM per l'applicativo esterno. \TBstrut \\ [2mm]
	VI\_14.3 & Scelta di node-influx come client esterno per la gestione della scrittura nel database InfluxDB. \TBstrut \\ [2mm]
\end{longtable}