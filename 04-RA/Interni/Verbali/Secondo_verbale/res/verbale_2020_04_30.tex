\section{Informazioni generali}
    \subsection{Informazioni incontro}
        \begin{itemize}
            \item \textbf{Luogo}: Videochiamata tramite Skype;
            \item \textbf{Data}: 2020-04-30;
            \item \textbf{Ora d'inizio}: 15.00;
            \item \textbf{Ora di fine}: 16.30;
            \item \textbf{Partecipanti}: 
            \begin{itemize}
                \item Corrizzato Vittorio;
                \item Dalla Libera Marco;
                \item Rampazzo Marco;
                \item Santagiuliana Vittorio;
                \item Schiavon Rebecca;
                \item Spreafico Alessandro;
                \item Toffoletto Massimo.
            \end{itemize}
        \end{itemize}
    \subsection{Argomenti trattati}
        Durante la riunione, i componenti del gruppo si sono confrontati riguardo i seguenti punti:
		\begin{itemize}
			\item implementazione della gestione degli errori nell'incremento 17;
			\item requisiti opzionali;
			\item adattamento plug-in a nuova versione di Grafana\glo.
		\end{itemize}
\section{Verbale}
    \subsection{Punto 1} 
    Come da pianificazione, durante il secondo periodo del macro periodo di validazione\glosp e collaudo, abbiamo deciso di sviluppare l'incremento 17, cioè l'implementazione della funzionalità di visualizzazione degli errori in caso di inserimento di file non validi ed errata mappatura dei predittori con il flusso di dati ricevuto dalla datasource.
    \subsection{Punto 2}
    Due componenti del gruppo hanno comunicato che il loro impegno in attività esterne al progetto avrà un peso maggiore negli ultimi periodi del progetto, perciò il gruppo ha deciso di aumentare l'impegno orario dei rimanenti componenti di qualche ora, pur rimanendo entro le 104 ore.
    \subsection{Punto 3}
    Recentemente è stata rilasciata una nuova versione di Grafana\glosp (6.7), la quale ha rimosso le dipendenze di AngularJS all'interno di Grafana\glo, causando problemi di retrocompatibità con il plugin. Procederemo quindi a fare modifiche mirate pere estendere il supporto all'ultima versione di Grafana\glo.
