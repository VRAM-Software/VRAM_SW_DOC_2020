\section{Informazioni generali}
    \subsection{Informazioni incontro}
        \begin{itemize}
            \item \textbf{Luogo}: Videochiamata tramite Skype;
            \item \textbf{Data}: 2020-04-23;
            \item \textbf{Ora d'inizio}: 09.30;
            \item \textbf{Ora di fine}: 11.00;
            \item \textbf{Partecipanti}: 
            \begin{itemize}
                \item Corrizzato Vittorio;
                \item Dalla Libera Marco;
                \item Rampazzo Marco;
                \item Santagiuliana Vittorio;
                \item Schiavon Rebecca;
                \item Spreafico Alessandro;
                \item Toffoletto Massimo.
            \end{itemize}
        \end{itemize}
    \subsection{Argomenti trattati}
    Durante la riunione, i componenti del gruppo si sono confrontati riguardo i seguenti punti:
    \begin{itemize}
    	\item implementazione indici di qualità dell'addestramento nell'incremento 8;
    	\item interfaccia grafica per gli indici di qualità;
    	\item scelta del valore k da usare nel calcolo degli indici di qualità.
    \end{itemize}
    
\section{Verbale}
    \subsection{Punto 1}
     Come da pianificazione, durante il primo periodo del macro periodo di validazione\glosp e collaudo, abbiamo deciso di sviluppare l'incremento 8, cioè l'implementazione del calcolo e visualizzazione degli indici di qualità dell'addestramento.
    \subsection{Punto 2}
    Il gruppo ha discusso della rappresentazione grafica degli indici di qualità e ha deciso che verranno rappresentati all'interno di rettangoli, i quali avranno un colore diverso in base al valore dell'indice.
    \subsection{Punto 3}
    Infine, il gruppo ha discusso in dettaglio il calcolo degli indici di qualità e ha stabilito che nella tecnica della convalidazione incrociata (nota in inglese come k-fold) k avrà un valore pari a 3. Il valore è stato scelto perché rappresenta un compromesso accettabile di affidabilità e velocità di calcolo.
        