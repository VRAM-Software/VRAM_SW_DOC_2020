\section{Informazioni generali}
    \subsection{Informazioni incontro}
        \begin{itemize}
            \item \textbf{Luogo}: Dipartimento di Matematica "Tullio Levi-Civita";
            \item \textbf{Data}: 2019-11-18;
            \item \textbf{Ora d'inizio}: 14.00;
            \item \textbf{Ora di fine}: 16.30;
            \item \textbf{Partecipanti}: \begin{itemize}
                \item Corrizzato Vittorio;
                \item Dalla Libera Marco;
                \item Rampazzo Marco;
                \item Santagiuliana Vittorio;
                \item Schiavon Rebecca;
                \item Spreafico Alessandro;
                \item Toffoletto Massimo.
            \end{itemize}
        \end{itemize}
    \subsection{Argomenti trattati}
        Durante la prima riunione, i componenti hanno discusso e deliberato le seguenti scelte:
        \begin{enumerate}
            \item nome e logo del gruppo;
            \item strumenti di supporto da utilizzare;
            \item discussione e scelta del capitolato\glo;
            \item divisione dei ruoli;
            \item discussione riguardante la normazione dei processi\glosp utili per l'inizio della produzione di documenti.
        \end{enumerate}
\section{Verbale}
    \subsection{Punto 1}
        Inizialmente il gruppo ha dibattuto per decidere un nome e un logo a esso associato. Ne è seguita una votazione in cui i componenti hanno
        concordato per \textit{VRAM Software} e per l'immagine a fronte di questo verbale (che verrà poi utilizzata nei successivi documenti).
    \subsection{Punto 2}
        In seguito, si è discusso sugli strumenti da utilizzare per una gestione ottimale del materiale di lavoro. La scelta è ricaduta sulle seguenti tecnologie:
        \begin{itemize}
            \item \textbf{Git}: come software di controllo versione;
            \item \textbf{GitHub}: come servizio di hosting;
            \item \textbf{Slack} e \textbf{Telegram}: come canali di comunicazione;
            \item \textbf{Google Calendar}: per organizzare riunioni e appuntare scadenze;
            \item \textbf{Google Drive}: per la condivisione di documenti o applicativi utili alla redazione o allo sviluppo.
        \end{itemize}
    \subsection{Punto 3}
        Successivamente i partecipanti hanno condiviso le loro opinioni riguardanti i vari capitolati\glo, e dopo aver esposto vantaggi e svantaggi di ognuno, hanno concordato sulla scelta di C4 (\textit{Predire in Grafana}\glo) proposto dall'azienda \textit{Zucchetti}.
    \subsection{Punto 4}
        A seguito della scelta del capitolato\glosp è stata effettuata una prima divisione dei ruoli per l'inizio della produzione della documentazione:
        \begin{itemize}
            \item \textbf{Responsabile}: Toffoletto Massimo;
            \item \textbf{Amministratore}: Corrizzato Vittorio;
            \item \textbf{Analisti}: Dalla Libera Marco, Rampazzo Marco e Spreafico Alessandro;
            \item \textbf{Verificatori}: Schiavon Rebecca e Santagiuliana Vittorio.
        \end{itemize}
    \subsection{Punto 5}
        Infine, è stato deciso di cominciare la stesura delle \textit{Norme di Progetto} in quanto necessarie per i documenti successivi poiché fornirà le regole a cui attenersi per la produzione di elaborati corretti e tra loro coerenti. Le sezioni da cui è cominciata la redazione sono state:
        \begin{itemize}
            \item \textit{Studio di Fattibilità};
            \item strumenti da utilizzare;
            \item documentazione;
            \item gestione della configurazione;
            \item processo\glosp di verifica;
            \item processi\glosp organizzativi;
        \end{itemize}