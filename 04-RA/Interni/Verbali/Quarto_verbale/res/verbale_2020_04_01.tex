\section{Informazioni generali}
    \subsection{Informazioni incontro}
        \begin{itemize}
            \item \textbf{Luogo}: Videochiamata tramite Skype;
            \item \textbf{Data}: 2020-04-01;
            \item \textbf{Ora d'inizio}: 15.00;
            \item \textbf{Ora di fine}: 19.00;
            \item \textbf{Partecipanti}: \begin{itemize}
                \item Corrizzato Vittorio;
                \item Dalla Libera Marco;
                \item Rampazzo Marco;
                \item Santagiuliana Vittorio;
                \item Schiavon Rebecca;
                \item Spreafico Alessandro;
                \item Toffoletto Massimo.
            \end{itemize}
        \end{itemize}
    \subsection{Argomenti trattati}
        Durante questo incontro il gruppo ha discusso sulle seguenti tematiche:
        \begin{itemize}
            \item sviluppo dell'incremento 6;
            \item decisioni in seguito all'incontro con il prof. Cardin;
            \item discussione riguardo alla stesura dell'allegato tecnico.
        \end{itemize}
\section{Verbale}
    \subsection{Punto 1}
        Il gruppo ha parlato dello sviluppo dell'incremento 13 pianificato per questo periodo. È stata constata la corretta implementazione delle funzionalità di scelta dell'algoritmo di predizione da addestrare e conseguente avvio dell'addestramento. I componenti del gruppo procederanno alla verifica fino alla fine del periodo per eventuali bug-fix per poi confermare il completamento dell'incremento.
    \subsection{Punto 2}
        In seguito ad un incontro con il prof. Cardin il gruppo ha avuto occasione di discutere dell'architettura del prodotto\glo. I membri, in questa riunione, hanno concordato sulla creazione di due diagrammi dei package distinti e sulla revisione di alcune scelte architetturali che potrebbero essere risultate forzate.
    \subsection{Punto 3}
        Visto la già avviata discussione riguardante l'architettura il gruppo ha iniziato a discutere sulla composizione dell'allegato tecnico che verrà presentato al prof. Cardin in occasione della Product Baseline. Sono quindi stati divisi i ruoli ed è stato deciso uno scheletro per suddetto documento.