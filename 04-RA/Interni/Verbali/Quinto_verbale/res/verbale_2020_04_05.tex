\section{Informazioni generali}
    \subsection{Informazioni incontro}
        \begin{itemize}
            \item \textbf{Luogo}: Videochiamata tramite Skype;
            \item \textbf{Data}: 2020-04-05;
            \item \textbf{Ora d'inizio}: 15.00;
            \item \textbf{Ora di fine}: 19.00;
            \item \textbf{Partecipanti}: \begin{itemize}
                \item Corrizzato Vittorio;
                \item Dalla Libera Marco;
                \item Rampazzo Marco;
                \item Santagiuliana Vittorio;
                \item Schiavon Rebecca;
                \item Spreafico Alessandro;
                \item Toffoletto Massimo.
            \end{itemize}
        \end{itemize}
    \subsection{Argomenti trattati}
        In questo meeting i membri hanno parlato dei seguenti argomenti:
        \begin{itemize}
            \item incremento 13;
            \item continuazione della discussione riguardo all'architettura;
            \item discussione riguardo alla stesura del manuale utente.
        \end{itemize}
\section{Verbale}
    \subsection{Punto 1}
        Inizialmente i componenti del gruppo hanno discusso sulla codifica e verifica dell'incremento 13 pianificato per il periodo di progettazione\glosp di dettaglio e codifica. Quest'ultimo, dopo una revisione di gruppo, è risultato completato e le funzioni di lettura del file JSON e configurazione del plug-in risultano usabili.
    \subsection{Punto 2}
        Successivamente è continuata la discussione riguardo alla struttura dell'architettura. Sono stati individuati come design pattern architetturali il Model View Controller per il plug-in e il Model View ViewModel per l'applicativo esterno. In seguito il gruppo approfondirà l'analisi sulla struttura del software per ricercare punti critici dove applicare altri design pattern creazionali, comportamentali o strutturali. 
    \subsection{Punto 3}
        In seguito è stato revisionato il proseguimento della stesura del manuale dello sviluppatore, il quale risulta ad un buon livello di completamento, ed è stata organizzata la redazione del manuale utente tramite l'organizzazione di uno scheletro e la divisione delle varie sezioni.