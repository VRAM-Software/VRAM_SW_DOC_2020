% \textbf = grassetto; \Large = font più grande
% \rowcolors{quanti colori alternare}{colore numero riga pari}{colore numero riga dispari}: colori alternati per riga
% \rowcolor{color}: cambia colore di una riga
% p{larghezza colonna}: p è un tipo di colonna di testo verticalmente allineata sopra, ci sarebbe anche m che è centrata a metà ma non è precisa per questo utilizzo TBStrut; la sintassi >{\centering} indica che il contenuto della colonna dovrà essere centrato
% \TBstrut fa parte di alcuni comandi che ho inserito in config.tex che permetto di aggiungere un po' di padding al testo
% \\ [2mm] : questra scrittura indica che lo spazio dopo una break line deve essere di 2mm
% 
\addtocontents{toc}{\protect\setcounter{tocdepth}{0}} %Inserire questo per escludere una sezione dall'indice.
%\hfill \break
%\textbf{\Large{Diario delle modifiche}} \\
\section*{Registro delle modifiche}
\rowcolors{2}{gray!25}{gray!15}
\begin{longtable} {
		>{\centering}p{17mm} 
		>{\centering}p{19.5mm}
		>{\centering}p{24mm} 
		>{\centering}p{24mm} 
		>{}p{32mm}}
	\rowcolor{gray!50}
	\textbf{Versione} & \textbf{Data} & \textbf{Nominativo} & \textbf{Ruolo} & \textbf{Descrizione} \TBstrut \\
	1.1.1 & 2020-02-18 & Schiavon Rebecca & \textit{Responsabile di progetto} & Approvazione documento. \TBstrut \\ [2mm]
	0.1.0 & 2020-02-17 & Corrizzato Vittorio e Santagiuliana Vittorio & \textit{Analista} e \textit{Verificatore} & Stesura e verifica documento. \TBstrut \\ [2mm]
\end{longtable}
\addtocontents{toc}{\protect\setcounter{tocdepth}{4}}