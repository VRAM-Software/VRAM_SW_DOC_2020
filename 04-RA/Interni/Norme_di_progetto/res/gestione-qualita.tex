\subsection{PRC-Q5 Garanzia della qualità}
\subsubsection{Scopo}
Lo scopo di questo processo\glosp è fornire adeguate garanzie di qualità in merito ai processi implementati durante lo svolgimento dell'intero progetto\glo.
\subsubsection{Aspettative}
Le aspettative del processo\glosp di garanzia della qualità sono riassunte nei seguenti punti che rappresentano gli obiettivi che vogliono essere raggiunti. Si vuole ottenere:
\begin{itemize}
	\item qualità per tutta la durata del ciclo di vita del software;
	\item garanzia di raggiungimento degli obiettivi di qualità prestabiliti;
	\item soddisfazione del cliente in merito al prodotto\glosp sviluppato;
	\item oggettività nella ricerca della qualità in modo da ottenerla in tutti i processi\glosp e in tutti i prodotti\glo;
	\item qualità misurabile attraverso delle metriche\glo.
\end{itemize}
\subsubsection{Descrizione}
Per garantire la qualità dello sviluppo di questo progetto\glosp si è fatto riferimento allo standard ISO 9000:2015. Esso afferma che il sistema di qualità, cioè l'insieme dei metodi e regole per garantire la qualità di un prodotto\glo, è costituito da tre passaggi:
\begin{itemize}
	\item sviluppo;
	\item controllo;
	\item miglioramento continuo.
\end{itemize}
Questi tre passaggi sono ripetuti per ogni processo\glosp e il loro scopo è raggiungere gli obiettivi di qualità prefissati per il prodotto\glosp software, i documenti e i processi.
Per assicurare che la qualità sia misurabile, fissiamo delle metriche associate alle attività che ci permettono di misurarla. I valori soglia e le misurazioni di tali metriche sono contenute nel documento: \textit{Piano di Qualifica v. 20.0.0}.
La misurazione di queste metriche deve essere fatta al termine di ogni singolo breve periodo indicato nella progettazione inserita nel documento \textit{Piano di Progetto v. 20.0.0} in modo da monitorare e apportare miglioramenti al nostro way of working\glosp costantemente.
\subsubsection{Attività}
\paragraph{Classificazione delle metriche}
In questo documento vengono definite delle metriche che garantiscono qualità di processo\glosp e di prodotto\glosp in modo misurabile.
Per avere un riferimento univoco alle metriche che permetta di individuarle facilmente all'interno dei documenti, viene utilizzato un codice identificativo. Questo codice avrà la seguente struttura: \\
\textbf{M-[Tipo][Numero]} \\
dove 
\begin{itemize}
	\item \textbf{M}: indica che si tratta di una metrica;
	\item \textbf{Tipo} può essere:
	\begin{itemize}
		\item \textbf{PROC} se la metrica si riferisce ad un processo;
		\item \textbf{PROD} se la metrica si riferisce ad un prodotto;
	\end{itemize}
	\item \textbf{Numero}: numero intero progressivo di due cifre.
\end{itemize}

\paragraph{Classificazione dei processi}
Per una migliore associazione dei processi con il documento \textit{Piano di Qualifica} viene definita una codifica con il seguente formato: \\
\\ \textbf{PRC-Q[Numero]} \\
\\ Dove:
\begin{itemize}
	\item \textbf{PRC-Q}: indica che si tratta di un processo;
	\item \textbf{Numero}: è un numero intero progressivo maggiore di zero.
\end{itemize}

\paragraph{Classificazione dei prodotti}
Per una migliore associazione dei prodotti con il documento \textit{Piano di Qualifica} viene definita una codifica con il seguente formato: \\
\\ \textbf{PRD-Q[Numero]} \\
\\ Dove:
\begin{itemize}
	\item \textbf{PRD-Q}: indica che si tratta di un prodotto;
	\item \textbf{Numero}: è un numero intero progressivo maggiore di zero.
\end{itemize}

\paragraph{Classificazione degli obiettivi di processo}
Per una migliore associazione degli obiettivi di qualità di processo con il documento \textit{Piano di Qualifica}, viene definita una codifica con il seguente formato: \\
\\ \textbf{OP-[Numero]} \\
\\ Dove:
\begin{itemize}
	\item \textbf{OP}: indica che si tratta di un obiettivo di qualità di processo;
	\item \textbf{Numero}: è un numero intero progressivo maggiore di zero.
\end{itemize}

\paragraph{Classificazione delle caratteristiche dei prodotti}
Per una migliore associazione delle caratteristiche dei prodotti con il documento \textit{Piano di Qualifica}, viene definita una codifica con il seguente formato: \\
\\ \textbf{CP-[Numero]} \\
\\ Dove:
\begin{itemize}
	\item \textbf{CP}: indica che si tratta di un caratteristica di prodotto;
	\item \textbf{Numero}: è un numero intero progressivo maggiore di zero.
\end{itemize}

\subsubsection{Obiettivi di qualità}
	\paragraph{OP-3 Monitoraggio della qualità}
		%\paragraph*{Obiettivo}\mbox{}\\ [1mm]
		L'obiettivo è garantire che un controllo della qualità al fine di perseguire il miglioramento continuo.
		\subparagraph{Metriche}\mbox{}\\ [1mm]
		Le metriche\glosp utilizzate sono:
			\begin{itemize}
				\item \textbf{M-PROC04 Percentuale di metriche soddisfatte}: indica la percentuale di metriche\glosp soddisfatte rispetto alla totalità di metriche\glosp utilizzare all'interno del progetto\glo;
				\begin{itemize}
					\item[] \textbf{Formula}: $\frac{numero \; di \; metriche \; soddisfatte}{numero \; di \; metriche \; totali}$.
				\end{itemize}
			\end{itemize}
%\subsubsection{Strumenti di supporto}
%Gli strumenti utilizzati per il processo\glosp di garanzia della qualità sono le metriche di qualità.

