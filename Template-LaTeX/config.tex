% Configurazioni varie, da inserire nel preambolo dopo i pacchetti

\hypersetup{hidelinks} %serve per nascondere riquadri rossi che circondano i link 

\lstset{literate= {à}{{\`a}}1 } %Permette di usare lettere accentate nei listings

\pagestyle{fancy} %Imposto stile pagina
\fancyhf{} %Reset, se lo tolgo LaTex mette impostazioni di default (p.es numerazione pagine di default)

\lhead{VRAM Software} %Left header che compare in ogni pagina
\rhead{\leftmark} %Nome della top-level structure (p.es. Section in article o Chapter in book) in ogni pagina
\lfoot{Template LaTeX V0.1} %Left foot in ogni pagina

%Setto il colore dei link
\hypersetup{
	colorlinks,
	linkcolor=[HTML]{404040},
	citecolor={purple!50!black},
	urlcolor={blue!50!black}
}

%Tabelle e tabulazione (può tornare utile)
%\setlength{\tablcolsep}{10pt}
%\renewcommand{\arraystretch}{1.4}

%Comando per aggiungere le pagine di ogni sezione
\newcommand{\newSection}[1]{%
	\newpage
	\input{res/sections/#1}
}

% Comandi per aggiungere padding a parole contenute nella tabella; è una specie di strut (un carattere invisibile)
\newcommand\Tstrut{\rule{0pt}{2.6ex}} % top padding
\newcommand\Bstrut{\rule[-0.9ex]{0pt}{0pt}} % bottom padding
\newcommand{\TBstrut}{\Tstrut\Bstrut} % top & bottom padding

\newcommand{\glo}{$_G$} %Comando per aggiungere il pedice G
\newcommand{\glosp}{$_G$ } %Comando per aggiungere il pedice G con spazio