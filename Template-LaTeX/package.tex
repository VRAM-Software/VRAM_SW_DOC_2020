% Tutti i pacchetti usati, da inserire nel preambolo prima delle configurazioni

\usepackage[T1]{fontenc} %Permette la sillabazione su qualsiasi testo contenente caratteri
\usepackage[utf8]{inputenc} %Serve per usare la codifica utf-8
\usepackage[english,italian]{babel} %Imposta italiano lingua principale, inglese secondaria. Es. serve per far apparire "indice" al posto di "contents"

\usepackage{graphicx} %Serve per includere le immagini

\usepackage{hyperref} %Gestisce i riferimenti/link. Es. Serve per rendere clickabili le sezioni dell'indice

\usepackage{float} %Serve per migliore la definizione di oggetti fluttuanti come figure e tabelle. Es. poter usare l'opzione [H] nelle figure ovvero tenere fissate le immagini che altrimenti LaTeX si sposta a piacere.

\usepackage{listings} %Serve per poter mettere snippets di codice nel testo

\usepackage{lastpage} %Serve per poter introdurre un'etichetta a cui si può fare riferimento Es. piè di pagina; poter fare " \rfoot{\thepage\ di \pageref{LastPage}} "

\usepackage{fancyhdr} %Per header e piè di pagina personalizzati

%Sono alcuni package che potranno esserci utili in futuro
%\usepackage{charter}
%\usepackage{eurosym}
\usepackage{subcaption}
%\usepackage{wrapfig}
%\usepackage{background}
\usepackage{longtable} % tabella che può continuare per più di una pagina
\usepackage[table]{xcolor} % ho dovuto aggiungere table in modo da poter colorare le row della tabella, dava: undefined control sequences
%\usepackage{colortbl}

\usepackage{dirtree} % usato per creare strutte tree-view in stile filesystem
\usepackage{xspace} % usato per inserire caratteri spazio