%Preambolo: la parte prima del \begin{document}

\documentclass[12pt,a4paper]{article} %formato del documento e grandezza caratteri

% Tutti i pacchetti usati, da inserire nel preambolo prima delle configurazioni

\usepackage[T1]{fontenc} %Permette la sillabazione su qualsiasi testo contenente caratteri
\usepackage[utf8]{inputenc} %Serve per usare la codifica utf-8
\usepackage[english,italian]{babel} %Imposta italiano lingua principale, inglese secondaria. Es. serve per far apparire "indice" al posto di "contents"

\usepackage{graphicx} %Serve per includere le immagini

\usepackage[hypertexnames=false]{hyperref} %Gestisce i riferimenti/link. Es. Serve per rendere clickabili le sezioni dell'indice

\usepackage{float} %Serve per migliore la definizione di oggetti fluttuanti come figure e tabelle. Es. poter usare l'opzione [H] nelle figure ovvero tenere fissate le immagini che altrimenti LaTeX si sposta a piacere.

\usepackage{listings} %Serve per poter mettere snippets di codice nel testo

\usepackage{lastpage} %Serve per poter introdurre un'etichetta a cui si può fare riferimento Es. piè di pagina; poter fare " \rfoot{\thepage\ di \pageref{LastPage}} "

\usepackage{fancyhdr} %Per header e piè di pagina personalizzati

%Sono alcuni package che potranno esserci utili in futuro
%\usepackage{charter}
%\usepackage{eurosym}
\usepackage{subcaption}
%\usepackage{wrapfig}
%\usepackage{background}
\usepackage{longtable} % tabella che può continuare per più di una pagina
\usepackage[table]{xcolor} % ho dovuto aggiungere table in modo da poter colorare le row della tabella, dava: undefined control sequences
%\usepackage{colortbl}

\usepackage{dirtree} % usato per creare strutte tree-view in stile filesystem
\usepackage{xspace} % usato per inserire caratteri spazio
\usepackage[official]{eurosym}
\usepackage{pdflscape} %include il file package.tex

% Configurazioni varie, da inserire nel preambolo dopo i pacchetti

\hypersetup{hidelinks} %serve per nascondere riquadri rossi che circondano i link 

\lstset{literate= {à}{{\`a}}1 } %Permette di usare lettere accentate nei listings

\pagestyle{fancy} %Imposto stile pagina
\fancyhf{} %Reset, se lo tolgo LaTex mette impostazioni di default (p.es numerazione pagine di default)

\lhead{\includegraphics[scale=0.25]{img/Logo_header.png}} %Left header che compare in ogni pagina
\rhead{\leftmark} %Nome della top-level structure (p.es. Section in article o Chapter in book) in ogni pagina

%Setto il colore dei link
\hypersetup{
	colorlinks,
	linkcolor=[HTML]{404040},
	citecolor={purple!50!black},
	urlcolor={blue!50!black}
}

%Tabelle e tabulazione (può tornare utile)
%\setlength{\tablcolsep}{10pt}
%\renewcommand{\arraystretch}{1.4}

%Comando per aggiungere le pagine di ogni sezione
\newcommand{\newSection}[1]{%
	\newpage
	\input{res/sections/#1}
}

% Comandi per aggiungere padding a parole contenute nella tabella; è una specie di strut (un carattere invisibile)
\newcommand\Tstrut{\rule{0pt}{2.6ex}} % top padding
\newcommand\Bstrut{\rule[-0.9ex]{0pt}{0pt}} % bottom padding
\newcommand{\TBstrut}{\Tstrut\Bstrut} % top & bottom padding

\newcommand{\glo}{$_G$} %Comando per aggiungere il pedice G
\newcommand{\glosp}{$_G$ } %Comando per aggiungere il pedice G con spazio

\begin{document}

% #### FRONTESPIZIO (frontmatter) ####
\setlength{\headheight}{39pt} %distanzia l'header
\pagenumbering{gobble} %Toglie il numero di pagina
\begin{titlepage}
	\begin{center}
		\includegraphics[scale=0.6]{img/logo.png} \\ %Logo
		\vspace{0.5cm} %Aggiunge uno spazio verticale di 0.5 cm
		
		{\LARGE Progetto "Predire in Grafana"} \\ %Autore, prende variabile definita nel config_latex.tex
		\vspace{0.5cm} %Attenzione a mettere il punto e NON la virgola
		
		{\Huge \textbf{\DocTitle}} \\ %Titolo, prende variabile definita nel config_latex.tex
		\vspace{0.5cm}
		
		\DocDate \\ %Data, prende variabile definita nel config_latex.tex
		\vspace{1cm}
		
		%Allineamento colonne: l=left r=right c=center, 
		%va specificato per ogni colonna
		%Se si vuole la riga tra colonne mettere "|"
		
		\begin{tabular}{r | l} %Elementi colonne separate da "&", le righe finiscono con "\\"
			Versione & \ver \\
			Approvazione & \app \\ %
			Redazione & \red \\
			Verifica & \test \\
			Stato & \stat \\
			Uso & \use \\
		    Destinato a & Azienda \\
						& Prof. Tullio Vardanega \\
						& Prof. Riccardo Cardin \\
			Email di riferimento& vram.software@gmail.com
		\end{tabular}
		\vfill
		\textbf{Descrizione} \\
		\DocDesc
	\end{center}
\end{titlepage}
\clearpage

% #### INDICE (tableofcontents )####
%\pagenumbering{Roman} %Pagine con i  numeri romani + reset a I
%\renewcommand{\footrulewidth}{1pt} %Di default footrulewidth==0 e quindi è invisibile
%\tableofcontents %Provoca la stampa dell'indice
%\label{LastPageRoman} %Indica l'ultima pagina dell'indice
%\rfoot{\thepage\ di \pageref{LastPageRoman}} %Pagina n di m, con numeri Romani
%\clearpage

% #### PARTE PRINCIPALE (mainmatter) ####

%\pagenumbering{arabic} %Pagine con i numeri arabi + reset a 1
%\rfoot{\thepage\ di \pageref{LastPage}} %Pagina n di m, con numeri Arabi; usa il pacchetto "lastpage", in caso non sia possibile usare tale pacchetto mettere al fondo dell'ultima pagina "\label{LastPage}"
%\clearpage