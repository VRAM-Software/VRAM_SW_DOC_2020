% Da inserire  subito dopo il "\begin{document}"

% #### FRONTESPIZIO (frontmatter) ####
\pagenumbering{gobble} %Toglie il numero di pagina
\begin{titlepage}
	\begin{center}
		\includegraphics[scale=0.9]{img/logo} \\ %Logo
		\vspace{1cm} %Aggiunge uno spazio verticale di 1 cm
		
		{\LARGE \DocAuthor} \\ %Autore, prende variabile definita nel main.tex
		\vspace{0.5cm} %Atenzione a mettere il punto e NON la virgola
		
		{\Huge \textbf{\DocTitle}} \\ %Titolo, prende variabile definita nel main.tex
		\vspace{0.5cm}
		
		\DocDate \\ %Data, prende variabile definita nel main.tex
		\vspace{1cm}
		
		%Allineamento colonne: l=left r=right c=center, 
		%va specificato per ogni colonna
		%Se si vuole la riga tra colonne mettere "|"
		
		\begin{tabular}{l | r} 
			Mario        & Rossi \\    %Elementi colonne separate da "&"
			Ermenegildo	 & Bianchi \\  %Le righe finiscono con "\\"
		\end{tabular}
		
		
	\end{center}
\end{titlepage}
\clearpage

% #### INDICE (tableofcontents )####
\pagenumbering{Roman} %Pagine con i  numeri romani + reset a I
\renewcommand{\footrulewidth}{1pt} %Di default footrulewidth==0 e quindi è invisibile
\tableofcontents %Provoca la stampa dell'indice
\label{LastPageRoman} %Indica l'ultima pagina dell'indice
\rfoot{\thepage\ di \pageref{LastPageRoman}} %Pagina n di m, con numeri Romani
\clearpage

% #### PARTE PRINCIPALE (mainmatter) ####

\pagenumbering{arabic} %Pagine con i numeri arabi + reset a 1
\rfoot{\thepage\ di \pageref{LastPage}} %Pagina n di m, con numeri Arabi; usa il pacchetto "lastpage", in caso non sia possibile usare tale pacchetto mettere al fondo dell'ultima pagina "\label{LastPage}"
\clearpage