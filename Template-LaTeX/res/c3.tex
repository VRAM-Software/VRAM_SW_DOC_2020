\section{Capitolato 3 - NaturalAPI}
\subsection{Informazioni generali}
\begin{itemize}
	\item Nome: NaturalAPI;
	\item Proponente: Teal Blue;
	\item Committente: Prof. Tullio Vardanega e Prof. Riccardo Cardin;
\end{itemize}
\subsection{Descrizione}
L'obbiettivo di questo capitolato è quello di far parlare a tutti gli $stakeholders_G$ un linguaggio comune, in modo da velocizzare ed evitare confusione durante la progettazione. Teal Blue vuole utilizzare le specifiche e i requisiti di un software, scritti in linguaggio naturale (Inglese, Italiano, ecc.), per generare $API_G$.
\subsection{Finalità del progetto}
Il prodotto da sviluppare è \textbf{NaturalAPI} un $toolkit_G$ che dovrà generare $API_G$ complete e test automatici a partire da un linguaggio naturale. Per la realizzazione di questo prodotto dovranno essere sviluppati tre $PoC_G$ o $feature_G$.
\begin{enumerate}
	\item \textbf{NaturalAPI Discover: }\textit{estrattore di $BDL_G$}. \\*
	L'estrattore \textit{Discover} ha lo scopo di estrarre entità, processi e combinazioni tra essi da un documento testuale di business.
	\item \textbf{NaturalAPI Design: }\textit{$parser_G$ di scenari e caratteristiche}. \\*
	Questa $PoC_G$ dovrà creare una $BAL_G$ $API_G$ in tempo reale a partire dai documenti di $Gherkin_G$ e da un $BDL_G$.
	\item \textbf{NaturalAPI Develop: }\textit{esportatore di linguaggio}. \\* 
	Questa $feature_G$ dovrà convertire un $BAL_G$ in casi di test e $API_G$ nel linguaggio di programmazione scelto, supportando la creazione e l'aggiornamento di nuove $repository_G$.
\end{enumerate} 
Ogni $feature_G$ sopraindicata dovrà essere accessibile attraverso almeno due dei seguenti modi: interfaccia da linea di comando, $GUI_G$ minimale o un'interfaccia web $REST_G$.
La parte logica del prodotto finale dovrà essere esportata in una delle seguenti modalità: come una libreria, come parte di un eseguibile o come un processo indipendente locale o remoto.  


\subsection{Tecnologie interessate}
\begin{itemize}
	\item $NLP_G$ o Natural Language Processing cioè un trattamento informatico del linguaggio naturale, che si occupa della realizzazione di sistemi in grado di comprendere il linguaggio naturale	
	\item $Dependency Parser_G$ cioè un parser che si occupa di analizzare la struttura grammaticale di una frase in linguaggio naturale per identificare le relazioni tra parole chiavi e parole che le modificano
	\item $BDD_G$ Behaviour Driven Development che è un processo di sviluppo software agile che ha, alla sua base, una continua comunicazione tra tutti gli $stakeholders_G$ di un progetto informatico
	\item $Hiptest_G$ e $Cucumber_G$ che sono degli strumenti software che supportano il processo $BDD_G$; il primo serve per eseguire test automatici, mentre il secondo legge le specificazioni software in linguaggio naturale, scritte con alcune regole di sintassi ($Gherkin_G$), e controlla che il software rispetti i requisiti
	\item Generazione $API_G$ e $DLS_G$ utilizzando:
	\begin{itemize}
		\item $OpenAPI_G$: uno standard per descrivere $API_G$
		\item $Swagger_G$: un $framework_G$ che aiuta a sviluppare servizi web $REST_G$
		\item $OWL_G$: un linguaggio web semantico
	\end{itemize}
	\item Un qualsiasi $framework_G$ a scelta come $Qt_G$, $React_G$, ecc.
\end{itemize} 
\subsection{Aspetti positivi}
\begin{itemize} 
	\item Il proponente non impone vincoli sui linguaggi di programmazione da utilizzare, questo permette  al gruppo di sviluppare un prodotto con un linguaggio o un $framework_G$ considerato più interessante. 
	\item I requisiti, le tecnologie e il modo di esportare il prodotto finale sono spiegati in modo preciso e esauriente.
	\item L'azienda Teal Blue si dimostra disponibile a incontri ed a una comunicazione aperta con il fornitore.
\end{itemize}
\subsection{Criticità e fattori di rischio}
\begin{itemize}
	\item Questo capitolato non ha suscitato molto interesse nel gruppo a causa dell'eccessiva astrattezza del capitolato e per il fatto che i concetti su cui prepararsi non erano, per il gruppo VRAM Software, stimolanti. 
	\item Le tecnologie da imparare sono molte ed è arduo quantificare il tempo necessario per raggiungere una preparazione sufficiente per gestire in modo produttivo le tecnologie elencate.
\end{itemize}
\subsection{Conclusione}
Il gruppo ha trovato interessante l'idea di Teal Blue di rendere la comunicazione tra tutti $stakeholders_G$ chiara e veloce, tuttavia abbiamo deciso di orientarci verso un progetto meno astratto.