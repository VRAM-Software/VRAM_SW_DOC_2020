\subsection{Capitolato 5 - Stalker}

\subsection{Informazioni}
\begin{itemize}
	\item \textbf{Nome}: Stalker
	\item \textbf{Proponente}: Imola Informatica
	\item \textbf{Committente}: Prof. Tullio Vardanega
\end{itemize}

\subsection{Descrizione}
Soluzione software composta di applicazione mobile che si appoggia su un'infrastruttura server, con lo scopo di monitorare le presenze in forma anonima o autenticata in uno o più luoghi circoscritti.
Gli ambiti di utilizzo possono variare, dal controllare l'affluenza in luoghi di interesse al verificare le presenze del personale nel luogo di lavoro.

\subsubsection{Requisiti applicazione mobile}
\begin{itemize}
	\item Recupero lista organizzazioni
	\item Possiblità di effettuare login tramite LDAP per organizzazioni che lo richiedono
	\item Storico accessi
	\item Visualizzazione in tempo reale della propria presenza all'interno di un luogo monitorato e cronometro del tempo trascorso al suo interno
\end{itemize}

\subsection{Requisiti interfaccia web lato server}
\begin{itemize}
	\item Funzionalità di login
	\item Creazione/modifica/eliminazione di organizzazioni
	\item Aggiunta/modifica/rimozione di luoghi, definiti da coordinate geografiche
	\item Configurazione collegamento server LDAP
	\item Invio di notifiche push alle applicazioni per segnalare l'aggiornamento delle liste di organizzazioni e luoghi
	\item Monitoraggio del numero di utenti presenti nei luoghi dell'organizzazione in tempo reale
	\item Ricerca sugli accessi dei dipendenti e creazione di report sulla frequentazione dei luoghi
	\item Gestione delle autorizzazioni per gli utenti dell'interfaccia web
\end{itemize}

\subsection{Requisiti dell'infrastruttura}
\begin{itemize}
	\item Comunicazioni solo all'entrata od uscita dall'area interessata
	\item Cifratura di tutte le comunicazioni tra app e server
	\item Architettura server scalabile (verticalmente e orizzontalmente) e tollerante a picchi di traffico
	\item Test di carico che dimostrino il funzionamento in varie situazioni
\end{itemize}

\subsubsection{Tecnologie interessate}
\begin{itemize}
	\item Java o Swift (applicazione mobile)
	\item NodeJS o python (back-end)
	\item utilizzo di protocolli asincroni per le comunicazioni tra app e server
	\item HTML5, CSS3 e Javascript (interfaccia web lato server)
	\item Utilizzo del pattern di Publisher/Subscriber, ovvero mittenti e destinatari dimessaggi dialogano attraverso un tramite(dispatcher)
	\item Utilizzo dell’IAAS Kubernetes o di un PAAS, Openshift o Rancher, per il rilascio delle componenti server
	\item API REST esposte dal server, o gRPC in alternativa
	\item GPS/sistemi ibridi di geolocalizzazione
	\item LDAP (Lightweighgt Directory Access Protocol)
	\item Test unitari e d'integrazione per tutte le componenti applicative
\end{itemize}
\subsubsection{Aspetti positivi}
\begin{itemize}
	\item Lo sviluppo di applicazioni mobili è una conoscenza richiesta
	\item Argomento interessante, anche grazie alla potenziale utilità nell'ambito della sicurezza
\end{itemize}
\subsubsection{Criticità e fattori di rischio}
\begin{itemize}
	\item I requisiti sono numerosi e importanti
	\item La precisione ottenibile con le tecnologie attuali non è sufficiente per rendere l’applicazione usabile nella realtà
	\item Nella presentazione è stato discusso il funzionamento del sistema GPS ma completamente tralasciata la funzionalità dei backend di geolocalizzazione presenti nei moderni sistemi mobile (ad esempio i Google Mobile Services per Android)
	\item Lo sviluppo lato server non è stato ben discusso durante la presentazione e potrebbe portare complicazioni
\end{itemize}
\subsubsection{Conclusioni}
Nonostante il nobile obiettivo, la difficile fattibilità di un prodotto concreto e la possiblità di complicazioni scoraggia il gruppo nella decisione di intraprendere questo capitolato.
