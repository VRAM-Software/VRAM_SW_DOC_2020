\subsection{Capitolato C2 - \textit{Etherless}}
    \subsection{Informazioni generali}
        \begin{itemize}
            \item \textbf{Nome}: Etherless;
            \item \textbf{Proponente}: Red Babel;
            \item \textbf{Committente}: Prof. Tullio Vardanega e Prof. Riccardo Cardin.
        \end{itemize}
    \subsection{Descrizione}
        Etherless è una \textit{Cloud Application Platform} che permette agli sviluppatori che la utilizzano di caricare nel cloud delle funzioni
        \textit{JavaScript}. Tali procedure possono poi essere acquistate da terzi tramite l'impiego della criptovaluta \textit{Ethereum} implementata
        grazie alla tecnologia della \textit{blockchain}.
    \subsection{Obiettivi di progetto}
        Il progetto si propone di aiutare gli sviluppatori fornendo:
        \begin{itemize}
            \item un framework \textit{Serverless} che gestisca il costo computazionale della funzione;
            \item un servizio di \textit{Smart Contracts} che sovrintenda il processo di pagamento.
        \end{itemize}
        Quest'ultima tecnologia sarà di supporto anche per l'acquirente in quanto assicurerà il completamento della transazione solo a lavoro verificato e completato.
    \subsection{Tecnologie interessate}
        Per l'implementazione delle varie funzionalità vengono date dall'azienda delle linee guida sulle tecnologie da utilizzare:
        \begin{itemize}
            \item \textbf{Typescript 3.6}: linguaggio di programmazione da impiegare, tramite l'approccio \textit{Promise} o \textit{async-await}, nello sviluppo della piattaforma \textit{Etherless};
            \item \textbf{Solidity}: linguaggio per la creazione e la gestione degli \textit{Smart Contracts};
            \item \textbf{AWS Lambda}: piattaforma computazione serverless fornita da \textit{Amazon} per la coordinazione degli eventi;
            \item \textbf{Serverless Framework}: framework Web per la creazione di applicazioni su \textit{AWS Lambda};
            \item \textbf{typescript-eslint}: strumento di analisi statica del codice \textit{Typescript}.
        \end{itemize}
        \textit{AWS Lambda} può essere corredata con altri componenti quali: \textit{AWS API Gateway} (per eventi HTTP), \textit{AWS DynamoDB} (database non relazionale)
        o \textit{AWS S3} (servizio di memorizzazione). Tutto questo poi può essere gestito tramite \textit{AWS CloudFormation} (piattaforma per l'organizzazione delle
        risorse AWS).
    \subsection{Requisiti di progetto}
        È richiesto di dividere lo sviluppo in tre fasi:
        \begin{itemize}
            \item \textbf{Local}: utilizzo dell'applicativo in ambiente locale, in cui può essere utilizzato \textit{Ethereum testrpc} di \textit{Truffle} per
            l'emulazione della blockchain;
            \item \textbf{Test}: utilizzo dell'applicativo in ambiente di testing, in cui può essere usata la soluzione proposta al punto precedente;
            \item \textbf{Staging}: utilizzo dell'applicativo in un ambiente pubblico, in tal caso si potrà usufruire di \textit{Ropsten Ethereum} come rete di testing.
        \end{itemize}
        Inoltre è obbligatorio separare il software in tre parti:
        \begin{itemize}
            \item \textbf{etherless-cli}: modulo attraverso il quale lo sviluppatore interagisce con Etherless;
            \item \textbf{etherless-smart}: modulo per l'interazione tra \textit{etherless-cli} e la parte server;
            \item \textbf{etherless-server}: modulo che ascolta gli eventi emessi da \textit{etherless-smart} per attivare le rispettive funzioni lambda. 
        \end{itemize}
        Ognuna di queste operazioni dovrà poi essere caricata e versionata tramite \textit{GitHub} o \textit{GitLab}.
    \subsection{Aspetti positivi}
    \begin{itemize}
        \item Ethereum e la tecnologia della blockchain più in generale sono tematiche molto attuali e innovative che suscitano interesse tra i componenti del gruppo
        anche per le possibili applicazioni future;
        \item L'azienda ha avvallato delle richieste chiare e precise, aspetto valutato positivamente dal gruppo.
    \end{itemize}
    \subsection{Criticità e fattori di rischio}
    \begin{itemize}
        \item L'azienda ha sede in Olanda, il rapporto con i proponenti potrebbe essere quindi meno efficace;
        \item Seppur siano proposte tematiche interessanti il gruppo ha presentato dei dubbi riguardanti la difficoltà e le tempistiche per l'approfondimento
        delle tecnologie presentate.
    \end{itemize}
    \subsection{Conclusioni}
    Il capitolato è apparso stimolante per quanto riguarda le tecnologie utilizzate e convincente nella sua esposizione. Tuttavia la distanza geografica e la
    mole di lavoro prevista sono risultate un ostacolo nell'effettiva realizzazione del software. Per questo motivi il gruppo ha deciso di vertere la sua scelta
    su un altro progetto.
