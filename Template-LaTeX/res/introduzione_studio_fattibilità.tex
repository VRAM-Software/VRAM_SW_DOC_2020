\section{Introduzione}
    \subsection{Scopo del documento}
        Nel seguente documento verranno analizzati i capitolati presentati nell'ambito del progetto di Ingegneria del Software. Saranno
        esposte le ragioni che hanno portato il gruppo alla scelta di C6 (\textit{ThiReMa - Things Relationship Management}) e
        alla conseguente esclusione delle altre proposte.
    \subsection{Glossario}
        Per evitare eventuali ambiguità relative a termini tecnici o altamente specifici questo documento ed i successivi verranno
        corredati da un \textit{Glossario} dove verranno illustrate tali parole. Le voci interessate saranno indicate da una 'G' a
        pedice.
    \subsection{Riferimenti}
        \subsection{Normativi}
            \begin{itemize}
                \item \textbf{Norme di Progetto}: \textit{Norme di Progetto}.
            \end{itemize}
        \subsection{Informativi}
            \begin{itemize}
                \item \textbf{Capitolato d'appalto C1 - Autonomous Highlights Platform}: https://www.math.unipd.it/~tullio/IS-1/2019/Progetto/C1.pdf
                \item \textbf{Capitolato d'appalto C2 - Etherless}: https://www.math.unipd.it/~tullio/IS-1/2019/Progetto/C2.pdf
                \item \textbf{Capitolato d'appalto C3 - NaturalAPI}: https://www.math.unipd.it/~tullio/IS-1/2019/Progetto/C3.pdf
                \item \textbf{Capitolato d'appalto C4 - Predire in Grafana}: https://www.math.unipd.it/~tullio/IS-1/2019/Progetto/C4.pdf
                \item \textbf{Capitolato d'appalto C5 - Stalker}: https://www.math.unipd.it/~tullio/IS-1/2019/Progetto/C5.pdf
                \item \textbf{Capitolato d'appalto C6 - ThiReMa - Things Relationship Management}: https://www.math.unipd.it/~tullio/IS-1/2019/Progetto/C6.pdf
            \end{itemize}
